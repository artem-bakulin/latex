\documentclass[a4paper,14pt]{extarticle}
\usepackage{cmap}				% To be able to copy-paste russian text from pdf			
\usepackage[utf8]{inputenc}
\usepackage[T1]{fontenc}
\usepackage[margin=1in]{geometry}
\usepackage[english, russian]{babel}

\usepackage[hyphens]{url}
\urlstyle{same}
\usepackage{hyperref}

\usepackage{multirow}
\usepackage{graphicx}
\usepackage{caption}
\usepackage{amsmath}
\usepackage{mathtools}
\usepackage[normalem]{ulem}

\usepackage[titles]{tocloft}
\setlength{\cftbeforesecskip}{2pt} % Remove spacing in table of contents

\numberwithin{equation}{section} % Use section number in equation numbers
\numberwithin{figure}{section}
\numberwithin{table}{section}

\usepackage{titlesec}
\newcommand{\sectionbreak}{\clearpage} % Start new page with each section

\usepackage{tikz}
\usepackage{pgfplots}
\usetikzlibrary{arrows}
\usepgfplotslibrary{groupplots,colorbrewer,dateplot,statistics}

%\def\ishtml{1}
\ifdefined\ishtml
  % HTML mode
  \newcommand{\urlnote}[2]{\href{#2}{#1}} % Make cool link 
  \newcommand{\smallsep}{thinspace} % to be replaced with unicode 8239 later
\else
  % PDF mode
   \usepackage{libertine}
   \usepackage{libertinust1math}
   \newcommand{\urlnote}[2]{#1\endnote{\url{#2}}}  % Put URLs to endnotes
   \newcommand{\smallsep}{\kern 0.1em}
\fi

% Move footnotes to end of document
\usepackage[backref=true]{enotez}
\DeclareTranslation{russian}{enotez-title}{Примечания}

\usepackage[
	output-decimal-marker={,},
	group-separator={\smallsep},
	group-minimum-digits=3
]{siunitx}

% Shoot me if I know a better way to make decimal groups of two
\newcommand{\rateone}[1]{\num{#1}}
\newcommand{\ratetwo}[2]{\num{#1}\smallsep#2}
\newcommand{\ratethree}[3]{\num{#1}\smallsep#2\smallsep#3}

\newcommand{\ru}[1]{\begin{otherlanguage}{russian}#1\end{otherlanguage}}
\newcommand{\en}[1]{\begin{otherlanguage}{english}#1\end{otherlanguage}}
\newcommand{\ruen}[2]{#1 (\en{#2})}

\usepackage[style=alphabetic, backend=biber]{biblatex}
\addbibresource{index.bib}
\renewcommand*{\bibfont}{\small}
\setcounter{biburllcpenalty}{9000}
\setcounter{biburlucpenalty}{9500}

\author{Артём Бакулин}
\date{\today}
\title{Индексное инвестирование (продолженние)}

\begin{document}

\section{Соотношение <<цена-прибыль>> и соотношение Шиллера}

%Предположим, что мы изучаем компанию, которая за прошлый год заработала \dollars{1000000} чистой прибыли. Эта компания выпустила \num{10000} акций, поэтому на каждую акцию проходится \dollars{100} чистой прибыли. Совет директоров компании может как выплатить эти деньги акционерам в качестве дивидендов, так и вложить их в дальнейшее развитие бизнеса.

Пожалуй, самая заезженная формула в инвестициях --- это соотношение \ruen{<<цена/прибыль>>}{Price/Earnings, P/E}. Чтобы посчитать отношение $P/E$ рынка акций, нужно разделить капитализацию рынка (суммарную стоимость всех акций всех компаний) на общую прибыль всех компаний. Например, если все акции всех компаний вместе взятые стоят \dollars{30} триллионов, и компании заработали в прошлом году \dollars{1} триллион, то соотношение $P/E$ равно 30. Покупая сейчас индекс акций, вы покупаете компании за 30 годовых прибылей.

Поскольку простая формула $P/E$ выглядит недостаточно солидно, обычно аналитики заменяют прибыль прошлого года $E$ на среднюю прибыль за последние 10 лет. Надо признать, что в в этом есть смысл.

Прибыли компаний меняются с течением времени. В какие-то годы экономика на подъёме и прибыли компаний оказываются выше среднего. В не столь удачные годы экономика замедляется и вместе с ней снижаются прибыли компаний. Если взять среднее за достаточно длинный период, лет 10--12, то мы сгладим колебания, связанные с экономическими циклами. Нужно только не забыть, что из-за инфляции \$1 10 лет назад и \$1 в прошлом году --- это немного разные доллары. Чтобы складывать яблоки с яблоками, нужно домножить прибыли прошлых лет на накопленную инфляцию.

Поздравляю, вы только что освоили секретное оружие финансовых аналитиков: соотношение \ruen{<<цена-прибыль>> с поправкой на цикличность}{cyclically-adjsuted price/earnings, CAPE}. Его ещё иногда называют \ruen{соотношеним Шиллера}{Shiller ratio} в честь одного из первоткрывателей, профессора \ruen{Роберта Шиллера}{Robert Shiller} \cite{campbell1988dividend}.

Теперь, когда мы разобрались, что такое соотношение Шиллера, самое время посмотреть, каким оно было в прошлом и какое оно сейчас. Пожалуйста, уведите от экрана беременных женщин и детей. Сейчас вы увидите шокирующий график, который, по мнению некоторых  финансовых аналитиков, предсказывает неминуемый крах фондового рынка США. Внимание на рисунок \ref{shiller_pe_historical_chart}.

\begin{figure}[ht]
\centering
\begin{tikzpicture}
\begin{axis}[
    width=\textwidth,
    date coordinates in=x,
    date ZERO=1880-01-01,
    xtick={1880-01-01,1900-01-01,1920-01-01,1940-01-01,1960-01-01,1980-01-01,2000-01-01,2020-01-01},
    minor xtick={1890-01-01,1910-01-01,1930-01-01,1950-01-01,1970-01-01,1990-01-01,2010-01-01},
    xticklabel=\year,
    grid=both,
    xmin=1880-01-01,
    xmax=2025-01-01,
    ylabel={Соотношение $P/E$ с поправкой на цикличность},
]
    
        \addplot[
            color = Set1-B,
            line width = 1pt
        ]
        table[
            x = month,
            y = shiller_cape,
            col sep = comma
        ]
        {data/shiller_cape.csv};        
\end{axis}
\end{tikzpicture}
\caption{Соотношение <<цена/прибыль>> с поправкой на цикличность (CAPE) для рынка акций США, 1881--2020. Данные: Robert Shiller Online.}
\label{shiller_pe_historical_chart}
\end{figure}

Текущее значение (35.0 по состоянию на март 2021) выглядит довольно высоким по историческим меркам. В прошлом лишь дважды акции стоили дороже, чем 30 годовых прибылей: в 1929-м аккурат перед Великой депрессией и в 2000-м на пике пузыря интернет-компаний. В первом случае инвесторам пришлось пережить падение рынка акций на 85\%, а во втором --- на 56\%. И вот история повтоярется в третий раз. Снова на рынке акций надулся пузырь. Мы стоим на пороге биржевого краха, после которого живые позавидуют мёртвым. Скажите, вам уже страшно?

Ирония в том, что аналитики продолжают ссылаться на авторитет Шиллера (нобелевский лауреат, как-никак), хотя он сам не так давно написал, что, возможно, никакого пузыря-то и нет. По его словам, высокое отношение $P/E$ может быть связано с низкими процентыми ставками и не обязательно является предвестником финансового шторма \cite{shiller2020cape}.

\section{Цены акций и процентные ставки. Формула Гордона}

Чтобы понять аргументацию Шиллера и соавторов, нам нужно разобраться, как связаны цены акций и процентные ставки.

Представим, что некая компания зарабатывает прибыль $E$, которую полностью выплачивает в виде дивидендов. Компания развивается, и каждый год прибыль растёт на $g$ процентов. Первая выплата $E$ доларов случится через год, через два года компания выплатит $E(1+g)$, через три года $E(1+g)^2$ и так далее до бесконечности. Компания никогда не разорится и будет ежегодно выплачивать  растущие дивиденды. Внимание, вопрос: сколько стоит такая компания?

На первый взгляд, компания в будущем выплатит акционерам бесконечное количество долларов, поэтому её цена --- тоже бесконечность. Это простое рассуждение неверно, потому что оно суммирует будущие доллары, которые вовсе не равны сегодняшним.

Во-первых, инфляция со временем съедает покупательную способность денег. Скорее всего, через 10 лет вы купите на \$100 меньше товаров, чем могли бы купить на \$100 сегодня.

Во-вторых, даже если инфляции нет вовсе, то всё равно люди при прочих равных предпочитают потребление сегодня потреблению в будущем. Поэтому доллары, полученные от нашей компании через 10 лет, менее ценны, чем доллары, полученные через 5 лет, и тем более менее ценны, чем сегодняшние доллары.

Чтобы корректно складывать будущие доллары, их нужно привести к сегодняшним, как если бы конвертировали мили и вёрсты в метры. Для этого нам понадобится так называемая процентная \ruen{ставка дисконтирования}{discount rate}. Если обозначить ставку дисконтирования $r$, то \ruen{текущая стоимость}{present value, PV} будущих $E$ долларов, которые мы получим через $T$ лет, равна
\begin{align*}
PV = \frac{E}{(1 +r)^T}
\end{align*}

Например, если ставка дисконтирования равна 10\%, то текущая стоимость \dollars{100}, которые вам выплатят через год, равна $\dollars{100}/(1 + 0.1) \approx \dollars{90.91}$. Если же доллары выплатят через 10 лет, то их текущая стоимость равна $\dollars{100} / (1+0.1)^{10} \approx \dollars{38.5}5$.

Другими словами, \dollars{38.55} сегодня имеют для вас такую же ценность, как и \dollars{100} через 10 лет. Чтобы убедить вас не потратить \dollars{38.55} сегодня, а вложить их на 10 лет, нужно пообещать вам доходность как минимум 10\% годовых (выплата должна быть как минимум \dollars{100}). В противном случае будущие доллары не перевесят сегодняшние, и вы откажетесь от инвестиций.

Вернёмся к нашей <<вечной>> компании. Составим таблицу \ref{gordon_growth_table}, в которой перечислим все будущие платежи и их текущую стоимость с учётом ставки дисконтирования $r$.

\begin{table}[ht]
\centering
\begin{tabular}{c|c|c}
Год & Выплата & Текущая стоимость \\
\hline
1 & $E$ & $E / (1+r)$ \\
2 & $E(1+g)$ & $E(1+g) / (1+r)^2$ \\
3 & $E(1+g)^2$ & $E(1+g)^2 / (1+r)^3$ \\
... & ... & ... \\
$n$ & $E(1+g)^{n-1}$ & $E(1+g)^{n-1} / (1+r)^n$ \\
... & ... & ...
\end{tabular}
\caption{Текущая стоимость платежей <<вечной>> компании.}
\label{gordon_growth_table}
\end{table}

Вспомним формулу суммы бесконечной геометрической прогрессии и запишем сумму всех выплат с учётом дисконтирования:
\begin{align}
P = \frac{E}{1 +r } + \frac{E(1+g)}{(1+r)^2} + ... + \frac{E(1+g)^{n-1}}{(1+r)^n} + ... = \dfrac{\dfrac{E}{1 + r}}{1 - \dfrac{1 + g}{1 + r}} = \frac{E}{r - g}
\label{gordon_formula_nominal_rates}
\end{align}

Формула (\ref{gordon_formula_nominal_rates}) называется формулой Гордона или \ruen{моделью роста Гордона}{Gordon growth model} \cite{gordon1956capital}. Она связывает цену акций $P$ с прибылью $E$, ставкой дисконтирования $r$ и темпом роста прибыли $g$.

Обратите внимание, что чем выше ставка дисконтирования $r$, тем дешевле акция. С другой стороны, чем выше темп роста прибыли $g$, тем дороже компания. Например, если компания вечно выплачивает дивиденды \dollars{100} при скорости роста $g=0\%$, то при ставке дисконтирования $r=10\%$ она стоит $\dollars{100} / 0.1 = \dollars{1000}$. Если при тех же выплатах ставка снижается до 5\%, то компания стоит $\dollars{100  / 0.05} = \dollars{200}$, то есть в два раза дороже.

\section{Соотношение <<цена-прибыль>> и реальные процентные ставки}

Добавим в нашу модель инфляцию. Пусть цены на все товары в экономике растут на $i$ процентов в год. Заменим ставку дисконтирования $r$ в номинальных процентах на реальную ставку $r^*$. Точно так же поступим со скоростью роста прибыли $g$: заменим номинальные проценты на реальные проценты сверх инфляции:
\begin{align}
\begin{cases}
r = r^* + i \\
g = g^* + i
\end{cases}
\Leftrightarrow
\quad
\begin{cases}
r^* = r - i \\
g^* = g - i
\end{cases}
\label{real_rates_formula}
\end{align}

Например, если номинальная процентная ставка $r$ равна 10\% (на каждый вложенный доллар вы получаете \$1.10-через год), а ожидаемая инфляция составляет 4\%, то реальная процентная ставка сверх инфляции равна 6\%. Если вы инвестировали сумму, равную цене одного эскимо, то через год вы получите сумму, эквивалентную 1.06 эскимо.

Подставим выражения (\ref{real_rates_formula}) в формулу Гордона  (\ref{gordon_formula_nominal_rates}). Обратите внимание, что инфляция $i$ сократилась:
\begin{align}
P = \frac{E}{r - g} = \frac{E}{(r^* + i) - (g^* + i)} =  \frac{E}{r^* - g^*} 
\label{gordon_formula_real_rates}
\end{align}

Двигаемся дальше. От чего зависит реальная ставка дисконтирования $r^*$? Её можно расписать как сумму двух компонет: \ruen{безрисковой}{risk-free}\ реальной ставки $f^*$ и премии за риск $\pi$:
\begin{align}
r^* = f^* + \pi
\label{risk_premium_formula}
\end{align}

Безрисковая реальная ставка $f^*$ отражает стоимость переноса потребления из сегодя в будущее. Например, если вы цените одно эскимо сегодня точно так же, как 1.06 эскимо через год, то ваша личная реальная процентая ставка $f^*$ равна $6\%$.

Премия за риск $\pi$ вознаграждает вас за неопределённость будущего. Редкая инвестиция является по-настоящему безрисковой. Иногда инвестиции оборачиваются потерями: вложили 1 эксимо, а через год получили только половинку. Чтобы компенсировать ваши страдания в плохом случае, ожидаемая доходность инвестиций должна быть выше, чем безрисковая ставка. Эта прибавка и будет премией за риск. Подробнее о премии за риск можно прочитать в \href{https://habr.com/ru/company/dbtc/blog/527050/}{одной из недавних статей}.

С учётом премии за риск (\ref{risk_premium_formula}), формула Гордона (\ref{gordon_formula_real_rates}) превращается в 
\begin{align}
P =\frac{E}{r^* - g^*} = \frac{E}{f^* + \pi - g^*}
\quad
\Rightarrow
\quad
P/E = \frac{1}{f^* + \pi - g^*}
\label{gordon_formula_with_risk_premium}
\end{align}

Формула (\ref{gordon_formula_with_risk_premium}) позволяет выделить три возможные причины роста соотношения <<цена--прибыль>> $P/E$ в последние годы. Во-первых, могла уменьшится безрисковая реальная ставка $f^*$. Во-вторых, могла уменьшиться премия за риск $\pi$, которую требуют инвесторы. Наконец, в-третьих, мог увеличиться ожидаемый темп роста прибыли $g^*$.

Так вот, согласно Шиллеру, именно низкая безрисковая реальная ставка $f^*$ объясняет высокое соотношение $P/E$.

\section{Реальные процентные ставки}

Можем ли мы залезть в голову инвесторам и посмотреть, какую безрисковую реальную ставку $f^*$ они закладывают в цены активов? Да, можем. С конца 90-х годов Казначейство США выпускает \ruen{государственные облигации, защищённые от инфляции}{Treasury Inflation-Protected Securities, TIPS}.

Например, если у вас есть бумага TIPS с номиналом \dollars{1000} и сроком погашения 10 лет, то правительство США обязуется выплатить вам \dollars{1000}, умноженные на накопленную за 10 лет инфляцию. Если инфляция за 10 лет составит 20\% (по 1.8\% в год), то вы получите \dollars{1200}, а если 100\% (по 7.2\% в год), то \dollars{2000}.

Мы можем посмотреть, по какой цене инвесторы продают и покупают бумаги TIPS на рынке, и вычислить, какую реальную доходность они рассчитывают получить. Поскольку дефолт по государственным облигациям США --- крайне маловероятное событие, то полученная доходность будет хорошим приближением теоретической безрисковой реальной ставки.

На рисунке \ref{treasury_yields_figure} показаны доходности TIPS и обычных гос. облигаций с 2003-го года. Текущая доходность TIPS отрицательная и составляет $-0.92\%$ годовых. Да, вы всё правильно поняли, инвесторы платят правительству США за привилегию дать ему деньги в долг. Чтобы через 10 лет получить от правительства США поправленый на инфляцию эквивалент сегодняшних \dollars{1000}, нужно сегодня дать ему в долг примерно $\dollars{1000} / (1 + 0.0092)^{10} \approx \dollars{1096}$.

К слову, доходность обычных гос. облигаций (без защиты от инфляции) составляет 1.26\% годовых. Можно сделать вывод, что в среднем участники рынка ожидают инфляцию на уровне $1.26\% - (-0.92\%) = 2.18\%$ в год. Именно при таком уровне инфляции ни инвесторы в обычные облигации, ни инвесторы в TIPS, не получат преимущества друг перед другом.

\begin{figure}[ht]
\centering
\begin{tikzpicture}
\begin{axis}[
    width=\textwidth,
    date coordinates in=x,
    date ZERO=1880-01-01,
    xtick={2003-01-01,2005-01-01,2007-01-01,2009-01-01,2011-01-01,2013-01-01,2015-01-01,2017-01-01,2019-01-01,2021-01-10},
    minor xtick={2004-01-01,2006-01-01,2008-01-01,2010-01-01,2012-01-01,2014-01-01,2016-01-01,2018-01-01,2020-01-01},
    xticklabel=\year,
    yticklabel={\pgfmathprintnumber{\tick}\%},
    grid=both,
    xmin=2003-01-01,
    xmax=2022-01-01,
    ylabel={Доходность (проценты годовых)},
    ylabel shift = -10pt,
    legend entries={
        Облигации с фиксированным купоном (T-Note),
        Облигации с защитой от инфляции (TIPS)
    },
    legend pos=north east,
    legend style={font=\small},
    legend cell align={left}
]
        
    \newcommand{\plotYield}[5]{
        \addplot[
            color = #2,
            line width = 1pt,
            mark = #3,
            mark repeat = 6,
            %mark phase = 6,
            mark options = {scale=2},
        ]
        table[
            x = DATE,
            y = #1,
            col sep = comma
        ]
        {data/#1.csv};   
        
        \node[
            circle,
            fill,
            inner sep = 2pt,
            color = #2
        ]
        at (axis cs: #4, #5) {};
        
        \node[
            anchor=south
        ]
        at (axis cs: #4, #5)
        {#5};
    }
        
    \plotYield{GS10}{Set1-A}{square}{2021-02-01}{1.26}
    \plotYield{FII10}{Set1-B}{none}{2021-02-01}{-0.92}
\end{axis}
\end{tikzpicture}
\caption{Доходность десятилетних гос. облигаций США: облигации с фиксированным купоном (T-Note) и облигации, защищённые от инфляции (TIPS). Данные: Federal Reserve Bank of St. Louis \cite{fredGS10}, \cite{fredFII10}.}
\label{treasury_yields_figure}
\end{figure}

К сожалению, облигации с защитой от инфляции появились не так давно. График на рисунке \ref{treasury_yields_figure} не просто так начинается с 2003-го года. Чтобы оценить реальные процентные ставки в далёком прошлом, нужно выкручиваться. Профессор Шиллер предлагает вычесть из текущей доходности обычных десятилетних облигаций инфляцию за предыдущие 10 лет. Например, если сейчас десятилетние облигации дают доходность 1.26\%, а инфляция за предыдущие 10 лет составила 1.71\% в год, то реальная доходность по Шиллеру равна $1.26\% - 1.71\% = -0.45\%$.

\begin{figure}[ht]
\centering
\begin{tikzpicture}
\begin{axis}[
    width=\textwidth,
    date coordinates in=x,
    date ZERO=1880-01-01,
    xtick={1880-01-01,1900-01-01,1920-01-01,1940-01-01,1960-01-01,1980-01-01,2000-01-01,2020-01-01},
    minor xtick={1890-01-01,1910-01-01,1930-01-01,1950-01-01,1970-01-01,1990-01-01,2010-01-01},
    xticklabel=\year,
    yticklabel={\pgfmathparse{100*\tick}\pgfmathprintnumber{\pgfmathresult}\%},
    grid=both,
    xmin=1880-01-01,
    xmax=2025-01-01,
    ylabel={Доходность (проценты годовых)},
]
    
        \addplot[
            color = Set1-B,
            line width = 1pt
        ]
        table[
            x = month,
            y = cape_excess_yield,
            col sep = comma
        ]
        {data/shiller_cape.csv};        
        
        \addplot[
            color = Set1-A,
            line width = 1pt,
            style = dashed
        ]
        table[
            x = month,
            y = subsequent_stock_return_10y,
            col sep = comma
        ]
        {data/shiller_cape.csv};        
\end{axis}
\end{tikzpicture}
\caption{Соотношение <<цена/прибыль>> с поправкой на цикличность (CAPE) для рынка акций США, 1881--2020. Данные: Robert Shiller Online.}
\end{figure}

\begin{figure}[ht]
\centering
\begin{tikzpicture}
\begin{axis}[
    width=\textwidth,
    date coordinates in=x,
    date ZERO=1880-01-01,
    xtick={1880-01-01,1900-01-01,1920-01-01,1940-01-01,1960-01-01,1980-01-01,2000-01-01,2020-01-01},
    minor xtick={1890-01-01,1910-01-01,1930-01-01,1950-01-01,1970-01-01,1990-01-01,2010-01-01},
    xticklabel=\year,
    yticklabel={\pgfmathparse{100*\tick}\pgfmathprintnumber{\pgfmathresult}\%},
    grid=both,
    xmin=1880-01-01,
    xmax=2025-01-01,
    ylabel={Соотношение $P/E$ с поправкой на цикличность},
]
    
        \addplot[
            color = Set1-B,
            line width = 1pt
        ]
        table[
            x = month,
            y = long_rate,
            col sep = comma
        ]
        {data/shiller_cape.csv};        
        
        \addplot[
            color = Set1-A,
            line width = 1pt
        ]
        table[
            x = month,
            y = real_rate_10y,
            col sep = comma
        ]
        {data/shiller_cape.csv};        
\end{axis}
\end{tikzpicture}
\caption{Соотношение <<цена/прибыль>> с поправкой на цикличность (CAPE) для рынка акций США, 1881--2020. Данные: Robert Shiller Online.}
\end{figure}

\section*{Список литературы}
\en{
\printbibliography[heading = none]
}

\end{document}
