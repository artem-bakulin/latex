\documentclass[a4paper,14pt]{extarticle}
\usepackage{cmap}				% To be able to copy-paste russian text from pdf			
\usepackage[utf8]{inputenc}
\usepackage[T1]{fontenc}
\usepackage[margin=1in]{geometry}
\usepackage[english, russian]{babel}

\usepackage[hyphens]{url}
\urlstyle{same}
\usepackage{hyperref}

\usepackage{multirow}
\usepackage{graphicx}
\usepackage{caption}
\usepackage{amsmath}
\usepackage{mathtools}
\usepackage[normalem]{ulem}

\usepackage[titles]{tocloft}
\setlength{\cftbeforesecskip}{2pt} % Remove spacing in table of contents

\numberwithin{equation}{section} % Use section number in equation numbers
\numberwithin{figure}{section}
\numberwithin{table}{section}

\usepackage{titlesec}
\newcommand{\sectionbreak}{\clearpage} % Start new page with each section

\usepackage{tikz}
\usepackage{pgfplots}
\usetikzlibrary{arrows}
\usepgfplotslibrary{groupplots,colorbrewer,dateplot,statistics}

%\def\ishtml{1}
\ifdefined\ishtml
  % HTML mode
  \newcommand{\urlnote}[2]{\href{#2}{#1}} % Make cool link 
  \newcommand{\smallsep}{thinspace} % to be replaced with unicode 8239 later
\else
  % PDF mode
   \usepackage{libertine}
   \usepackage{libertinust1math}
   \newcommand{\urlnote}[2]{#1\endnote{\url{#2}}}  % Put URLs to endnotes
   \newcommand{\smallsep}{\kern 0.1em}
\fi

% Move footnotes to end of document
\usepackage[backref=true]{enotez}
\DeclareTranslation{russian}{enotez-title}{Примечания}

\usepackage[
	output-decimal-marker={,},
	group-separator={\smallsep},
	group-minimum-digits=3
]{siunitx}

% Shoot me if I know a better way to make decimal groups of two
\newcommand{\rateone}[1]{\num{#1}}
\newcommand{\ratetwo}[2]{\num{#1}\smallsep#2}
\newcommand{\ratethree}[3]{\num{#1}\smallsep#2\smallsep#3}

\newcommand{\ru}[1]{\begin{otherlanguage}{russian}#1\end{otherlanguage}}
\newcommand{\en}[1]{\begin{otherlanguage}{english}#1\end{otherlanguage}}
\newcommand{\ruen}[2]{#1 (\en{#2})}

\usepackage[style=alphabetic, backend=biber]{biblatex}
\addbibresource{index.bib}
\renewcommand*{\bibfont}{\small}
\setcounter{biburllcpenalty}{9000}
\setcounter{biburlucpenalty}{9500}

\author{Артём Бакулин}
\date{\today}
\title{Индексstock market p/eное инвестирование (продолжение)}

\begin{document}

\section*{Соотношение <<цена-прибыль>> и соотношение Шиллера}

%Предположим, что мы изучаем компанию, которая за прошлый год заработала \dollars{1000000} чистой прибыли. Эта компания выпустила \num{10000} акций, поэтому на каждую акцию проходится \dollars{100} чистой прибыли. Совет директоров компании может как выплатить эти деньги акционерам в качестве дивидендов, так и вложить их в дальнейшее развитие бизнеса.

Пожалуй, самая заезженная формула в инвестициях --- это соотношение \ruen{<<цена/прибыль>>}{Price/Earnings, P/E}. Чтобы посчитать отношение $P/E$ рынка акций, нужно разделить капитализацию рынка (суммарную стоимость всех акций всех компаний) $P$ на общую прибыль всех компаний $E$. Например, если все акции всех компаний из индекса S\&P\,500 вместе взятые стоят \dollars{30} триллионов, и компании заработали в прошлом году \dollars{0.75} триллиона, то соотношение $P/E$ равно $30/0.75 = 40$. Покупая сейчас индекс акций, вы покупаете компании за 40 годовых прибылей.

Поскольку простая формула $P/E$ выглядит недостаточно солидно, обычно аналитики заменяют прибыль прошлого года $E$ на среднюю прибыль за последние 10 лет. Надо признать, что в этом есть смысл. Прибыли компаний меняются с течением времени. В какие-то годы экономика на подъёме, и прибыли компаний оказываются выше среднего. В не столь удачные годы экономика замедляется, и вместе с ней снижаются прибыли компаний. Если взять среднее за достаточно длинный период, лет 10--12, то мы сгладим колебания, связанные с экономическими циклами.

Нужно только не забыть, что из-за инфляции \$1 10 лет назад и \$1 в прошлом году --- это немного разные доллары. Чтобы складывать яблоки с яблоками, нужно домножить прибыли прошлых лет на накопленную инфляцию. Обозначим $E_{1}$ суммарную прибыль всех компаний за прошлый год, $E_2$ --- суммарную прибыль за предпоследний год, и так далее до $E_{10}$ --- годовой прибыли компаний 10 лет назад. Также обозначим инфляцию за последние 10 лет $I_{10}$, за последние 9 лет --- $I_9$ и так далее до $I_1$. Например, если $I_5=1.2$, то за пять последних лет цены выросли в 1.2 раза. Тогда мы сможем посчитать среднюю прибыль компаний за последние 10 лет  $E'$:
\begin{align}
E' &= \frac{E_1I_1 + E_2I_2 + ... + E_{10}I_{10}}{10}
\end{align}

Если эту среднюю прибыль за 10 лет $E'$ подставить в <<обычное>> соотношение $P/E$, то получится соотношение \ruen{<<цена-прибыль>> с поправкой на цикличность}{cyclically-adjsuted price/earnings, CAPE}. Его ещё иногда называют \ruen{соотношением Шиллера}{Shiller ratio} в честь одного из первооткрывателей, профессора \ruen{Роберта Шиллера}{Robert Shiller} \cite{campbell1988dividend}:
\begin{align}
\text{CAPE} = \frac{P}{E'} = \frac{10P}{E_1I_1 + E_2I_2 + ... + E_{10}I_{10}}
\label{cape_formula}
\end{align}

Теперь, когда мы разобрались, что такое соотношение Шиллера, самое время посмотреть, каким оно было в прошлом и какое оно сейчас. Пожалуйста, уведите от экрана беременных женщин и детей. Сейчас вы увидите шокирующий график, который, по мнению некоторых  финансовых аналитиков, предсказывает неминуемый крах фондового рынка США. Внимание на рисунок \ref{shiller_pe_historical_chart}.



\newcommand{\dotWithNumber}[5] {
        \node[
            circle,
            fill,
            inner sep = 2pt,
            color = #3
        ]
        at (axis cs: #1, #2) {};
        
        \node[
            anchor=#5
        ]
        at (axis cs: #1, #2)
        {#4};
}



\begin{figure}[ht]
\centering
\begin{tikzpicture}
\begin{axis}[
    width=\textwidth,
    date coordinates in=x,
    date ZERO=1880-01-01,
    xtick={1880-01-01,1900-01-01,1920-01-01,1940-01-01,1960-01-01,1980-01-01,2000-01-01,2020-01-01},
    minor xtick={1890-01-01,1910-01-01,1930-01-01,1950-01-01,1970-01-01,1990-01-01,2010-01-01},
    xticklabel=\year,
    grid=both,
    xmin=1880-01-01,
    xmax=2026-01-01,
    ylabel={Соотношение $P/E$ с поправкой на цикличность (CAPE)},
]
    
    \newcommand{\capeDotAndNumber}[2]{
         \dotWithNumber{#1}{#2}{Set1-B}{#2}{south}
    }
    
        \addplot[
            color = Set1-B,
            line width = 1pt
        ]
        table[
            x = month,
            y = shiller_cape,
            col sep = comma
        ]
        {data/shiller_cape.csv};        

        \capeDotAndNumber{1929-09-01}{32.6}        
        \capeDotAndNumber{1999-12-01}{44.2}
        \capeDotAndNumber{2021-03-01}{35.0}
\end{axis}
\end{tikzpicture}
\caption{Соотношение <<цена/прибыль>> с поправкой на цикличность (CAPE) для рынка акций США. Данные: Robert Shiller \cite{shillerOnline}.}
\label{shiller_pe_historical_chart}
\end{figure}

Текущее значение (35.0 по состоянию на март 2021) выглядит довольно высоким по историческим меркам. В прошлом лишь дважды акции стоили дороже, чем 30 годовых прибылей: в 1929-м аккурат перед Великой депрессией и в конце 1990-х на пике пузыря интернет-компаний. И вот история повторяется в третий раз. Снова на рынке акций надулся пузырь. Мы стоим на пороге финансовой катастрофы, после которой живые позавидуют мёртвым. Скажите, вам уже страшно?

Впрочем, я бы не спешил накрываться белой простынёй и медленно ползти на кладбище. Ирония в том, что аналитики продолжают ссылаться на авторитет Шиллера (нобелевский лауреат, как-никак), хотя он сам не так давно написал, что, возможно, никакого пузыря-то и нет. По его словам, высокое отношение CAPE может быть связано с низкими процентными ставками и не обязательно является предвестником финансового шторма \cite{shiller2020cape}.

\section*{Цены акций и процентные ставки. Формула Гордона}

Чтобы понять аргументацию Шиллера и соавторов, нам нужно разобраться, как связаны цены акций и процентные ставки.

Представим, что некая компания зарабатывает прибыль $E$, которую полностью выплачивает в виде дивидендов. Компания развивается, и каждый год прибыль растёт на $g$ процентов. Первая выплата $E$ долларов случится через год, через два года компания выплатит $E(1+g)$, через три года $E(1+g)^2$ и так далее до бесконечности. Компания никогда не разорится и будет ежегодно выплачивать  растущие дивиденды. Внимание, вопрос: сколько стоит такая компания?

На первый взгляд, компания в будущем выплатит акционерам бесконечное количество долларов, поэтому её цена --- тоже бесконечность. Это простое рассуждение неверно, потому что оно суммирует будущие доллары, которые вовсе не равны сегодняшним.

Во-первых, инфляция со временем съедает покупательную способность денег. Скорее всего, через 10 лет вы купите на \$100 меньше товаров, чем могли бы купить на \$100 сегодня.

Во-вторых, даже если инфляции нет вовсе, то всё равно люди при прочих равных предпочитают потребление сегодня потреблению в будущем. Поэтому доллары, полученные от нашей компании через 10 лет, менее ценны, чем доллары, полученные через 5 лет, и тем более менее ценны, чем сегодняшние доллары.

Чтобы корректно складывать будущие доллары, их нужно привести к сегодняшним, как если бы переводили мили и вёрсты в метры. Для этого нам понадобится так называемая процентная \ruen{ставка дисконтирования}{discount rate}. Если обозначить ставку дисконтирования $r$, то \ruen{текущая стоимость}{present value, PV} будущих $E$ долларов, которые мы получим через $T$ лет, равна
\begin{align*}
PV = \frac{E}{(1 +r)^T}
\end{align*}

Например, если ставка дисконтирования равна 10\%, то текущая стоимость \dollars{100}, которые вам выплатят через год, равна $\dollars{100}/(1 + 0.1) \approx \dollars{90.91}$. Если же доллары выплатят через 10 лет, то их текущая стоимость равна $\dollars{100} / (1+0.1)^{10} \approx \dollars{38.5}5$.

Другими словами, \dollars{38.55} сегодня имеют для вас такую же ценность, как и \dollars{100} через 10 лет. Чтобы убедить вас не потратить \dollars{38.55} сегодня, а вложить их на 10 лет, нужно пообещать вам доходность как минимум 10\% годовых (выплата должна быть как минимум \dollars{100}). В противном случае будущие доллары не перевесят сегодняшние, и вы откажетесь от инвестиций.

Вернёмся к нашей <<вечной>> компании. Составим таблицу \ref{gordon_growth_table}, в которой перечислим все будущие платежи и их текущую стоимость с учётом ставки дисконтирования $r$.

\begin{table}[ht]
\centering
\begin{tabular}{c|c|c}
Год & Выплата & Текущая стоимость \\
\hline
1 & $E$ & $E / (1+r)$ \\
2 & $E(1+g)$ & $E(1+g) / (1+r)^2$ \\
3 & $E(1+g)^2$ & $E(1+g)^2 / (1+r)^3$ \\
... & ... & ... \\
$n$ & $E(1+g)^{n-1}$ & $E(1+g)^{n-1} / (1+r)^n$ \\
... & ... & ...
\end{tabular}
\caption{Текущая стоимость платежей <<вечной>> компании.}
\label{gordon_growth_table}
\end{table}

Вспомним формулу суммы бесконечной геометрической прогрессии и запишем сумму всех выплат с учётом дисконтирования:
\begin{align}
P = \frac{E}{1 +r } + \frac{E(1+g)}{(1+r)^2} + ... + \frac{E(1+g)^{n-1}}{(1+r)^n} + ... = \dfrac{\dfrac{E}{1 + r}}{1 - \dfrac{1 + g}{1 + r}} = \frac{E}{r - g}
\label{gordon_formula_nominal_rates}
\end{align}

Формула (\ref{gordon_formula_nominal_rates}) называется формулой Гордона или \ruen{моделью роста Гордона}{Gordon growth model} \cite{gordon1956capital}. Она связывает цену акций $P$ с прибылью $E$, ставкой дисконтирования $r$ и темпом роста прибыли $g$.

Обратите внимание, что чем выше ставка дисконтирования $r$, тем дешевле акция. С другой стороны, чем выше темп роста прибыли $g$, тем дороже компания. Например, если компания вечно выплачивает дивиденды \dollars{100} при скорости роста $g=0\%$, то при ставке дисконтирования $r=10\%$ она стоит $\dollars{100} / 0.1 = \dollars{1000}$. Если при тех же выплатах ставка снижается до 5\%, то компания стоит $\dollars{100  / 0.05} = \dollars{200}$, то есть в два раза дороже.

\section*{Соотношение <<цена-прибыль>> и реальные процентные ставки}

Добавим в нашу модель инфляцию. Пусть цены на все товары в экономике растут на $i$ процентов в год. Заменим ставку дисконтирования $r$ в номинальных процентах на реальную ставку $r^*$. Точно так же поступим со скоростью роста прибыли $g$: заменим номинальные проценты на реальные проценты сверх инфляции:
\begin{align}
\begin{cases}
r = r^* + i \\
g = g^* + i
\end{cases}
\Leftrightarrow
\quad
\begin{cases}
r^* = r - i \\
g^* = g - i
\end{cases}
\label{real_rates_formula}
\end{align}

Например, если номинальная процентная ставка $r$ равна 10\% (на каждый вложенный доллар вы получаете \$1.10 через год), а ожидаемая инфляция составляет 4\%, то реальная процентная ставка сверх инфляции равна 6\%. Если вы инвестировали сумму, равную цене одного эскимо, то через год вы получите сумму, эквивалентную 1.06 эскимо.

Подставим выражения (\ref{real_rates_formula}) в формулу Гордона  (\ref{gordon_formula_nominal_rates}). Обратите внимание, что инфляция $i$ сократилась:
\begin{align}
P = \frac{E}{r - g} = \frac{E}{(r^* + i) - (g^* + i)} =  \frac{E}{r^* - g^*} 
\label{gordon_formula_real_rates}
\end{align}

Двигаемся дальше. От чего зависит реальная ставка дисконтирования $r^*$? Её можно расписать как сумму двух слагаемых: \ruen{безрисковой}{risk-free}\ реальной ставки $f^*$ и премии за риск $\pi$:
\begin{align}
r^* = f^* + \pi
\label{risk_premium_formula}
\end{align}

Безрисковая реальная ставка $f^*$ отражает стоимость переноса потребления из сегодня в будущее. Например, если вы цените одно эскимо сегодня точно так же, как 1.06 эскимо через год, то ваша личная реальная процентная ставка $f^*$ равна $6\%$.

Премия за риск $\pi$ вознаграждает вас за неопределённость будущего. Редкая инвестиция является по-настоящему безрисковой. Иногда инвестиции оборачиваются потерями: вложили 1 эскимо, а через год получили только половинку. Чтобы компенсировать ваши страдания в плохом случае, ожидаемая доходность инвестиций должна быть выше, чем безрисковая ставка. Эта прибавка и будет премией за риск. Подробнее о премии за риск можно прочитать в \href{https://habr.com/ru/company/dbtc/blog/527050/}{одной из недавних статей}.

С учётом премии за риск (\ref{risk_premium_formula}), формула Гордона (\ref{gordon_formula_real_rates}) превращается в 
\begin{align}
P =\frac{E}{r^* - g^*} = \frac{E}{f^* + \pi - g^*}
\quad
\Rightarrow
\quad
P/E = \frac{1}{f^* + \pi - g^*}
\label{gordon_formula_with_risk_premium}
\end{align}

Формула (\ref{gordon_formula_with_risk_premium}) позволяет выделить три возможные причины роста соотношения <<цена--прибыль>> $P/E$ в последние годы. Во-первых, могла уменьшится безрисковая реальная ставка $f^*$. Во-вторых, могла уменьшиться премия за риск $\pi$, которую требуют инвесторы. Наконец, в-третьих, мог увеличиться ожидаемый темп роста прибыли $g^*$.

Так вот, согласно Шиллеру, именно низкая безрисковая реальная ставка $f^*$ объясняет высокое соотношение $P/E$.

\section*{Реальные процентные ставки}

Можем ли мы залезть в голову инвесторам и посмотреть, какую безрисковую реальную ставку $f^*$ они закладывают в цены активов? Да, можем. С конца 90-х годов Казначейство США выпускает \ruen{государственные облигации, защищённые от инфляции}{Treasury Inflation-Protected Securities, TIPS}.

Например, если у вас есть бумага TIPS с номиналом \dollars{1000} и сроком погашения 10 лет, то правительство США обязуется выплатить вам \dollars{1000}, умноженные на накопленную за 10 лет инфляцию. Если инфляция за 10 лет составит 20\% (по 1.8\% в год), то вы получите \dollars{1200}, а если 100\% (по 7.2\% в год), то \dollars{2000}.

Мы можем посмотреть, по какой цене инвесторы продают и покупают бумаги TIPS на рынке, и вычислить, какую реальную доходность они рассчитывают получить. Поскольку дефолт по государственным облигациям США --- крайне маловероятное событие, то полученная доходность будет хорошим приближением теоретической безрисковой реальной ставки.

На рисунке \ref{treasury_yields_figure} показаны доходности TIPS и обычных гос. облигаций с 2003-го года. Текущая доходность TIPS отрицательная и составляет $-0.92\%$ годовых. Да, вы всё правильно поняли: инвесторы платят правительству США за привилегию дать ему деньги в долг. Чтобы через 10 лет получить от правительства США поправленный на инфляцию эквивалент сегодняшних \dollars{1000}, нужно сегодня дать ему в долг примерно $\dollars{1000} / (1 + 0.0092)^{10} \approx \dollars{1096}$.

К слову, доходность обычных гос. облигаций (без защиты от инфляции) составляет 1.26\% годовых. Можно сделать вывод, что в среднем участники рынка ожидают инфляцию на уровне $1.26\% - (-0.92\%) = 2.18\%$ в год. Именно при таком уровне инфляции ни инвесторы в обычные облигации, ни инвесторы в TIPS не получат преимущества друг перед другом.


\newcommand{\plotThickZeroAxis}{
        \draw[very thick] (axis cs:1880-01-01, 0) -- (axis cs: 2050-01-01, 0);   
}

\begin{figure}[ht]
\centering
\begin{tikzpicture}
\begin{axis}[
    width=\textwidth,
    date coordinates in=x,
    date ZERO=2003-01-01,
    xtick={2003-01-01,2005-01-01,2007-01-01,2009-01-01,2011-01-01,2013-01-01,2015-01-01,2017-01-01,2019-01-01,2021-01-10},
    minor xtick={2004-01-01,2006-01-01,2008-01-01,2010-01-01,2012-01-01,2014-01-01,2016-01-01,2018-01-01,2020-01-01,2022-01-01},
    xticklabel=\year,
    yticklabel={\pgfmathprintnumber{\tick}\%},
    grid=both,
    xmin=2003-01-01,
    xmax=2022-01-01,
    ylabel={Доходность (проценты годовых)},
    ylabel shift = -10pt,
    legend entries={
        Облигации с фиксированным купоном (T-Note),
        Облигации с защитой от инфляции (TIPS)
    },
    legend pos=north east,
    legend style={font=\small},
    legend cell align={left}
]
        
    \newcommand{\plotYield}[5]{
        \addplot[
            color = #2,
            line width = 1pt,
            mark = #3,
            mark repeat = 6,
            %mark phase = 6,
            mark options = {scale=2},
        ]
        table[
            x = DATE,
            y = #1,
            col sep = comma
        ]
        {data/#1.csv};   
        
        \dotWithNumber{#4}{#5}{#2}{#5}{south}
    }
        

	
    \plotYield{GS10}{Set1-A}{square}{2021-02-01}{1.26}
    \plotYield{FII10}{Set1-B}{none}{2021-02-01}{-0.92}
    
    \plotThickZeroAxis
\end{axis}
\end{tikzpicture}
\caption{Доходность десятилетних гос. облигаций США: облигации с фиксированным купоном (T-Note) и облигации, защищённые от инфляции (TIPS). Данные: Federal Reserve Bank of St. Louis \cite{fredGS10}, \cite{fredFII10}.}
\label{treasury_yields_figure}
\end{figure}

К сожалению, облигации с защитой от инфляции появились не так давно. График на рисунке \ref{treasury_yields_figure} не просто так начинается с 2003-го года. Чтобы оценить реальные процентные ставки в далёком прошлом, нужно выкручиваться. Профессор Шиллер предлагает вычесть из текущей доходности обычных десятилетних облигаций инфляцию за предыдущие 10 лет. Например, если сейчас десятилетние облигации дают доходность 1.45\%, а инфляция за предыдущие 10 лет составила 1.61\% в год, то реальная доходность по Шиллеру равна $1.45\% - 1.61\% = -0.16\%$.

На рисунке \ref{long_run_interest_rates} представлены данные за 140 лет: номинальная доходность обычных десятилетних облигаций (красная линия) и их реальная доходность за вычетом предшествующей десятилетней инфляции (синяя линия). Как видите, мы уже не можем сказать, что сегодняшние реальные ставки --- беспрецедентно низкие в истории. Но всё равно периодов, когда безрисковая реальная ставка уходила ниже нуля, не так много.



\begin{figure}[ht]
\centering
\begin{tikzpicture}
\begin{axis}[
    width=\textwidth,
    date coordinates in=x,
    date ZERO=1880-01-01,
    xtick={1880-01-01,1900-01-01,1920-01-01,1940-01-01,1960-01-01,1980-01-01,2000-01-01,2020-01-01},
    minor xtick={1890-01-01,1910-01-01,1930-01-01,1950-01-01,1970-01-01,1990-01-01,2010-01-01,2030-01-01},
    xticklabel=\year,
    yticklabel={\pgfmathparse{100*\tick}\pgfmathprintnumber{\pgfmathresult}\%},
    grid=both,
    xmin=1880-01-01,
    xmax=2035-01-01,
    ymin=-0.04,
    ymax=0.16,
    ylabel={Доходность (проценты годовых)},
    ylabel shift = -10pt,
    legend entries={
        Номинальная доходность,
        Реальная доходность сверх инфляции
    },
    legend pos=north west,
    legend style={font=\small},
    legend cell align={left}
]         

 \newcommand{\plotShillerData}[7]{    
        \addplot[
            color = #2,
            line width = 1pt,
            mark = #3,
            mark repeat = 120,
            mark phase = 60,
            mark options = {scale=2},
        ]
        table[
            x = month,
            y = #1,
            col sep = comma
        ]
        {data/shiller_cape.csv};   
        
	   \dotWithNumber{#4}{#5}{#2}{#6}{#7}
    }

    
    \plotShillerData{long_rate}{Set1-A}{square}{2021-03-01}{0.0145}{1.45}{south west}
    \plotShillerData{real_rate_10y}{Set1-B}{none}{2021-03-01}{-0.0016}{-0.16}{south west}
        
    \plotThickZeroAxis
\end{axis}
\end{tikzpicture}
\caption{Номинальная и реальная доходность десятилетних гос. облигаций США. Данные: Robert Shiller \cite{shillerOnline}.}
\label{long_run_interest_rates}
\end{figure}

Как мы видим, в прошлом реальная процентная ставка изменялась в довольно широких пределах от -3.6\% в 1951 г. до 7.6\% в 1892 г. Конечно, из формулы (\ref{gordon_formula_with_risk_premium}) следует, что такие значительные колебания безрисковой реальной ставки $f^*$ могут вызвать значительные изменения соотношения $P/E$, даже если прибыли компаний $E$, скорость роста дивидендов $g^*$ и премия за риск $\pi$ не меняются.

\section*{Избыточная доходность CAPE}

Как быть? Из уравнения (\ref{gordon_formula_with_risk_premium}) можно выразить премию за риск $\pi$. Напомню, что премия за риск показывает, насколько доходность акций выше доходности безрисковых облигаций. Это именно тот параметр, который интересуют инвесторов, когда они выбирают пропорцию между акциями и облигациями.
\begin{align*}
\pi =\underbrace{\left(\frac{1}{P/E} - f^*\right)}_{\mathclap{\text{избыточная доходность P/E}}} + g^*
\end{align*}

По нашей модели получается, что премия за риск (доходность акций) тем выше, чем больше разность $1/(P/E) - f^*$, которая называется \ruen{избыточная доходность P/E}{P/E excess yield}. Например, если отношение $P/E$ равно 20 (компании стоят 20 годовых прибылей), то отношение $1/(P/E)$ равно 0.05 или 5\% (инвестиции в компанию приносят 5\% в год). Если реальная безрисковая ставка $f^*$ при этом рана 1\%, то избыточная доходность P/E равна $5\% - 1\% = 4\%$.

Если заменить прибыль последнего года $E$ на среднюю прибыль последних 10 лет $E'$, как формуле CAPE \ref{cape_formula}, то получится величина, которую Шиллер и соавторы называют \ruen{избыточной доходностью CAPE}{excess CAPE yield, ECY}:
\begin{align*}
\pi \approx \underbrace{\left(\frac{1}{\text{CAPE}} - f^*\right)}_{\mathclap{\text{избыточная доходность CAPE}}} + g^*
\end{align*}

Если предположить, что $g^*=0$ (прибыль компаний растёт на инфляцию), то избыточная доходность CAPE может подсказать, какой будет будущая премия за риск. Высокое отношение $CAPE$ в знаменателе тянет премию за риск вниз, а низкая безрисковая реальная ставка $f^*$ тянет премию за риск вверх. Аналитики правы, когда говорят, что высокое отношение $P/E$ или $CAPE$  \textit{при прочих равных} может быть предвестником низкой доходности акций (низкой премии за риск). Но они упускают из виду безрисковую ставку $f^*$, которая может компенсировать рост отношения CAPE.

Рисунок \ref{excess_cape_yield_chart} показывает, как в прошлом изменялась избыточная доходность CAPE (сплошная синяя линия). Красная пунктирная линия --- это годовая доходность рынка акций сверх инфляции в последующие 10 лет. Например, последняя точка, для которой у нас есть все данные --- март 2011-го года. Тогда соотношение CAPE было равно 22.9, номинальная доходность 10-летних облигация составляла 3.41\%, инфляция в предыдущие 10 лет составила 2.41\%. Таким образом, реальная безрисковая процентная ставка по Шиллеру была равна $3.41\% - 2.41\% = 1.0\%$, а избыточная доходность CAPE составляла $1/22.9 - 1\% = 3.36\%$. За следующие 10 лет до марта 2021-го рынок акций дал доходность сверх инфляции 11.8\% в год --- довольно много по историческим меркам.



\begin{figure}[ht]
\centering
\begin{tikzpicture}
\begin{axis}[
    width=\textwidth,
    date coordinates in=x,
    date ZERO=1880-01-01,
    xtick={1880-01-01,1900-01-01,1920-01-01,1940-01-01,1960-01-01,1980-01-01,2000-01-01,2020-01-01},
    minor xtick={1890-01-01,1910-01-01,1930-01-01,1950-01-01,1970-01-01,1990-01-01,2010-01-01},
    xticklabel=\year,
    yticklabel={\pgfmathparse{100*\tick}\pgfmathprintnumber{\pgfmathresult}\%},
    grid=both,
    xmin=1880-01-01,
    xmax=2030-01-01,
    ymin=-0.06,
    ymax=0.24,
    legend entries={
        Избыточная доходность CAPE,
        Доходность акций в последующие 10 лет
    },
    legend pos=north east,
    legend style={font=\small},
    legend cell align={left}
]
    
     \newcommand{\plotShillerData}[7]{    
        \addplot[
            color = #2,
            style = #3,
            line width=1pt
        ]
        table[
            x = month,
            y = #1,
            col sep = comma
        ]
        {data/shiller_cape.csv};   
        
	   \dotWithNumber{#4}{#5}{#2}{#6}{#7}
    }
    
    \newcommand{\plotShillerMedian}[2]{
         \draw[color=#2, line width=1pt, style=dashed] (axis cs: 1880-01-01, #1) -- (axis cs: 2050-01-01, #1);
  }

    \plotShillerData{cape_excess_yield}{Set1-B}{solid}{2021-03-01}{0.03}{3.0}{west}
    \plotShillerData{subsequent_stock_return_10y}{Set1-A}{dashed}{2011-03-01}{0.118}{11.8}{west}

    \plotShillerMedian{0.0683}{Set1-A}
          \plotShillerMedian{0.0350}{Set1-B}
     
    \dotWithNumber{1930-04-01}{-0.0119}{Set1-B}{-1.2}{west}
    \dotWithNumber{2000-01-01}{-0.0152}{Set1-B}{-1.5}{west}
     
    \plotThickZeroAxis
\end{axis}
\end{tikzpicture}
\caption{Избыточная доходность CAPE и годовая доходность рынка акций США сверх инфляции в последующие 10 лет. Данные: Robert Shiller \cite{shillerOnline}.}
\label{excess_cape_yield_chart}
\end{figure}

Когда вы читаете график \ref{excess_cape_yield_chart}, не забывайте, что в избыточной доходности CAPE сам показатель CAPE стоит в знаменателе. Высокие значения CAPE на графике \ref{shiller_pe_historical_chart} соответствуют низким значениям избыточной доходности CAPE. В частности, перед Великой депрессией избыточная доходность CAPE опускалась до -1.2\%, а перед сдуванием пузыря интернет компаний --- до -1.5\%.

А что можно сказать о текущем уровне избыточной доходности CAPE? Да, собственно, ничего интересного. Текущее значение 3.0\% чуть ниже исторической медианы 3.5\%, но именно <<чуть>>. Мы и близко не подошли к уровням, которые предшествовали Великой депрессии или краху интернет-компаний. Как показано в таблице \ref{cape_excess_yield_quantiles_table}, мы сейчас даже выше, скажем,  чем медианный уровень за последние 35 лет 2.63\%. Если рынок акций и <<перегрет>>, то не сильнее, чем обычно.

\begin{table}[ht]
\centering
\begin{tabular}{l|r|r|r|r|r|r|r}
Период       & Мин.      & 5\%       & 25\%   & 50\%    & 75\%      & 95\% & Макс. \\ \hline
1881--1915 & -2.58\% & -0.92\% & 0.25\% & 1.46\% &   4.74\% & 6.89\%  & 8.83\% \\
1916--1950 & -1.19\% &  0.16\% & 2.88\% & 8.53\% & 12.55\%  & 19.19\% & 23.53\% \\
1951--1985 &  0.54\% &  0.97\% & 2.45\% & 5.38\% &   8.12\%  & 10.37\% & 12.01\% \\
1986--2020 & -1.52\% &  0.21\% & 1.54\% & 2.63\% &   3.79\%  & 5.63\%  & 7.26\% \\ \hline
1881--2020 & -2.58\% & -0.30\% & 1.55\% & 3.50\% &   6.69\%  & 13.26\% & 23.53\%
\end{tabular}
\caption{Квантили избыточной доходности CAPE для рынка акций США, 1881--2020. Данные: Robert Shiller \cite{shillerOnline}.}
\label{cape_excess_yield_quantiles_table}
\end{table}

Что ещё можно сказать о графике \ref{excess_cape_yield_chart}? Невооружённым глазом видно, что за низкими значениями избыточной доходности CAPE часто следует не самое лучшее десятилетие для рынка акций. Чтобы формализовать это наблюдение, я воспользусь методом из статьи \ruen{Клиффорда Аснесса}{Clifford Asness}, основателя фонда \en{AQR}\ \cite{asness2012old}.

Упорядочим все исторические значения избыточной доходности CAPE по возрастанию и разобьём из на 10 равных групп --- децилей. Для каждого дециля посмотрим на среднюю доходность рынка акций США сверх инфляции в последующие 10 лет. Ограничимся данными с 1926-го года, потому что именно с этого года доступны наиболее качественные данные по ценам акций. Для простоты будем смотреть только на CAPE по состоянию на начало каждого года. Тогда у нас получится таблица \ref{cape_excess_yield_and_stock_returns_table}.

\begin{table}[ht]
\centering
\begin{tabular}{r|r|r|r|r}
\multicolumn{2}{c|}{Изб.\,дох-ть\,CAPE} &
\multicolumn{3}{c}{Десятилетняя\,дох-ть\,акций} \\
\hline
Мин. & Макс. & Средняя & Худшая & Лучшая \\
\hline
-1.5\% &  0.7\% &  0.6\% & -3.0\% &  5.5\% \\
 0.8\% &  1.4\% &  1.4\% & -4.4\% &  9.0\% \\
 1.7\% &  1.9\% &  3.9\% & -3.5\% & 10.1\% \\
 2.0\% &  2.7\% &  5.3\% & -1.7\% & 15.0\% \\
\hline
 2.8\% &  3.3\% &  5.5\% & -2.7\% & 14.6\% \\
\hline
 3.4\% &  4.9\% &  8.4\% &  2.4\% & 15.4\% \\
 5.1\% &  6.5\% &  9.0\% &  5.6\% & 11.0\% \\
 6.6\% &  8.0\% & 10.4\% &  5.8\% & 13.5\% \\
 8.4\% &  9.6\% & 10.2\% &  6.5\% & 12.4\% \\
10.1\% & 13.0\% & 14.0\% &  9.1\% & 17.6\%        
\end{tabular}
\caption{Децили избыточной доходности CAPE и последующая десятилетняя доходность рынка акций США сверх инфляции, 1926--2011. Данные: Robert Shiller \cite{shillerOnline}.}
\label{cape_excess_yield_and_stock_returns_table}
\end{table}

Начнём с очевидного наблюдения: по мере того, как мы забираемся во всё более высокие децили доходности CAPE, последующая доходность рынка акций в среднем растёт. Только при переходе между восьмой и девятой строчкой средняя доходность снижается на 0.2\%. С другой стороны, мы не видим такой же очевидной монотонности для лучшего случая. Удачное десятилетие может случиться почти после любого уровня избыточной доходности CAPE.

Означает ли это, что обвал фондового рынка отменяется? Нет, не означает. Если вы смотрели <<Волка с Уолл-стрит>> с Ди Каприо, то, возможно, помните, что фильм начинается с биржевого краха --- <<чёрного понедельника>> 19 октября 1987 года. На тот момент показатель избыточной доходности CAPE составлял ничем не примечательные 3.4\%. Однако это не помешало индексу Dow Jones обвалиться на 22\% в течение всего лишь одного торгового дня. 

Вообще, у меня для вас две новости насчёт краха рынка, и обе плохие. Во-первых, аналитики совершенно правы в том, что в будущем нас ждёт крах рынка акций. Во-вторых, никто не в силах предсказать, когда он случится, с какого уровня и на сколько процентов. Возможно, обвал на 40\% уже произошёл за то время, пока я готовлю статью к публикации (и тогда вам стоит вернуться на машине времени назад и продать акции заранее). Возможно, нас ждёт ещё несколько лет роста рынка акций, после которых рынок откатится назад, но всё равно останется выше уровней начала 2021-го года (и тогда сегодняшним владельцам акций нечего бояться).

Практика показывает, что предсказание доходности рынка акций, так называемый market timing, --- на удивление бесперспективное занятие. С этим трудно смириться, но это так. Акции --- рискованный актив. Иногда они непредсказуемо падают, и эти обвалы неизбежны так же, как снег зимой. Именно за риск неожиданного обвала вы и зарабатываете премию за риск в спокойные времена. Если каждый инвестор мог бы предсказать будущую доходность рынка акций с помощью отношения P/E или другой простецкой дроби <<что-то на что-то>>, чтобы вовремя выскочить из акций и убежать в безрисковые облигации, то никакой премии за риск бы не было.

Поэтому я считаю, что информационная ценность предсказания <<в будущем нас ждёт обвал рынка акций>> равна нулю. Ну да, в январе следующего года в Москве, скорее всего, будет идти снег. Что с того? Можно подумать, кто-то в этом сомневается. Вот если бы прогнозист уточнил, сколько миллиметров снега выпадет в конкретный день, чтобы можно было заранее спланировать лыжную прогулку, то тогда от его прогнозов был бы толк. Однако такая точность прогноза находится за гранью возможностей современной науки. Точно так же нельзя предсказать будущую доходность рынка акций с приемлемой точностью, достаточной для какого бы то ни было практического применения.

\section*{Список литературы}
\en{
\printbibliography[heading = none]
}

\end{document}
