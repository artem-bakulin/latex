\documentclass{beamer}

\usepackage{cmap}				% To be able to copy-paste russian text from pdf
\usepackage[T2A]{fontenc}
\usepackage[utf8]{inputenc}
\usepackage[russian]{babel}
\usepackage{textpos}
\usepackage{ragged2e}
\usepackage{amssymb}
\usepackage{ulem}
\usepackage{tikz}
\usepackage{pgfplots}
\usepackage{color}
\usepackage{cancel}
\usepackage{multirow}
\pgfplotsset{compat=1.17}
\usetikzlibrary{arrows,snakes,backgrounds,shapes}
\usepgfplotslibrary{groupplots,colorbrewer,dateplot,statistics}
\usepackage{animate}

\usepackage{amsfonts}
\usepackage{amsmath}
\usepackage{amssymb}
\usepackage{graphicx}
\usepackage{setspace}


\title{Индексное инвестирование}
\author{Артём Бакулин}
\date{\today}

\begin{document}


\begin{frame}{Домашнее задание (Python)}
\justify
Срок сдачи 10.04.2021 (суббота) 23:59 MSK.

\justify
Постройте границу эффективности для портфелей, составленных из классов активов (asset classes), используя исторические ожидаемые доходности, дисперсии и корреляции.

\justify
Выберите 5--6 классов активов на свой вкус на сайте Portfolio Visualizer (там же можно перепроверить свои ответы):

{\scriptsize \url{https://www.portfoliovisualizer.com/efficient-frontier}}

\justify
Для решения задачи квадратичного программирования (QP) используйте пакет cvxopt:

{\scriptsize \url{https://cvxopt.org/userguide/coneprog.html\#quadratic-programming}}

\justify
Описание математической задачи в статье автора (раздел 1.6):
{\scriptsize \url{https://artem-bakulin.github.io/latex/papers/index-investing/}}
\end{frame}



\begin{frame}{Исходные данные}
\justify
Пример исходных данных с сайта Portfolio Visualizer (можете использовать их, можете взять другие):

\centering
\vspace{\baselineskip}
\scriptsize
\begin{tabular}{l|r|r|r|r|r|r|r}

 & \multicolumn{2}{c|}{Доходность} & \multicolumn{5}{c}{Корреляция} \\ 
\cline{2-8}
Актив                   & Сред.  & Откл.  & Lar.  & Sm.   & ex-US   & US B.    & G.B. \\ \hline
US Large Cap            & 11.3\% & 15.9\% & 1.00  & 0.84  & 0.83    & 0.02  & 0.26\\
US Small Cap            & 12.5\% & 19.3\% & 0.84  & 1.00  & 0.79    & -0.05 & 0.22\\
Global ex-US Stock      & 6.9\%  & 16.8\% & 0.83  & 0.79  & 1.00    & 0.01 & 0.42\\
Total US Bond Market       & 5.1\%  & 3.6\%  & 0.02  & -0.05 & 0.01    & 1.00 & 0.60\\
Global Bonds (unhedged) & 5.9\%  & 6.8\%  & 0.26  & 0.22  & 0.42    & 0.60 & 1.00
\end{tabular}
\end{frame}


\begin{frame}{Квадратичное программирование в cvxopt}
\justify
Метод cvxopt.solvers.qp решает задачу оптимизации следующего вида:
\begin{align*}
\begin{cases}
\dfrac{1}{2}x^TPx + q^Tx \to \min \\
Ax = b \\
Gx \le h 
\end{cases}
\end{align*}

\justify
Здесь $x$ --- искомый вектор весов активов в портфеле (в нашем примере он имеет длину 5). Матрицы $P$, $q$, $G$, $h$, $A$, $b$ задаются условиями задачи. Наша цель --- правильно составить эти матрицы.
\end{frame}



\begin{frame}{Шаг 1. Матрица ковариации $P$}
\justify
\begin{align*}
P = \begin{bmatrix}
\sigma_1^2 & \rho_{1,2}\sigma_1\sigma_2 & \cdots & \rho_{1,n}\sigma_1\sigma_n \\
\rho_{2,1}\sigma_2\sigma_1 & \sigma_2^2 & \cdots & \rho_{2,n}\sigma_2\sigma_n \\
\vdots & \vdots & \ddots & \vdots \\
\rho_{n,1}\sigma_n\sigma_1 & \rho_{n,2}\sigma_n\sigma_2 & \cdots & \sigma_n^2
\end{bmatrix},
\quad
q = 0
\end{align*}

Здесь $\sigma_i$ --- стандартное отклонение доходности актива $i$, $\rho_{i,j}$ --- корреляция активов $i$ и $j$. 

\justify
С такими матрицами $P$ и $q$ солвер будет искать решение (вектор $x$ весов активов в портфеле), которое минимизирует дисперсию портфеля.
\end{frame}



\begin{frame}{Шаг 2. Ограничения с равенствами $A$ и $b$}
\justify
Во-первых, мы хотим, чтобы в сумме все веса в векторе $x$ были равны 1. Тогда мы инвестируем в активы весь капитал, но не больше.
\begin{align*}
A_1 = \begin{bmatrix}
1 & 1 \dots 1
\end{bmatrix},
\quad
b_1 = \begin{bmatrix}1\end{bmatrix}
\end{align*}
\justify
Во-вторых, мы хотим, чтобы ожидаемая доходность портфеля равнялась $r$ (этот параметр мы будем задавать извне):
\begin{align*}
A_2 = \begin{bmatrix}
\mu_1 & \mu_2 \dots \mu_n
\end{bmatrix},
\quad
b_2 = \begin{bmatrix}r\end{bmatrix}
\end{align*}
Здесь $\mu_i$ --- ожидаемая доходность актива $i$.

\justify
В солвер нужно отдать блочную матрицу ограничений. Тогда он будет искать портфели, которые инвестирую весь капитал и дают нужную доходность $r$:
\begin{align*}
A = \begin{bmatrix}A_1 \\ A_2\end{bmatrix}, \quad b = \begin{bmatrix}b_1 \\ b_2\end{bmatrix}
\end{align*} 
\end{frame}


\begin{frame}{Шаг 3. Ограничения с неравенствами $G$ и $h$}
\justify
Потребуем, чтобы веса активов в портфеле были неотрицательными (нельзя продавать активы в короткую). Для этого нужны матрицы ограничений-неравенств:
\begin{align*}
G =
\begin{bmatrix}
    -1 &      0 & \cdots & 0 \\
     0 &     -1 & \cdots & 0 \\
\vdots & \vdots & \ddots & \vdots \\
     0 &      0 & \cdots & -1
\end{bmatrix},
\quad
h = \begin{bmatrix}0 \\ 0 \\ \vdots \\ 0\end{bmatrix}
\end{align*}

\justify
Условие $Gx \le h$ говорит солверу искать решения $x$, в которых все элементы неотрицательные.
\end{frame}


\begin{frame}{Шаг 4. Оптимальные портфели}
\justify
Теперь для каждого значения целевой доходности $r$ вы можете составить задачу оптимизации, которая ищет вектор $x$ весов портфеля, который а) имеет требуемую доходность и б) имеет минимально возможную дисперсию.

\justify
Для некоторых значений $r$ (скажем, от 2\% до 12\% с шагом 2\%) распечатайте табличку: доходность, стандартное отклонение, веса активов в портфеле. 

\justify
Изменяя $r$ с небольшим шагом (например, 0.05\%), постройте график границы эффективности, как на слайдах. На оси $x$ стандартное отклонение оптимального портфеля, на оси $y$ его доходность.

\justify
Повторите упражнение с дополнительным ограничением: вес каждого актива не может быть меньше 5\%, вес никакого актива не должен превышать 50\%.
\end{frame}

\end{document}


