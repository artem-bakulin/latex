\documentclass{beamer}

\usepackage{cmap}				% To be able to copy-paste russian text from pdf
\usepackage[T2A]{fontenc}
\usepackage[utf8]{inputenc}
\usepackage[russian]{babel}
\usepackage{textpos}
\usepackage{ragged2e}
\usepackage{amssymb}
\usepackage{ulem}
\usepackage{tikz}
\usepackage{pgfplots}
\usepackage{color}
\usepackage{cancel}
\usepackage{multirow}
\pgfplotsset{compat=1.17}
\usetikzlibrary{arrows,snakes,backgrounds,shapes}
\usepgfplotslibrary{groupplots,colorbrewer,dateplot,statistics}
\usepackage{animate}

\usepackage{amsfonts}
\usepackage{amsmath}
\usepackage{amssymb}
\usepackage{graphicx}
\usepackage{setspace}

\title{Индексное инвестирование}
\author{Артём Бакулин}
\date{\today}
\usetheme{Warsaw}
\usecolortheme{beaver}

\begin{document}



\begin{frame}
\titlepage
\end{frame}



\begin{frame}{Содержание}
\justifying

\begin{itemize}
\justifying
\item Рациональные инвесторы и премия за риск.
\item Портфельная оптимизация.
\item Capital Asset Pricing Model (CAPM).
\item Метрики для оценки инвестиций.
\item Факторные модели и Arbitrage Pricing Theory (APT).
\item Индексное инвестирование и биржевые фонды (ETFs).
\end{itemize}
\end{frame}



\begin{frame}{Рациональные инвесторы}
\justify
Обычно инвесторы любят доходность и не любят риск. Из двух
инвестиций с одинаковым риском лучше та, которая даёт большую доходность. Из двух инвестиций с одинаковой доходностью лучше та, которая несёт меньший риск.

\vspace{\baselineskip}
Является ли избегание риска (risk aversion) нерациональным поведением слабых духом? Нет! Рациональный инвестор тоже может не любить риск.
\end{frame}



\begin{frame}{Функция полезности}
\justify
Рациональный homo economicus максимизирует функцию полезности (utility function). Из двух альтернатив он всегда выбирает ту, которая даёт большую полезность.

\vspace{\baselineskip}
Предположим, что функция полезности рационального инвестора --- десятичный логарифм количества долларов на счету. Чем больше миллиардов, тем лучше, но каждый следующий миллиард приносит меньше счастья, чем предыдущий. Реалистично? Более чем!
\end{frame}



\begin{frame}{Сложный выбор}
\justify
Инвестор имеет на счету \$100\,000, которые дают полезность $\lg 100\,000 = 5.0$. Он должен вложить их в один из двух инструментов: в безрисковые облигации или рискованные акции.

\vspace{\baselineskip}
\begin{tabular}{c|c|c|c|c|c}
\multirow{2}{*}{Сценарий} & \multirow{2}{*}{Вер-ть} & \multicolumn{2}{c|}{Облигации} & \multicolumn{2}{c}{Акции} \\
\cline{3-6}
        &      & Капитал    & Полез-ть & Капитал      & Полез-ть \\ \hline
Хороший & 50\% & \$105\,000 & 5.021    & \$125\,000 & 5.097 \\
Плохой  & 50\% & \$105\,000 & 5.021    & \$85\,000  & 4.929 \\ \hline
Среднее &      & \$105\,000 & \alert{5.021} & \$105\,000 & \alert{5.013}
\end{tabular}

\vspace{\baselineskip}
Рациональный инвестор выберет менее рискованную альтернативу! Это верно не только для логарифма, но и для любой выпуклой вверх функции полезности (когда каждый следующий доллар радует меньше предыдущего).
\end{frame}



\begin{frame}{Премия за риск}
\justify
Что нужно сделать, чтобы рациональный инвестор выбрал рискованные акции или хотя бы воспринимал альтернативы безразлично? Например, пообещать более высокую доходность в хорошем сценарии и более высокое матожидание!

\vspace{\baselineskip}
\begin{tabular}{c|c|c|c|c|c}
\multirow{2}{*}{Сценарий} & \multirow{2}{*}{Вер-ть} & \multicolumn{2}{c|}{Облигации} & \multicolumn{2}{c}{Акции} \\
\cline{3-6}
        &      & Капитал    & Полез-ть & Капитал      & Полез-ть \\ \hline
Хороший & 50\% & \$105\,000 & 5.021    & \alert{\$129\,706} & \alert{5.113} \\
Плохой  & 50\% & \$105\,000 & 5.021    & \$85\,000  & 4.929 \\ \hline
Среднее &      & \$105\,000 & 5.021    & \alert{\$107\,353} & \alert{5.021}
\end{tabular}

\vspace{\baselineskip}
Дополнительные \$2\,353 в мат. ожидании --- премия за риск (risk premium), которая компенсирует инвестору дискомфорт от возможных потерь в плохом сценарии.
\end{frame}



\begin{frame}{Полная доходность}
\justify
Пусть $P_t$ --- цена актива (акции, облигации, квартиры) в момент времени $t$, а $D_t$ --- денежная выплата (дивиденд , купон, аренда квартиры) в этот же момент времени. Нас интересует полная доходность --- рост цены (capital gain) плюс дивиденды. Считаем, что налогов нет.
\begin{align*}
R_{t+1} = \dfrac{P_{t+1} + D_{t+1}}{P_t} - 1
\end{align*}
Следствие: ниже цена сегодня --- выше будущая доходность, и наоборот. Например, если инвесторы требуют премию за риск, то она будет скорее отражаться в более низкой цене сегодня, чем в будущих выплатах (которые мы не всегда знаем).
\end{frame}



\begin{frame}{Теория вероятностей}
\justify
Пусть $X$ и $Y$ --- случайные величины, доходности двух активов.

\vspace{\baselineskip}
Мат. ожидание линейно:
\begin{align*}
\mathbb{E}(\alpha X + \beta Y) = \alpha\mathbb{E}X + \beta\mathbb{E}Y
\end{align*}

Дисперсия и стандартное отклонение:
\begin{align*}
Var(X) = \sigma_{X}^2 = \mathbb{E}\left[(X - \mathbb{E}X)^2\right]
\end{align*}

Ковариация и корреляция:
\begin{align*}
Cov(X, Y) = \mathbb{E}\left[(X - \mathbb{E}X)(Y - \mathbb{E}Y)\right] = \rho\sigma_X\sigma_Y
\end{align*}


Дисперсия линейной комбинации случайных величин:
\begin{align*}
Var(\alpha X + \beta Y) &= \alpha^2 Var(X) + \beta^2 Var(Y) + 2Cov(X, Y) = \\
&= \alpha^2\sigma_X^2 + \beta^2 \sigma_Y^2 + 2\rho\sigma_X\sigma_Y
\end{align*}\end{frame}



\begin{frame}{О пользе корреляций - 1}
\justify
Две акции $X$ и $Y$ имеют одинаковые ожидаемые доходности и стандартные отклонения: $\mu = 5\%$ и $\sigma = 10\%$. Корреляция доходностей $\rho=0.4$. Инвестор может вложить долю $w$ своего богатства в $X$ и долю $(1 - w)$ в $Y$.

\vspace{\baselineskip}
Ожидаемая доходность не зависит от $w$:
\begin{align*}
\mathbb{E}(wX + (1 - w)Y) &= w\mathbb{E}X + (1-w)\mathbb{E}Y \\ &= w\mu + (1 - w)\mu = \mu = 5\%
\end{align*}

Может ли выбор $w$ уменьшить стандартное отклонение, т.е. риск?
\end{frame}



\begin{frame}{О пользе корреляций - 2}
\center
\begin{tikzpicture}
\begin{axis}[
  width = \textwidth * 0.7,
  xlabel={\small $w$ --- доля инвестиций в акцию $X$},
  ylabel={\small Ст. откл. портфеля, \%},
  xlabel near ticks,
  ylabel near ticks,
  xmin=0, xmax=1,
  ymin=0, ymax=10
]
   \addplot[
        color = Set1-B,
		  line width = 1pt,
		  samples at = {0,0.05,...,1}
	]
	{10 * sqrt(x^2 + (1-x)^2 + 2*0.4*x*(1-x))};

	\draw[
		color=black,
		dashed
	]
	(axis cs: 0.5, 0) -- (axis cs: 0.5, 8.366) -- (axis cs: 0, 8.366);

	\node[
		anchor=north west
	]
	at (axis cs: 0.5, 8.366)
	{(0.5, 8.37\%)};

    \node[
        color=Set1-B,
        circle,
        fill, 
        inner sep=2pt
    ]
    at (axis cs: 0.5, 8.366) {};
\end{axis}
\end{tikzpicture}

\justify
Вывод: диверсификация уменьшает риск при той же доходности, если корреляция активов отлична от 1.
\end{frame}



\begin{frame}{Оптимизация портфеля по Марковицу - 1}
\centering
\begin{tabular}{l|r|r|r|r|r|r}
 & \multicolumn{2}{c|}{Доходность} & \multicolumn{4}{c}{Корреляция} \\ \cline{2-7}
Актив         & Сред.  & Ст.~откл. & Акц. & Обл. & Нед. & Зол. \\ \hline
Акции         & 10.9\% & 15.2\%    & 1.00  & 0.00   & 0.59    & 0.04 \\
Облигации     & 5.2\%  & 3.6\%     & 0.00  & 1.00   & 0.19    & 0.28 \\
Недвижимость  & 10.8\% & 19.2\%    & 0.59  & 0.19   & 1.00    & 0.13 \\
Золото        & 7.0\%  & 15.6\%    & 0.04  & 0.28   & 0.13    & 1.00
\end{tabular}

\justify
\scriptsize
Средние годовые доходности, стандартные отклонения и корреляции по данным сайта Portfolio Visualizer (1994-2020).
\end{frame}



\begin{frame}{Оптимизация портфеля по Марковицу - 2}
\justify
Пусть у нас есть $n$ активов с ожидаемыми доходностями $\mu_i$, стандартными отклонениями $\sigma_i$, корреляциями $\rho_{i,j}$. Инвестиционный портфель задан весами $x_i$. Какие веса активов в портфеле минимизируют риск при фиксированной ожидаемой доходности $r$?

\begin{align*}
x = \begin{bmatrix}
x_1, \\
x_2, \\
\vdots \\
x_n
\end{bmatrix},
\mu = \begin{bmatrix}
\mu_1, \\
\mu_2, \\
\vdots \\
\mu_n
\end{bmatrix},
S = \begin{bmatrix}
\sigma_1^2 & \rho_{1,2}\sigma_1\sigma_2 & \cdots & \rho_{1,n}\sigma_1\sigma_n \\
\rho_{2,1}\sigma_2\sigma_1 & \sigma_2^2 & \cdots & \rho_{2,n}\sigma_2\sigma_n \\
\vdots & \vdots & \ddots & \vdots \\
\rho_{n,1}\sigma_n\sigma_1 & \rho_{n,2}\sigma_n\sigma_2 & \cdots & \sigma_n^2
\end{bmatrix}
\end{align*}

Задача квадратичного программирования (quadratic programming):
\begin{align*}
\begin{cases}
x^TS x \to \min \\
\mu^Tx = r \\
\sum x_i = 1 \\
x \ge 0
\end{cases}
\end{align*}
\end{frame}



\newcommand{\drawAssetNode}[3]{
    \node[
        circle,
        fill,
        inner sep=2pt
    ] at (axis cs: #1, #2) {};
    \node[
        anchor=north
    ]
    at (axis cs: #1, #2)
    {\scriptsize #3};
}

\newcommand{\drawPortfolioNode}[8]{
    \node[
        anchor=#8,
        inner sep=1pt
    ] at (axis cs: #1, #2) {
    	  \setlength\tabcolsep{2pt}
	     \scriptsize
	     \begin{tabular}{|l|r|}
		  \hline
		  \multicolumn{2}{|c|}{#3} \\ \hline
		  Акц. & #4\% \\
		  Обл. & #5\% \\
		  Нед. & #6\%  \\
		  Зол. & #7\% \\
		  \hline
		  \end{tabular}
    };
	 
	 \node[
	     circle,
	     fill,
	     inner sep=2pt,
	     color=Set1-B
    ] at (axis cs:#1, #2) {};
}



\begin{frame}{Граница эффективности - 1}

\centering
\begin{tikzpicture}
\begin{axis}[
    width=\textwidth,
    height=\textheight - 1cm,
    xlabel={\small Стандартное отклонение, \%},
    ylabel={\small Ожидаемая доходность, \%},
    xlabel near ticks,
    ylabel near ticks,
    xmin=0, xmax=21,
    ymin=0, ymax=14
]

\addplot[
    line width=1pt,
    color=Set1-B
]
table[
    x=std_dev,
    y=target_return,
    col sep=comma
]
{data/efficient_frontier_plot_data.csv};

\drawAssetNode{15.2}{10.9}{Акции}

\drawAssetNode{3.6}{5.2}{Облигации}

\drawAssetNode{19.2}{10.8}{Недвиж.}

\drawAssetNode{15.6}{7.0}{Золото}

\drawPortfolioNode{4.40}{6.5}{Портфель 1}{21.2}{74.6}{0.5}{3.7}{south east}

\drawPortfolioNode{12.0}{9.95}{Портфель 2}{63.2}{3.5}{14.4}{18.9}{south east}

\drawPortfolioNode{12.0}{6.5}{Портфель 3}{0.0}{30.0}{0.0}{70.0}{north east}

\draw[dashed] (axis cs: 12.0, 9.95) -- (axis cs: 12.0, 0);
\draw[dashed] (axis cs: 4.40, 6.5) -- (axis cs: 12.0, 6.5);
\draw[dashed] (axis cs: 4.40, 0) -- (axis cs: 4.40, 6.5);

\end{axis}
\end{tikzpicture}
\end{frame}



\begin{frame}{Граница эффективности - 2}
\centering
\begin{tikzpicture}
\begin{axis}[
  width=\textwidth,
  height=\textwidth/2,
  xlabel={\small Стандартное отклонение, \%},
  ylabel={\small Доля актива, \%},
  xlabel near ticks,
  ylabel near ticks,
  xmin=3.5, xmax=15.2,
  ymin=0, ymax=100,
  stack plots=y
]

\node at (axis cs:9.3, 25) {\small Акции};
\node at (axis cs:9.3, 63) {\small Облигации};
\node at (axis cs:9.3, 82) {\small Недвиж.};
\node at (axis cs:9.3, 93) {\small Золото};

\addplot[fill=Pastel2-C] table[x=std_dev, y=stocks, col sep=comma] {data/efficient_frontier_plot_data.csv} \closedcycle;

\addplot[fill=Pastel2-D] table[x=std_dev, y=bonds, col sep=comma] {data/efficient_frontier_plot_data.csv} \closedcycle;

\addplot[fill=Pastel2-E] table[x=std_dev, y=reit, col sep=comma] {data/efficient_frontier_plot_data.csv} \closedcycle;

\addplot[fill=Pastel2-F] table[x=std_dev, y=gold, col sep=comma] {data/efficient_frontier_plot_data.csv} \closedcycle;
\end{axis}
\end{tikzpicture}
\justify
Граница эффективности (efficient frontier) --- линия, на которой лежат портфели, имеющие минимальный риск при заданном уровне доходности. Обычно портфели на границе содержат комбинацию базовых активов, а не какой-то один из них.
\end{frame}



\begin{frame}{Оптимизация с безрисковым активом - 1}
\justify
Предположим, что на рынке есть безрисковый (risk-free) $F$, который имеет стандартное отклонение $0\%$. Пример: короткие Treasury Bills (ни единого разрыва за 230 лет).

\justify
Предположим также, что инвесторы могут занимать деньги под безрисковую процентную ставку.
\end{frame}


\renewcommand{\drawAssetNode}[4]{
    \node[circle,fill,inner sep=2pt] at (axis cs: #1, #2) {};
    \node[anchor=#4] at (axis cs: #1, #2) {\small #3};
}

\renewcommand{\drawPortfolioNode}[8]{\node[anchor=#8] at (axis cs: #1, #2) {
		\scriptsize \begin{tabular}{|l|r|}
		\hline
		\multicolumn{2}{|c|}{#3} \\ \hline
		Акц. & #4\% \\
		Обл. & #5\% \\
		Нед. & #6\%  \\
		Зол. & #7\% \\
		\hline
		\end{tabular}
	};
	\node[circle, fill, inner sep=2pt, color=Set1-B] at (axis cs:#1, #2) {};
}

\newcommand{\drawPortfolioNodeTwo}[6]{
\node[anchor=#6] at (axis cs: #1, #2) {
		\scriptsize 
		\setlength\tabcolsep{2pt}
		\begin{tabular}{|l|r|}
		\hline
		\multicolumn{2}{|c|}{#3} \\ \hline
		Портф. T & #4\% \\
		Безриск. & #5\% \\
		\hline
		\end{tabular}
	};
	\node[circle, fill, inner sep=2pt, color=Set1-B] at (axis cs:#1, #2) {};
}

\begin{frame}{Оптимизация с безрисковым активом - 2}
\centering
\begin{tikzpicture}
\begin{axis}[
    width=\textwidth,
    height=\textheight-1cm,
    xlabel={\small Стандартное отклонение, \%},
    ylabel={\small Ожидаемая доходность, \%},
    xlabel near ticks,
    ylabel near ticks,
    xmin=0, xmax=14,
    ymin=0, ymax=10
]

\addplot[line width=1pt, color=Set1-B, domain=0:21] {4 + x * (6.75 - 4) / 4.83};

\drawPortfolioNodeTwo{1.45}{4.83}{Портфель А}{30}{70}{north west}

\drawPortfolioNodeTwo{7.25}{8.12}{Портфель B}{150}{-50}{north west}

\drawAssetNode{0.0}{4.0}{F}{north west}

\drawPortfolioNode{4.83}{6.75}{Портфель T}{24.2}{69.5}{1.5}{4.8}{south east}

\addplot[line width=1pt, color=Set1-B, dashed] table[x=std_dev, y=target_return, col sep=comma] {data/efficient_frontier_plot_data.csv};

\end{axis}
\end{tikzpicture}
\end{frame}



\begin{frame}{Оптимизация с безрисковым активом - 3}
\justify
Теорема о двух фондах: любой портфель на новой границе эффективности можно представить как линейную комбинацию безрискового актива $F$ и касательного (tangent) портфеля рискованных активов $T$.

\vspace{\baselineskip}
В зависимости от предпочтительного баланса риска и доходности, все инвесторы будут держать разные количества $F$ и $T$. Однако никто не будет держать отличную от $T$ комбинацию рискованных активов, потому что такая комбинация будет заведомо хуже (меньше доход при том же риске или больший риск при той же доходности).

\vspace{\baselineskip}
Все инвесторы в совокупности распределят свои деньги между рискованными активами в соответствии с весами в портфеле $T$. Портфель $T$ --- рыночный портфель (market portfolio)!
\end{frame}



\begin{frame}{Оптимизация с безрисковым активом - 4}
\justify
Предположим, что все инвесторы владеют \$200 млрд. \$100 млрд. они вложили в безрисковый актив, а \$100 млрд. --- в рискованные активы.

\justify
Кто-то вложит в рискованную активы большую долю капитала, кто-то меньшую. Но каждый вложит в золото 4.8\% от своих инвестиций в рискованные активы, потому что иначе он окажется под границей эффективности. Суммарно все инвесторы купят золота на \$4.8 млрд.

\justify
Сколько золота купят инвесторы? Всё, что есть на рынке! Ни один слиток не останется бесхозным. Если в природе существует 4.8 миллиона унций золота, то спрос и предложение уравновесятся на цене \$1000 за унцию.
\end{frame}



\begin{frame}{На пути к CAPM}
\justify
Обозначим $R_{free}$ и $R_{mkt}$ доходности безрискового актива и рыночного портфеля, а $\sigma_{mkt}$ --- стандартное отклонение рыночного портфеля. Инвестор вложил долю $\beta$ своего богатства в рыночный портфель, а долю $1-\beta$ --- в безрисковый актив. Какую среднюю доходность $\mathbb{E}(R)$ он получит?

\begin{align*}
\mathbb{E}(R) &= \mathbb{E}\left[ \beta R_{mkt} + (1 - \beta)R_{free} \right] \Leftrightarrow \\
\mathbb{E}(R) - R_{free} &= \beta (\mathbb{E}(R_{mkt}) - R_{free})
\end{align*}

Избыточная доходность (excess return) составного портфеля зависит от $\beta$ --- чувствительности к рыночному риску (market risk). Это верно для портфелей на границе эффективности. Но вдруг это верно для любого актива в экономике?
\end{frame}



\begin{frame}{Capital Asset Pricing Model}
Можно доказать, что ожидаемая доходность $\mathbb{E}(R_{asset})$ любого актива на рынке равна
\begin{align*}
\mathbb{E}(R_{asset}) = R_{free} + \dfrac{Cov(R_{asset}, R_{mkt})}{Var(R_{mkt})} \left (\mathbb{E}(R_{mkt}) - R_{free} \right)
\end{align*}
Обозначим
\begin{align*}
\beta_{asset} = \dfrac{Cov(R_{asset}, R_{mkt})}{Var(R_{mkt})} = \rho_{asset,mkt}\dfrac{\sigma_{asset}}{\sigma_{mkt}}
\end{align*}
Тогда:
\begin{align*}
\mathbb{E}(R_{asset}) - R_{free} = \beta_{asset}(\underbrace{\mathbb{E}(R_{mkt}) - R_{free}}_{\text{премия за риск}})
\end{align*}
Это формулировка Capital Asset Pricing Model (CAPM).
\end{frame}



\begin{frame}{Систематический риск}
\justify
\begin{align*}
&\mathbb{E}(R_{asset}) - R_{free} = \beta_{asset}(\mathbb{E}(R_{mkt}) - R_{free}) \\
&\beta_{asset} = \dfrac{Cov(R_{asset}, R_{mkt})}{Var(R_{mkt})} = \rho_{asset,mkt}\dfrac{\sigma_{asset}}{\sigma_{mkt}}
\end{align*}

Доходность любого актива зависит от его <<беты>>. <<Бета>> 1.5 означает, что если рынок растёт (падает) на 1\%, то актив в среднем растёт (падает) на 1.5\%. Придумайте примеры акций с $\beta<1$ и $\beta>1$?

\vspace{\baselineskip}
Можно сказать, что <<бета>> отражает чувствительность актива к некому глобальному рыночному риску (market risk) или систематическому риску (systematic risk). Насколько актив растёт вместе с рынком в хорошие времена, и насколько он падает вместе с рынком в плохие.
\end{frame}



\begin{frame}{Идиосинкратический риск}
\justify
Каждый актив (каждая акция) несёт свой собственный специфический риск. Самолёт авиакомпании может упасть, завод корпорации может сгореть. Этот идиосинкратический (idiosyncratic) риск можно уменьшить до нуля диверсификацией (diversify away). Рациональный инвестор держит в портфеле тысячи активов, поэтому не страдает от этого риска.

\vspace{\baselineskip}
Вывод: идиосинкратический риск не входит в цену активов! Вы не зарабатываете дополнительную премию за риск, если держите в портфеле всего одну (пусть даже самую любимую) акцию. Все остальные инвесторы диверсифицировали этот риск и поэтому не требуют премию за риск (т.е. скидку в цене).

\vspace{\baselineskip}
Систематический риск нельзя диверсифицировать. Инвесторы должны с ним мириться, поэтому требуют и получают компенсацию.
\end{frame}



\begin{frame}{Систематический риск и предел диверсификации - 1}
\justify
Пусть есть $n$ акций с одинаковыми ожидаемыми доходностями $\mu=5\%$ и стандартными отклонениями $\sigma = 20\%$. Все акции попарно связаны коэффициентом корреляции $\rho=0.25$. Мы вложили по доле $1/n$ в каждую акцию. Чему равно стандартное отклонение портфеля?

\begin{align*}
\sqrt{Var\left(\sum\limits_{i=1}^{n}\frac{1}{n}X_i \right)} &=
\sqrt{\frac{1}{n^2}\left(\sum\limits_{i=1}^{n}Var(X_i) + 2\sum\limits_{1 \le i < j \le n}Cov(X_i, X_j)\right)} = \\
&= \sqrt{\frac{1}{n^2}\left(n\sigma^2 + n(n-1)\rho\sigma^2\right)} = \\
&= \sigma\sqrt{\frac{1}{n} + \frac{n-1}{n}\rho} \to \sigma\sqrt{\rho}
\end{align*}
\end{frame}



\begin{frame}{Систематический риск и предел диверсификации - 2}
\centering
\begin{tikzpicture}
\begin{axis}[
    width=\textwidth,
    height=\textheight-1cm,
    xlabel={\small Количество акций в портфеле},
    ylabel={\small Стандартное отклонение портфеля, \%},
    xlabel near ticks,
    ylabel near ticks,
    xmin=0, xmax=50,
    ymin=0, ymax=20
]

\addplot[line width=1pt, color=Set1-B, domain=1:50, samples at={1,1.2,...,50}] {20 * sqrt(1/x + (x-1)/x*0.25)};

\addplot[line width=1pt, dashed, color=Set1-B, domain=0:50]{10};

\draw[<->,>=triangle 90] (axis cs: 45, 0) -- (axis cs: 45, 10) node[pos=0.5,anchor=east]{\scriptsize{Недиверсифицируемый (систематический) риск}};

\draw[<->,>=triangle 90] (axis cs: 45, 10) -- (axis cs: 45, 20) node[pos=0.5,anchor=east]{\scriptsize{Диверсифицируемый (идиосинкратический) риск}};

\end{axis}
\end{tikzpicture}
\end{frame}



\begin{frame}{Роль ковариации с рынком}
\justify
Какую из двух акций выбрать?

\vspace{\baselineskip}
\centering
\begin{tabular}{l|r|r|r}
Состояние мира & Вероятность & Акция 1  & Акция 2 \\ \hline
\only<1>{Состояние 1}\only<2->{Потеря работы}    & 50\%        & \$1\,000 & \$500 \\
\only<1>{Состояние 2}\only<2->{Премия \$10\,000}    & 50\%        & \$500    & \$1\,000 \\ \hline
Мат. ожидание  &             & \$750    & \$750
\end{tabular}

\only<3>{
\justify
Инвестор с выпуклой вверх функцией полезности ценит активы, которые приносят доход в плохие времена, когда остальные активы падают. Такой актив сглаживает удар по потреблению и поэтому стоит дорого.

\justify
Если актив приносит доход в хорошие времена, когда каждый дополнительный доллар не так ценен, то инвестор требует премию за риск.
}
\end{frame}



\begin{frame}{Прокси рыночного портфеля}
\justify
В рамках CAPM в рыночный портфель входят не только акции, а вообще все активы в экономике, включая недвижимость, самолёты и пароходы. Далеко не все эти активы являются торгуемыми. Для практического применения CAPM нужно выбрать <<прокси>> --- портфель торгуемых активов. Обычно это фондовый индекс.

\vspace{\baselineskip}
\begin{itemize}
\justifying
\item Standard and Poor's 500 --- 500 крупнейших компаний США.
\item Russel 2000 --- 2000 крупнейших компаний США.
\item STOXX 600 --- 600 крупнейших компаний Европы.
\item ММВБ (рубли) и РТС (доллары) --- 42 крупнейшие компании России.
\item MSCI World --- 1600 крупнейших компаний мира.
\end{itemize}
\end{frame}



\begin{frame}{Фондовый индекс}
\justify
Фондовый индекс --- индикатор состояния рынка, отражающий изменение цены заданной корзины ценных бумаг. Самый распространённый тип индекса --- индекс, взвешенный по капитализации (cap-weighted).

\begin{align*}
Index = \dfrac{\sum P_iQ_i}{D}
\end{align*}
Здесь $P_i$ --- цена бумаги $i$, $Q_i$ --- число бумаг $i$ в свободном обращении (free-float).

\vspace{\baselineskip}
Делитель $D$ обычно подбирают так, чтобы в момент самого первого расчёта индекс был круглым числом (например, 100). Впоследствии его подправляют при изменении состава индекса, разделении акций, и т.п.
\end{frame}



\begin{frame}{Рыночная премия за риск - 1}
\begin{align*}
\mathbb{E}(R_{asset}) - R_{free} = \beta_{asset}(\mathbb{E}(R_{mkt}) - R_{free})
\end{align*}
\justify
Ожидаемая избыточная доходность инвестиций пропорциональна мере систематического риска $\beta_{asset}$ и рыночной премии за риск (market risk premium) $\mathbb{E}(R_{mkt}) - R_{free}$. О какого порядка премии мы говорим?
\end{frame}



    \newcommand{\addGrowthPlot}[4]{
        \addplot[
            color = #2,
            line width = 1pt, 
            mark = #3,
            mark repeat = 120,
            mark phase = 396,
            mark options = {scale=2},
            style = #4
        ]
        table[
            x = date,
            y = #1,
            col sep = comma
        ]
        {data/fama_french_cumulative_growth_data.csv};
    }
    
    \newcommand{\addFlatLine}[5] {
        \draw[
            red,
            thick
        ]
        (axis cs: #1, #3) -- (axis cs: #2, #3)
        node[
            pos=#5,
            anchor=south
        ]
        {\scriptsize #4};
    }
    
    \newcommand{\addLossLine}[5] {
        \draw[
            red,
            thick
        ]
        (axis cs: #1, #3) -- (axis cs: #2, #4)
        node[
            anchor=west
        ]
        {\scriptsize #5\%};
    }

\begin{frame}{Рыночная премия за риск - 2}
\centering

\begin{tikzpicture}
\begin{axis}[
    width=\textwidth,
    height=\textheight - 1cm,
    date coordinates in=x,
    date ZERO=1926-06-30,
    xtick={1930-01-01,1940-01-01,1950-01-01,1960-01-01,1970-01-01,1980-01-01,1990-01-01,2000-01-01,2010-01-01,2020-01-01},
    minor xtick={1930-01-01,1950-01-01,1970-01-01,1990-01-01,2010-01-01},
    xticklabel=\year,
    grid=both,
    xmin=1926-12-31,
    xmax=2025-01-01,
    ymode=log,
    ymax=10000,
    log ticks with fixed point,
    ylabel={\small Рост \$1 начальных инвестиций},
    ylabel shift = -10pt,
    legend entries={
        Рынок акций,
        Акции минус облигации,
        Безрисковые облигации,
        Инфляция (CPI-U)
    },
    legend pos=north west,
    legend style={font=\scriptsize},
    legend cell align={left}
]
    
    \addGrowthPlot{mkt}{Set1-B}{none}{solid}
    \addGrowthPlot{mkt_rf}{Set1-C}{none}{solid}
    \addGrowthPlot{rf}{Set1-D}{none}{solid}
    \addGrowthPlot{cpi}{Set1-E}{none}{dashed}

    \addFlatLine{1929-08-31}{1945-02-28}{2.10}{1929--45}{0.5}
    \addFlatLine{1968-11-30}{1983-04-30}{26.2}{1968--83}{0.9}
    \addFlatLine{2000-03-31}{2013-01-31}{137}{2000--13}{0.5}

    \addLossLine{1929-08-31}{1932-06-30}{2.10}{0.322}{-85}
    \addLossLine{1968-11-30}{1974-09-30}{26.2}{11.6}{-56}
    \addLossLine{2000-03-31}{2009-02-28}{137}{62.7}{-54}
\end{axis}
\end{tikzpicture}

\scriptsize{Данные: Kenneth French Data Library.}
\end{frame}



\begin{frame}{Рыночная премия за риск - 3}
\centering
\begin{tabular}{l|r|r|r|r}
Период & Сред. (ст. откл.) & $t$-тест & $p$-знач. & 99\% дов. инт. \\
\hline
1927--1959 & 11.3\% (24.9\%) & 2.61 &  1.4\% & [-0.6\%, 23.2\%] \\
1960--1989 &  5.2\% (16.9\%) & 1.69 & 10.3\% & [-3.3\%, 13.7\%] \\
1990--2020 &  9.4\% (17.9\%) & 2.90 &  0.7\% & [ 0.5\%, 18.2\%] \\
1960--2020 &  7.3\% (17.4\%) & 3.28 &  0.2\% & [ 1.4\%, 13.2\%] \\ \hline
1927--2020 &  8.7\% (20.3\%) & 4.16 & <0.1\% & [ 3.2\%, 14.2\%] 
\end{tabular}
{\scriptsize Годовые доходности сверх безрисковой процентной ставки.

Данные: Kenneth French Data Library. \par }

\justify
Внимание! Есть точка зрения, что историческая доходность рынка США слишком высока и является загадкой (equity risk premium puzzle). Инвесторы должны очень-очень не любить риск, чтобы спрос и предложение уравновесились на такой премии за риск. Не имеем ли мы дело с эффектом выжившего (survivorship bias)?
\end{frame}



\begin{frame}{Избегание риска - 1}
\justify
Вы должны вложить весь свой капитал на срок 15 лет. Какой актив выбрать: A или B?

\centering
\begin{tikzpicture}
\begin{groupplot}[
    group style = {group size = 2 by 1},
    width = \textwidth / 2,
	 ybar,
	 ymin = -30, ymax = 50,
	 xmin = 0.5, xmax = 50.5,
	 xlabel={\small Номер сценария},
	 ylabel={\small Годовая доходность, \%},
	 xlabel near ticks,
	 ylabel near ticks,
	 xtick={1, 10,20,30,40,50}
]
    
    \nextgroupplot[title = {Актив A}]
    \addplot[bar width=1pt, fill, color=Set1-B] table[x=mkt_1y_rank, y=sample_mkt_1y, col sep=comma] {data/simulated_market_annual_returns.csv};
    
    \nextgroupplot[title = {Актив B}, ylabel={}]
    \addplot[bar width=1pt, fill, color=Set1-B] table[x=rf_1y_rank, y=sample_rf_1y, col sep=comma] {data/simulated_market_annual_returns.csv};
    
\end{groupplot}
\end{tikzpicture}
\end{frame}



\begin{frame}{Избегание риска - 2}
\justify
Вы должны вложить весь свой капитал на срок 15 лет. Какой актив выбрать: C или D?

\centering
\begin{tikzpicture}
\begin{groupplot}[
    group style = {group size = 2 by 1},
    width = \textwidth / 2,
	 ybar,
	 ymin = 0, ymax = 20,
	 xmin = 0.5, xmax = 50.5,
	 xlabel={\small Номер сценария},
	 ylabel={\small Годовая доходность, \%},
	 xlabel near ticks,
	 ylabel near ticks,
	 xtick={1, 10,20,30,40,50}
]
    
    \nextgroupplot[title = {Актив C}]
    \addplot[bar width=1pt, fill, color=Set1-B] table[x=mkt_15y_rank, y=sample_mkt_15y, col sep=comma] {data/simulated_market_annual_returns.csv};
    
    \nextgroupplot[title = {Актив D}, ylabel={}]
    \addplot[bar width=1pt, fill, color=Set1-B] table[x=rf_15y_rank, y=sample_rf_15y, col sep=comma] {data/simulated_market_annual_returns.csv};
    
\end{groupplot}
\end{tikzpicture}
\end{frame}



\begin{frame}{Избегание риска - 3}
\justify
Активы A и C --- рынок акций США. Активы B и D --- безрисковые облигации. Графики A и B показывают доходность за один наудачу выбранный год. Графики С и D показывают доходность за наудачу выбранные 15 лет подряд.

\justify
Ричард Талер считает, что существует эффект <<близорукого избегания риска>>. Мы не можем сложить в уме 15 случайных годовых доходностей и оценить распределение суммы. Инвесторы интуитивно умножают годовые колебания на 15 и переоценивают риск. На деле шанс получить отрицательную доходность на горизонте 15 лет не так велик, потому что центральная предельная теорема усредняет краткосрочные колебания.

\justify
Следствие. Чем реже вы проверяете состояние счёта, тем более рискованный (и более доходный) портфель вы можете себе позволить.
\end{frame}



\begin{frame}{Отношение Шарпа}
\justify
Отношение Шарпа (Sharpe ratio) --- отношение средней избыточной доходности к её стандартному отклонению.
\begin{align*}
\overline{R_{excess}} &= \dfrac{1}{n}\sum\limits_{t=1}^{n}(R_{portfolio,t} - R_{free,t})\\
Sharpe &= \dfrac{\overline{R_{excess}}}{\sqrt{\dfrac{1}{n-1}\sum\limits_{t=1}^{n}(R_{portfolio,t} - R_{free,t} - \overline{R_{excess}})^2}}
\end{align*}

Хозяйке на заметку: касательный (tangent) портфель имеет максимальное отношение Шарпа среди всех портфелей, составленных из рискованных активов.
\end{frame}



\begin{frame}{Отношение Сортино}
\justify
Отношение Сортино (Sortino information ratio) --- отношение доходности сверх минимально допустимой $R_{min}$ (minimum acceptable return) к волатильности <<вниз>>.
\begin{align*}
\overline{R_{excess}} &= \dfrac{1}{n}\sum\limits_{t=1}^{n}(R_{portfolio,t} - R_{min,t}) \\
Sortino &= \dfrac{\overline{R_{excess}}}{\sqrt{\dfrac{1}{n-1}\sum\limits_{t=1}^{n}\left(\min[0, R_{portfolio,t} - R_{min, t}]\right)^2}}
\end{align*}
В качестве минимально допустимой доходности можно использовать как константу (например, 0), так и переменную величину (например, безрисковую доходность).
\end{frame}



\begin{frame}{Геометрическое среднее}
\justify
Арифметическое среднее доходностей --- лучшая оценка того, что случится с вашими инвестициями в следующий период (месяц, год). Если вы оцениваете инвестиции на несколько периодов, то используйте геометрическое среднее.
\begin{align*}
Geom Mean = \left(\prod\limits_{t=1}^{n}(1 + R_{portfolio,t} - R_{free,t}) \right)^{\dfrac{1}{n}} - 1
\end{align*}

\centering
\begin{tabular}{l|r|r|r|r|r}
\multirow{2}{*}{Период} &
Арифм. & 
\multirow{2}{*}{Ст. откл.} &
Геом. &
Отнош. &
Отнош. \\
& сред. & & сред. & Шарпа & Сортино \\ 
\hline
1927--1959 & 11.3\% & 24.9\% & 8.3\% & 0.45 & 1.00 \\
1960--1989 &  5.2\% & 16.9\% & 3.8\% & 0.31 & 0.58 \\
1990--2020 &  9.4\% & 17.9\% & 7.8\% & 0.52 & 1.06 \\
1960--2020 &  7.3\% & 17.4\% & 5.8\% & 0.42 & 0.82 \\ \hline
1927--2020 &  8.7\% & 20.3\% & 6.7\% & 0.43 & 0.89
\end{tabular}

{\scriptsize Годовые доходности сверх безрисковой процентной ставки.

Данные: Kenneth French Data Library. \par }
\end{frame}



\begin{frame}{Прогнозирование рыночной премии за риск}
\justify
Можно ли предсказать, какой будет equity risk premium в следующем году, чтобы если она будет отрицательной, то пересидеть год в безрисковом активе? Скорее нет, чем да.

\justify
Welch, Goyal (2008): отношения P/E (цена акции к прибыли), D/P (дивиденды к цене акции) и другие не предсказывают рыночную доходность out of sample. То есть они не лучше предсказания <<в следующем году доходность будет равна средней доходности в прошлом>>.

\justify
В 2015--2019 годах многие аналитики строили графики дробей <<что-то на что-то>> и предсказывали обвал рынка. Пришёл март 2020 г. и обвал на 35\% случился. Случился ли он из-за <<перегретости рынка>> или из-за пандемии?
\end{frame}




\begin{frame}{Пример: cyclically adjusted price-earnings (CAPE)}
\centering
\begin{tikzpicture}
\begin{axis}[
    width=\textwidth,
    height=\textheight - 1cm,
    date coordinates in=x,
    date ZERO=1880-01-01,
    xtick={1880-01-01,1900-01-01,1920-01-01,1940-01-01,1960-01-01,1980-01-01,2000-01-01,2020-01-01},
    minor xtick={1890-01-01,1910-01-01,1930-01-01,1950-01-01,1970-01-01,1990-01-01,2010-01-01},
    xticklabel=\year,
    grid=both,
    xmin=1880-01-01,
    xmax=2025-01-01,
    ylabel={\small Cyclically adjusted P/E},
    %ylabel shift = -10pt,
    %legend entries={
    %    Рынок акций,
    %    Акции минус облигации,
    %    Безрисковые облигации,
    %    Инфляция (CPI-U)
    %},
    %legend pos=north west,
    %legend style={font=\scriptsize},
    %legend cell align={left}
]
    
        \addplot[
            color = Set1-B,
            line width = 1pt
        ]
        table[
            x = month,
            y = shiller_cape,
            col sep = comma
        ]
        {data/shiller_cape.csv};
\end{axis}
\end{tikzpicture}
{\scriptsize Данные: Robert Shiller.}
\end{frame}



\begin{frame}{Цена к прибыли и другие соотношения}
\justify
Пусть компания выплачивает дивиденды $D$ раз в год и тратит на дивиденды долю $f$ чистой прибыли $E$ (earnings). Дивиденды и прибыль растут на $g$ процентов в год. Ставка дисконтирования $r$. Тогда цена акции $P$ равна:
\begin{align*}
P = \frac{D}{r -g} = \frac{fE}{r-g} \Rightarrow \frac{P}{E} = \frac{f}{r -g}
\end{align*}
Ставку дисконтирования $r$ можно оценить по CAPM:
\begin{align*}
r = r_f + \beta\pi,
\end{align*}
где $r_f$ --- безрисковая ставка, $\pi$ --- рыночная премия за риск.

\justify
Вывод: ниже процентные ставки --- выше отношение P/E акций при той же прибыли, дивидендах и ожиданиях роста. Высокие отношения P/E --- зеркальное отражение самых низких в истории процентных ставок.
\end{frame}



\begin{frame}{Оценка успеха управляющих - 1}
\justify
Лектор составил портфель акций на Московской бирже в ноябре 2015 г. За 5 лет он заработал 100\%. Три паевых фонда, красный, жёлтый и зелёный, заработали 104\%, 85\% и 84\%. Молодцы ли лектор и управляющие фондами?

\vspace{\baselineskip}
\only<2->{Безрисковая процентная ставка (ключевая ставка ЦБ) за тот же период дала бы $50\%$. Индекс ММВБ полной доходности (включая дивиденды минус налоги) вырос на $117\%$.

\vspace{\baselineskip}
Какой риск взяли на себя лектор и управляющие, и получили ли они компенсацию за этот риск?}
\end{frame}



\begin{frame}{Оценка успеха управляющих - 2}
\center
\begin{tikzpicture}
\begin{axis}[
  width=\textwidth,
  height=\textheight - 1cm,
  date coordinates in=x,
  date ZERO=2015-10-31,
  xtick={2016-01-01,2017-01-01,2018-01-01,2019-01-01,2020-01-01, 2021-01-01},
  xticklabel=\year,
  xmin=2015-10-31,
  xmax=2021-01-01,
  grid=major,
  ylabel={\small{Рост 1 р. начальных инвестиций}},
  xlabel near ticks,
  ylabel near ticks,
  legend entries = {
      Индекс ММВБ,
      Ставка ЦБ,
      Лектор,
      Красный,
      Жёлтый,
      Зелёный
  },
  legend pos=north west,
  legend style={font=\tiny},
  legend cell align={left}
]
\addplot[color=black, mark=none, thick] table[x=date, y=benchmark_growth, col sep=comma]{data/fund_growth.csv};
\addplot[color=black, mark=none, dashed, thick] table[x=date, y=risk_free_growth, col sep=comma]{data/fund_growth.csv};
\addplot[color=blue, mark=none, thick] table[x=date, y=personal_growth, col sep=comma]{data/fund_growth.csv};
\addplot[color=red, mark=none, thick] table[x=date, y=red_growth, col sep=comma]{data/fund_growth.csv};
\addplot[color=orange, mark=none, thick] table[x=date, y=yellow_growth, col sep=comma]{data/fund_growth.csv};
\addplot[color=green, mark=none, thick] table[x=date, y=green_growth, col sep=comma]{data/fund_growth.csv};
\end{axis}
\end{tikzpicture}
\end{frame}



\begin{frame}{Регрессия для оценки результатов - 1}
\justify
CAPM говорит, что избыточная доходность портфеля прямо пропорциональна рыночному риску:
\begin{align*}
\mathbb{E}(R_{fund}) - R_{free} = \beta_{portfolio}(\mathbb{E}(R_{mkt}) - R_{free})
\end{align*}

\justify
Посчитаем ежемесячные доходности рынка, портфеля и безрискового актива, а затем оценим линейную регрессию:
\begin{align*}
R_{fund,t} - R_{free,t} = \alpha + \beta(R_{mkt,t} - R_{free,t}) + \epsilon_t, \epsilon_t \sim \mathcal{N}(0, \sigma_{\epsilon}^2)
\end{align*}
По CAPM, $\alpha=0$. Если управляющий имеет положительную <<альфу>>, то он зарабатывает больше, чем можно было бы ожидать при данном уровне систематического риска.
\end{frame}



\begin{frame}{Регрессия для оценки результатов - 2}
\center
\begin{tikzpicture}
\begin{axis}[
  width=\textwidth,
  height=\textheight - 1.5cm,
  legend entries = {
    Лектор,
    Красный фонд,
    $y = 0.12 + 0.60x$,
    $y = -0.05 + 0.92x$
  },
  legend pos=north west,
  legend style={font=\tiny},
  legend cell align={left},
  xlabel={\small{Месячная избыточная доходность индекса}},
  ylabel={\small{Месячная избыточная доходность фонда}},
  xlabel near ticks,
  ylabel near ticks,
  %xtick={-0.1, -0.05, 0, 0.05, 0.1},
  %ytick={-0.1, -0.05, 0, 0.05, 0.1},
  grid=major,
  xmin=-11,xmax=11,
  ymin=-11,ymax=11
]

\addplot[color=blue, only marks] table[x=benchmark_excess_return, y=personal_excess_return, col sep=comma]{data/fund_excess_return.csv};

\addplot[color=red, only marks] table[x=benchmark_excess_return, y=red_excess_return, col sep=comma]{data/fund_excess_return.csv};

\addplot[color=blue, domain=-12:12] {0.1163 + 0.599451*x};

\addplot[color=red, domain=-12:12] {-0.04577 + 0.9202177*x};

\end{axis}
\end{tikzpicture}
\end{frame}



\begin{frame}{Регрессия для оценки результатов - 3}
\center
\begin{tabular}{l|c|c|c|c|c}
\multirow{2}{*}{Фонд}    & $\hat{\alpha}$ & $\hat{\beta}$ & \multirow{2}{*}{$R^2$} & Отнош. & Отнош. \\
& \scriptsize{(ст. откл.)} & \scriptsize{(ст. откл.)} & & Шарпа & Сортино \\ \hline
Лектор  &  0.12\% \scriptsize{(0.29\%)} & 0.60 \scriptsize{(0.07)}   &  0.54 & 0.60 & 1.08 \\
Красный & -0.05\% \scriptsize{(0.18\%)} & 0.92 \scriptsize{(0.05)}   &  0.87 & 0.55 & 0.90 \\
Жёлтый  & -0.20\% \scriptsize{(0.26\%)} & 0.91 \scriptsize{(0.07)}   &  0.75 & 0.38 & 0.62 \\
Зелёный & -0.28\% \scriptsize{(0.16\%)} & 1.00 \scriptsize{(0.04)}   &  0.91 & 0.37 & 0.58 \\ \hline
Индекс  & 0.00\% \scriptsize{(0.00\%)}  & 1.00 \scriptsize{(0.00)}   &  1.00 & 0.64 & 1.02
\end{tabular}
\justify
{\scriptsize(59 месячных доходностей 11.2015--09.2020. Значения $\hat{\alpha}$ и $\hat{\beta}$ для месячных доходностей. Отношения Шарпа и Сортино сконвертированы в годовое выражение умножением на $\sqrt{12}$.)}

\vspace{\baselineskip}
Вывод: инвесторы фондов зарабатывают меньше, чем могли бы при том же уровне систематического риска.

\vspace{\baselineskip}
Кстати, лучшее отношение Шарпа --- у индекса. Совпадение?
\end{frame}



\begin{frame}{Дополнительные факторы риска}
\justify
Согласно CAPM, единственный риск, за который можно заработать премию --- это систематический рыночный риск. Так ли это на самом деле? Вдруг лектор с положительной <<альфой>> (забудем на минуту, что она не статистически значимая) на самом деле заработал на другом риске?

\vspace{\baselineskip}
Несколько наблюдений об американском рынке акций:
\begin{itemize}
\justifying
\item Эффект размера (size effect). Акции небольших компаний растут лучше, чем акции крупных.
\item Эффект стоимости (value effect). Акции компаний с высоким отношением бухгалтерского капитала (book value) к рыночной капитализации (market cap) растут лучше, чем акции компаний с низким отношением.
\item Инерция (momentum). Акции, которые выросли в прошлом году, растут лучше, чем акции, упавшие в прошлом году.
\end{itemize}
\end{frame}



\begin{frame}{Факторные портфели}
\justify
Составим синтетические портфели, которые будут зарабатывать деньги за счёт эффектов размера, стоимости и инерции.

\begin{itemize}
\justifying
\item SMB (Small Minus Big): покупаем 50\% самых маленьких акций, продаём 50\% самых больших.
\item HML (High book/market Minus Low book/market): покупаем 33\% акций с высоким отношением book/market, продаём 33\% акций с низким отношением.
\item MOM (MOMentum): покупаем 30\% лучших акций прошлого года, продаём 30\% худших.
\end{itemize}
Оказывается, что все три портфеля имеют положительную историческую доходность. При этом сами стратегии довольно простые, скорее даже механические!
\end{frame}


    \newcommand{\addFactorPlot}[3]{
        \addplot[
            color=#2,
            mark=#3,
            line width=1pt,
            mark repeat=120,
            mark phase=36,
            mark options={scale=2}
        ]
        table [
            x = date,
            y = #1,
            col sep=comma
        ]
        {data/fama_french_cumulative_growth_data.csv};
    }
    
\begin{frame}{Доходности факторных портфелей - 1}
\centering
    \begin{tikzpicture}
    \begin{axis}[
        width=\textwidth,
        height=\textheight - 1cm,
        date coordinates in=x,
        date ZERO=1926-06-30,
        xtick={1930-01-01,1940-01-01,1950-01-01,1960-01-01,1970-01-01,1980-01-01,1990-01-01,2000-01-01,2010-01-01,2020-01-01},
        minor xtick={1930-01-01,1950-01-01,1970-01-01,1990-01-01,2010-01-01},
        xticklabel=\year,
        grid=both,
        xmin=1926-12-31,
        xmax=2025-01-01,
        ymode=log,
        ymax=1000,
        log ticks with fixed point,
        ylabel={\small Рост \$1 начальных инвестиций},
        ylabel shift = -10pt,
        legend entries={
            MOM (MOMentum),
            HML (High B/M Minus Low B/M),
            SMB (Small Minus Big)
        },
        legend pos=north west,
        legend style={font=\scriptsize},
        legend cell align={left}
    ]


    \addFactorPlot{mom}{Set1-A}{none}

    \addFactorPlot{hml}{Set1-B}{none}
    
    \addFactorPlot{smb}{Set1-C}{none}

    \end{axis}
    \end{tikzpicture}
{\scriptsize Данные: Kenneth French Data Library.}
\end{frame}



\begin{frame}{Доходности факторных портфелей - 2}
\centering
\small
\begin{tabular}{l|l|r|r|r|c}
Фактор & Период & Сред.\,(ст.\,откл.) & $t$-тест & $p$-знач. & 99\%\,дов.\,инт.\\ \hline
SMB & 1927--59 &  3.4\% (13.6\%) & 1.43 & 16.4\% & [-3.1\%,  9.9\%] \\    
    & 1960--89 &  3.2\% (13.8\%) & 1.28 & 21.2\% & [-3.7\%, 10.2\%] \\
    & 1990--20 &  1.3\% ( 9.5\%) & 0.79 & 43.6\% & [-3.3\%,  6.0\%] \\    
    & 1960--20 &  2.3\% (11.8\%) & 1.51 & 13.7\% & [-1.7\%,  6.3\%] \\ 
    & 1927--20 &  2.7\% (12.4\%) & 2.08 &  4.0\% & [-0.7\%,  6.0\%] \\    
\hline
HML & 1927--59 &  5.0\% (14.1\%) & 2.05 &  4.8\% & [-1.7\%, 11.7\%] \\    
    & 1960--89 &  6.1\% (11.0\%) & 3.04 &  0.5\% & [ 0.6\%, 11.6\%] \\     
    & 1990--20 &  1.1\% (15.1\%) & 0.41 & 68.2\% & [-6.4\%,  8.6\%] \\    
    & 1960--20 &  3.6\% (13.4\%) & 2.08 &  4.2\% & [-1.0\%,  8.1\%] \\     
    & 1927--20 &  4.1\% (13.6\%) & 2.91 &  0.4\% & [ 0.4\%,  7.8\%] \\      
\hline
MOM & 1927--59 &  7.4\% (17.8\%) & 2.40 &  2.3\% & [-1.1\%, 15.9\%] \\    
    & 1960--89 & 10.5\% (12.8\%) & 4.49 & <0.1\% & [ 4.0\%, 16.9\%] \\ 
    & 1990--20 &  6.0\% (16.3\%) & 2.06 &  4.8\% & [-2.0\%, 14.1\%] \\  
    & 1960--20 &  8.2\% (14.7\%) & 4.35 & <0.1\% & [ 3.2\%, 13.2\%] \\  
    & 1927--20 &  7.9\% (15.8\%) & 4.87 & <0.1\% & [ 3.7\%, 12.2\%] \\  
\hline
\end{tabular}

{\scriptsize Годовые доходности 1927--2020. Данные: Kenneth French Data Library}
\end{frame}



\begin{frame}{Модель Фамы-Френча-Кархарта}
\justify
Вернёмся к анализу успеха управляющих и дополним регрессию новыми факторами.
\begin{align*}
R_{fund,t} - R_{free,t} &= \alpha + \beta_{mkt}(R_{mkt,t} - R_{free,t}) + \\ &+ \beta_{smb}R_{smb,t} + \beta_{hml}R_{hml,t} + \beta_{mom}R_{mom,t} + \epsilon_t
\end{align*}
Если регрессия покажет, что $\beta_{smb} > 0$, то это означает, что управляющий сделал ставку на эффект размера и заработал (или не заработал) премию за риск.

\vspace{\baselineskip}
Fama and French (2010): средняя четырёхфакторная <<альфа>> ПИФов акций на рынке США -1.1\% в год с учётом комиссий.
\end{frame}



\begin{frame}{Другие факторы риска}
\justify
Факторный анализ позволяет понять, на какой риск делает ставку ваш управляющий (или вы сами). <<Нет альфы, есть только скрытые беты>>.

\vspace{\baselineskip}
Примеры объясняющих факторов, не связанных с акциями:
\begin{itemize}
\justifying
\item Форма кривой процентных ставок (доходность 10Y облигаций минус доходность 1Y облигаций).
\item Кредитный спред (доходность облигаций с рейтингом BBB минус доходность Treasury).
\item Валютный кэрри-трейд (валюты с высокими процентными ставками минус валюты с низкими процентными ставками).
\item Проданные опционы (доходность стрэддла или пут-опциона на индекс).
\end{itemize}
\end{frame}



\begin{frame}{Факторы риска на рынке России - 1}
\justify
На сайте ИПЭИ РАНХиГС ({\small \url{https://ipei.ranepa.ru/ru/capm-ru}}) собраны временные ряды факторных портфелей, составленных из российский акций.
\begin{itemize}
\justifying
\item MKT-RF: рынок минус безрисковая ставка (как в CAPM).
\item SMB: как у Фамы-Френча-Кархарта.
\item HML: как у Фамы-Френча-Кархарта.
\item MOM: как у Фамы-Френча-Кархарта.
\item LIQ (LIQuidity): неликвидные акции минус ликвидные.
\item DP (Dividend/Price): акции с высокой дивидендной доходностью (низким отношением дивиденд/цена) минус акции с низкой дивидендной доходностью.
\item SOE (State-Owned Enterprise): частные компании минус компании с гос. участием.
\end{itemize}
\end{frame}



\begin{frame}{Факторы риска на рынке России - 2}
\center
\begin{tikzpicture}
\begin{axis}[
  width=\textwidth,
  height=\textheight - 1cm,
  date coordinates in=x,
  date ZERO=2004-12-31,
  xtick={2005-01-01,2008-01-01,2011-01-01,2014-01-01,2017-01-01,2020-01-01},
  xticklabel=\year,
  xmin=2005-01-01,
  xmax=2021-01-01,
  ymin=0,
  ytick={1, 2, 3, 4, 5, 6, 7, 8, 9, 10},
  grid=major,
  ylabel={\small{Рост 1 р. начальных инвестиций}},
  xlabel near ticks,
  ylabel near ticks,
  legend entries = {
      MKT-RF,
      SMB,
      HML,
      LIQ,
      DP,
      SOE
  },
  legend pos=north west,
  legend style={font=\tiny},
  legend cell align={left}
]
\addplot[color=black, mark=none, thick] table[x=month, y=rmrf_ru, col sep=comma]{data/ru_factors_cumulative_growth_data.csv};
\addplot[color=red, mark=none, thick] table[x=month, y=smb_ru, col sep=comma]{data/ru_factors_cumulative_growth_data.csv};
\addplot[color=orange, mark=none, thick] table[x=month, y=hml_ru, col sep=comma]{data/ru_factors_cumulative_growth_data.csv};
\addplot[color=green, mark=none, thick] table[x=month, y=liq_ru, col sep=comma]{data/ru_factors_cumulative_growth_data.csv};
\addplot[color=violet, mark=none, thick] table[x=month, y=dy_ru, col sep=comma]{data/ru_factors_cumulative_growth_data.csv};
\addplot[color=blue, mark=none, thick] table[x=month, y=soe_ru, col sep=comma]{data/ru_factors_cumulative_growth_data.csv};
\end{axis}
\end{tikzpicture}

{\scriptsize Данные: ИПЭИ РАНХиГС.}
\end{frame}



\begin{frame}{Факторы риска на рынке России - 3}
\center
\begin{tikzpicture}
\begin{axis}[
  width=\textwidth,
  height=\textheight - 1cm,
  date coordinates in=x,
  date ZERO=2004-12-31,
  xtick={2015-01-01,2016-01-01,2017-01-01,2018-01-01,2019-01-01,2020-01-01},
  xticklabel=\year,
  xmin=2015-01-01,
  xmax=2020-06-30,
  ymin=0.5, ymax=2.5,
  grid=major,
  ylabel={\small{Рост 1 р. начальных инвестиций}},
  xlabel near ticks,
  ylabel near ticks,
  legend entries = {
      MKT-RF,
      SMB,
      HML,
      LIQ,
      DP,
      SOE
  },
  legend pos=north west,
  legend style={font=\tiny},
  legend cell align={left}
]
\addplot[color=black, mark=none, thick] table[x=month, y=rmrf_ru, col sep=comma]{data/ru_factors_cumulative_growth_data_2015.csv};
\addplot[color=red, mark=none, thick] table[x=month, y=smb_ru, col sep=comma]{data/ru_factors_cumulative_growth_data_2015.csv};
\addplot[color=orange, mark=none, thick] table[x=month, y=hml_ru, col sep=comma]{data/ru_factors_cumulative_growth_data_2015.csv};
\addplot[color=green, mark=none, thick] table[x=month, y=liq_ru, col sep=comma]{data/ru_factors_cumulative_growth_data_2015.csv};
\addplot[color=violet, mark=none, thick] table[x=month, y=dy_ru, col sep=comma]{data/ru_factors_cumulative_growth_data_2015.csv};
\addplot[color=blue, mark=none, thick] table[x=month, y=soe_ru, col sep=comma]{data/ru_factors_cumulative_growth_data_2015.csv};
\end{axis}
\end{tikzpicture}

{\scriptsize Данные: ИПЭИ РАНХиГС.}
\end{frame}



\begin{frame}{Факторы риска на рынке России - 4}
\centering
\begin{tabular}{l|r|r|r|r|c}
Фактор & Сред. & Ст.\,откл. & $t$-тест & $p$-знач. & 99\%\,дов.\,инт. \\
\hline
MKT-RF &  1.50\% & 6.2\% &  3.35 & <0.1\% & [ 0.3\%, 2.7\%] \\
SMB    &  1.29\% & 4.9\% &  3.62 & <0.1\% & [ 0.4\%, 2.2\%] \\ 
SOE    &  0.64\% & 3.7\% &  2.25 &  2.6\% & [-0.1\%, 1.4\%] \\
HML    &  0.44\% & 6.8\% &  0.90 & 36.9\% & [-0.8\%, 1.7\%] \\
MOM    &  0.38\% & 6.4\% &  0.82 & 41.2\% & [-0.8\%, 1.6\%] \\  
PE     &  0.23\% & 4.2\% &  0.75 & 45.3\% & [-0.6\%, 1.0\%] \\
DY     &  0.10\% & 5.6\% &  0.25 & 80.0\% & [-0.9\%, 1.2\%] \\
LIQ    & -0.07\% & 4.4\% & -0.22 & 82.3\% & [-0.9\%, 0.8\%] \\ \hline
\end{tabular}

\centering
{\scriptsize Месячные доходности 2005--2020. SOE: 2007--2020. Данные: ИПЭИ РАНХиГС.}

\justify
Статистически значимые факторы --- весь рынок (MKT-RF) и размер (SMB). С натяжкой ---  частные компании против государственных (SOE). Стоимость (HML), инерция (MOM), дивидендная доходность (DP), неликвидность (LIQ), мультипликатор P/E (PE), похоже, не дают статистически значимой дополнительной доходности.
\end{frame}



\begin{frame}{Факторный анализ доходностей}
\justify
Вернёмся к вопросу положительной <<альфы>> лектора. Что будет, если мы добавим в регрессию дополнительные факторы? Сравним две регрессии:
\begin{align*}
R_{fund,t} - R_{free,t} &= \alpha + \beta_{mkt}(R_{mkt,t} - R_{free,t}) + \epsilon_t
\\
R_{fund,t} - R_{free,t} &= \alpha + \beta_{mkt}(R_{mkt,t} - R_{free,t}) + \beta_{soe}R_{soe,t} + \epsilon_t
\end{align*}

\centering
\begin{tabular}{l|l|l}
Параметр            & Регрессия 1                   & Регрессия 2 \\ \hline
$\hat{\alpha}$      & 0.12\% {\scriptsize (0.32\%)} & 0.04\% {\scriptsize (0.30\%)} \\
$\hat{\beta}_{mkt}$ & 0.58 {\scriptsize (0.096)}    & 0.68 {\scriptsize (0.098)} \\
$\hat{\beta}_{soe}$ & ---                           & 0.28 {\scriptsize (0.10)} \\
$R^2$               & 0.42                          & 0.49 \\ \hline
\end{tabular}

{\scriptsize (50 месячных наблюдений 11.2015--12.2019)}

\justify
<<Альфа>> уменьшилась в 3 раза. Неужели лектор просто не любит госкомпании? Остальные факторы не улучшают регрессию.


\end{frame}



\begin{frame}{Arbitrage Pricing Theory - 1}
\justify
Предположим, что есть $k$ торгуемых факторов систематического риска с доходностями $R_i$. Тогда ожидаемая избыточная доходность любого актива должна быть равна
\begin{align*}
\mathbb{E}(R_{asset}) - R_{free} = \beta_1\mathbb{E}(R_1) + ... + \beta_k\mathbb{E}(R_k)
\end{align*}

Пусть это не так, и есть актив с положительной $\alpha$ в регрессии
\begin{align*}
R_{asset,t} - R_{free,t} = \alpha + \beta_1R_{1,t} + ... + \beta_kR_{k,t} + \epsilon_t, \epsilon_t \sim \mathcal{N}(0, \sigma_{\epsilon}^2)
\end{align*}

Инвестор, который стремится совершить арбитраж (заработать деньги из воздуха), купит актив и продаст все факторы в соответствии с весами $\beta_i$:
\begin{align*}
R_{asset} - \beta_1 R_{1} - ... - \beta_k R_k = R_{free} + \alpha + \epsilon
\end{align*}
\end{frame}



\begin{frame}{Arbitrage Pricing Theory - 2}
\justify
\begin{align*}
R_{asset} - \beta_1 R_{1} - ... - \beta_k R_k = R_{free} + \alpha + \epsilon
\end{align*}
Правая часть -- случайная величина со средним $R_{free} + \alpha$ и стандартным отклонением $\sigma_{\epsilon}$. Таким образом, отношение Шарпа равно $\alpha / \sigma_{\epsilon}$.

\vspace{\baselineskip}
Если регрессия, из которой мы вычленили $\alpha$ и $\beta_i$, хорошо объясняет доходность актива (имеет высокий $R^2$), то стандартное отклонение ошибки регрессии $\sigma_\epsilon$ будет достаточно малым.

\vspace{\baselineskip}
Чем лучше факторы объясняют актив, тем выше отношение Шарпа, которое заработает арбитражёр, и тем быстрее арбитражёры исправят ошибку рынка.
\end{frame}



\begin{frame}{Индексное инвестирование}
\justify
Активное управление портфелем, выбор акций --- игра с нулевой суммой. Чтобы кто-то имел положительную <<альфу>> относительно индекса, кто-то должен иметь отрицательную <<альфу>>. Немногие управляющие бьют индекс, а успех в прошлом чаще всего не означает успех в будущем.

\vspace{\baselineskip}
Возможно, средний инвестор получит лучший результат (более высокую доходность при том же уровне риска), если купит все акции из индекса и будет зарабатывать рыночную премию за риск. Однако, купить каждую из 500 акций индекса самостоятельно --- довольно проблематично.

\vspace{\baselineskip}
Индексный паевой фонд (index mutual fund) принимает деньги инвесторов и покупает на них акции, составляющие индекс, в той же пропорции, что и в индексе.
\end{frame}



\begin{frame}{Индексное инвестирование}
\justify
Задача индексного фонда --- дать возможность инвестору купить <<бету>> с минимальными накладными расходами. Меньше расходы на управление (не нужна армия аналитиков) --- лучше результат инвестора.

\vspace{\baselineskip}
Задача инвестора --- не искать управляющего, который обеспечит <<альфу>>, и не пытаться создать <<альфу>> самостоятельно, а выбрать <<бету>> (или <<беты>>) и смириться со средней рыночной премией за риск.


\vspace{\baselineskip}
Типичная комиссия за управление индексным фондом в США (не в России) --- единицы или десятки базисных пунктов (1 б.п. = 0.01\%) в год. Например, самый крупный фонд на индекс S\&P\,500 от Vanguard стоит 0.03\% (3 сотых процента) в год.
\end{frame}



\begin{frame}{Биржевые фонды}
\justify
Биржевые фонды (exchange traded funds) --- паевые фонды, паи (акции) которых торгуются на бирже. Задача ETF такая же, как у обычного индексного фонда --- дать инвестору возможность купить индекс с минимальными расходами.

\justify
Цена на пай ETF определяется спросом и предложением на рынке. Гипотетически, на идеальном рынке цена (market price) пая ETF всегда в точности равна цене акций, которыми владеет ETF (net asset value, NAV).

\justify
Чтобы помочь невидимой руке рынка уравнять цену пая с ценой акций из индекса, фонд назначает авторизованных участников (authorized participants). Эти участники имеют право получать и гасить паи фонда в обмен на корзины акций из индекса.
\end{frame}



\begin{frame}{Роль авторизованных участников - 1}
\justify
Ситуация 1: пай фонда стоит дороже корзины акций. Авторизованный участник покупает акции с рынка, меняет их на пай фонда, продаёт пай на рынке.

\vspace{\baselineskip}
\centering
\begin{tikzpicture}
\draw (0, 1.2) node{Ромашка ETF};
\draw[rounded corners] (-1.3, 1.6) rectangle (1.3, -1);
\draw (0, 0.6) node[rectangle,rounded corners,draw,minimum width=2.2cm]{GOOG};
\draw (0, 0) node[rectangle,rounded corners,draw,minimum width=2.2cm]{DBK.DE};
\draw (0, -0.6) node[rectangle,rounded corners,draw,minimum width=2.2cm]{AAPL};

\draw (4.1, 0.3) node[rectangle,rounded corners,draw,minimum width=2.2cm,minimum height=2.6cm]{\begin{tabular}{c}Автори-\\зованный\\участник\end{tabular}};

\draw[->,>=triangle 90] (1.3, 1.2) -- (3, 1.2) node[pos=0.5,anchor=south]{пай};

\draw[->,>=triangle 90] (3, -0.6) -- (1.3, -0.6) node[pos=0.5,anchor=north]{\scriptsize \begin{tabular}{c}
GOOG \\ DBK.DE \\ AAPL
\end{tabular}};

\draw[->,>=triangle 90] (5.2, 1.2) -- (6.9, 1.2) node[pos=0.5,anchor=south]{пай};

\draw[->,>=triangle 90] (6.9, -0.6) -- (5.2, -0.6) node[pos=0.5,anchor=north]{\scriptsize \begin{tabular}{c}
GOOG \\ DBK.DE \\ AAPL
\end{tabular}};

\draw (8, 0.3) node[rectangle,rounded corners,draw,minimum width=2.2cm,minimum height=2.6cm]{Рынок};

\end{tikzpicture}

\justify
Цена корзины акций растёт, цена пая фонда снижается.

\end{frame}



\begin{frame}{Роль авторизованных участников - 2}
\justify
Ситуация 2: пай фонда стоит дешевле корзины акций. Авторизованный участник покупает пай фонда на рынке, меняет его на корзину акций, продаёт акции на рынке.

\vspace{\baselineskip}
\centering
\begin{tikzpicture}
\draw (0, 1.2) node{Ромашка ETF};
\draw[rounded corners] (-1.3, 1.6) rectangle (1.3, -1);
\draw (0, 0.6) node[rectangle,rounded corners,draw,minimum width=2.2cm]{GOOG};
\draw (0, 0) node[rectangle,rounded corners,draw,minimum width=2.2cm]{DBK.DE};
\draw (0, -0.6) node[rectangle,rounded corners,draw,minimum width=2.2cm]{AAPL};

\draw (4.1, 0.3) node[rectangle,rounded corners,draw,minimum width=2.2cm,minimum height=2.6cm]{\begin{tabular}{c}Автори-\\зованный\\участник\end{tabular}};

\draw[<-,>=triangle 90] (1.3, 1.2) -- (3, 1.2) node[pos=0.5,anchor=south]{пай};

\draw[<-,>=triangle 90] (3, -0.6) -- (1.3, -0.6) node[pos=0.5,anchor=north]{\scriptsize \begin{tabular}{c}
GOOG \\ DBK.DE \\ AAPL
\end{tabular}};

\draw[<-,>=triangle 90] (5.2, 1.2) -- (6.9, 1.2) node[pos=0.5,anchor=south]{пай};

\draw[<-,>=triangle 90] (6.9, -0.6) -- (5.2, -0.6) node[pos=0.5,anchor=north]{\scriptsize \begin{tabular}{c}
GOOG \\ DBK.DE \\ AAPL
\end{tabular}};

\draw (8, 0.3) node[rectangle,rounded corners,draw,minimum width=2.2cm,minimum height=2.6cm]{Рынок};

\end{tikzpicture}

\justify
Цена корзины акций снижается, цена пая фонда растёт.

\end{frame}



\begin{frame}{Специфические риски биржевых фондов}
\justify
Механизм авторизованных участников может дать сбой в кризис. Лектор имеет радость владеть ETF-ом облигаций, который просел относительно net asset value на 6\% в марте 2020 г. Если бы в этот момент лектору срочно понадобились деньги, было бы очень обидно.

\justify
Чтобы заработать что-то большее, чем минимальная комиссия за управление, многие фонды одалживают ценные бумаги желающим продать их в короткую (margin lending). Гипотетически, возможен сценарий, когда фонду не вернут бумаги и он понесёт убытки.

\justify
Некоторые фонды (особенно фонды золота и биржевых товаров) являются <<синтетическими>>. Внутри --- не корзина товаров, а дериватив, total return swap с крупным банком. У инвестора фонда возникает кредитный риск на банк.
\end{frame}



\begin{frame}{Роль активного управления}
\justify
Могут ли все-все-все инвесторы инвестировать только в индекс? Вряд ли. Кто же будет оценивать отдельные акции относительно друг друга, покупать хорошие компании и продавать плохие?

\justify
На рынке есть неэффективности (inefficiencies) и несовершенства (imperfections). У вас будет шанс заработать премию за риск, который не могут или не хотят взять другие, если на вас не распространяются некоторые регуляторные ограничения, если у вас другая чувствительность к риску, или есть конкурентное преимущество в технологиях.

\justify
Впрочем, затраты на поиск и эксплуатацию несовершенств рынка могут превысить ожидаемую премию за риск.

\justify
Подробнее в Pedersen <<Efficiently Inefficient>> (2015).
\end{frame}



\begin{frame}{Выводы и рекомендации}
\begin{itemize}
\justifying
\item Диверсифицируйтесь! По классам активов, по валютам, по брокерам и банкам, по странам.
\item Не вкладывайте всю сумму разом. Регулярные небольшие взносы уменьшают риск просидеть без доходности 15 лет.
\item Всегда оценивайте, какой риск вы (или ваш управляющий) берёте на себя, вознаграждается ли этот риск рынком, и почему вы готовы его держать, а продавец актива --- нет.
\item Если вы не торгуете 8/5, то задумайтесь об индексном инвестировании через дешёвый ПИФ или ETF.
\item Выберите такой уровень риска (например, соотношение акции/облигации), при котором вы, с вашей личной чувствительностью к риску, сможете крепко спать по ночам.
\end{itemize}
\end{frame}



\begin{frame}{Личный опыт}
\justify
Портфель <<Сам себе пенсионный фонд>>: Vanguard VT (FTSE Global All Cap), FinEx FXUS (MSCI USA), FXCN (MSCI China), FXDE (MSCI Germany). Автоматическое пополнение на 10\% от зарплаты и премий (когда они есть).

\justify
Портфель <<Когда нет индексного фонда>>: акции на Московской бирже. AFLT, AKRN, GCHE, GMKN, LKOH, MOEX, MTSS, TATNP, UPRO.

\justify
Портфель <<Облигации>>: инфляционные ОФЗ-ИН 52001.

\justify
Портфель <<Отток капитала>>: iShares IVV (S\&P\,500), iShares IGSB (1-3Y Corp Bonds), Xtrackers DX2X (STOXX 600), Xtrackers D5BG (EUR Corp Bonds).

\justify
Акции/облигации: 35/65. Рубли/доллары/евро: 33/33/33.
\end{frame}



\begin{frame}{Дальнейшее чтение}
\begin{itemize}
\justifying


\item Горяев, Чумаченко. <<Финансовая грамота>> (2012).
\item Pedersen. <<Efficiently Inefficient>> (2015).
\item Cochrane. <<Asset Pricing: Revised Edition>> (2005).
\item Welch. <<Investments>> (2009).
\item Bali, Engle, Murray. <<Empirical Asset Pricing>> (2016).
\end{itemize}
\end{frame}

\end{document}


