\documentclass{beamer}

\usepackage{cmap}				% To be able to copy-paste russian text from pdf
\usepackage[T2A]{fontenc}
\usepackage[utf8]{inputenc}
\usepackage[russian]{babel}
\usepackage{textpos}
\usepackage{ragged2e}
\usepackage{amssymb}
\usepackage{ulem}
\usepackage{tikz}
\usepackage{pgfplots}
\usepackage{color}
\usepackage{cancel}
\usepackage{multirow}
\usepackage{multicol}
\pgfplotsset{compat=1.17}
\usetikzlibrary{arrows,snakes,backgrounds,shapes}
\usepgfplotslibrary{groupplots,colorbrewer,dateplot,statistics}
\usepackage{animate}

\usepackage{amsfonts}
\usepackage{amsmath}
\usepackage{amssymb}
\usepackage{graphicx}
\usepackage{setspace}
\usepackage{mathtools}

\usepackage{qrcode}


\title{Индексное инвестирование}
\author{Артём Бакулин}
\date{декабрь 2021 г.}

\def\usebankstyle{1}

\ifdefined\usebankstyle

	% Remove navigation bar from the bottom-right corner
	\beamertemplatenavigationsymbolsempty
	
	% Use black dash rather than blue triangle for each \item
	\setbeamertemplate{itemize item}{\color{black}\textemdash}
	
	\definecolor{bank-light-blue}{HTML}{00A3E0}
	\definecolor{bank-dark-blue}{RGB}{0, 24, 168}
	
	\setbeamercolor{frametitle}{fg=bank-light-blue}
	
	\setbeamertemplate{frametitle}{
	    \begin{beamercolorbox}[ht=0.9cm]{frametitle}
	    \strut \insertframetitle \strut
	    \end{beamercolorbox}
	}
	
	\addtobeamertemplate{frametitle}{}{%
	\begin{textblock*}{50mm}(\textwidth-0.68cm,-0.68cm)
	\includegraphics[height=0.68cm,width=0.68cm]{img/1280px_bank_logo.png}
	\end{textblock*}}
	
	\setbeamertemplate{footline}{
	    \begin{beamercolorbox}[leftskip=1cm, rightskip=1cm]{footline}
	    		\vspace{0.2cm}
			\begin{minipage}[t]{\textwidth-2cm}
				\hrule
				\vspace{1pt}			
				{\color{bank-dark-blue} Deutsche Bank}
				\hfill
				\insertauthor, \insertdate
				\hfill
				\insertpagenumber
				\newline
				{\color{bank-light-blue} Technology Center}
			\end{minipage}
	    \end{beamercolorbox}
	}

	\newcommand{\inserttitleframe}{
		% Background image is defined inside these curcly braces
		{
			\usebackgroundtemplate{\includegraphics[width=\paperwidth]{img/bank_background.jpg}}
			\begin{frame}[plain]
				\vspace{3.5cm}
				
				%\hspace{-0.4cm} {\footnotesize Математические модели в инвестиционных банках}
				
				\vspace{0.5cm}
				
				\hspace{-0.4cm} {\huge \inserttitle}
				
				\vspace{0.2cm}
				
				\hspace{-0.4cm} {\scriptsize \color{bank-light-blue} \#PositiveImpact}
				
				\vspace{0.2cm}
				
				\hspace{-0.4cm} {\footnotesize \insertauthor, \insertdate}
			\end{frame}
		}
	}
	
	\newcommand{\insertdisclaimerframe}{
		\begin{frame}{Disclaimer}
			\small
			\justify
			Данный материал не является предложением или предоставлением
			какой-либо услуги. Данный материал предназначен исключительно для
			информационных и иллюстративных целей и не предназначен для
			распространения в рекламных целях. Любой анализ третьих сторон не 		
			предполагает какого-либо одобрения или рекомендации. Мнения, 
			выраженные в	данном материале, являются актуальными на текущий момент,
			появляются только в этом материале и могут быть изменены без 
			предварительного уведомления. Эта информация предоставляется с 
			пониманием того, что в отношении материала, предоставленного здесь, вы
			будете принимать самостоятельное решение в отношении любых действий в
			связи с настоящим материалом, и это решение является основанным на 
			вашем	собственном суждении, и что вы способны понять и оценить 
			последствия этих действий. ООО <<ТехЦентр Дойче Банка>> не несет 
			никакой ответственности за любые убытки любого рода, относящихся к
			 этому материалу.
		\end{frame}
	}

\else
	\usetheme{Warsaw}
	\usecolortheme{beaver}

	\newcommand{\inserttitleframe}{
		\begin{frame}
		\titlepage
		\end{frame}
	}
	
	\newcommand{\insertdisclaimerframe}{
	}
	
	% remove navigation bar
	\setbeamertemplate{navigation symbols}{}

	% add page counter
	\setbeamertemplate{page number in head/foot}[totalframenumber] 

\fi


\newcommand{\qrcodeminipage}[1]{
	\begin{minipage}{0.3\textwidth}
		\qrcode[height=\textwidth]{#1}
	\end{minipage}
}

\newcommand{\en}[1]{\begin{otherlanguage}{english}#1\end{otherlanguage}}


\begin{document}



\inserttitleframe



\begin{frame}{Чего не будет в этой лекции}
\begin{itemize}
\justifying
\item Как выбрать самые перспективные акции.
\item Как предугадать крах рынка акций.
\item Как торговать внутри дня и зарабатывать 1\% в день.
\item Как научиться видеть <<фигуры>> на графиках котировок.
\item Как писать торговых роботов.
\item Как разбогатеть быстро и без риска.
\end{itemize}
\end{frame}


\begin{frame}{Содержание}
\begin{itemize}
\justifying
\item Рациональные инвесторы и премия за риск.
\item Диверсификация и портфельная оптимизация.
\item Capital Asset Pricing Model (CAPM).
\item Метрики для оценки инвестиций.
\item Факторные модели и Arbitrage Pricing Theory (APT).
\item Индексное инвестирование и биржевые фонды (ETFs).
\end{itemize}
\end{frame}



    \newcommand{\addGrowthPlot}[4]{
        \addplot[
            color = #2,
            line width = 1pt, 
            mark = #3,
            mark repeat = 120,
            mark phase = 36,
            mark options = {scale=1.5},
            style = #4
        ]
        table[
            x = date,
            y = #1,
            col sep = comma
        ]
        {data/fama_french_cumulative_growth_data.csv};
    }

\begin{frame}{Акции против облигаций}
\centering

\begin{tikzpicture}
\begin{axis}[
    width=\textwidth,
    height=\textheight - 1cm,
    date coordinates in=x,
    date ZERO=1926-06-30,
    xtick={1930-01-01,1940-01-01,1950-01-01,1960-01-01,1970-01-01,1980-01-01,1990-01-01,2000-01-01,2010-01-01,2020-01-01},
    minor xtick={1930-01-01,1950-01-01,1970-01-01,1990-01-01,2010-01-01},
    xticklabel=\year,
    grid=both,
    xmin=1926-12-31,
    xmax=2025-01-01,
    ymode=log,
    ymax=20000,
    log ticks with fixed point,
    ylabel={\small Рост \$1 начальных инвестиций},
    ylabel shift = -10pt,
    legend entries={
        Рынок акций США,
        Безрисковые облигации,
        Инфляция (CPI-U)
    },
    legend pos=north west,
    legend style={font=\scriptsize},
    legend cell align={left}
]
    
    \addGrowthPlot{mkt}{Set1-A}{none}{solid}
    \addGrowthPlot{rf}{Set1-B}{square}{solid}
    \addGrowthPlot{cpi}{Set1-C}{none}{dashed}
\end{axis}
\end{tikzpicture}

\scriptsize{Данные: Kenneth French Data Library.}
\end{frame}



\begin{frame}{Риск и доходность - 1}
\justify
Вы должны вложить крупную сумму (например, один годовой доход) в одну из двух ценных бумаг. Какую из них вы выберете?

\justify
\centering
\begin{minipage}[t]{0.6\textwidth}
	\begin{tabular}{l|r|r}
	                              & Бумага A & Бумага B  \\ \hline
	Цена сегодня                  & \$100    & \$100     \\ \hline
	\multirow{2}{*}{Вер-ть 50\%}  & \$105    & \$130     \\ 
	                              & ($+5\%$) & ($+30\%$) \\ \hline
	\multirow{2}{*}{Вер-ть 50\%}  & \$105    & \$80      \\
	                              & ($+5\%$) & ($-20\%$)  \\ \hline
	\multirow{2}{*}{Среднее}      & \$105    & \$105     \\
	                              & ($+5\%$) & ($+5\%$)
	\end{tabular}
\end{minipage}
\hfill
\qrcodeminipage{https://www.menti.com/ssoimunbud}
\end{frame}



\begin{frame}{Риск и доходность - 2}
\justify
Вы должны вложить крупную сумму (например, один годовой доход) в одну из двух ценных бумаг. Какую из них вы выберете?

\justify
\centering
\begin{minipage}[t]{0.6\textwidth}
	\begin{tabular}{l|r|r}
	                              & Бумага A & Бумага B     \\ \hline
	Цена сегодня                  & \$100    & \alert{\$85} \\ \hline
	\multirow{2}{*}{Вер-ть 50\%}  & \$105    & \$130        \\ 
	                              & ($+5\%$) & ($+52.9\%$)  \\ \hline
	\multirow{2}{*}{Вер-ть 50\%}  & \$105    & \$80         \\
	                              & ($+5\%$) & ($-5.9\%$)  \\ \hline
	\multirow{2}{*}{Среднее}      & \$105    & \$105        \\
	                              & ($+5\%$) & (\alert{$+23.5\%$})
	\end{tabular}
\end{minipage}
\hfill
\qrcodeminipage{https://www.menti.com/ssoimunbud}
\end{frame}



\begin{frame}{Избегание риска}
\justify
Обычно инвесторы любят доходность и не любят риск. Из двух инвестиций с 
одинаковым риском лучше та, которая даёт большую доходность. Из двух инвестиций 
с одинаковой доходностью лучше та, которая несёт меньший риск.

\justify
Проще всего было бы объяснить избегание риска (\en{risk aversion}) 
нерациональостью слабых духом \en{homo sapiens}. Можно ли построить модель, в
которой рациональный робот (\en{homo economicus}) тоже будет не любить риск?
\end{frame}



\begin{frame}{Счастье и пончики}
\justify
Счастье --- в пончиках (точнее, в логарифме их количества). Каждый следующий 
пончик менее ценен, чем предыдущий. Не столь приятно получить четвёртый пончик, 
как обидно потерять третий и остаться с двумя.

\justify
\centering
\begin{tikzpicture}
	\begin{axis} [
		width = \textwidth,
		height = \textheight - 3.2cm,
		xmin = 0, xmax = 7,
		ymin = -1, ymax = 2,
		grid = major,
		xlabel = {\small Количество шоколадных пончиков с шоколадом},
		ylabel = {\small Уровень счастья},
		ylabel near ticks
	]
		\addplot[
			color = Set1-B,
			thick,
			samples at = {0.1,0.11,...,7}
		]
		{ln(x)};
		
		\draw[thick] 
			(3, 1.099) -- 
			(4, 1.099) -- 
			node[
				pos = 0.5,
				anchor = west
			]
			{0.28}
			(4, 1.386);

		\draw[thick] 
			(2, 0.693) -- 
			(3, 0.693) -- 
			node[
				pos = 0.5,
				anchor = west
			]
			{0.41}
			(3, 1.099);

		\node[
			fill = Set1-B,
			circle,
			inner sep = 2pt
		]
		at (3, 1.099) {};
	\end{axis}
\end{tikzpicture}
\end{frame}



\begin{frame}{Напоминание: мат. ожидание}
\justify
Математическое ожидание (\en{expected value}) случайной величины $\mathcal{X}: 
\Omega \to \mathbb{R}$, заданной на вероятностном пространстве $(\Omega, 
\mathcal{F}, \mathbb{P})$ --- интеграл Лебега
\begin{align*}
\mathbb{E}\mathcal{X} = \int\limits_\Omega\mathcal{X}(\omega)d\mathbb{P}(\omega)
\end{align*}

\justify
Если у случайной величины $\mathcal{X}$ есть плотность $f(x)$, то
\begin{align*}
\mathbb{E}\mathcal{X} = \int\limits_{-\infty}^{+\infty}x\cdot f(x)dx
\end{align*}

\justify
Если дискретная случайная величина $\mathcal{X}$ принимает значения $x_1,...,x_n$ с вероятностями $p_1,...,p_n$, то
\begin{align*}
\mathbb{E}\mathcal{X} = \sum\limits_{i=1}^{n} p_i x_i
\end{align*}
\end{frame}



\begin{frame}{Напоминание: мат. ожидание - 2}
\justify
Математическое ожидание линейно (даже если случайные величины не являются 
независимыми):
\begin{align*}
\mathbb{E}(a\mathcal{X} + b\mathcal{Y}) =
a\mathbb{E}\mathcal{X} + b\mathbb{E}\mathcal{Y}
\end{align*}
для любых случайных величин $\mathcal{X}$ и $\mathcal{Y}$, для которых 
существует мат. ожидание.

\justify
Пример: пусть $\mathcal{X}$ --- результат броска игрального кубика, на котором 
грани от 1 до 6 выпадают с вероятностями $1/6$.
\begin{align*}
\mathbb{E}\mathcal{X} =
\frac{1}{6} \cdot 1 + \frac{1}{6} \cdot 2 + ... + \frac{1}{6} \cdot 6 = 3.5
\end{align*}

\justify
Если у нас два кубика (как в <<Монополии>>), то мат. ожидание суммы очков равно
\begin{align*}
\mathbb{E}(\mathcal{X} + \mathcal{X})
= 2\mathbb{E}\mathcal{X}
= 2 \cdot 3.5 = 7
\end{align*}
\end{frame}



\begin{frame}{Функция полезности}
\justify
Предположим, что рациональный инвестор оптимизирует функцию полезности (\en{
utility function})\ $u(x)$, где $x$ --- уровень богатства. Среди всех 
альтернатив он выбирает ту, которая обещает более высокую среднюю ожидаемую
полезность.
\begin{align*}
\mathbb{E}\Big(u(x)\Big) \to \max
\end{align*}

\justify
Предположим, что функция полезности рационального инвестора --- десятичный 
логарифм количества долларов на счету. Чем больше миллиардов, тем лучше, но 
каждый следующий миллиард приносит меньше счастья, чем предыдущий.
\end{frame}



\begin{frame}{Сложный выбор}
\justify
Инвестор имеет на счету \$100\,000, которые дают полезность
$\lg 100\,000 = 5.0$. Он должен вложить их в один из двух инструментов: в 
безрисковые облигации или рискованные акции.

\justify
\centering
\begin{tabular}{c|c|c|c|c|c}
\multirow{2}{*}{Сценарий} & \multirow{2}{*}{Вер-ть} & \multicolumn{2}{c|}{Облигации} & \multicolumn{2}{c}{Акции} \\
\cline{3-6}
        &      & Капитал    & Полез-ть & Капитал      & Полез-ть \\ \hline
Хороший & 50\% & \$105\,000 & 5.021    & \$125\,000 & 5.097 \\
Плохой  & 50\% & \$105\,000 & 5.021    & \$85\,000  & 4.929 \\ \hline
Среднее &      & \$105\,000 & \alert{5.021} & \$105\,000 & \alert{5.013}
\end{tabular}

\justify
Рациональный инвестор выберет менее рискованную альтернативу! Это верно не 
только для логарифма, но и для любой выпуклой вверх функции полезности (когда 
каждый следующий доллар радует меньше предыдущего).
\end{frame}



\begin{frame}{Премия за риск}
\justify
Что нужно сделать, чтобы рациональный инвестор выбрал рискованные акции или хотя бы воспринимал альтернативы безразлично? Например, пообещать более высокую доходность в хорошем сценарии и более высокое матожидание!

\justify
\centering
\begin{tabular}{c|c|c|c|c|c}
\multirow{2}{*}{Сценарий} & \multirow{2}{*}{Вер-ть} & \multicolumn{2}{c|}{Облигации} & \multicolumn{2}{c}{Акции} \\
\cline{3-6}
        &      & Капитал    & Полез-ть & Капитал      & Полез-ть \\ \hline
Хороший & 50\% & \$105\,000 & 5.021    & \alert{\$129\,706} & \alert{5.113} \\
Плохой  & 50\% & \$105\,000 & 5.021    & \$85\,000  & 4.929 \\ \hline
Среднее &      & \$105\,000 & 5.021    & \alert{\$107\,353} & \alert{5.021}
\end{tabular}

\justify
Дополнительные \$2\,353 в мат. ожидании --- премия за риск (risk premium), 
которая компенсирует инвестору дискомфорт от возможных потерь в плохом 
сценарии.
\end{frame}



\begin{frame}{Полная доходность}
\justify
Пусть $P_t$ --- цена актива (акции, облигации, квартиры) в момент времени $t$, 
а $D_t$ --- денежная выплата (дивиденд , купон, аренда квартиры) в этот же 
момент времени. Нас интересует полная доходность --- рост цены (capital gain) 
плюс дивиденды. Считаем, что налогов нет.
\begin{align*}
R_{t+1} = \dfrac{P_{t+1} + D_{t+1}}{P_t} - 1
\end{align*}

\justify
Следствие: ниже цена сегодня --- выше будущая доходность.

\justify
Хорошая компания (например, с высокими будущими дивидендами) --- не обязательно 
хорошая инвестиция. Если все согласны, что компания хорошая, то сегодняшняя 
цена акции будет выше, а будущая доходность --- ниже.
\end{frame}



\begin{frame}{Ожидаемая доходность}
\justify
Будущая цена, будущие дивиденды и будущая доходность --- случайные величины. 
Перейдём к математическим ожиданиям.
\begin{align*}
\mathcal{R}_{t+1} 
&= \frac{\mathcal{P}_{t+1} + \mathcal{D}_{t+1}}{P_t} - 1   
\quad \Rightarrow \\
\mathbb{E}\mathcal{R}_{t+1}
&= \frac{\mathbb{E}\mathcal{P}_{t+1} + \mathbb{E}\mathcal{D}_{t+1}}{P_t} - 1 
\quad \Rightarrow \\
P_t &=
\frac{\mathbb{E}\mathcal{P}_{t+1} + \mathbb{E}\mathcal{D}_{t+1}}{1 + \mathbb{E}\mathcal{R}_{t+1}}
\end{align*}

\justify
Текущая цена акции --- ожидаемые будущие выплаты, продисконтированные по 
ожидаемой (требуемой участниками рынка) доходности.

\justify
Если участники рынка требуют премию за риск, то она будет отражена в более 
низкой сегодняшней цене (более высокой требуемой доходности). 
\end{frame}



\begin{frame}{Напоминание: дисперсия}
\justify
Дисперсия (\en{variance}) случайной величины $\mathcal{X}$ --- мера разброса 
относительно мат. ожидания:
\begin{align*}
\operatorname{Var}(\mathcal{X})
= \mathbb{E}\Big[(\mathcal{X} - \mathbb{E}\mathcal{X})^2\Big]
\end{align*}

\justify
Стандартное отклонение (\en{standard deviation}) или волатильность 
(\en{volatility})--- квадратный корень из дисперсии:
\begin{align*}
\sigma_{\mathcal{X}} = \sqrt{\operatorname{Var}(\mathcal{X})}
\end{align*}

\justify
Если $\mathcal{X}$ --- результат броска кубика, то
\begin{align*}
\operatorname{Var}(\mathcal{X}) 
= \frac{1}{6}(1 - 3.5)^2 + \frac{1}{6}(2 - 3.5)^2 + ... + \frac{1}{6}(6-3.5)^2 
\approx 2.92
\end{align*}
\end{frame}



\begin{frame}{Напоминание: ковариация и корреляция}
\justify
Ковариация (\en{covariance}) показывает линейную связь между случайными величинами: насколько чаще более высокие значения одной выпадают одновременно 
с более высокими значениями другой.
\begin{align*}
\operatorname{Cov}(\mathcal{X}_1, \mathcal{X}_2) =
\mathbb{E}\Big[(\mathcal{X}_1 - \mathbb{E}\mathcal{X}_1)
(\mathcal{X}_2 - \mathbb{E}\mathcal{X}_2)\Big]
\end{align*}

\justify
Коэффициент корреляции Пирсона $\rho$ (\en{Pearson correlation}):
\begin{align*}
\rho =
\frac{
	\operatorname{Cov}(\mathcal{X}_1, \mathcal{X}_2)
}{
	\sqrt{\operatorname{Var}(\mathcal{X}_1)\operatorname{Var}(\mathcal{X}_2)}
}
\end{align*}

\justify
Если обозначить $\sigma_1$ и $\sigma_2$ стандартные 
отклонения, то
\begin{align*}
\operatorname{Cov}(\mathcal{X}_1, \mathcal{X}_2) 
= \rho\sigma_1\sigma_2
\end{align*}

\justify
Ковариация линейна:
\begin{align*}
\operatorname{Cov}(\mathcal{X}, a\mathcal{Y} + b\mathcal{Z}) =
a\operatorname{Cov}(\mathcal{X}, \mathcal{Y}) +
b\operatorname{Cov}(\mathcal{X}, \mathcal{Z})
\end{align*}
\end{frame}



\begin{frame}{Напоминание: дисперсия суммы}
\justify
Дисперсия линейной комбинации:
\begin{align*}
\operatorname{Var}(a\mathcal{X}_1 + b\mathcal{X}_2)
&= a^2\operatorname{Var}(\mathcal{X}_1)
+ b^2\operatorname{Var}(\mathcal{X}_2) 
+ 2ab\operatorname{Cov}(\mathcal{X}_1, \mathcal{X}_2) = \\
&= a^2\sigma_1^2 + b^2\sigma_2^2 + 2ab\rho\sigma_1\sigma_2
\end{align*}

\justify
Пример: $\mathbb{E}\mathcal{X}_1 = -1$, $\mathbb{E}\mathcal{X}_2 = 1$,
$\sigma_1 = \sigma_2 = 1$, $\rho = 0.5$, $a=b=0.5$.
\begin{minipage}[t]{0.5\textwidth}
	\begin{tikzpicture}
		\begin{axis}[
			width = \textwidth,
			height = \textwidth,
			xmin = -4, xmax = 2,
			ymin = -2, ymax = 4,
			grid = major
		]
		
		\addplot[
			only marks,
			color = Set1-B
		]
		table[
			x = x1,
			y = x2,
			col sep = comma
		]
		{data/sum_of_normal_scatter.csv};
		
		\draw[thick] (0, -5) -- (0, 5);
		\draw[thick] (-5, 0) -- (5, 0);
		\end{axis}
	\end{tikzpicture}
\end{minipage}
\hfill
\begin{minipage}[t]{0.5\textwidth}
	\begin{tikzpicture}
		\begin{axis}[
			width = \textwidth,
			height = \textwidth,
			grid = major,
			xmin = -4, xmax = 4,
			ymin = 0, ymax = 0.5
		]
		
		\addplot[
			dashed,
			thick,
			color = Set1-B
		]
		table[
			x = x,
			y = x1,
			col sep = comma
		]
		{data/sum_of_normal_density.csv};
		
		
		\addplot[
			dashed,
			thick,
			color = Set1-B
		]
		table[
			x = x,
			y = x2,
			col sep = comma
		]
		{data/sum_of_normal_density.csv};
		
		\addplot[
			thick,
			color = Set1-A
		]
		table[
			x = x,
			y = avg,
			col sep = comma
		]
		{data/sum_of_normal_density.csv};
		
		\end{axis}
	\end{tikzpicture}
\end{minipage}
\end{frame}



\begin{frame}{Оптимизация среднего и дисперсии}
\justify
Предположим, что инвесторы измеряют риск дисперсией (или стандартным 
отклонением) доходности. Чем выше матожидание доходности, тем лучше. Чем выше 
дисперсия (стандартное отклонение), тем хуже.

\justify
(*) Например, инвестор имеет экспоненциальную функцию полезности
$u(x) = 1 -e^{-\lambda x}$, где $x$ --- богатство, $\lambda$ --- степень 
нелюбви к риску. Путь будущее богатство --- нормально распределённая случайная 
величина $\mathcal{X} \sim \mathcal{N}(\mu, \sigma^2)$. Тогда инвестор будет 
максимизировать выражение
\begin{align*}
\mathbb{E}\mathcal{X} - \frac{\lambda}{2}\operatorname{Var}(\mathcal{X})
\to \max
\end{align*}
\end{frame}



\begin{frame}{О пользе корреляций - 1}
\justify
Две акции $X$ и $Y$ имеют одинаковые ожидаемые доходности и стандартные отклонения: $\mu = 5\%$ и $\sigma = 10\%$. Корреляция доходностей $\rho=0.4$. Инвестор может вложить долю $w$ своего богатства в $X$ и долю $(1 - w)$ в $Y$.

\justify
Ожидаемая доходность не зависит от $w$:
\begin{align*}
\mathbb{E}(w\mathcal{X} + (1 - w)\mathcal{Y}) 
&= w\mathbb{E}\mathcal{X} + (1-w)\mathbb{E}\mathcal{Y} = \\
&= w\mu + (1 - w)\mu = \mu = 5\%
\end{align*}

\justify
Может ли выбор $w$ уменьшить стандартное отклонение, т.е. риск?
\end{frame}



\begin{frame}{О пользе корреляций - 2}
\center
\begin{tikzpicture}
\begin{axis}[
  width = \textwidth * 0.667,
  xlabel={\small $w$ --- доля инвестиций в акцию $X$},
  ylabel={\small Ст. откл. портфеля, \%},
  xlabel near ticks,
  ylabel near ticks,
  xmin=0, xmax=1,
  ymin=0, ymax=10
]
   \addplot[
        color = Set1-B,
		  line width = 1pt,
		  samples at = {0,0.05,...,1}
	]
	{10 * sqrt(x^2 + (1-x)^2 + 2*0.4*x*(1-x))};

	\draw[
		color=black,
		dashed
	]
	(axis cs: 0.5, 0) -- (axis cs: 0.5, 8.366) -- (axis cs: 0, 8.366);

	\node[
		anchor=north west
	]
	at (axis cs: 0.5, 8.366)
	{(0.5, 8.37\%)};

    \node[
        color=Set1-B,
        circle,
        fill, 
        inner sep=2pt
    ]
    at (axis cs: 0.5, 8.366) {};
\end{axis}
\end{tikzpicture}

\justify
Вывод: диверсификация уменьшает риск при той же доходности, если корреляция активов отлична от 1.
\end{frame}



\begin{frame}{Портфель из двух активов - 1}
\justify
В нашем распоряжении два актива: индекс акций и индекс облигаций. Предположим,
что мы точно знаем будущие распределения доходностей активов и их корреляцию.
Какие портфели мы могли бы из них составить?

\justify
\centering
\begin{tabular}{l|r|r|r|r}
 & \multicolumn{2}{c|}{Доходность} & \multicolumn{2}{c}{Корреляция} \\ \cline{2-5}
Актив         & Сред.  & Ст.~откл. & Акц.  & Обл. \\ \hline
Акции         & 10.9\% & 15.2\%    & 1.00  & 0.00 \\
Облигации     & 5.2\%  & 3.6\%     & 0.00  & 1.00
\end{tabular}

\justify{\scriptsize 
Средние годовые доходности, стандартные отклонения и корреляции по данным сайта Portfolio Visualizer (1994--2020).
}

\justify
(*) Прошлые доходности не предсказывают будущие доходности.
\end{frame}



\newcommand{\drawAssetNode}[3]{
    \node[
        circle,
        fill,
        inner sep=2pt
    ] at (axis cs: #1, #2) {};
    \node[
        anchor=north
    ]
    at (axis cs: #1, #2)
    {\scriptsize #3};
}

\newcommand{\drawPortfolioNode}[6]{
    \node[
        anchor=#6,
        inner sep=1pt
    ] at (axis cs: #1, #2) {
    	  \setlength\tabcolsep{2pt}
	     \scriptsize
	     \begin{tabular}{|l|r|}
		  \hline
		  \multicolumn{2}{|c|}{#3} \\ \hline
		  Акц. & #4\% \\
		  Обл. & #5\% \\
		  \hline
		  \end{tabular}
    };
	 
	 \node[
	     circle,
	     fill,
	     inner sep=2pt,
	     color=Set1-B
    ] at (axis cs:#1, #2) {};
}



\begin{frame}{Портфель из двух активов - 2}
\centering
\begin{tikzpicture}
\begin{axis}[
    width=\textwidth,
    height=\textheight - 1cm,
    xlabel={\small Стандартное отклонение (волатильность), \%},
    ylabel={\small Ожидаемая доходность, \%},
    xlabel near ticks,
    ylabel near ticks,
    xmin=0, xmax=16,
    ymin=0, ymax=14
]

\addplot[
    line width=1pt,
    color=Set1-B
]
table[
    x=std_dev,
    y=target_return,
    col sep=comma
]
{data/efficient_frontier_plot_data_two_assets.csv};

\drawAssetNode{15.2}{10.9}{Акции}

\drawAssetNode{3.6}{5.2}{Облигации}

\drawPortfolioNode{5.4}{7.0}{Портфель 1}{31.6}{68.4}{south east}

\drawPortfolioNode{12.8}{10}{Портфель 2}{84.2}{15.8}{south east}

\end{axis}
\end{tikzpicture}
\end{frame}



\begin{frame}{Портфель из нескольких активов}
\centering
\begin{tabular}{l|r|r|r|r|r|r}
 & \multicolumn{2}{c|}{Доходность} & \multicolumn{4}{c}{Корреляция} \\ \cline{2-7}
Актив         & Сред.  & Ст.~откл. & Акц. & Обл. & Нед. & Зол. \\ \hline
Акции         & 10.9\% & 15.2\%    & 1.00  & 0.00   & 0.59    & 0.04 \\
Облигации     & 5.2\%  & 3.6\%     & 0.00  & 1.00   & 0.19    & 0.28 \\
Недвижимость  & 10.8\% & 19.2\%    & 0.59  & 0.19   & 1.00    & 0.13 \\
Золото        & 7.0\%  & 15.6\%    & 0.04  & 0.28   & 0.13    & 1.00
\end{tabular}

\justify
{\scriptsize
Средние годовые доходности, стандартные отклонения и корреляции по данным сайта 
Portfolio Visualizer (1994--2020).
}

\justify
Какой портфель минимизирует риск при заданной ожидаемой доходности или 
максимизирует ожидаемую доходность при заданном уровне риска?
\end{frame}



\renewcommand{\drawAssetNode}[3]{
    \node[
        circle,
        fill,
        inner sep=2pt
    ] at (axis cs: #1, #2) {};
    \node[
        anchor=north
    ]
    at (axis cs: #1, #2)
    {\scriptsize #3};
}

\renewcommand{\drawPortfolioNode}[8]{
    \node[
        anchor=#8,
        inner sep=1pt
    ] at (axis cs: #1, #2) {
    	  \setlength\tabcolsep{2pt}
	     \scriptsize
	     \begin{tabular}{|l|r|}
		  \hline
		  \multicolumn{2}{|c|}{#3} \\ \hline
		  Акц. & #4\% \\
		  Обл. & #5\% \\
		  Нед. & #6\%  \\
		  Зол. & #7\% \\
		  \hline
		  \end{tabular}
    };
	 
	 \node[
	     circle,
	     fill,
	     inner sep=2pt,
	     color=Set1-B
    ] at (axis cs:#1, #2) {};
}



\begin{frame}{Граница эффективности - 1}

\centering
\begin{tikzpicture}
\begin{axis}[
    width=\textwidth,
    height=\textheight - 1cm,
    xlabel={\small Стандартное отклонение (волатильность), \%},
    ylabel={\small Ожидаемая доходность, \%},
    xlabel near ticks,
    ylabel near ticks,
    xmin=0, xmax=21,
    ymin=0, ymax=14
]

\addplot[
    line width=1pt,
    color=Set1-B
]
table[
    x=std_dev,
    y=target_return,
    col sep=comma
]
{data/efficient_frontier_plot_data.csv};

\drawAssetNode{15.2}{10.9}{Акции}

\drawAssetNode{3.6}{5.2}{Облигации}

\drawAssetNode{19.2}{10.8}{Недвиж.}

\drawAssetNode{15.6}{7.0}{Золото}

\drawPortfolioNode{4.40}{6.5}{Портфель 1}{21.2}{74.6}{0.5}{3.7}{south east}

\drawPortfolioNode{12.0}{9.95}{Портфель 2}{63.2}{3.5}{14.4}{18.9}{south east}

\drawPortfolioNode{12.0}{6.5}{Портфель 3}{0.0}{30.0}{0.0}{70.0}{north east}

\draw[dashed] (axis cs: 12.0, 9.95) -- (axis cs: 12.0, 0);
\draw[dashed] (axis cs: 4.40, 6.5) -- (axis cs: 12.0, 6.5);
\draw[dashed] (axis cs: 4.40, 0) -- (axis cs: 4.40, 6.5);

\end{axis}
\end{tikzpicture}
\end{frame}



\begin{frame}{Граница эффективности - 2}
\centering
\begin{tikzpicture}
\begin{axis}[
  width=\textwidth,
  height=\textwidth/2,
  xlabel={\small Стандартное отклонение (волатильность), \%},
  ylabel={\small Доля актива, \%},
  xlabel near ticks,
  ylabel near ticks,
  xmin=3.5, xmax=15.2,
  ymin=0, ymax=100,
  stack plots=y
]

\node at (axis cs:9.3, 25) {\small Акции};
\node at (axis cs:9.3, 63) {\small Облигации};
\node at (axis cs:9.3, 82) {\small Недвиж.};
\node at (axis cs:9.3, 93) {\small Золото};

\addplot[fill=Pastel2-C] table[x=std_dev, y=stocks, col sep=comma] {data/efficient_frontier_plot_data.csv} \closedcycle;

\addplot[fill=Pastel2-D] table[x=std_dev, y=bonds, col sep=comma] {data/efficient_frontier_plot_data.csv} \closedcycle;

\addplot[fill=Pastel2-E] table[x=std_dev, y=reit, col sep=comma] {data/efficient_frontier_plot_data.csv} \closedcycle;

\addplot[fill=Pastel2-F] table[x=std_dev, y=gold, col sep=comma] {data/efficient_frontier_plot_data.csv} \closedcycle;
\end{axis}
\end{tikzpicture}
\justify
Граница эффективности (efficient frontier) --- линия, на которой лежат 
портфели, имеющие минимальный риск при заданном уровне доходности. Обычно 
портфели на границе содержат комбинацию базовых активов, а не какой-то один из 
них.
\end{frame}



\begin{frame}{Оптимизация портфеля по Марковицу - 1}
\justify
Пусть у нас есть $n$ активов с ожидаемыми доходностями $\mu_i$, стандартными 
отклонениями $\sigma_i$, корреляциями $\rho_{i,j}$. Инвестиционный портфель 
задан весами $x_i$. Какие веса активов в портфеле минимизируют риск при 
фиксированной ожидаемой доходности $r$?
\begin{align*}
x = \begin{bmatrix}
x_1 \\
x_2 \\
\vdots \\
x_n
\end{bmatrix}
\mu = \begin{bmatrix}
\mu_1 \\
\mu_2 \\
\vdots \\
\mu_n
\end{bmatrix}
S = \begin{bmatrix}
\sigma_1^2 & \rho_{1,2}\sigma_1\sigma_2 & \cdots & \rho_{1,n}\sigma_1\sigma_n \\
\rho_{2,1}\sigma_2\sigma_1 & \sigma_2^2 & \cdots & \rho_{2,n}\sigma_2\sigma_n \\
\vdots & \vdots & \ddots & \vdots \\
\rho_{n,1}\sigma_n\sigma_1 & \rho_{n,2}\sigma_n\sigma_2 & \cdots & \sigma_n^2
\end{bmatrix}
\end{align*}

\justify
Задача квадратичного программирования (\en{quadratic programming}):
\begin{align*}
\begin{cases}
x^TS x \to \min \\
\mu^Tx = r \\
\sum x_i = 1 \\
x \ge 0
\end{cases}
\end{align*}
\end{frame}



\begin{frame}{Оптимизация портфеля по Марковицу - 2}
\begin{align*}
\begin{cases}
\alert{x^TS x \to \min} \\
\mu^Tx = r \\
\sum x_i = 1 \\
x \ge 0
\end{cases}
\end{align*}

\justify
$x^TSx$ --- это компактная запись дисперсии портфеля. Например, если у нас 
всего два актива ($n=2$), то
\begin{align*}
x^TSx &=
\begin{bmatrix}x_1 & x_2\end{bmatrix}
\begin{bmatrix}\sigma_1^2 & \rho\sigma_1\sigma_2 \\
\rho\sigma_1\sigma_2 & \sigma_2^2\end{bmatrix}
\begin{bmatrix}
x_1 \\ x_2
\end{bmatrix} = \\
&= x_1^2\sigma_1^2 + x_2^2\sigma_2^2 + 2x_1x_2\rho\sigma_1\sigma_2
\end{align*}

\justify
Это формула дисперсии линейной комбинации. Если минимизировать это выражение, 
то мы найдём такие веса $x_1$ и $x_2$, при которых дисперсия портфеля 
минимальна. 
\end{frame}



\begin{frame}{Оптимизация портфеля по Марковицу - 3}
\begin{align*}
\begin{cases}
x^TS x \to \min \\
\alert{\mu^Tx = r} \\
\sum x_i = 1 \\
x \ge 0
\end{cases}
\end{align*}

\justify
Ограничение $\mu^Tx = r$ заставляет искать только такие доли активов $x_i$, при 
которых ожидаемая доходность портфеля равна $r$:
\begin{align*}
\mu^Tx =
\begin{bmatrix}
\mu_1 & \mu_2 & ... & \mu_n
\end{bmatrix}
\begin{bmatrix}
x_1 \\ x_2 \\ ... \\ x_n
\end{bmatrix} =
\mu_1 x_1 + \mu_2 x_2 + ... + \mu_n x_n
\end{align*}
\end{frame}



\begin{frame}{Оптимизация портфеля по Марковицу - 4}
\begin{align*}
\begin{cases}
x^TS x \to \min \\
\mu^Tx = r \\
\alert{\sum x_i = 1} \\
\alert{x \ge 0}
\end{cases}
\end{align*}

\justify
Ограничение $\sum x_i = 1$ заставляет искать только такие портфели (веса 
$x_i$), в которых суммарный вес всех активов равен 1 (мы должны инвестировать 
весь капитал, но не больше, чем у нас есть).

\justify
Ограничение $x \ge 0$ оставляет только те портфели (веса $x_i$), в которых все 
веса положительные (можно только покупать активы, но нельзя продавать в 
короткую).
\end{frame}



\begin{frame}{Оптимизация портфеля по Марковицу - 5}
\justify
Достоинства:
\begin{itemize}
\justifying
\item Понятная математическая модель.
\item Наглядная демонстрация выгод диверсификации.
\end{itemize}

\justify
Недостатки:
\begin{itemize}
\justifying
\item Результат неустойчив  и сильно зависит от ожидаемых доходностей.
\item Предполагается, что инвесторы измеряют риск стандартным отклонением, что не совсем правда.
\end{itemize}

\justify
Усложнения и дальнейшие идеи:
\begin{itemize}
\justifying
\item Модель Блэка-Литтермана вычисляет доходности из рыночных цен и позволяет 
их немного подправить.
\item Можно оптимизировать не дисперсию, а дисперсию <<вниз>> или
\en{Value at Risk (VaR)}.
\end{itemize}
\end{frame}



\begin{frame}{Оптимизация с безрисковым активом - 1}
\justify
Предположим, что на рынке есть безрисковый (risk-free) актив $F$, который имеет 
стандартное отклонение $0\%$. Пример: короткие \en{Treasury Bills}\ (ни единого 
разрыва за 230 лет).

\justify
Предположим также, что инвесторы могут занимать деньги под безрисковую 
процентную ставку.

\justifying
\centering
\begin{tabular}{l|r|r|r|r|r|r|r}
 & \multicolumn{2}{c|}{Доходность} & \multicolumn{5}{c}{Корреляция} \\ 
 \cline{2-8}
Актив    & Сред.  & Ст.~откл. & Акц. & Обл. & Нед. & Зол. & \alert{Без.} \\ \hline
Акции    & 10.9\% & 15.2\%    & 1.00 & 0.00 & 0.59 & 0.04 & \alert{0.00} \\
Облиг.   & 5.2\%  & 3.6\%     & 0.00 & 1.00 & 0.19 & 0.28 & \alert{0.00} \\
Недвиж.  & 10.8\% & 19.2\%    & 0.59 & 0.19 & 1.00 & 0.13 & \alert{0.00} \\
Золото   & 7.0\%  & 15.6\%    & 0.04 & 0.28 & 0.13 & 1.00 & \alert{0.00} \\
\alert{Безриск.} & \alert{4.0\%}  & \alert{0.0\%} & \alert{0.00} & \alert{0.00} & \alert{0.00} & \alert{0.00} & \alert{1.00}
\end{tabular}
\end{frame}



\renewcommand{\drawAssetNode}[4]{
    \node[
        circle,
        fill,
        inner sep=2pt
    ] at (axis cs: #1, #2) {};
    \node[
        anchor=#4
    ]
    at (axis cs: #1, #2)
    {\scriptsize #3};
}

\renewcommand{\drawPortfolioNode}[8]{\node[anchor=#8] at (axis cs: #1, #2) {
		\scriptsize \begin{tabular}{|l|r|}
		\hline
		\multicolumn{2}{|c|}{#3} \\ \hline
		Акц. & #4\% \\
		Обл. & #5\% \\
		Нед. & #6\%  \\
		Зол. & #7\% \\
		\hline
		\end{tabular}
	};
	\node[circle, fill, inner sep=2pt, color=Set1-B] at (axis cs:#1, #2) {};
}

\newcommand{\drawPortfolioNodeFive}[9]{\node[anchor=#9] at (axis cs: #1, #2) {
		\scriptsize \begin{tabular}{|l|r|}
		\hline
		\multicolumn{2}{|c|}{#3} \\ \hline
		Акц. & #4\% \\
		Обл. & #5\% \\
		Нед. & #6\% \\
		Зол. & #7\% \\
		Без. & #8\% \\
		\hline
		\end{tabular}
	};
	\node[circle, fill, inner sep=2pt, color=Set1-B] at (axis cs:#1, #2) {};
}


\begin{frame}{Оптимизация с безрисковым активом - 2}
\centering
\begin{tikzpicture}
\begin{axis}[
    width=\textwidth,
    height=\textheight-1cm,
    xlabel={\small Стандартное отклонение (волатильность), \%},
    ylabel={\small Ожидаемая доходность, \%},
    xlabel near ticks,
    ylabel near ticks,
    xmin=0, xmax=14,
    ymin=0, ymax=10
]

\draw[line width=1pt, color=Set1-B] (0, 4) -- (12, 6.5);

\drawAssetNode{0.0}{4.0}{Безриск. ($F$)}{north west}

\drawPortfolioNode{12.0}{6.5}{Портфель $\Pi$}{0.0}{30.0}{0.0}{70.0}{south east}

\drawPortfolioNodeFive{3}{4.625}{$0.75F + 0.25\Pi$}{0.0}{7.5}{0.0}{17.5}{75.0}{north west}

\end{axis}
\end{tikzpicture}
\end{frame}



\begin{frame}{Оптимизация с безрисковым активом - 3}
\centering
\begin{tikzpicture}
\begin{axis}[
    width=\textwidth,
    height=\textheight-1cm,
    xlabel={\small Стандартное отклонение (волатильность), \%},
    ylabel={\small Ожидаемая доходность, \%},
    xlabel near ticks,
    ylabel near ticks,
    xmin=0, xmax=14,
    ymin=0, ymax=10
]

\draw[line width=1pt, color=Set1-B] (0, 4) -- (20, 14);

\drawAssetNode{0.0}{4.0}{Безриск. ($F$)}{north west}

\drawPortfolioNode{3.60}{5.80}{Портфель $\Pi$}{10.5}{89.3}{0.0}{0.2}{north west}

\drawPortfolioNodeFive{7.2}{7.6}{$-F + 2\Pi$}{21.0}{178.6}{0.0}{0.4}{$-100.0$}{north west}

\addplot[line width=1pt, color=Set1-B, dashed] table[x=std_dev, y=target_return, col sep=comma] {data/efficient_frontier_plot_data.csv};

\end{axis}
\end{tikzpicture}
\end{frame}




\renewcommand{\drawAssetNode}[4]{
    \node[circle,fill,inner sep=2pt] at (axis cs: #1, #2) {};
    \node[anchor=#4] at (axis cs: #1, #2) {\small #3};
}

\renewcommand{\drawPortfolioNode}[8]{\node[anchor=#8] at (axis cs: #1, #2) {
		\scriptsize \begin{tabular}{|l|r|}
		\hline
		\multicolumn{2}{|c|}{#3} \\ \hline
		Акц. & #4\% \\
		Обл. & #5\% \\
		Нед. & #6\%  \\
		Зол. & #7\% \\
		\hline
		\end{tabular}
	};
	\node[circle, fill, inner sep=2pt, color=Set1-B] at (axis cs:#1, #2) {};
}

\newcommand{\drawPortfolioNodeTwo}[6]{
\node[anchor=#6] at (axis cs: #1, #2) {
		\scriptsize 
		\setlength\tabcolsep{2pt}
		\begin{tabular}{|l|r|}
		\hline
		\multicolumn{2}{|c|}{#3} \\ \hline
		Портф. T & #4\% \\
		Безриск. & #5\% \\
		\hline
		\end{tabular}
	};
	\node[circle, fill, inner sep=2pt, color=Set1-B] at (axis cs:#1, #2) {};
}


\begin{frame}{Оптимизация с безрисковым активом - 4}
\centering
\begin{tikzpicture}
\begin{axis}[
    width=\textwidth,
    height=\textheight-1cm,
    xlabel={\small Стандартное отклонение (волатильность), \%},
    ylabel={\small Ожидаемая доходность, \%},
    xlabel near ticks,
    ylabel near ticks,
    xmin=0, xmax=14,
    ymin=0, ymax=10
]

\addplot[line width=1pt, color=Set1-B, domain=0:21] {4 + x * (6.75 - 4) / 4.83};

\drawPortfolioNodeFive{1.45}{4.83}{$0.7F + 0.3T$}{7.3}{20.9}{0.4}{1.4}{$70.0$}{north west}

\drawPortfolioNodeFive{7.25}{8.12}{$-0.5F + 1.5T$}{36.3}{104.2}{2.3}{7.2}{$-50.0$}{north west}

\drawAssetNode{0.0}{4.0}{F}{north west}

\drawPortfolioNode{4.83}{6.75}{Портфель T}{24.2}{69.5}{1.5}{4.8}{south east}

\addplot[line width=1pt, color=Set1-B, dashed] table[x=std_dev, y=target_return, col sep=comma] {data/efficient_frontier_plot_data.csv};

\end{axis}
\end{tikzpicture}
\end{frame}



\begin{frame}{Касательный портфель}
\justify
Теорема о двух фондах: новая граница эффективности --- прямая линия. Любой 
портфель на этой прямой можно представить как линейную комбинацию безрискового 
актива $F$ и касательного (\en{tangent}) портфеля рискованных активов $T$.

\justify
Каждый инвестор, который хочет оказаться на границе эффективности, должен 
вложить какую-то часть в безрисковый актив (или наоборот, занять деньги под 
безрисковую ставку), а остальной капитал распределить между рискованны активами
пропорционально их весам в касательном портфеле $T$.

\justify
Как узнать веса активов в касательном портфеле $T$, не решая задачу 
квадратичного программирования?
\end{frame}



\begin{frame}{Рыночный портфель - 1}
\justify
Предположим, что все инвесторы на рынке решают одну и ту же задачу оптимизации.
Они имеют одинаковое мнение об ожидаемых доходностях, дисперсиях и корреляциях.
Они скармливают в алгоритм квадратичного программирования одни и те же входные 
значения и получают один и тот же ответ.

\justify
Предположим, что все инвесторы готовы вложить \$100 триллионов в рискованные 
активы. Каждый из них знает, что в касательном портфеле $T$ правильная доля 
золота $4.8\%$. Тогда все инвесторы будут готовы вложить в золото \$4.8 
триллиона.

\justify
Если суммарный спрос на золото \$4.8 триллиона, а всего в природе существует 
4.8 миллиардов унций золота, то невидимая рука рынка установит цену золота на 
уровне \$1000 за унцию так, чтобы ни одна унция не осталась без хозяина.
\end{frame}



\begin{frame}{Рыночный портфель - 2}
\justify
И каждый инвестор по отдельности, и все инвесторы сообща распределяют капитал 
по рискованным активам пропорционально весам в касательном портфеле $T$. Если 
Петя вложил в золото 4.8\% от \$100, а Вася --- 4.8\% от \$100\,000, то вместе 
они вложили в золото 4.8\% от своего общего капитала \$100\,100.

\justify
Следовательно, рыночная капитализация актива (суммарная стоимость всех единиц 
актива) должна быть пропорциональна его весу в касательном портфеле $T$.

\justify
\centering
\begin{tabular}{l|r|r}
Актив        & Капитализация & Вес в портфеле $T$ \\ \hline
Акции        &  \$24.2 трлн. &  24.2\%             \\
Облигации    &  \$69.5 трлн. &  69.5\%             \\
Недвижимость &   \$1.5 трлн. &   1.5\%             \\
Золото       &   \$4.8 трлн. &   4.8\%             \\ \hline
Итого        & \$100.0 трлн. & 100.0\% 
\end{tabular}
\end{frame}



\begin{frame}{Рыночный портфель - 3}
\centering
\begin{tabular}{l|r|r}
Актив        & Капитализация & Доля рынка \\ \hline
Акции        &  \$24.2 трлн. &  24.2\%    \\
Облигации    &  \$69.5 трлн. &  69.5\%    \\
Недвижимость &   \$1.5 трлн. &   1.5\%    \\
Золото       &   \$4.8 трлн. &   4.8\%    
\end{tabular}

\justify
Если вы --- маленький частный инвестор, который не может повлиять своими 
сделками на цены активов, то самый простой способ узнать веса в касательном 
портфеле $T$ --- посмотреть на рыночные капитализации и поверить, что остальные 
инвесторы не дураки и уже решили задачу оптимизации за вас.

\justify
Единственное решение, которое вам надо принять --- сколько денег вложить в 
безрисковый актив. После этого вам нужно купить на оставшийся капитал
все-все-все рискованные активы пропорционально их рыночной капитализации. Это 
единственный способ оказаться на границе эффективности, а не под ней.
\end{frame}



\begin{frame}{Отношение Шарпа - 1}
\justify
Отношение Шарпа (\en{Sharpe ratio}) --- отношение ожидаемой избыточной 
доходности портфеля $\mathcal{R}$ сверх безрисковой ставки $R_f$ к стандартному 
отклонению доходности $\sigma$.
\begin{align*}
S = \frac{\mathbb{E}\mathcal{R} - R_f}{\sigma}
\end{align*}

\justify
Отношение Шарпа --- <<цена>> избыточной доходности.

\justify
Если отношение Шарпа равно 0.5, каждый дополнительный 1\% волатильности 
<<покупает>> дополнительные 0.5\% ожидаемой доходности сверх безрисковой 
ставки. Чтобы заработать на 1\% больше, нужно будет терпеть волатильность 
портфеля на 2\% выше.
\end{frame}



\begin{frame}{Отношение Шарпа - 2}
\centering
\begin{tikzpicture}
\begin{axis}[
    width=\textwidth,
    height=\textheight-1cm,
    xlabel={\small Стандартное отклонение (волатильность), \%},
    ylabel={\small Ожидаемая доходность, \%},
    xlabel near ticks,
    ylabel near ticks,
    xmin=0, xmax=14,
    ymin=0, ymax=10
]

\addplot[line width=1pt, color=Set1-B, domain=0:21] {4 + x * (6.75 - 4) / 4.83};

%\drawPortfolioNodeFive{1.45}{4.83}{$0.7F + 0.3T$}{7.3}{20.9}{0.4}{1.4}{$70.0$}{north west}

%\drawPortfolioNodeFive{7.25}{8.12}{$-0.5F + 1.5T$}{36.3}{104.2}{2.3}{7.2}{$-50.0$}{north west}

%\drawAssetNode{0.0}{4.0}{F}{north west}

\drawPortfolioNode{4.83}{6.75}{Портфель T}{24.2}{69.5}{1.5}{4.8}{south east}

\draw[<->, >= triangle 45] (4.83, 0) -- (4.83, 4) node[pos=0.5, anchor=west] 
{\small $R_f = 4\%$};

\draw[<->, >= triangle 45] (4.83, 4) -- (4.83, 6.75) node[pos=0.5, anchor=west]
{\small $\mathbb{E}\mathcal{R}_m - R_f = 2.75\%$};

\draw[<->, >= triangle 45] (0, 4) -- (4.83, 4) node[pos=0.5, anchor=north]
{\small $\sigma_m=4.83\%$};

\draw[dashed] (4.83, 6.75) -- (14, 6.75);

\draw (1.5, 4) arc [
	start angle = 0,
	end angle = 29.65,
	x radius = 1.5,
	y radius = 1.5
];

\draw (6.33, 6.75) arc [
	start angle = 0,
	end angle = 29.65,
	x radius = 1.5,
	y radius = 1.5
] node [pos=0.8, anchor=west] {\small $\tg s = \dfrac{2.75}{4.83} \approx 0.57$};

\end{axis}
\end{tikzpicture}
\end{frame}



\begin{frame}{Отношение Шарпа - 3}
\justify
Инвестор решил оказаться на границе эффективности и вложил долю капитала $w$ в 
рыночный портфель с доходностью $\mathcal{R}_{m}$ и стандартным отклонением
$\sigma_{m}$, а долю $1-w$ --- в безрисковый актив.
\begin{align*}
S = \frac{w\mathbb{E}\mathcal{R}_{m} + (1-w)R_f - R_f}{w\sigma_{m}} = 
\frac{w(\mathcal{R}_{m} - R_f)}{w\sigma_{m}} =
\frac{\mathcal{R}_{m} - R_f}{\sigma_{m}}
\end{align*}

\justify
Все портфели на прямой границе эффективности имеют одинаковое отношение Шарпа.

\justify
Отношение Шарпа касательного рыночного портфеля --- угловой коэффициент 
(тангенс угла наклона) прямой границы эффективности.
\end{frame}



\begin{frame}{На пути к CAPM - 1}
\justify
Инвестор вложил $w$ в рыночный портфель и $1-w$ в безрисковый актив. На рынке
появился совершенно новый актив с доходностью $\mathcal{R}_{a}$, 
стандартным отклонением $\sigma_{a}$ и корреляцией со всем рынком $\rho$.

\justify
Инвестор может переложить небольшую долю $\varepsilon$ из безрискового актива в 
новый рискованный актив.

\justify
Новая ожидаемая доходность:
\begin{align*}
\mathbb{E}\mathcal{R} =
w\mathbb{E}\mathcal{R}_{m} + 
\varepsilon\mathbb{E}\mathcal{R}_a +
(1 - w - \varepsilon)R_{f}
\end{align*}

\justify
Новое стандартное отклонение:
\begin{align*}
\sigma = \sqrt{
	w^2\sigma_m^2 +
	2w\varepsilon\rho\sigma_m\sigma_a + \varepsilon^2\sigma_a^2
}
\end{align*}
\end{frame}



\begin{frame}{На пути к CAPM - 2}
\justify
Новое отношение Шарпа:
\begin{align*}
S(\varepsilon) = \frac{
	w(\mathbb{E}\mathcal{R}_{m} - R_f) + 
	\varepsilon(\mathbb{E}\mathcal{R}_a - R_f)
}{
	\sqrt{
		w^2\sigma_m^2 +
		2w\varepsilon\rho\sigma_m\sigma_a + \varepsilon^2\sigma_a^2
	}
}
\end{align*}

\justify
Возьмём производную в точке $\varepsilon = 0$:
\begin{align*}
\frac{\partial S}{\partial \varepsilon} \Biggr\rvert_{\varepsilon = 0} = 
\frac{1}{w\sigma_m} \left(
	\mathbb{E}\mathcal{R}_a - R_f -
	\frac{\rho\sigma_{a}}{\sigma_m}(\mathbb{E}\mathcal{R}_m - R_f)
\right)
\end{align*}

\justify
Чтобы рынок был в равновесии, производная должна быть равна нулю! Если она 
положительная (добавление актива улучшает отношение Шарпа), то все бросятся 
покупать актив. Спрос вырастет, цена увеличится, ожидаемая доходность упадёт.
\end{frame}



\begin{frame}{Capital Asset Pricing Model (CAPM)}
\justify
Чтобы рынок был в равновесии, для любого актива должно выполняться соотношение
\begin{align*}
\underbrace{\mathbb{E}\mathcal{R}_{a} - R_f}_{\mathclap{
\substack{\text{изб. дох-ть} \\ \text{актива}}}} = \dfrac{\rho\sigma_a}{\sigma_m}
\underbrace{\Big(\mathbb{E}\mathcal{R}_m - R_f\Big)}_{\mathclap{
\substack{\text{рыночная премия} \\ \text{за риск}}}}
\end{align*}

\justify
Это модель ценообразования капитальных активов (\en{Capital Asset Pricing 
Model, CAPM}). Коэффициент связи доходности актива с доходностью рынка обычно 
называют <<бета>>:
\begin{align*}
\beta_a = \dfrac{\rho\sigma_a}{\sigma_m} =
\dfrac{\operatorname{Cov}(\mathcal{R}_a, \mathcal{R}m)}
{\operatorname{Var}(\mathcal{R}_m)}
\end{align*}

\justify
Ожидаемая доходность актива зависит не от его дисперсии как таковой, а от 
ковариации со всем остальным рынком!
\end{frame}



\begin{frame}{Систематический риск}
\justify
\begin{align*}
&\mathbb{E}(\mathcal{R}_{a}) - R_{f} =
\beta_{a}\big(
	\mathbb{E}\mathcal{R}_{mkt} - R_{f}		
\big) \\
&\beta_a = \dfrac{\rho\sigma_a}{\sigma_m} =
\dfrac{\operatorname{Cov}(\mathcal{R}_a, \mathcal{R}m)}
{\operatorname{Var}(\mathcal{R}_m)}
\end{align*}

\justify
Доходность любого актива зависит от его <<беты>>. <<Бета>> 1.5 означает, что 
если рынок растёт (падает) на 1\%, то актив в среднем растёт (падает) на 1.5\%. 

\justify
<<Бета>> отражает чувствительность актива к глобальному 
рыночному риску (\en{market risk}) или систематическому риску (\en{systematic 
risk}). Насколько актив растёт вместе с рынком в хорошие времена, и насколько 
он падает вместе с рынком в плохие?

\justify
Придумайте примеры акций с $\beta<1$ и $\beta>1$?
\end{frame}



\begin{frame}{<<Беты>> некоторых акций}
\centering
\begin{tabular}{l|l|r}
Компания         & Тикер   & <<Бета>> \\ \hline
Lufthansa        & LHA.DE  & 1.70     \\
Infineon         & IFX.DE  & 1.49     \\
Deutsche Bank    & DBK.DE  & 1.44     \\
BMW              & BMW.DE  & 1.25     \\
Siemens          & SIE.DE  & 1.09     \\
Deutsche Telekom & DTE.DE  & 0.75     \\
Henkel           & HEN3.DE & 0.65    \\
Merck            & MRK.DE  & 0.45     
\end{tabular}

\justify
\centering
{\scriptsize Данные: Yahoo Finance}

\justify
Чем сильнее бизнес компании страдает в кризис, тем выше <<бета>>.

\justify
(*) <<Бета>> отдельной компании зависит не только от индустрии, но и от 
соотношения акционерного капитала и долговой нагрузки.
\end{frame}



\begin{frame}{Идиосинкратический риск}
\justify
Каждый актив (каждая акция) несёт свой собственный специфический риск. Самолёт 
авиакомпании может упасть, завод корпорации может сгореть. Этот 
идиосинкратический (\en{idiosyncratic}) риск можно уменьшить до нуля 
диверсификацией (\en{diversify away}). Рациональный инвестор держит в портфеле 
тысячи активов, поэтому не страдает от этого риска.

\justify
Вывод: идиосинкратический риск не входит в цену активов! Вы не зарабатываете 
дополнительную премию за риск, если держите в портфеле всего одну (пусть даже 
самую любимую) акцию. Все остальные инвесторы диверсифицировали этот риск и 
поэтому не требуют премию за риск (т.е. скидку в цене).

\justify
Систематический риск нельзя диверсифицировать. Инвесторы должны с ним мириться, 
поэтому требуют и получают компенсацию.
\end{frame}



\begin{frame}{Предел диверсификации - 1}
\justify
Пусть есть $n$ акций с одинаковыми ожидаемыми доходностями $\mu=5\%$ и 
стандартными отклонениями $\sigma = 20\%$. Все акции попарно связаны 
коэффициентом корреляции $\rho=0.25$. Мы вложили по доле $1/n$ в каждую акцию. 
Чему равно стандартное отклонение портфеля?

\begin{align*}
\sqrt{Var\left(\sum\limits_{i=1}^{n}\frac{1}{n}X_i \right)} &=
\sqrt{\frac{1}{n^2}\left(\sum\limits_{i=1}^{n}Var(X_i) + 2\sum\limits_{
1 \le i < j \le n}Cov(X_i, X_j)\right)} = \\
&= \sqrt{\frac{1}{n^2}\left(n\sigma^2 + n(n-1)\rho\sigma^2\right)} = \\
&= \sigma\sqrt{\frac{1}{n} + \frac{n-1}{n}\rho} \to \sigma\sqrt{\rho}
\end{align*}
\end{frame}



\begin{frame}{Предел диверсификации - 2}
\centering
\begin{tikzpicture}
\begin{axis}[
    width=\textwidth,
    height=\textheight-1cm,
    xlabel={\small Количество акций в портфеле},
    ylabel={\small Стандартное отклонение портфеля, \%},
    xlabel near ticks,
    ylabel near ticks,
    xmin=0, xmax=50,
    ymin=0, ymax=20
]

\addplot[line width=1pt, color=Set1-B, domain=1:50, samples at={1,1.2,...,50}] {20 * sqrt(1/x + (x-1)/x*0.25)};

\addplot[line width=1pt, dashed, color=Set1-B, domain=0:50]{10};

\draw[<->,>=triangle 90] (axis cs: 45, 0) -- (axis cs: 45, 10) node[pos=0.5,anchor=east]{\scriptsize{Недиверсифицируемый (систематический) риск}};

\draw[<->,>=triangle 90] (axis cs: 45, 10) -- (axis cs: 45, 20) node[pos=0.5,anchor=east]{\scriptsize{Диверсифицируемый (идиосинкратический) риск}};

\end{axis}
\end{tikzpicture}
\end{frame}



\begin{frame}{Роль корреляции с рынком - 1}
\justify
Какую из двух акций выбрать?

\justify
\centering
\begin{tabular}{l|r|r|r}
Состояние мира    & Вер-ть & Акция A  & Акция B \\ \hline
Потеря работы     & 50\%   & \$1\,000 & \$500 \\
Премия 12 зарплат & 50\%   & \$500    & \$1\,000 \\ \hline
Среднее           &        & \$750    & \$750
\end{tabular}

\justify
\centering
\qrcodeminipage{https://www.menti.com/ou76f57662}
\end{frame}



\begin{frame}{Роль корреляции с рынком - 2}
\justify
Инвесторам нравятся активы, которые дают дополнительный доход в <<плохие 
времена>>. Например, всем нравятся те активы, которые дают страховку от потери
работы.

\justify
Для рядового инвестора шанс потерять работу в кризис выше, чем в спокойное 
время. Он будет искать такие активы, которые не так сильно упадут в кризис, 
когда каждый дополнительный доллар будет на счету.

\justify
Спрос на активы, которые не сильно падают в кризис вместе с рынком (имеют 
низкую <<бету>>), выше, их цена --- выше, а ожидаемая доходность --- ниже.

\justify
Максимальную ожидаемую доходность дают те активы, которые обвалятся ровно в тот
момент, когда вам будет больнее всего (это активы с высокой <<бетой>>).
\end{frame}



\begin{frame}{Прокси рыночного портфеля}
\justify
В рамках CAPM в рыночный портфель входят не только акции, а вообще все активы в экономике, включая недвижимость, самолёты и пароходы. Далеко не все эти активы являются торгуемыми. Для практического применения CAPM нужно выбрать <<прокси>> --- портфель торгуемых активов. Обычно это фондовый индекс.

\vspace{\baselineskip}
\begin{itemize}
\justifying
\item Standard and Poor's 500 --- 500 крупнейших компаний США.
\item Russel 2000 --- 2000 крупнейших компаний США.
\item STOXX 600 --- 600 крупнейших компаний Европы.
\item ММВБ (рубли) и РТС (доллары) --- 43 крупнейшие компании России.
\item MSCI World --- 1600 крупнейших компаний мира.
\end{itemize}
\end{frame}



\begin{frame}{Фондовый индекс - 1}
\justify
Фондовый индекс --- индикатор состояния рынка, отражающий изменение цены 
заданной корзины ценных бумаг. Самый распространённый тип индекса --- индекс, 
взвешенный по капитализации (cap-weighted).
\begin{align*}
Index = \dfrac{\sum P_iQ_i}{D}
\end{align*}
Здесь $P_i$ --- цена бумаги $i$, $Q_i$ --- число бумаг $i$ в свободном 
обращении (free-float). $P_iQ_i$ --- рыночная капитализация одной бумаги,
весь числитель --- капитализация рынка.

\justify
Делитель $D$ обычно подбирают так, чтобы в момент самого первого расчёта индекс 
был круглым числом (например, 100). Впоследствии его подправляют при изменении 
состава индекса, разделении акций, и т.п.
\end{frame}



\begin{frame}{Фондовый индекс - 2}
\justify
Индекс капитализации любимых банков лектора:

\justify
\centering
\begin{tabular}{l|r|r|r|r}
                  & День 1 & День 2 & Перерыв & День 3 \\ \hline
Немецкий банк     & \$100  & \$105  & \$105   & \$110 \\
Американский банк & \$200  & \$180  & \$180   & \$190 \\
Французский банк  &        &        & \$50    & \$55  \\
Английский банк   &        &        & \$100   & \$95  \\ \hline
Капитализация     & \$300  & \$285  & \$435   & \$450 \\
Делитель          & 3.0    & 3.0    & 4.5789  & 4.5789 \\ \hline
Индекс            & 100.0  & 95.0   & 95.0    & 98.3
\end{tabular}

\justify
Если на рынок выходят новые компании, то рост капитализации всего рынка не 
увеличивает богатство акционеров уже представленных на рынке компаний. Индекс 
акций не растёт вслед за капитализацией.
\end{frame}



\begin{frame}{Индексы полной доходности}
\justify
<<Обычные>> ценовые индексы (\en{price index}), такие как S\&P\,500 или индекс 
МосБиржи, не учитывают дивиденды. После выплаты дивидендов цена акции обычно 
снижается на размер выплаты. Ценовой индекс тоже пропорционально уменьшается.

\justify
Индексы полной доходности (\en{total return index}) учитывают дивиденды. 
Динамика индекса полной доходности показывает гипотетический рост капитала 
инвестора, который всегда реинвестирует дивиденды в индекс.

\justify
\begin{itemize}
\item IMOEX --- индекс МосБиржи.
\item MCFTR --- индекс полной доходности без учёта налогов.
\item MCFTRN --- индекс полной доходности с учётом налогов на дивиденды по 
ставке для нерезидентов.
\item MCFTRR --- индекс полной доходности с учётом налогов на дивиденды по
ставке для организаций-резидентов.
\end{itemize}
\end{frame}



\begin{frame}{Рыночная премия за риск - 1}
\begin{align*}
\mathbb{E}\mathcal{R} - R_f = \beta(\mathbb{E}\mathcal{R}_m - R_f)
\end{align*}

\justify
Ожидаемая избыточная доходность инвестиций пропорциональна мере 
систематического риска $\beta$ и рыночной премии за риск (market risk premium) 
$\mathbb{E}\mathcal{R}_m - R_f$. О какого порядка премии мы говорим?
\end{frame}



    \renewcommand{\addGrowthPlot}[4]{
        \addplot[
            color = #2,
            line width = 1pt, 
            mark = #3,
            mark repeat = 120,
            mark phase = 396,
            mark options = {scale=2},
            style = #4
        ]
        table[
            x = date,
            y = #1,
            col sep = comma
        ]
        {data/fama_french_cumulative_growth_data.csv};
    }
    
    \newcommand{\addFlatLine}[5] {
        \draw[
            red,
            thick
        ]
        (axis cs: #1, #3) -- (axis cs: #2, #3)
        node[
            pos=#5,
            anchor=south
        ]
        {\scriptsize #4};
    }
    
    \newcommand{\addLossLine}[5] {
        \draw[
            red,
            thick
        ]
        (axis cs: #1, #3) -- (axis cs: #2, #4)
        node[
            anchor=west
        ]
        {\scriptsize #5\%};
    }

\begin{frame}{Рыночная премия за риск - 2}
\centering

\begin{tikzpicture}
\begin{axis}[
    width=\textwidth,
    height=\textheight - 1cm,
    date coordinates in=x,
    date ZERO=1926-06-30,
    xtick={1930-01-01,1940-01-01,1950-01-01,1960-01-01,1970-01-01,1980-01-01,1990-01-01,2000-01-01,2010-01-01,2020-01-01},
    minor xtick={1930-01-01,1950-01-01,1970-01-01,1990-01-01,2010-01-01},
    xticklabel=\year,
    grid=both,
    xmin=1926-12-31,
    xmax=2025-01-01,
    ymode=log,
    ymax=20000,
    log ticks with fixed point,
    ylabel={\small Рост \$1 начальных инвестиций},
    ylabel shift = -10pt,
    legend entries={
        Рынок акций США,
        Акции минус облигации,
        Безрисковые облигации,
        Инфляция (CPI-U)
    },
    legend pos=north west,
    legend style={font=\scriptsize},
    legend cell align={left}
]
    
    \addGrowthPlot{mkt}{Set1-B}{none}{solid}
    \addGrowthPlot{mkt_rf}{Set1-C}{none}{solid}
    \addGrowthPlot{rf}{Set1-D}{none}{solid}
    \addGrowthPlot{cpi}{Set1-E}{none}{dashed}

    \addFlatLine{1929-08-31}{1945-02-28}{2.10}{1929--45}{0.5}
    \addFlatLine{1968-11-30}{1983-04-30}{26.2}{1968--83}{0.9}
    \addFlatLine{2000-03-31}{2013-01-31}{137}{2000--13}{0.5}

    \addLossLine{1929-08-31}{1932-06-30}{2.10}{0.322}{-85}
    \addLossLine{1968-11-30}{1974-09-30}{26.2}{11.6}{-56}
    \addLossLine{2000-03-31}{2009-02-28}{137}{62.7}{-54}
\end{axis}
\end{tikzpicture}

\scriptsize{Данные: Kenneth French Data Library.}
\end{frame}



\begin{frame}{Рыночная премия за риск - 3}
\centering
\begin{tabular}{l|r|r|r|r}
Период & Сред. (ст. откл.) & $t$-тест & $p$-знач. & 99\% дов. инт. \\
\hline
1927--1959 & 11.3\% (24.9\%) & 2.61 &  1.4\% & [-0.6\%, 23.2\%] \\
1960--1989 &  5.2\% (16.9\%) & 1.69 & 10.3\% & [-3.3\%, 13.7\%] \\
1990--2020 &  9.4\% (17.9\%) & 2.90 &  0.7\% & [ 0.5\%, 18.2\%] \\
1960--2020 &  7.5\% (17.4\%) & 3.42 &  0.1\% & [ 1.7\%, 13.4\%] \\ \hline
1927--2020 &  8.8\% (20.2\%) & 4.26 & <0.1\% & [ 3.4\%, 14.3\%] 
\end{tabular}
{\scriptsize Годовые доходности сверх безрисковой процентной ставки.

Данные: Kenneth French Data Library.}

\end{frame}



\begin{frame}{Загадка премии за риск}
\justify
Mehra, Prescott (1985): инвесторы должны неправдоподбно сильно не любить риск,
чтобы премия за риск была такой же высокой, как в реальности. Модельный 
инвестор готов заплатить 49\% капитала, лишь бы не играть в лотерею, которая с 
шансами 50/50 либо уполовинит капитал, либо удвоит его. Это загадка (\en{equity 
risk premium puzzle}).

\justify
Рациональные объяснения:
\begin{itemize}
\justifying
\item Функция полезности неправильная.
\item Эффект выжившего. Инвесторы закладывают в цену риск финансовой
катастрофы, который пока не реализовался.
\end{itemize}

\justify
Поведенческое объяснение:
\begin{itemize}
\item Люди --- не рациональные роботы.
\end{itemize}
\end{frame}



\begin{frame}{Избегание риска - 1}
\justify
Опрос для следующих двух слайдов:

\justify
\centering
\qrcodeminipage{https://www.menti.com/mjujee2hfe}
\end{frame}



\begin{frame}{Избегание риска - 2}
\justify
Вы должны вложить весь свой капитал на срок 15 лет. Какой актив выбрать: A или
B?

\justify
\centering
\begin{tikzpicture}
\begin{groupplot}[
    group style = {group size = 2 by 1},
    width = \textwidth / 2,
	 ybar,
	 ymin = -30, ymax = 50,
	 xmin = 0.5, xmax = 50.5,
	 xlabel={\small Номер сценария},
	 ylabel={\small Годовая доходность, \%},
	 xlabel near ticks,
	 ylabel near ticks,
	 xtick={1, 10,20,30,40,50}
]
    
    \nextgroupplot[title = {Актив A}]
    \addplot[bar width=1pt, fill, color=Set1-B] table[x=mkt_1y_rank, y=sample_mkt_1y, col sep=comma] {data/simulated_market_annual_returns.csv};
    
    \nextgroupplot[title = {Актив B}, ylabel={}]
    \addplot[bar width=1pt, fill, color=Set1-B] table[x=rf_1y_rank, y=sample_rf_1y, col sep=comma] {data/simulated_market_annual_returns.csv};
    
\end{groupplot}
\end{tikzpicture}

\justify
На графиках --- доходность в один случайно выбранный год.
\end{frame}



\begin{frame}{Избегание риска - 3}
\justify
Вы должны вложить весь свой капитал на срок 15 лет. Какой актив выбрать: C или D?

\justify
\centering
\begin{tikzpicture}
\begin{groupplot}[
    group style = {group size = 2 by 1},
    width = \textwidth / 2,
	 ybar,
	 ymin = 0, ymax = 20,
	 xmin = 0.5, xmax = 50.5,
	 xlabel={\small Номер сценария},
	 ylabel={\small Годовая доходность, \%},
	 xlabel near ticks,
	 ylabel near ticks,
	 xtick={1, 10,20,30,40,50}
]
    
    \nextgroupplot[title = {Актив C}]
    \addplot[bar width=1pt, fill, color=Set1-B] table[x=mkt_15y_rank, y=sample_mkt_15y, col sep=comma] {data/simulated_market_annual_returns.csv};
    
    \nextgroupplot[title = {Актив D}, ylabel={}]
    \addplot[bar width=1pt, fill, color=Set1-B] table[x=rf_15y_rank, y=sample_rf_15y, col sep=comma] {data/simulated_market_annual_returns.csv};
    
\end{groupplot}
\end{tikzpicture}

\justify
На графиках --- средняя доходность за 15 лет.
\end{frame}



\begin{frame}{Избегание риска - 4}
\justify
Активы A и C --- рынок акций США. Активы B и D --- безрисковые облигации. 
Графики A и B показывают доходность за один наудачу выбранный год. Графики С и 
D показывают доходность за наудачу выбранные 15 лет подряд.

\justify
Ричард Талер считает, что существует эффект <<близорукого избегания риска>>. Мы 
не можем сложить в уме 15 случайных годовых доходностей и оценить распределение 
суммы. Инвесторы интуитивно умножают годовые колебания на 15 и переоценивают 
риск. На деле шанс получить отрицательную доходность на горизонте 15 лет не так 
велик, потому что центральная предельная теорема усредняет краткосрочные 
колебания.

\justify
Следствие. Чем реже вы проверяете состояние счёта, тем более рискованный (и 
более доходный) портфель вы можете себе позволить.
\end{frame}



\begin{frame}{Долгосрочные инвестиции - 1}
\centering
\small
\begin{tabular}{l|r|r|r|r|r|r|r}
Годы & Мин.    &     5\% &   25\% &   50\% &   75\% &   95\% & Макс. \\ \hline
1    & -62.1\% & -25.4\% & -3.2\% & 10.0\% & 21.3\% & 39.9\% & 175.7\% \\
2    & -48.7\% & -18.0\% &  0.4\% &  8.1\% & 16.6\% & 28.2\% &  58.1\% \\
3    & -38.5\% & -11.9\% &  0.7\% &  8.1\% & 13.3\% & 23.6\% &  40.4\% \\ 
5    & -15.5\% &  -7.2\% &  1.3\% &  8.0\% & 12.1\% & 19.0\% &  33.8\% \\
10   &  -5.0\% &  -2.6\% &  3.5\% &  7.2\% & 10.6\% & 14.3\% &  18.1\% \\
15   &  -1.9\% &   0.3\% &  3.6\% &  7.0\% & 10.0\% & 13.4\% &  14.9\% \\
20   &   0.3\% &   2.0\% &  4.1\% &  7.0\% &  9.4\% & 12.0\% &  13.6\% \\
30   &   4.4\% &   4.7\% &  5.9\% &  7.0\% &  8.1\% &  9.7\% &  11.5\% 
\end{tabular}

\justify
Квантили исторических доходностей рынка акций США сверх инфляции в зависимости 
от срока владения (проценты годовых), 1927--2020. Данные: Kenneth French Data 
Library, St. Louis  Fed.
\end{frame}



\newcommand{\addQuantilePlot}[3] {
    \addplot[
        color=#2,
        style=#3,
        thick
    ] 
    table [
        x=holding_period,
        y=#1,
        col sep=comma
    ]
    {data/us_mkt_holding_periods.csv};
}



\begin{frame}{Долгосрочные инвестиции - 2}
\centering
\begin{tikzpicture}
\begin{axis}[
    width = \textwidth,
    height = \textheight - 1.5cm,
    xmin=1,
    xmax=30,
    xtick={1, 5, 10, ..., 30},
    xlabel={\scriptsize Период владения, годы},
    ymin = -20,
    ymax = 40,
    ytick = {-20, -10, 0, 10, 20, 30, 40},
	 minor y tick num = 1,
    ylabel={\scriptsize Годовая доходность сверх инфляции},
    ylabel shift = -5pt,
    yticklabel={\pgfmathprintnumber{\tick}\%},
    grid=both,
    legend entries={
        Мин. / Макс.,
        Квантили 5\% / 95\%,
        Квантили 25\% / 75\%,
        Медиана
    },
    legend pos=north east,
    legend style={font=\scriptsize},
    legend cell align={left},
    ticklabel style = {font=\scriptsize}
]

    \addQuantilePlot{0}   {Set1-A} {solid}
    \addQuantilePlot{5}   {Set1-A} {dashed}
    \addQuantilePlot{25}  {Set1-B} {dashed}
    \addQuantilePlot{50}  {Set1-B} {solid}
    \addQuantilePlot{75}  {Set1-B} {dashed}
    \addQuantilePlot{95}  {Set1-A} {dashed}
    \addQuantilePlot{100} {Set1-A} {solid}
    
    \draw[thick, color=black] (axis cs: 1, 0) -- (axis cs: 30, 0);
\end{axis}
\end{tikzpicture}
{\scriptsize Данные: Kenneth French Data Library, St. Louis Fed.}
\end{frame}



\begin{frame}{Регулярные инвестиции - 1}
\justify
Каждый месяц лектор откладывает на пенсию 10\% от зарплаты (и от премий, когда 
они есть, хнык-хнык). Все деньги он инвестирует в широкий рынок акций США.

\justify
Предположим, что на дистанции зарплата лектора растёт на инфляцию (снова хнык-
хнык). За 30 лет лектор отложит эквивалент 3 годовых доходов.

\justify
На сколько вырастет покупательная способность сбережений за счёт 
инвестирования?
\end{frame}

\newcommand{\addQuantilePlotRegular}[3] {
    \addplot[
        color=#2,
        style=#3,
        thick
    ] 
    table [
        x=holding_period,
        y=#1,
        col sep=comma
    ]
    {data/us_mkt_regular_investment.csv};
}

\begin{frame}{Регулярные инвестиции - 2}
\centering
\begin{tikzpicture}
\begin{axis}[
    width = \textwidth,
    height = \textheight - 1.5cm,
    xmin=1,
    xmax=30,
    xtick={1, 5, 10, ..., 30},
    xlabel={\scriptsize Период инвестиций, годы},
    %ymin = -20,
    %ymax = 40,
    ytick = {0, 1, 2, 3, 4, 5, 6, 7, 8},
	 %minor y tick num = 1,
    ylabel={\scriptsize Рост сверх инфляции},
    %ylabel shift = -5pt,
    grid=both,
    %ymode=log,
    ymin=0,
    ymax=8,
    log ticks with fixed point,
    legend entries={
        Мин. / Макс.,
        Квантили 5\% / 95\%,
        Квантили 25\% / 75\%,
        Медиана
    },
    legend pos=north west,
    legend style={font=\tiny},
    legend cell align={left},
    ticklabel style = {font=\scriptsize}
]

    \addQuantilePlotRegular{0}   {Set1-A} {solid}
    \addQuantilePlotRegular{5}   {Set1-A} {dashed}
    \addQuantilePlotRegular{25}  {Set1-B} {dashed}
    \addQuantilePlotRegular{50}  {Set1-B} {solid}
    \addQuantilePlotRegular{75}  {Set1-B} {dashed}
    \addQuantilePlotRegular{95}  {Set1-A} {dashed}
    \addQuantilePlotRegular{100} {Set1-A} {solid}
    
    \draw[thick, color=black] (axis cs: 1, 1) -- (axis cs: 30, 1);
\end{axis}
\end{tikzpicture}
{\scriptsize Данные: Kenneth French Data Library, St. Louis Fed.}
\end{frame}



\begin{frame}{Прогнозирование рыночной премии за риск}
\justify
Можно ли предсказать, какой будет equity risk premium в следующем году, чтобы 
если она будет отрицательной, то пересидеть год в безрисковом активе? Скорее 
нет, чем да.

\justify
Welch, Goyal (2008): отношения P/E (цена акции к прибыли), D/P (дивиденды к 
цене акции) и другие не предсказывают рыночную доходность out of sample. То 
есть они не лучше предсказания <<в следующем году доходность будет равна 
средней доходности в прошлом>>.

\justify
В 2015--2019 годах многие аналитики строили графики дробей <<что-то на что-то>> 
и предсказывали обвал рынка. Пришёл март 2020 г., и обвал случился. 
Случился ли он из-за <<перегретости рынка>> или из-за пандемии?
\end{frame}



\begin{frame}{Market timing - 1}
\justify
\en{Market timing} --- попытки угадать краткосрочные колебания рынка, чтобы 
пересидеть плохие периоды в безрисковых облигациях.

\justify
Проблема: распределение дневных доходностей имеет толстый правый хвост: 
небольшое число торговых дней дают очень большую доходность.

\justify
\en{JPMorgan Retirement Guide (2021)}: если бы вы в 2001 году вложили \$10\,000 
в индекс S\&P\,500, то вы бы заработали за 20 лет \$42\,231 (7.5\% годовых). 
Если вы пропустили 10 удачных дней, то вы заработали \$19\,347 (3.3\%).

\justify
\en{``Time in market beats market timing``}.
\end{frame}



\begin{frame}{Market timing - 2}
\justify
Начинается самая страшная эпидемия со времён <<испанки>>. Рынок упал на 25\%. 
Покупаем?

\justify
\centering
\begin{tikzpicture}
	\begin{axis}[
		width = \textwidth,
		height = \textheight - 2cm,
		date coordinates in = x,
		date ZERO = 2020-01-01,
		xticklabel = {\day/\month},
		xmin = 2020-01-31,
		xmax = 2020-04-15
	]
	
	\addplot [
		color = Set1-B,
		line width = 1pt
	]
	table [
		x = DATE,
		y = SP500,
		col sep = comma
	]
	{data/SP500_covid.csv};
	\end{axis}
\end{tikzpicture}
\end{frame}



\begin{frame}{Market timing - 3}
\justify
Начинается самая страшная эпидемия со времён <<испанки>>. Рынок упал на 25\%. 
Покупаем?

\justify
\centering
\begin{tikzpicture}
	\begin{axis}[
		width = \textwidth,
		height = \textheight - 2cm,
		date coordinates in = x,
		date ZERO = 2020-01-01,
		xticklabel = {\day/\month},
		xmin = 2020-01-31,
		xmax = 2020-04-15
	]
	
	\addplot [
		color = Set1-B,
		line width = 1pt
	]
	table [
		x = DATE,
		y = SP500,
		col sep = comma
	]
	{data/SP500_covid.csv};
	
	\addplot [
		color = Set1-B,
		line width = 1pt,
		dashed
	]
	table [
		x = DATE,
		y = SP500,
		col sep = comma
	]
	{data/SP500_covid_extended.csv};
	\end{axis}
\end{tikzpicture}
\end{frame}


\begin{frame}{Market timing - 4}
\justify
Даже если у вас есть фундаментальная модель, которая кричит, что рынок пере- 
или недооценён, вы всё равно окажетесь в условиях огромной неопределённости.
Что, если ваша модель не учитывает новый фактор (например, совершенно новый и
толком не изученный вирус)?

\justify
Кроме того, дно рынка (самые низкие цены и самая высокая премия за риск) будет 
именно в момент максимальной неопределённости и страха.

\justify
Если вы покупаете активы <<на дне>>, то кто-то вам их продаёт. Этот кто-то 
уверен, что впереди нас ждёт ещё одно дно. Почему вы считаете, что он глупее,
чем вы?
\end{frame}



\begin{frame}{P/E Шиллера (CAPE) - 1}
\justify
Пусть все компании на рынке зарабатывают прибыль $E$ в год, а текущая цена 
(капитализация) равна $P$. Отношение <<цена-прибыль>> $P/E$ часто используют 
как индикатор переоценённости или недооценённости акций.

\justify
P/E Шиллера или P/E с поправкой на цикличность (\en{cyclically adjusted P/E, 
CAPE}) использует среднюю прибыль компаний за последние 10 лет с поправкой
на инфляцию.
\end{frame}



\newcommand{\dotWithNumber}[5] {
        \node[
            circle,
            fill,
            inner sep = 2pt,
            color = #3
        ]
        at (axis cs: #1, #2) {};
        
        \node[
            anchor=#5
        ]
        at (axis cs: #1, #2)
        {#4};
}

\begin{frame}{P/E Шиллера (CAPE) - 2}
\centering
\begin{tikzpicture}
\begin{axis}[
    width=\textwidth,
    height=\textheight - 1cm,
    date coordinates in=x,
    date ZERO=1880-01-01,
    xtick={1880-01-01,1900-01-01,1920-01-01,1940-01-01,1960-01-01,1980-01-01,2000-01-01,2020-01-01},
    minor xtick={1890-01-01,1910-01-01,1930-01-01,1950-01-01,1970-01-01,1990-01-01,2010-01-01},
    xticklabel=\year,
    grid=both,
    xmin=1880-01-01,
    xmax=2030-01-01,
    ylabel={\small Cyclically adjusted P/E},
    %ylabel shift = -10pt,
    %legend entries={
    %    Рынок акций,
    %    Акции минус облигации,
    %    Безрисковые облигации,
    %    Инфляция (CPI-U)
    %},
    %legend pos=north west,
    %legend style={font=\scriptsize},
    %legend cell align={left}
]
    
        \addplot[
            color = Set1-B,
            line width = 1pt
        ]
        table[
            x = month,
            y = shiller_cape,
            col sep = comma
        ]
        {data/shiller_cape.csv};
        
        \dotWithNumber{1929-09-01}{32.6}{Set1-B}{\small 32.6}{south}        
        
        \dotWithNumber{1999-12-01}{44.2}{Set1-B}{\small 44.2}{south}
        
        \dotWithNumber{2021-09-01}{38.3}{Set1-B}{\small 38.3}{south}
        
\end{axis}
\end{tikzpicture}
{\scriptsize Данные: Robert Shiller.}
\end{frame}



\begin{frame}{Модель Гордона}
\justify
Пусть акция приносит дивиденды $D$. Первая выплата случится через год. Начиная 
со второго года выплаты растут на $g$ процентов в год. Ставка дисконтирования 
будущих платежей $r$. Чему равна цена такой акции?

\begin{align*}
P = \frac{D}{1+r} + \frac{D(1+g)}{(1+r)^2} + \frac{D(1+g)^2}{(1+r)^3} + ... = 
\frac{D}{r - g}
\end{align*}

\justify
Эта формула известна как модель Гордона (\en{Gordon growth model}).
\end{frame}



\begin{frame}{Процентные ставки}
\justify
Предположим, что дивиденды $D$ равны чистой прибыли $E$. Ставку дисконтирования 
$r$ распишем как в CAPM через безрисковую ставку $r_f$ и премию за риск $\pi$:
\begin{align*}
r = r_f + \pi
\end{align*}

\justify
Перейдём к реальной ставке $r_f^*$ и реальной скорости роста дивидендов $g^*$
сверх инфляции $i$:
\begin{align*}
r_f = r_f^* + i, \quad g = g^* + i
\end{align*}
Тогда:
\begin{align*}
P = \frac{D}{r - g} = \frac{E}{r_f^* + i + \pi - (g^* + i)} 
\quad\Rightarrow\quad \frac{P}{E} = \frac{1}{r_f^* + \pi - g^*}
\end{align*}

\justify
Вывод: ниже реальная ставка $r_f^*$ --- выше отношение P/E при той же премии за 
риск $\pi$ и темпе роста дивидендов $g^*$.
\end{frame}



\newcommand{\plotThickAxis}[1]{
        \draw[very thick]
        (axis cs:1880-01-01, #1) -- (axis cs: 2050-01-01, #1);   
}

\newcommand{\plotThickZeroAxis}{
    \plotThickAxis{0}
}

\begin{frame}{Реальные процентные ставки - 1}
\centering
\begin{tikzpicture}
\begin{axis}[
    width=\textwidth,
    height=\textheight - 1cm,
    date coordinates in=x,
    date ZERO=2000-01-01,
    xtick={2000-01-01,2005-01-01,2010-01-01,2015-01-01,2020-01-01},
    minor xtick={1890-01-01,1910-01-01,1930-01-01,1950-01-01,1970-01-01,1990-01-01,2010-01-01},
    xticklabel=\year,
    yticklabel={\pgfmathprintnumber{\tick}\%},
    grid=both,
    xmin=2003-01-01,
    xmax=2024-12-01,
    ymin=-1.5,
    ymax=5.5,
    ylabel={\small Доходность облигаций},
    ylabel shift = -10pt,
    legend entries={
        10Y Treasury Notes,
        10Y Treasury Inflation-Protected Securities
    },
    legend pos=north east,
    legend style={font=\scriptsize},
    legend cell align={left}
]
    
        \addplot[
            color = Set1-A,
            line width = 1pt
        ]
        table[
            x = DATE,
            y = GS10,
            col sep = comma
        ]
        {data/GS10.csv};
        
        \addplot[
            color = Set1-B,
            line width = 1pt
        ]
        table[
            x = DATE,
            y = FII10,
            col sep = comma
        ]
        {data/FII10.csv};
        
        \dotWithNumber{2021-11-01}{1.56}{Set1-A}{\small 1.56}{south west}

        \dotWithNumber{2021-11-01}{-1.06}{Set1-B}{\small $-1.06$}{south west}

		  \plotThickZeroAxis

\end{axis}
\end{tikzpicture}
{\scriptsize Данные: St. Louis Fed (GS10, FII10)}.
\end{frame}



 \newcommand{\plotShillerData}[7]{    
        \addplot[
            color = #2,
            line width = 1pt,
            mark = #3,
            mark repeat = 120,
            mark phase = 60,
            mark options = {scale=2},
        ]
        table[
            x = month,
            y = #1,
            col sep = comma
        ]
        {data/shiller_cape.csv};   
        
	   \dotWithNumber{#4}{#5}{#2}{#6}{#7}
    }


\begin{frame}{Реальные процентные ставки - 2}
\centering
\begin{tikzpicture}
\begin{axis}[
    width = \textwidth,
    height = \textheight - 1cm,
    date coordinates in=x,
    date ZERO=1880-01-01,
    xtick={1880-01-01,1900-01-01,1920-01-01,1940-01-01,1960-01-01,1980-01-01,2000-01-01,2020-01-01},
    minor xtick={1890-01-01,1910-01-01,1930-01-01,1950-01-01,1970-01-01,1990-01-01,2010-01-01,2030-01-01},
    xticklabel=\year,
    yticklabel={\pgfmathparse{100*\tick}\pgfmathprintnumber{\pgfmathresult}\%},
    grid=both,
    xmin=1880-01-01,
    xmax=2040-01-01,
    ymin=-0.04,
    ymax=0.16,
    ylabel={\small Доходность облигаций},
    ylabel shift = -5pt,
    legend entries={
        Номинальная дох-ть,
        Реальная дох-ть
    },
    legend pos=north west,
    legend style={font=\small},
    legend cell align={left}
]         
    
    \plotShillerData{long_rate}{Set1-A}{none}{2021-09-01}{0.0129}{\small 1.29}{west}
    \plotShillerData{real_rate_10y}{Set1-B}{none}{2021-09-01}{-0.006}{\small -0.60}{west}
        
    \plotThickZeroAxis
\end{axis}
\end{tikzpicture}
\centering
{\scriptsize Данные: Robert Shiller.}
\end{frame}



\begin{frame}{Избыточная доходность CAPE - 1}
\justify
Предположим, что $g^* = 0$, то есть дивиденды и прибыль компаний растут на 
инфляцию.
\begin{align*}
P = \frac{E}{r_f^* + \pi} \quad\Rightarrow\quad \pi = \frac{E}{P} - r_f^*
\end{align*}

\justify
Отношение $E/P$ называется CAPE yield. Разность earnings yield и 
безрисковой реальной ставки, которая называется excess CAPE yield (ECY), может 
подсказать (а может и не подсказать) будущую премию за риск $\pi$.

\justify
Верно ли, что из высокого (относительно истории) отношения $P/E$ следует низкий 
excess CAPE yield и низкая премия за риск? Не обязательно.
\end{frame}



\begin{frame}{Избыточная доходность CAPE - 2}
\centering
\begin{tikzpicture}
\begin{axis}[
    width=\textwidth,
    height=\textheight - 1cm,
    date coordinates in=x,
    date ZERO=1880-01-01,
    xtick={1880-01-01,1900-01-01,1920-01-01,1940-01-01,1960-01-01,1980-01-01,2000-01-01,2020-01-01},
    minor xtick={1890-01-01,1910-01-01,1930-01-01,1950-01-01,1970-01-01,1990-01-01,2010-01-01,2030-01-01},
    xticklabel=\year,
    grid=both,
    xmin=1880-01-01,
    xmax=2035-01-01,
    ymin = -0.05,
    ymax = 0.25,
    ylabel={\small Excess CAPE Yield (ECY)},
    yticklabel={\pgfmathparse{100*\tick}\pgfmathprintnumber{\pgfmathresult}\%}
    %ylabel shift = -10pt,
    %legend entries={
    %    Рынок акций,
    %    Акции минус облигации,
    %    Безрисковые облигации,
    %    Инфляция (CPI-U)
    %},
    %legend pos=north west,
    %legend style={font=\scriptsize},
    %legend cell align={left}
]
    
        \addplot[
            color = Set1-B,
            line width = 1pt
        ]
        table[
            x = month,
            y = cape_excess_yield,
            col sep = comma
        ]
        {data/shiller_cape.csv};
        
        \dotWithNumber{2021-11-01}{0.0322}{Set1-B}{\small 3.2}{west}
        
         \dotWithNumber{1930-04-01}{-0.0119}{Set1-B}{\small $-1.2$}{west}
    		\dotWithNumber{2000-01-01}{-0.0152}{Set1-B}{\small $-1.5$}{west}
     
        
	    \plotThickZeroAxis
\end{axis}
\end{tikzpicture}
{\scriptsize Данные: Robert Shiller.}
\end{frame}



\begin{frame}{Избыточная доходность CAPE - 3}
\centering
\begin{tabular}{l|r|r|r|r|r}
Период     &    5\% &   5\% &  50\% &   75\% &   95\% \\ \hline 
1881--1915 & -0.9\% & 0.2\% & 1.5\% &  4.7\% &  6.9\% \\ 
1916--1950 &  0.2\% & 2.9\% & 8.5\% & 12.6\% & 19.2\% \\
1951--1985 &  1.0\% & 2.5\% & 5.4\% &  8.1\% & 10.4\% \\
1986--2020 &  0.2\% & 1.5\% & \alert{2.6\%} &  3.8\% &  5.6\% \\ \hline
1881--2020 & -0.3\% & 1.5\% & \alert{3.5\%} &  6.7\% & 13.3\%
\end{tabular}

\centering
{\scriptsize Распределение CAPE Excess Yield. Данные: Robert Shiller.}

\justify
Shiller (2020): текущее значение 3.2\% (сентябрь 2021 г.) чуть ниже 
исторической медианы. Акции стоят дорого в терминах P/E, но не так дорого по 
сравнению с облигациями.

\justify
Значит, можно не опасаться обвала рынка в ближайшие лет 5? Нет, не значит. 
Акции --- по-прежнему рискованный инструмент. Возьмите столько риска, сколько 
комфортно лично вам.
\end{frame}



\begin{frame}{CAPE и прогнозирование премии за риск}
\centering
\begin{tabular}{r|r|r|r|r}
\multicolumn{2}{c|}{Изб.\,дох-ть\,CAPE} &
\multicolumn{3}{c}{Десятилетняя\,дох-ть\,акций} \\
\hline
Мин. & Макс. & Средняя & Худшая & Лучшая \\
\hline
-1.5\% &  0.7\% &  0.2\%  & -5.9\% &  8.9\% \\
 0.7\% &  1.5\% &  2.3\%  & -4.4\% & 11.9\% \\
 1.5\% &  1.9\% &  3.8\%  & -4.0\% & 15.2\% \\
 2.0\% &  2.6\% &  4.2\%  & -3.2\% & 16.1\% \\ 
\hline
 \alert{2.6\%} &  \alert{3.3\%} &  \alert{5.5\%}  & \alert{-4.0\%} & \alert{15.8\%} \\
\hline
 3.3\% &  4.8\% &  8.0\%  & -1.2\% & 15.4\% \\
 4.8\% &  6.4\% &  9.1\%  &  1.9\% & 14.3\% \\
 6.4\% &  8.2\% &  9.6\%  &  3.8\% & 14.3\% \\
 8.2\% & 10.0\% &  10.7\% &  6.0\% & 14.3\% \\ 
10.0\% & 14.4\% &  14.4\% &  7.8\% & 18.8\%  
\end{tabular}

\centering
{\scriptsize Excess CAPE Yield и будущая премия за риск, 1927--2010. Данные: Robert Shiller.}

\justify
Проблема: в 1950 году мы не могли знать, что значение 3.2\% попадёт в 5-й 
дециль с 1927 по 2010 год.
\end{frame}



\begin{frame}{Бэктестинг - 1}
\justify
Бэктестинг (\en{backtesting}) --- проверка стратегии на исторических данных. 
Как стратегия торговала бы в момент времени $T$, если бы знала все 
данные, известные к моменту времени $T$?

\justify
Стратегия CAPE:
\begin{itemize}
\item Каждый месяц вычисляем избыточную доходность CAPE.
\item Находим её квантиль $Q$ в историческом распределении за последние 40 лет
(480 наблюдений).
\item Доля акций на следующий месяц $\max(\min((Q - 5\%) / (95\%-5\%), 1), 0)$.
\end{itemize}

\justify
Стратегия 50/50: раз в месяц вкладываем 50\% в акции и 50\% в безрисковый 
актив.
\end{frame}



\newcommand{\plotThickOneAxis}{
   \plotThickAxis{1.0}   
}

\renewcommand{\addGrowthPlot}[4]{
        \addplot[
            color = #2,
            line width = 1pt, 
            mark = #3,
            mark repeat = 120,
            mark phase = 276,
            mark options = {scale=2},
            style = #4
        ]
        table[
            x = date,
            y = #1,
            col sep = comma
        ]
        {data/cape_strategy_growth_1927.csv};
    }

\begin{frame}{Бэктестинг - 2}

\centering
\begin{tikzpicture}
\begin{axis}[
    width = \textwidth,
    height = \textheight - 1cm,
    date coordinates in=x,
    date ZERO=1926-06-30,
    xtick={1930-01-01,1940-01-01,1950-01-01,1960-01-01,1970-01-01,1980-01-01,1990-01-01,2000-01-01,2010-01-01,2020-01-01},
    minor xtick={1930-01-01,1950-01-01,1970-01-01,1990-01-01,2010-01-01},
    xticklabel=\year,
    grid=both,
    xmin=1926-12-31,
    xmax=2025-01-01,
    ymode=log,
    log ticks with fixed point,
    ylabel={Рост \$1 сверх безрисковой ставки},
    ylabel shift = -5pt,
    legend entries={
            CAPE,
            50/50,
            Преимущество CAPE
        },
    legend pos=north west,
    legend style={font=\scriptsize},
    legend cell align={left}
    ]
    \addGrowthPlot{strategy_growth}{Set1-A}{square}{solid}
    \addGrowthPlot{benchmark_growth}{Set1-B}{o}{solid}
    \addGrowthPlot{overperformance}{Set1-C}{none}{solid}
    \plotThickAxis{1.0}
\end{axis}
\end{tikzpicture}
{\scriptsize Данные: Kenneth French Data Library, Robert Shiller.}
\end{frame}



\renewcommand{\addGrowthPlot}[4]{
        \addplot[
            color = #2,
            line width = 1pt, 
            mark = #3,
            mark repeat = 60,
            mark phase = 60,
            mark options = {scale=2},
            style = #4
        ]
        table[
            x = date,
            y = #1,
            col sep = comma
        ]
        {data/cape_strategy_growth_1990.csv};
    }
    
\begin{frame}{Бэктестинг - 3}
\centering
\begin{tikzpicture}
\begin{axis}[
    width=\textwidth,
    height = \textheight - 1cm,
    date coordinates in=x,
    date ZERO=1926-06-30,
    xtick={1990-01-01,1995-01-01,2000-01-01,2005-01-01,2010-01-01,2015-01-01,2020-01-01},
    xticklabel=\year,
    grid=both,
    xmin=1990-01-01,
    xmax=2022-01-01,
        ylabel={Рост \$1 сверх безрисковой ставки},
        ylabel shift = -5pt,
        legend entries={
            Стратегия CAPE,
            Стратегия 50/50,
            Преимущество CAPE
        },
        legend pos=north west,
        legend style={font=\scriptsize},
        legend cell align={left}
    ]
    \addGrowthPlot{strategy_growth}{Set1-A}{square}{solid}
    \addGrowthPlot{benchmark_growth}{Set1-B}{o}{solid}
    \addGrowthPlot{overperformance}{Set1-C}{none}{solid}
    \plotThickAxis{1.0}

\end{axis}
\end{tikzpicture}
{\scriptsize Данные: Kenneth French Data Library, Robert Shiller.}
\end{frame}



\begin{frame}{Бэктестинг - 4}
\centering
{\small
\begin{tabular}{l|r|r|r|r|r|r}
& \multicolumn{2}{c|}{1927--1959} & \multicolumn{2}{c|}{1960--1989} & \multicolumn{2}{c}{1990--2020} \\ 
\cline{2-7}
                   & CAPE    & 50/50 & CAPE    & 50/50 & CAPE    & 50/50 \\   \hline
Ср. дох-ть &  7.8\% &   5.4\%  &   3.4\% &   2.4\%   &   4.2\% &   4.3\%   \\
Ст. откл.  &  12.3\% &   11.7\%  &  8.0\% &  7.8\%   &  7.4\% &  7.6\%   \\
От. Шарпа  &   0.64  &   0.46    &   0.43  &   0.31    &   0.56  &   0.56    \\
Просадка   & -39.5\% & -58.4\%   & -33.0\% & -32.0\%   & -37.0\% & -30.3\% \\
Доля акций &  46.5\% & 50.0\%   &  40.0\% & 50.0\%   &  34.3\% & 50.0\%  \\ 
\cline{2-7}
Мед. ECY   & \multicolumn{2}{c|}{6.0\%} & \multicolumn{2}{c|}{3.9\%} & \multicolumn{2}{c}{2.3\%} 
\end{tabular}
}

\justify
Даже если поверить, что в первой половине XX стратегия CAPE обгоняла стратегию
50/50, в наши дни преимущество пропало. Даже если есть статистическая связь 
между избыточной доходностью CAPE и премией за риск, её не так-то просто 
превратить в прибыльную торговую стратегию.
\end{frame}



\begin{frame}{Так сколько риска взять?}
\justify
Прежде чем инвестировать в акции, обеспечьте себе финансовую подушку на 
3--6--12 месяцев без работы.

\justify
Возьмите столько систематического риска, сколько комфортно лично вам, с вашей 
личной чувствительностью к риску. Учтите, что в истории были прецеденты, когда 
акции теряли по 80\%. Если вы можете без последствий для душевного здоровья 
пережить просадку портфеля на 40\%, купите акций на $40\% / 80\% = 50\%$.

\justify
Не переоценивайте свою терпимость к риску.

\end{frame}



\begin{frame}{Горизонт инвестирования}
\justify
Соизмеряйте долю акций (и других рискованных активов) с временным горизонтом 
инвестирования. На горизонтах до 10--15 лет акции --- крайне рискованный 
инструмент.

\justify
Оцените, какую доходность вам нужно получить, чтобы за заданное количество лет
накопить на достижение цели (оплатить образование детям, купить личный 
самолёт). Зная долгосрочную доходность акций (5--6\% сверх инфляции), вычислите 
требуемый вес акций в портфеле.

\justify
Популярное правило большого пальца для пенсионных накоплений: доля облигаций в 
портфеле равна возрасту. Это не догма. Например, если вы хотите передать
капитал по наследству, то ваш горизонт инвестирования длиннее, чем ваша 
ожидаемая продолжительность жизни.
\end{frame}



\begin{frame}{Человеческий капитал}
\justify
Человеческий капитал --- сумма дисконтированных будущих доходов человека. Для 
многих людей, особенно молодых, человеческий капитал намного больше финансового 
капитала.

\justify
Кто вы: акция или облигация? Как ваш человеческий капитал связан с рынком 
акций?
\begin{itemize}
\justifying
\item Вы --- исследователь или профессор на постоянной ставке. Вы преподаёте и 
в кризис, и во время роста. Ваш доход не сильно связан с финансовым рынком. Вы 
--- облигация.
\item Вы --- инвестбанкир. <<Бета>> акций вашего работодателя 1.5, как и 
<<беты>> других потенциальных работодателей. В кризис вы запросто останетесь 
без работы. Вы --- акция.
\end{itemize}

\justify
Не забывайте о человеческом капитале. Если вы --- акция, то сместите финансовый 
капитал в сторону облигаций.
\end{frame}



\begin{frame}{Усреднение при покупке акций}
\justify
Допустим, у вас есть крупная сумма, которую вы готовы инвестировать в акции.
Что лучше?
\begin{itemize}
\justifying
\item Вложить всю сумму разом.
\item Разбить на 12-24 частей и покупать акции частями раз в месяц.
\end{itemize}

\justify
Vanguard (2012): выгоднее вложить все деньги сразу. В прошлом вложить все 
деньги разом было выгоднее в 66\% случаев. Акции на дистанции в среднем растут.
Если вы инвестируете частями, вы недополучаете часть премии за риск.

\justify
Люди --- не рациональный роботы, и им страшно вкладывать всю сумму разом. Если
вам страшно, то покупать акции частями раз в месяц не глядя на цены --- хорошее
механическое правило.
\end{frame}



\begin{frame}{Отношение Шарпа}
\justify
Отношение Шарпа (Sharpe ratio) --- отношение средней избыточной доходности 
портфеля к её стандартному отклонению.
\begin{align*}
\overline{R_{excess}}
&= \dfrac{1}{n}\sum\limits_{t=1}^{n}(R_{portfolio,t} - R_{free,t})\\
Sharpe &= \dfrac{\overline{R_{excess}}}
{\sqrt{\dfrac{1}{n-1}\sum\limits_{t=1}^{n}(R_{portfolio,t} - R_{free,t} - 
\overline{R_{excess}})^2}}
\end{align*}

\justify
Хозяйке на заметку: касательный (\en{tangent}) портфель имеет максимальное 
отношение Шарпа среди всех портфелей, составленных из рискованных активов.
\end{frame}



\begin{frame}{Отношение Сортино}
\justify
Отношение Сортино (Sortino information ratio) --- отношение доходности сверх 
минимально допустимой $R_{min}$ (minimum acceptable return) к волатильности 
<<вниз>>.
\begin{align*}
\overline{R_{excess}} &=
\dfrac{1}{n}\sum\limits_{t=1}^{n}(R_{portfolio,t} - R_{min,t}) \\
Sortino &= \dfrac{
	\overline{R_{excess}}
}{
	\sqrt{\dfrac{1}{n-1}\sum\limits_{t=1}^{n}
		\left(\min[0, R_{portfolio,t} - R_{min, t}]\right)^2}
}
\end{align*}

\justify
В качестве минимально допустимой доходности можно использовать как константу 
(например, 0), так и переменную величину (например, безрисковую доходность).
\end{frame}



\begin{frame}{Геометрическое среднее - 1}
\justify
Арифметическое среднее доходностей --- лучшая оценка того, что случится с 
вашими инвестициями в следующий период (месяц, год). Если вы оцениваете 
инвестиции на несколько периодов, то используйте геометрическое среднее.
\begin{align*}
Geom Mean = \left(
\prod\limits_{t=1}^{n}(1 + R_{portfolio,t} - R_{free,t})
\right)^{\dfrac{1}{n}} - 1
\end{align*}

\justify
Пример: за два года портфель сходил на 20\% вверх, а потом на 20\% вниз.
\begin{itemize}
\justifying
\item Арифметическое среднее: $(+20\%-20\%)/2 = 0\%$.
\item Геометрическое среднее: $\sqrt{(1+0.2)(1-0.2)} - 1 \approx -2\%$. 
\end{itemize}
\end{frame}



\begin{frame}{Геометрическое среднее - 2}
\centering
\begin{tabular}{l|r|r|r|r|r}
\multirow{2}{*}{Период} &
Арифм. & 
\multirow{2}{*}{Ст. откл.} &
Геом. &
Отнош. &
Отнош. \\
& сред. & & сред. & Шарпа & Сортино \\ 
\hline
1927--1959 & 11.3\% & 24.9\% & 8.3\% & 0.45 & 1.00 \\
1960--1989 &  5.2\% & 16.9\% & 3.8\% & 0.31 & 0.58 \\
1990--2020 &  9.4\% & 17.9\% & 7.8\% & 0.52 & 1.06 \\
1960--2020 &  7.5\% & 17.4\% & 6.0\% & 0.43 & 0.86 \\ \hline
1927--2020 &  8.8\% & 20.2\% & 6.8\% & 0.44 & 0.91
\end{tabular}

{\scriptsize Годовые доходности сверх безрисковой процентной ставки.

Данные: Kenneth French Data Library. \par }

\justify
Хозяйке на заметку: если актив следует геометрическому броуновскому движению,
то геометрическое среднее отличается от арифметического среднего на $
\sigma^2/2$, где $\sigma$ --- стандартное отклонение.
\end{frame}


\begin{frame}{Международная диверсификация}
\justify
Что, если рынок акций США --- это случайное исключение из правил? А вдруг он 
повторит судьбу рынка Японии, который стагнирует уже 30 лет?

\centering
\begin{tabular}{l|r|r|r|r}
\multirow{2}{*}{Регион} &
Геом. & 
Арифм. &
\multirow{2}{*}{Ст. откл.} &
Отнош. \\
& сред. & сред. & & Шарпа \\ \hline
Европа    & 4.3\% & 6.1\% & 19.7\% & 0.31 \\
Япония    & 4.2\% & 8.7\% & 29.2\% & 0.30 \\
Швейцария & 4.6\% & 6.4\% & 19.4\% & 0.33 \\
UK        & 5.5\% & 7.3\% & 19.6\% & 0.37 \\
США       & 6.5\% & 8.5\% & 19.9\% & 0.43 \\ \hline
Мир       & 5.2\% & 6.6\% & 17.4\% & 0.38
\end{tabular}

\centering
{\scriptsize Годовые доходности рынков акций сверх инфляции (реальная доходность),
1900--2019. Данные: Credit Suisse Global Investment Returns Yearbook 2020.}

\justify
Вывод: диверсифицируйтесь не только по компаниям и отраслям, но и по странам.
\end{frame}




\begin{frame}{Оценка успеха управляющих - 1}
\justify
Лектор составил портфель акций на Московской бирже в ноябре 2015 г. За 5 лет он 
заработал 114\%. Три паевых фонда, красный, жёлтый и зелёный, заработали 136\%, 
114\% и 107\%. Молодцы ли лектор и управляющие фондами?

\justify
Безрисковая процентная ставка (ключевая ставка ЦБ) за тот же период дала бы 
$51\%$. Индекс ММВБ полной доходности (включая дивиденды минус налоги) вырос на 
$150\%$.

\justify
Какой риск взяли на себя лектор и управляющие, и получили ли они компенсацию за 
этот риск?
\end{frame}



\begin{frame}{Оценка успеха управляющих - 2}
\center
\begin{tikzpicture}
\begin{axis}[
  width=\textwidth,
  height=\textheight - 1cm,
  date coordinates in=x,
  date ZERO=2015-10-31,
  xtick={2016-01-01,2017-01-01,2018-01-01,2019-01-01,2020-01-01, 2021-01-01},
  xticklabel=\year,
  xmin=2015-10-31,
  xmax=2021-02-01,
  grid=major,
  ylabel={\small{Рост 1 р. начальных инвестиций}},
  xlabel near ticks,
  ylabel near ticks,
  legend entries = {
      Индекс ММВБ,
      Ставка ЦБ,
      Лектор,
      Красный,
      Жёлтый,
      Зелёный
  },
  legend pos=north west,
  legend style={font=\tiny},
  legend cell align={left}
]
\addplot[color=Set1-D, mark=none, thick] table[x=date, y=benchmark_growth, col sep=comma]{data/fund_growth.csv};
\addplot[color=Set1-D, mark=none, dashed, thick] table[x=date, y=risk_free_growth, col sep=comma]{data/fund_growth.csv};
\addplot[color=Set1-B, mark=none, thick] table[x=date, y=personal_growth, col sep=comma]{data/fund_growth.csv};
\addplot[color=Set1-A, mark=none, thick] table[x=date, y=red_growth, col sep=comma]{data/fund_growth.csv};
\addplot[color=Set1-E, mark=none, thick] table[x=date, y=yellow_growth, col sep=comma]{data/fund_growth.csv};
\addplot[color=Set1-C, mark=none, thick] table[x=date, y=green_growth, col sep=comma]{data/fund_growth.csv};
\end{axis}
\end{tikzpicture}
\end{frame}



\begin{frame}{Регрессия для оценки результатов - 1}
\justify
CAPM говорит, что избыточная доходность портфеля прямо пропорциональна 
рыночному риску:
\begin{align*}
\mathbb{E}\mathcal{R}_{fund} - R_{free} = 
\beta_{portfolio}(\mathbb{E}\mathcal{R}_{mkt} - R_{free})
\end{align*}

\justify
Посчитаем ежемесячные доходности рынка, портфеля и безрискового актива, а затем 
оценим линейную регрессию:
\begin{align*}
R_{fund,t} - R_{free,t} =
\alpha + \beta(R_{mkt,t} - R_{free,t}) + \varepsilon_t,
\quad
\varepsilon_t \sim \mathcal{N}(0, \sigma_{\varepsilon}^2)
\end{align*}

\justify
По CAPM, $\alpha=0$. Если управляющий имеет положительную <<альфу>>, то он 
зарабатывает больше, чем можно было бы ожидать при данном уровне 
систематического риска.
\end{frame}



\begin{frame}{Регрессия для оценки результатов - 2}
\center
\begin{tikzpicture}
\begin{axis}[
  width=\textwidth,
  height=\textheight - 1cm,
  legend entries = {
    Лектор,
    Красный фонд,
    $y = +0.07 + 0.61x$,
    $y = -0.01 + 0.89x$
  },
  legend pos=north west,
  legend style={font=\tiny},
  legend cell align={left},
  xlabel={\small{Месячная избыточная доходность индекса, \%}},
  ylabel={\small{Месячная изб. дох. фонда, \%}},
  xlabel near ticks,
  ylabel near ticks,
  %xtick={-0.1, -0.05, 0, 0.05, 0.1},
  %ytick={-0.1, -0.05, 0, 0.05, 0.1},
  grid=major,
  xmin=-11,xmax=11,
  ymin=-11,ymax=11
]

\addplot[color=Set1-B, only marks] table[x=benchmark_excess_return, y=personal_excess_return, col sep=comma]{data/fund_excess_return.csv};

\addplot[color=Set1-A, only marks] table[x=benchmark_excess_return, y=red_excess_return, col sep=comma]{data/fund_excess_return.csv};

\addplot[color=Set1-B, domain=-12:12] {0.07 + 0.61*x};

\addplot[color=Set1-A, domain=-12:12] {-0.01 + 0.89*x};

\end{axis}
\end{tikzpicture}
\end{frame}



\begin{frame}{Регрессия для оценки результатов - 3}
\center
\begin{tabular}{l|r|r|r|r|r}
\multirow{2}{*}{Фонд}    & $\hat{\alpha}$ & $\hat{\beta}$ & \multirow{2}{*}{$R^2$} & Отнош. & Отнош. \\
& \scriptsize{(ст. откл.)} & \scriptsize{(ст. откл.)} & & Шарпа & Сортино \\ \hline
Лектор  &  0.07\% \scriptsize{(0.27\%)} & 0.61 \scriptsize{(0.06)}   &  0.61 & 0.65 & 1.20 \\
Красный & -0.01\% \scriptsize{(0.18\%)} & 0.89 \scriptsize{(0.04)}   &  0.89 & 0.68 & 1.20 \\
Жёлтый  & -0.17\% \scriptsize{(0.25\%)} & 0.91 \scriptsize{(0.06)}   &  0.80 & 0.52 & 0.93 \\
Зелёный & -0.28\% \scriptsize{(0.16\%)} & 0.98 \scriptsize{(0.04)}   &  0.92 & 0.48 & 0.81 \\ \hline
Индекс  & 0.00\% \scriptsize{(0.00\%)}  & 1.00 \scriptsize{(0.00)}   &  1.00 & 0.73 & 1.29
\end{tabular}
\justify
{\scriptsize(62 месячных доходностей 11.2015--12.2020. Значения $\hat{\alpha}$ 
и $\hat{\beta}$ для месячных доходностей. Отношения Шарпа и Сортино 
сконвертированы в годовое выражение умножением на $\sqrt{12}$.)}

\justify
Если купить фонд лектора с <<плечом>> $1/0.61 = 1.64$, то можно получить 
<<бету>> 1.0, как в индексе, но получать дополнительную доходность $0.07\% 
\cdot 1.64 \cdot 12 = 1.37\%$ в год при том же систематическом риске. Слишком 
хорошо, чтобы быть правдой.
\end{frame}



\begin{frame}{Дополнительные факторы риска}
\justify
Согласно CAPM, единственный риск, за который можно заработать премию --- это 
систематический рыночный риск.

\justify
Так ли это на самом деле? Вдруг лектор с положительной <<альфой>> (забудем на 
минуту, что она не статистически значимая) на самом деле заработал на другом 
риске?

\end{frame}



\begin{frame}{<<Аномалии>> на рынке акций}
\justify
<<Аномалии>> (\en{anomalies}) --- статистические закономерности, которые 
наблюдаются на рынке акций, но которые нельзя объяснить моделью CAPM.

\justify
Эффект размера (\en{size effect}). Акции небольших компаний в среднем дают 
более высокую доходность, чем акции крупных компаний.

\justify
Эффект стоимости (\en{value effect}). Скучные акции <<стоимости>> (\en{value}) 
в среднем дают более высокую доходность, чем гламурные акции роста 
(\en{growth}).

\justify
Инерция (\en{momentum}). Акции, которые выросли в прошлом году, растут лучше, 
чем акции, упавшие в прошлом году.
\end{frame}



\begin{frame}{Эффект размера - 1}
\justify
Разобьём все акции на пять групп по возрастанию рыночной капитализации. 
Посмотрим на среднюю избыточную доходность сверх безрисковой процентной ставки.

\justify
\centering
\begin{tabular}{r|r|r|r|r}
Мал. 20\% &      2 &      3 &      4 & Бол. 20\% \\ \hline
   14.0\% & 12.5\% & 11.7\% & 10.7\% &     8.3\%
\end{tabular}

\centering
{\scriptsize Средняя годовая доходность акций сверх безрисковой ставки в
зависимости от рыночной капитализации, 1927--2020 гг. Данные Kenneth French
Data Library.}
\end{frame}



\begin{frame}{Эффект размера - 2}
\justify
Акции с более высокой <<бетой>> должны давать более высокую доходность. Что, 
если у маленьких акций просто более высокая <<бета>>? Разобьём все акции на 25 
групп: 5 по возрастанию размера и 5 по возрастанию <<беты>>.

\justify
\centering
\begin{tabular}{l|r|r|r|r|r}
              &   Мал. &      2 &      3 &     4  & Бол. \\ \hline
Низк. $\beta$ & 10.4\% &  9.3\% &  8.9\% &  8.9\% & 6.8\% \\
2             & 12.0\% & 11.3\% & 11.4\% & 10.6\% & 7.4\% \\
3             & 12.6\% & 12.4\% & 10.8\% &  9.9\% & 7.4\% \\
4             & 13.8\% & 11.7\% & 10.6\% &  8.3\% & 7.8\% \\
Выс. $\beta$  & 11.9\% &  9.2\% & 10.0\% & 10.7\% & 7.3\%
\end{tabular}

\centering
{\scriptsize Средняя годовая доходность акций сверх безрисковой ставки в
зависимости от капитализации и <<беты>>, 1964--2020 гг. Данные Kenneth French 
Data Lubrary.}

\justify
Загадка: что случилось с акциями с высокой <<бетой>>?
\end{frame}



\begin{frame}{Эффект стоимости - 1}
\justify
Принцип <<стоимостного>> инвестирования (\en{value investing}): ищите 
недоценённые бумаги с помощью фундаментального анализа, то есть на основании
финансовых показателей.

\justify
Некоторые простейшие показатели:
\begin{itemize}
\justifying
\item Цена к прибыли (\en{price / earnings, P/E}).
\item Дивидендная доходность (\en{dividend yield, D/P}).
\item Бухгалтерский капитал к рыночной капитализации

(\en{book\,value / market\,value, B/M}).
\end{itemize}

\justify
Показатель B/M удобен тем, что обычно ни числитель, ни знаменатель обычно не 
обращаются в ноль и не бывают отрицательными. Прибыль же может быть 
отрицательной, а дивиденды --- нулевыми.
\end{frame}




\begin{frame}{Эффект стоимости - 2}
\justify
Разобьём компании на 5 групп по возрастанию показателя B/M.

\justify
\centering
\begin{tabular}{r|r|r|r|r}
Низк. B/M &     2 &     3 &      4 & Выс. B/M \\ \hline
    8.6\% & 8.4\% & 9.8\% & 10.3\% &   13.1\% 
\end{tabular}

\centering
{\scriptsize Средняя годовая доходность акций сверх безрисковой ставки в
зависимости от отношения book/market, 1927--2020 гг. Данные Kenneth French 
Data Library.}

\justify
Компании с высоким B/M --- скучные компании <<стоимости>>. Например,
поставщики коммунальных услуг с многочисленными физическими активами на балансе
(высокое значение \en{book value}).

\justify
Компании с низким B/M --- гламурные компании <<роста>>. Например,
интернет-гиганты, нематериальные активы которых не отражены в бухгалтерском
капитале (низкое значение \en{book value}). 

\end{frame}



\begin{frame}{Факторные портфели SMB и HML}
\justify
Fama, French (1993): разобьём акции на 6 групп в зависимости от размера и 
отношения B/M.

\justify
\centering
\begin{tabular}{l|c|c}
                & Мал. 50\%     & Бол. 50\% \\ \hline
30\% низк. B/M  & Small Growth  & Big Growth \\
40\% сред. B/M  & Small Neutral & Big Neutral \\
30\% высок. B/M & Small Value   & Big Value
\end{tabular}

\justify
Портфель SMB (Small Minus Big): покупаем все Small части, продаём в короткую
все Big части.

\justify
Портфель HML (High B/M Minus Low B/M): покупаем две Value части, продаём в 
короткую две Growth части.
\end{frame}



\begin{frame}{Факторный портфель MOM}
\justify
Эффект инерции: прошлые победители продолжают побежать. Сгруппируем акции в 
зависимости от размера и доходности за предшествующий год.

\justify
\centering
\begin{tabular}{l|c|c}
             & Мал. 50\%     & Бол. 50\% \\ \hline
30\% худших  & Small Losers  & Big Losers \\
40\% средних & Small Medium  & Big Medium \\
30\% лучших  & Small Winners & Big Winners
\end{tabular}

\justify
Факторный портфель MOM (\en{MOMentum}), он же WML (Winners Minus Losers) или
UMD (Up Minus Down): покупаем обе части Winners, продаём в короткую обе части
Losers.
\end{frame}



    \newcommand{\addFactorPlot}[3]{
        \addplot[
            color=#2,
            mark=#3,
            line width=1pt,
            mark repeat=120,
            mark phase=36,
            mark options={scale=2}
        ]
        table [
            x = date,
            y = #1,
            col sep=comma
        ]
        {data/fama_french_cumulative_growth_data.csv};
    }
    
\begin{frame}{Доходности факторных портфелей - 1}
\centering
    \begin{tikzpicture}
    \begin{axis}[
        width=\textwidth,
        height=\textheight - 1cm,
        date coordinates in=x,
        date ZERO=1926-06-30,
        xtick={1930-01-01,1940-01-01,1950-01-01,1960-01-01,1970-01-01,1980-01-01,1990-01-01,2000-01-01,2010-01-01,2020-01-01},
        minor xtick={1930-01-01,1950-01-01,1970-01-01,1990-01-01,2010-01-01},
        xticklabel=\year,
        grid=both,
        xmin=1926-12-31,
        xmax=2025-01-01,
        ymode=log,
        ymax=1000,
        log ticks with fixed point,
        ylabel={\small Рост \$1 начальных инвестиций},
        ylabel shift = -10pt,
        legend entries={
            MOM (MOMentum),
            HML (High B/M Minus Low B/M),
            SMB (Small Minus Big)
        },
        legend pos=north west,
        legend style={font=\scriptsize},
        legend cell align={left}
    ]


    \addFactorPlot{mom}{Set1-A}{none}

    \addFactorPlot{hml}{Set1-B}{none}
    
    \addFactorPlot{smb}{Set1-C}{none}

    \end{axis}
    \end{tikzpicture}
{\scriptsize Данные: Kenneth French Data Library.}
\end{frame}


    \renewcommand{\addFactorPlot}[3]{
        \addplot[
            color=#2,
            mark=#3,
            line width=1pt,
            mark repeat=120,
            mark phase=36,
            mark options={scale=2}
        ]
        table [
            x = date,
            y = #1,
            col sep=comma
        ]
        {data/fama_french_cumulative_growth_data_1990.csv};
    }

\begin{frame}{Доходности факторных портфелей - 2}
\centering
    \begin{tikzpicture}
    \begin{axis}[
        width=\textwidth,
        height=\textheight - 1cm,
        date coordinates in=x,
        date ZERO=1990-01-01,
        xtick={1990-01-01,1995-01-01,2000-01-01,2005-01-01,2010-01-01,2015-01-01,2020-01-01},
        %minor xtick={1930-01-01,1950-01-01,1970-01-01,1990-01-01,2010-01-01},
        xticklabel=\year,
        grid=both,
        xmin=1989-12-31,
        xmax=2023-01-01,
        ymode=log,
        ymax=10,
        log ticks with fixed point,
        ylabel={\small Рост \$1 начальных инвестиций},
        ylabel shift = -10pt,
        legend entries={
            MOM,
            HML,
            SMB
        },
        legend pos=north west,
        legend style={font=\scriptsize},
        legend cell align={left}
    ]
    \addFactorPlot{mom}{Set1-A}{none}
    \addFactorPlot{hml}{Set1-B}{none}
    \addFactorPlot{smb}{Set1-C}{none}

    \end{axis}
    \end{tikzpicture}
{\scriptsize Данные: Kenneth French Data Library.}
\end{frame}



\begin{frame}{Доходности факторных портфелей - 3}
\centering
\small
\begin{tabular}{l|l|r|r|r|c}
Фактор & Период & Сред.\,(ст.\,откл.) & $t$-тест & $p$-знач. & 99\%\,дов.\,инт.\\ \hline
SMB & 1927--59 &  3.4\% (13.6\%) & 1.43 & 16.4\% & [-3.1\%,  9.9\%] \\    
    & 1960--89 &  3.2\% (13.8\%) & 1.28 & 21.2\% & [-3.7\%, 10.2\%] \\
    & 1990--20 &  1.3\% ( 9.5\%) & 0.79 & 43.6\% & [-3.3\%,  6.0\%] \\    
    & 1960--20 &  2.3\% (11.8\%) & 1.51 & 13.7\% & [-1.7\%,  6.3\%] \\ 
    & 1927--20 &  2.7\% (12.4\%) & 2.08 &  4.0\% & [-0.7\%,  6.0\%] \\    
\hline
HML & 1927--59 &  5.0\% (14.1\%) & 2.05 &  4.8\% & [-1.7\%, 11.7\%] \\    
    & 1960--89 &  6.1\% (11.0\%) & 3.04 &  0.5\% & [ 0.6\%, 11.6\%] \\     
    & 1990--20 &  1.1\% (15.1\%) & 0.41 & 68.2\% & [-6.4\%,  8.6\%] \\    
    & 1960--20 &  3.6\% (13.4\%) & 2.08 &  4.2\% & [-1.0\%,  8.1\%] \\     
    & 1927--20 &  4.1\% (13.6\%) & 2.91 &  0.4\% & [ 0.4\%,  7.8\%] \\      
\hline
MOM & 1927--59 &  7.4\% (17.8\%) & 2.40 &  2.3\% & [-1.1\%, 15.9\%] \\    
    & 1960--89 & 10.5\% (12.8\%) & 4.49 & <0.1\% & [ 4.0\%, 16.9\%] \\ 
    & 1990--20 &  6.0\% (16.3\%) & 2.06 &  4.8\% & [-2.0\%, 14.1\%] \\  
    & 1960--20 &  8.2\% (14.7\%) & 4.35 & <0.1\% & [ 3.2\%, 13.2\%] \\  
    & 1927--20 &  7.9\% (15.8\%) & 4.87 & <0.1\% & [ 3.7\%, 12.2\%] \\  
\hline
\end{tabular}

{\scriptsize Годовые доходности 1927--2020. Данные: Kenneth French Data Library}
\end{frame}



\renewcommand{\addFactorPlot}[3]{
        \addplot[
            color=#2,
            mark=#3,
            line width=1pt,
            mark repeat=120,
            mark phase=36,
            mark options={scale=2}
        ]
        table [
            x = date,
            y = #1,
            col sep=comma
        ]
        {data/fama_french_international_cumulative_growth_data.csv};
    }

\begin{frame}{Факторы на рынке развитых стран (без США)}
\centering
    \begin{tikzpicture}
    \begin{axis}[
        width=\textwidth,
        height=\textheight - 1cm,
        date coordinates in=x,
        date ZERO=1991-01-01,
        xtick={1991-01-01,1995-01-01,2000-01-01,2005-01-01,2010-01-01,2015-01-01,2020-01-01},
        %minor xtick={1930-01-01,1950-01-01,1970-01-01,1990-01-01,2010-01-01},
        xticklabel=\year,
        grid=both,
        xmin=1990-12-31,
        xmax=2023-01-01,
        ymode=log,
        ymax=10,
        log ticks with fixed point,
        ylabel={\small Рост \$1 начальных инвестиций},
        ylabel shift = -10pt,
        legend entries={
            MOM,
            HML,
            MKT-RF,
            SMB
        },
        legend pos=north west,
        legend style={font=\scriptsize},
        legend cell align={left}
    ]
    \addFactorPlot{mom}{Set1-A}{none}
    \addFactorPlot{hml}{Set1-B}{none}
    \addFactorPlot{mkt_rf}{Set1-D}{none}
    \addFactorPlot{smb}{Set1-C}{none}

    \end{axis}
    \end{tikzpicture}
{\scriptsize Данные: Kenneth French Data Library.}
\end{frame}



\begin{frame}{Поведенческое объяснение}
\justify
Почему простой фильтр по финансовым показателям позволяет выбрать акции, 
которые заработают дополнительную доходность?

\justify
Поведенческая интерпретация: люди --- не роботы.
\begin{itemize}
\justifying
\item Инвесторы обращают внимание на громкие имена больших компаний.
\item Инвесторы слишком оптимистично оценивают перспективы <<гламурных>>
компаний роста, экстраполируя рост прибыли слишком далеко в будущее.
\item Инвесторы любят акции-победители и продолжают их покупать. Они разгоняют
цены до тех пор, пока краткосрочная инерция не сменяется возвратом к среднему.
\end{itemize}

\end{frame}



\begin{frame}{Рациональное объяснение - 1}
\justify
Маленькие акции и акции стоимости несут дополнительный риск. Их более высокая
доходность --- премия за риск.

\justify
\centering
\begin{tabular}{l|r|r|r|r}
Период рецессии    & MKT-RF  & SMB & HML & MOM \\ \hline
07.1990 -- 03.1991 & $1.9\%$ & $\alert{-3.4\%}$ & \alert{$-5.9\%$} & 6.2\% \\
03.2000 -- 11.2000 & $\alert{-15.7\%}$ & $\alert{-28.2\%}$ & 51.5\% & $\alert{-9.0\%}$ \\
12.2007 -- 06.2009 & $\alert{-35.4\%}$ & 10.7\% & \alert{$-11.1\%$} & $\alert{-30.9\%}$ \\
02.2021 -- 04.2021 & $\alert{-9.6\%}$ & $\alert{-1.5\%}$ & $\alert{-18.3\%}$ & 2.0\%
\end{tabular}
\centering
{\scriptsize Данные: Kenneth French Data Library.}

\justify
Инвесторы требуют дополнительную премию за те риски, которые реализуются в
<<плохие времена>>. Если маленькие акции и акции стоимости сильнее обваливаются 
тогда, когда среднему инвестору и так плохо, то они буду зарабатывать
дополнительную премию за риск.
\end{frame}



\begin{frame}{Рациональное объяснение - 2}
\justify
Маленькие компании (S) менее финансово устойчивы, и в кризис их вероятность
разориться выше, чем у больших компаний.

\justify
Компании стоимости (H) имеют больше физического капитала (станки, здания), 
который трудно переориентировать или продать. В худшем случае технологическая 
инновация может обесценить все материальные активы.

\justify
Инвесторы экономят на налогах. Они продают подешевевшие акции, чтобы снизить 
налоговую базу. Это снижает цену проигравших акций ещё сильнее.

\justify
В любом случае каждой акцией кто-то владеет. Мы не можем все вместе 
инвестировать только в маленькие акции стоимости, которые недавно показали
хороший результат.
\end{frame}



\begin{frame}{Модель Фамы-Френча-Кархарта}
\justify
Вернёмся к анализу успеха управляющих и дополним регрессию новыми факторами.
\begin{align*}
R_{fund,t} - R_{free,t} &= \alpha + \beta_{mkt}(R_{mkt,t} - R_{free,t}) + \\
&+ \beta_{smb}R_{smb,t} + \beta_{hml}R_{hml,t} + \beta_{mom}R_{mom,t}
 + \varepsilon_t
\end{align*}

\justify
Если регрессия покажет, что $\beta_{smb} > 0$, то это означает, что управляющий 
сделал ставку на эффект размера и заработал (или не заработал) премию за риск.

\justify
Fama, French (2010): средняя четырёхфакторная <<альфа>> паевых фондов акций на 
рынке США $-1.1\%$ в год с учётом комиссий.
\end{frame}



\begin{frame}{Факторы риска на рынке России - 1}
\justify
На сайте ИПЭИ РАНХиГС ({\small \url{https://ipei.ranepa.ru/ru/capm-ru}}) 
собраны временные ряды факторных портфелей, составленных из российский акций.
\begin{itemize}
\justifying
\item MKT-RF: рынок минус безрисковая ставка.
\item SMB: как у Фамы-Френча-Кархарта.
\item HML: как у Фамы-Френча-Кархарта.
\item MOM: как у Фамы-Френча-Кархарта.
\item LIQ (LIQuidity): неликвидные акции минус ликвидные.
\item PE (Price/Earnings): акции с низким отношением P/E (цена к прибыли) минус 
акции с высоким отношением P/E.
\item DP (Dividend/Price): акции с высокой дивидендной доходностью минус акции 
с низкой дивидендной доходностью.
\item SOE (State-Owned Enterprise): частные компании минус компании с гос. 
участием.
\end{itemize}
\end{frame}



\begin{frame}{Факторы риска на рынке России - 2}
\center
\begin{tikzpicture}
\begin{axis}[
  width=\textwidth,
  height=\textheight - 1cm,
  date coordinates in=x,
  date ZERO=2004-12-31,
  xtick={2005-01-01,2008-01-01,2011-01-01,2014-01-01,2017-01-01,2020-01-01},
  xticklabel=\year,
  xmin=2005-01-01,
  xmax=2022-01-01,
  ymode=log,
  log ticks with fixed point,
  extra y ticks = {0.5, 2, 3, 4, 5},
  extra y tick labels = {0.5, 2, 3, 4, 5},
  grid=both,
  ylabel={\small{Рост 1 р. начальных инвестиций}},
  xlabel near ticks,
  ylabel near ticks,
  legend entries = {
      MKT-RF,
      SMB,
      HML,
      MOM,
      SOE
  },
  legend pos=north west,
  legend style={font=\tiny, row sep=0pt},
  legend cell align={left}
]
\addplot[color=Set1-A, mark=none, thick] table[x=month, y=rmrf_ru_fixed, col sep=comma]{data/ru_factors_cumulative_growth_data.csv};
\addplot[color=Set1-B, mark=none, thick] table[x=month, y=smb_ru, col sep=comma]{data/ru_factors_cumulative_growth_data.csv};
\addplot[color=Set1-C, mark=none, thick] table[x=month, y=hml_ru, col sep=comma]{data/ru_factors_cumulative_growth_data.csv};
\addplot[color=Set1-D, mark=none, thick] table[x=month, y=mom_ru, col sep=comma]{data/ru_factors_cumulative_growth_data.csv};
\addplot[color=Set1-E, mark=none, thick] table[x=month, y=soe_ru, col sep=comma]{data/ru_factors_cumulative_growth_data.csv};
\end{axis}
\end{tikzpicture}

{\scriptsize Данные: МосБиржа и ЦБ РФ (MKT-RF), ИПЭИ РАНХиГС (остальные факторы).}
\end{frame}



\begin{frame}{Факторы риска на рынке России - 3}
\center
\begin{tikzpicture}
\begin{axis}[
  width=\textwidth,
  height=\textheight - 1cm,
  date coordinates in=x,
  date ZERO=2004-12-31,
  xtick={2015-01-01,2016-01-01,2017-01-01,2018-01-01,2019-01-01,2020-01-01},
  xticklabel=\year,
  xmin=2015-01-01,
  xmax=2021-02-01,
  %ymode=log,
  grid=major,
  %log ticks with fixed point,
  %extra y ticks = {0.5, 2, 3, 4, 5},
  %extra y tick labels = {0.5, 2, 3, 4, 5},
  ylabel={\small{Рост 1 р. начальных инвестиций}},
  xlabel near ticks,
  ylabel near ticks,
  legend entries = {
      MKT-RF,
      SMB,
      HML,
      MOM,
      SOE
  },
  legend pos=north west,
  legend style={font=\tiny},
  legend cell align={left}
]
\addplot[color=Set1-A, mark=none, thick] table[x=month, y=rmrf_ru_fixed, col sep=comma]{data/ru_factors_cumulative_growth_data_2015.csv};
\addplot[color=Set1-B, mark=none, thick] table[x=month, y=smb_ru, col sep=comma]{data/ru_factors_cumulative_growth_data_2015.csv};
\addplot[color=Set1-C, mark=none, thick] table[x=month, y=hml_ru, col sep=comma]{data/ru_factors_cumulative_growth_data_2015.csv};
\addplot[color=Set1-D, mark=none, thick] table[x=month, y=mom_ru, col sep=comma]{data/ru_factors_cumulative_growth_data_2015.csv};
\addplot[color=Set1-E, mark=none, thick] table[x=month, y=soe_ru, col sep=comma]{data/ru_factors_cumulative_growth_data_2015.csv};
\end{axis}
\end{tikzpicture}

{\scriptsize Данные: ИПЭИ РАНХиГС.}
\end{frame}



\begin{frame}{Факторы риска на рынке России - 4}
\centering
\begin{tabular}{l|r|r|r|r|c}
Фактор & Сред. & Ст.\,откл. & $t$-тест & $p$-знач. & 99\%\,дов.\,инт. \\
\hline
MKT-RF  &  0.72\% & 7.0\% &  1.43 & 15.5\% & [-0.6\%, 2.0\%] \\
SMB     &  1.29\% & 4.9\% &  3.62 & <0.1\% & [ 0.4\%, 2.2\%] \\ 
SOE     &  0.64\% & 3.7\% &  2.25 &  2.6\% & [-0.1\%, 1.4\%] \\
HML     &  0.44\% & 6.8\% &  0.90 & 36.9\% & [-0.8\%, 1.7\%] \\
MOM     &  0.38\% & 6.4\% &  0.82 & 41.2\% & [-0.8\%, 1.6\%] \\  
PE      &  0.23\% & 4.2\% &  0.75 & 45.3\% & [-0.6\%, 1.0\%] \\
DP      &  0.10\% & 5.6\% &  0.25 & 80.0\% & [-0.9\%, 1.2\%] \\
LIQ     & -0.07\% & 4.4\% & -0.22 & 82.3\% & [-0.9\%, 0.8\%] \\ \hline
\end{tabular}

\justify
{\scriptsize Месячные доходности 2005--2020. SOE: 2007--2020. Данные: МосБиржа 
и ЦБ РФ (MKT-RF), ИПЭИ РАНХиГС (остальные факторы).}

\justify
Статистически значимый фактор --- размер (SMB). С натяжкой ---  частные 
компании против государственных (SOE). Остальные факторы, похоже, не дают 
статистически значимой дополнительной доходности.
\end{frame}



\begin{frame}{Факторный анализ доходностей}
\justify
Вернёмся к вопросу положительной <<альфы>> лектора. Что будет, если мы добавим 
в регрессию дополнительные факторы? Сравним две регрессии:
\begin{align*}
R_{fund,t} - R_{free,t} &=
\alpha + \beta_{mkt}(R_{mkt,t} - R_{free,t}) + \varepsilon_t
\\
R_{fund,t} - R_{free,t} &=
\alpha + \beta_{mkt}(R_{mkt,t} - R_{free,t}) + \beta_{soe}R_{soe,t} + 
\varepsilon_t
\end{align*}

\centering
\begin{tabular}{l|l|l}
Параметр            & Регрессия 1                   & Регрессия 2 \\ \hline
$\hat{\alpha}$      & 0.07\% {\scriptsize (0.27\%)} & $-0.04\%$
{\scriptsize (0.27\%)} \\
$\hat{\beta}_{mkt}$ & 0.61 {\scriptsize (0.06)}     & 0.68
{\scriptsize (0.07)} \\
$\hat{\beta}_{soe}$ & ---                           & 0.18
{\scriptsize (0.09)} \\
$R^2$               & 0.61                          & 0.63 \\ \hline
\end{tabular}

{\scriptsize (62 месячных наблюдения 11.2015--12.2020)}

\justify
<<Альфа>> исчезла. Лектор просто не любит госкомпании. Остальные 
факторы не улучшают регрессию.
\end{frame}



\begin{frame}{Arbitrage Pricing Theory - 1}
\justify
На рынке всегда есть арбитражёры (\en{arbitrageurs}) --- участник, которые 
пытаются найти неправильные цены и заработать на этом почти без риска. Когда
они покупают недооценённые активы и продают переоценённые, они помогают
невидимой руке рынка сделать цены <<правильными>>.

\justify
Предположим, что есть $k$ торгуемых факторов систематического риска с 
доходностями $\mathcal{R}_i$. Тогда ожидаемая избыточная доходность любого 
актива должна быть равна
\begin{align*}
\mathbb{E}\mathcal{R}_{asset} - R_{free} =
\beta_1\mathbb{E}\mathcal{R}_1 + ... + \beta_k\mathbb{E}\mathcal{R}_k
\end{align*}

\justify
Пусть это не так, и есть актив с положительной $\alpha$ в регрессии
\begin{align*}
R_{asset,t} - R_{free,t} =
\alpha + \beta_1R_{1,t} + ... + \beta_kR_{k,t} 
+ \varepsilon_t,\quad
\varepsilon_t \sim \mathcal{N}(0, \sigma_{\varepsilon}^2)
\end{align*}
\end{frame}



\begin{frame}{Arbitrage Pricing Theory - 2}
\justify
\justify
Инвестор, который стремится совершить арбитраж (заработать деньги из воздуха), 
купит актив и продаст все факторы в соответствии с весами $\beta_i$:
\begin{align*}
\mathcal{R}_{asset} - \beta_1 \mathcal{R}_{1} - ... - \beta_k \mathcal{R}_k = 
R_{free} + \alpha + \varepsilon
\end{align*}

\justify
Правая часть -- случайная величина со средним $R_{free} + \alpha$ и стандартным 
отклонением $\sigma_{\varepsilon}$. Таким образом, отношение Шарпа равно $
\alpha / \sigma_{\varepsilon}$.

\justify
Если регрессия, из которой мы вычислили $\alpha$ и $\beta_i$, хорошо объясняет 
доходность актива (имеет высокий $R^2$), то стандартное отклонение ошибки 
регрессии $\sigma_\varepsilon$ будет достаточно малым.

\justify
Чем лучше факторы объясняют актив, тем выше отношение Шарпа, которое заработает 
арбитражёр, и тем быстрее арбитражёры исправят ошибку рынка.
\end{frame}



\begin{frame}{Другие факторы риска}
\justify
Факторный анализ позволяет понять, на какой риск делает ставку ваш управляющий 
(или вы сами). <<Нет альфы, есть только скрытые беты>>.

\justify
Примеры объясняющих факторов, не связанных с акциями:
\begin{itemize}
\justifying
\item Форма кривой процентных ставок (доходность 10Y облигаций минус доходность 
1Y облигаций).
\item Кредитный спред (доходность облигаций с рейтингом BBB минус доходность 
Treasury).
\item Валютный кэрри-трейд (валюты с высокими процентными ставками минус валюты 
с низкими процентными ставками).
\item Проданные опционы (доходность стрэддла или пут-опциона на индекс).
\end{itemize}
\end{frame}



\begin{frame}{Облигации и фиксированный доход}
\justify
Облигации называют инструментами с фиксированным доходом (\en{fixed 
income}). Если вы держали облигацию до погашения, и с эмитентом ничего
не случилось, то ваш доход фиксирован. Однако траектория может вам не 
понравится.

\centering
\begin{tikzpicture}
\begin{axis}[	
	width = \textwidth,
	height = \textheight - 3cm,
	ymin = 0.8, ymax = 1,
	xmin = 0, xmax = 10,
	xlabel = {\small Время (годы)},
	ylabel = {\small Цена (\% номинала)},
	grid = major,
	yticklabel={\pgfmathparse{100*\tick}\pgfmathprintnumber{\pgfmathresult}},
]

	\addplot[Set1-B, line width = 1pt]
	table[x=t, y=price, col sep = comma]
	{data/bond_price_random_path.csv};
\end{axis}
\end{tikzpicture}
\end{frame}



\begin{frame}{Доходность облигаций - 1}
\justify
Доходность облигации к погашению (\en{yield to maturity, YTM}) --- такая ставка 
дисконтирования, при которой сумма всех предстоящих платежей по облигации
равна текущей рыночной цене.

\justify
Пусть облигация выплачивает купоны $C_i$ в моменты времени $T_i$ и также 
выплачивает номинал $N$ в момент $T_n$. Текущая рыночная цена $P$. Доходность
$y$ --- это решение уравнения
\begin{align*}
P = \frac{C_1}{(1+y)^{T_1}} + \frac{C_2}{(1+y)^{T_2}}
+ .. + \frac{C_{n-1}}{(1+y)^{T_{n-1}}} + \frac{C_n + N}{(1+y)^{T_n}}
\end{align*}

\justify
Выше доходность --- ниже цена!

\justify
Например, доходность облигации 5\%. Если эмитент не разорится, то покупка такой 
облигации будет примерно равна депозиту под 5\% годовых.
\end{frame}



\begin{frame}{Доходность облигаций - 2}
\justify
Доходность облигации можно разбить на несколько слагаемых:
\begin{itemize}
\item Безрисковая реальная ставка.
\item Ожидаемая инфляция за срок жизни облигации.
\item Математическое ожидание потерь при дефолте.
\item Премия за риск долгосрочной инфляции.
\item Премия за риск дефолта (кредитный риск).
\end{itemize}

\justify
Движение рыночных процентных ставок --- главный фактор риска, который влияет
на все облигации.
\end{frame}



\begin{frame}{Премия за срок}
\justify
Премия за срок (\en{term premium}\ или \en{bond risk premium}) вознаграждает
инвесторов за риск владения <<длинными>> облигациями. Длинные облигации
сильнее реагируют на изменение процентных ставок в экономике. Их владельцы
сильнее страдают от неожиданной инфляции.

\justify
\centering
\begin{tabular}{r|r|r|r|r|r|r}
<1Y    & 1Y--2Y & 2Y--3Y & 3Y--4Y & 4Y--5Y & 5Y--10Y & >10Y \\ \hline
0.59\% & 1.05\% & 1.39\% & 1.60\% & 1.61\% & 1.91\% & 2.97\%
\end{tabular}

\justify
\centering
{\scriptsize Избыточные доходности гос. облигаций США, 1952--2011. Источник:
\en{Ang (2014)}.}
\end{frame}



\begin{frame}{Премия за кредитный риск - 1}
\justify
Премия за кредитный риск (\en{credit premium}) компенсирует инвесторам риск
того, что иногда случаются дефолты по облигациям. Рискованные облигации должны
давать доходность, которая не только компенсирует мат. ожидание потерь от
дефолтов, но и даёт премию за риск.

\justify
\centering
\begin{tabular}{r|r|r|r|r}
AAA    & AA     & A      & BBB    & CCC \\ \hline
0.32\% & 0.42\% & 0.43\% & 1.04\% & 0.86\%  
\end{tabular}

\justify
\centering
{\scriptsize Избыточная доходность корпоративных облигаций сверх гос. 
облигаций, 1987--2011. Данные: \en{Ang (2014)}.}

\justify
Следствие. Если вам предлагают еврооблигации ООО <<Рога и копыта>> с 
доходностью 10\% в долларах, то помните, что лишь 1\% --- это премия за риск,
которую вы заработаете в мат. ожидании, а 9\% компенсируют ожидаемые
потери при дефолте.
\end{frame}



\begin{frame}{Премия за кредитный риск - 2}
\centering
\begin{tikzpicture}
	\begin{axis}[
		width = \textwidth,
		height = \textheight - 1cm,
		date coordinates in = x,
		xticklabel = {\year},
		xtick = {1990-01-01, 1995-01-01, 2000-01-01, 2005-01-01, 2010-01-01, 2015-01-01, 2020-01-01},
		ylabel = {\small Рост \$1 начальных инвестиций},
		ylabel near ticks,
		ymode = log,
		log ticks with fixed point,
  		extra y ticks = {0.5, 2, 3, 4, 5},
 		extra y tick labels = {0.5, 2, 3, 4, 5},
		xmin = 1986-10-31,
		xmax = 2021-12-31,
		ymin = 0.9,
		ymax = 16,
		grid = both,
		legend entries = {
			BofA High Yield Index,
			BofA Corporate Index
		},
		legend pos=north west,
      legend style={font=\small},
      legend cell align={left}
	]
    
    \addplot[color = Set1-A, thick, mark = *, mark phase = 9, mark repeat = 30]  
    table[x = month, y = high_yield_growth, col sep = comma]
    {../derivatives/bofa_bond_indices.csv};
		
    \addplot[color = Set1-B, thick, mark = square, mark phase = 9,
             mark repeat = 30]
    table[x = month, y = corp_growth, col sep = comma]
    {../derivatives/bofa_bond_indices.csv};
	\end{axis}
\end{tikzpicture}

\centering
\scriptsize Данные: Bank of America, St Louis Fed.
\end{frame}



\begin{frame}{Облигации для частных инвесторов}
\justify
Не инвестируйте в длинные облигации, если только у вас нет долгосрочных 
обязательств, зафиксированных в номинальном выражении. Оставьте эту премию
за риск институциональным инвесторам.

\justify
Не инвестируйте в <<мусорные>> облигации с низким кредитным рейтингом, если
только вы не рискованный инвестор, который уже на 100\% в акциях, и которому
всё равно не хватает острых ощущений.

\justify
Если вы хотите заработать премию за кредитный риск, то не инвестируйте в
одну-две-три облигации. Обязательно составьте диверсифицированный портфель. 
Помните, что при дефолте по облигации вы вполне можете потерять 100\% 
вложенного в неё капитала.
\end{frame}



\begin{frame}{Ребалансировка портфеля}
\justify
Допустим, вы выбрали соотношение акции/облигации 60/40. Случился кризис, акции 
упали в два раза и пропорция стала 45/55. Что делать: продать оставшиеся акции, 
оставить как есть, докупить ещё акций?

\justify
Исторически, ребалансировка к постоянному соотношению (продать часть 
подорожавших облигаций и купить подешевевшие акции) давала лучшее соотношение 
риска и доходности.

\justify
Почему? Все участники рынка не могут ребалансироваться 
одновременно (на каждого продавца есть свой покупатель).
\end{frame}



\begin{frame}{Почему ребалансировка работает?}
\justify
Поведенческое объяснение: сложно в разгар кризиса купить обвалившийся актив. 

Рациональное объяснение: ребалансировка неявно продаёт волатильность. Вы 
действуете так же, как если бы динамически реплицировали проданный стрэддл. 
Продажа волатильности --- в среднем прибыльная стратегия, потому что инвесторы 
готовы платить за опцион-страховку.

\justify
Замечания о ребалансировке из Ang (2014):
\begin{itemize}
\justifying
\item Ребалансируйте редко (раз в год, максимум раз в квартал).
\item Докупайте широкие международные индексы, а не отдельные акции или индексы 
отдельных стран. Пример: Российская империя, Китай и т.д.
\item Не забывайте о налогах.
\end{itemize}
\end{frame}



\begin{frame}{Индексное инвестирование - 1}
\justify
Активное управление портфелем, выбор акций --- игра с нулевой суммой. Чтобы 
кто-то имел положительную <<альфу>> относительно индекса, кто-то должен иметь 
отрицательную <<альфу>>. Немногие управляющие бьют индекс, а успех в прошлом 
чаще всего не означает успех в будущем.

\justify
Возможно, средний инвестор получит лучший результат (более высокую доходность 
при том же уровне риска), если купит все акции из индекса и будет зарабатывать 
рыночную премию за риск. Однако, купить каждую из 500 акций индекса 
самостоятельно --- довольно проблематично.

\justify
Индексный паевой фонд (index mutual fund) принимает деньги инвесторов и 
покупает на них акции, составляющие индекс, в той же пропорции, что и в 
индексе.
\end{frame}



\begin{frame}{Индексное инвестирование - 2}
\justify
Задача индексного фонда --- дать возможность инвестору купить <<бету>> с 
минимальными накладными расходами. Меньше расходы на управление (не нужна армия 
аналитиков) --- лучше результат инвестора.

\justify
Задача инвестора --- не искать управляющего, который обеспечит <<альфу>>, и не 
пытаться создать <<альфу>> самостоятельно, а выбрать <<бету>> (или <<беты>>) и 
смириться со средней рыночной премией за риск.

\justify
Типичная комиссия за управление индексным фондом в США (не в России) --- 
единицы или десятки базисных пунктов (1 б.п. = 0.01\%) в год. Например, самый 
крупный фонд на индекс S\&P\,500 от Vanguard стоит 0.03\% (3 сотых процента) в 
год.
\end{frame}



\begin{frame}{Биржевые фонды}
\justify
Биржевые фонды (exchange traded funds) --- паевые фонды, паи (акции) которых 
торгуются на бирже. Задача ETF такая же, как у обычного индексного фонда --- 
дать инвестору возможность купить индекс с минимальными расходами.

\justify
Цена на пай ETF определяется спросом и предложением на рынке. Гипотетически, на 
идеальном рынке цена (market price) пая ETF всегда в точности равна цене акций, 
которыми владеет ETF (net asset value, NAV).

\justify
Чтобы помочь невидимой руке рынка уравнять цену пая с ценой акций из индекса, 
фонд назначает авторизованных участников (authorized participants). Эти 
участники имеют право получать и гасить паи фонда в обмен на корзины акций из 
индекса.
\end{frame}



\begin{frame}{Роль авторизованных участников - 1}
\justify
Ситуация 1: пай фонда стоит дороже корзины акций. Авторизованный участник 
покупает акции с рынка, меняет их на пай фонда, продаёт пай на рынке.

\vspace{\baselineskip}
\centering
\begin{tikzpicture}
\draw (0, 1.2) node{Ромашка ETF};
\draw[rounded corners] (-1.3, 1.6) rectangle (1.3, -1);
\draw (0, 0.6) node[rectangle,rounded corners,draw,minimum width=2.2cm]{GOOG};
\draw (0, 0) node[rectangle,rounded corners,draw,minimum width=2.2cm]{DBK.DE};
\draw (0, -0.6) node[rectangle,rounded corners,draw,minimum width=2.2cm]{AAPL};

\draw (4.1, 0.3) node[rectangle,rounded corners,draw,minimum width=2.2cm,minimum height=2.6cm]{\begin{tabular}{c}Автори-\\зованный\\участник\end{tabular}};

\draw[->,>=triangle 90] (1.3, 1.2) -- (3, 1.2) node[pos=0.5,anchor=south]{пай};

\draw[->,>=triangle 90] (3, -0.6) -- (1.3, -0.6) node[pos=0.5,anchor=north]{\scriptsize \begin{tabular}{c}
GOOG \\ DBK.DE \\ AAPL
\end{tabular}};

\draw[->,>=triangle 90] (5.2, 1.2) -- (6.9, 1.2) node[pos=0.5,anchor=south]{пай};

\draw[->,>=triangle 90] (6.9, -0.6) -- (5.2, -0.6) node[pos=0.5,anchor=north]{\scriptsize \begin{tabular}{c}
GOOG \\ DBK.DE \\ AAPL
\end{tabular}};

\draw (8, 0.3) node[rectangle,rounded corners,draw,minimum width=2.2cm,minimum height=2.6cm]{Рынок};

\end{tikzpicture}

\justify
Цена корзины акций растёт, цена пая фонда снижается.

\end{frame}



\begin{frame}{Роль авторизованных участников - 2}
\justify
Ситуация 2: пай фонда стоит дешевле корзины акций. Авторизованный участник 
покупает пай фонда на рынке, меняет его на корзину акций, продаёт акции на 
рынке.

\vspace{\baselineskip}
\centering
\begin{tikzpicture}
\draw (0, 1.2) node{Ромашка ETF};
\draw[rounded corners] (-1.3, 1.6) rectangle (1.3, -1);
\draw (0, 0.6) node[rectangle,rounded corners,draw,minimum width=2.2cm]{GOOG};
\draw (0, 0) node[rectangle,rounded corners,draw,minimum width=2.2cm]{DBK.DE};
\draw (0, -0.6) node[rectangle,rounded corners,draw,minimum width=2.2cm]{AAPL};

\draw (4.1, 0.3) node[rectangle,rounded corners,draw,minimum width=2.2cm,minimum height=2.6cm]{\begin{tabular}{c}Автори-\\зованный\\участник\end{tabular}};

\draw[<-,>=triangle 90] (1.3, 1.2) -- (3, 1.2) node[pos=0.5,anchor=south]{пай};

\draw[<-,>=triangle 90] (3, -0.6) -- (1.3, -0.6) node[pos=0.5,anchor=north]{\scriptsize \begin{tabular}{c}
GOOG \\ DBK.DE \\ AAPL
\end{tabular}};

\draw[<-,>=triangle 90] (5.2, 1.2) -- (6.9, 1.2) node[pos=0.5,anchor=south]{пай};

\draw[<-,>=triangle 90] (6.9, -0.6) -- (5.2, -0.6) node[pos=0.5,anchor=north]{\scriptsize \begin{tabular}{c}
GOOG \\ DBK.DE \\ AAPL
\end{tabular}};

\draw (8, 0.3) node[rectangle,rounded corners,draw,minimum width=2.2cm,minimum height=2.6cm]{Рынок};

\end{tikzpicture}

\justify
Цена корзины акций снижается, цена пая фонда растёт.

\end{frame}



\begin{frame}{Специфические риски биржевых фондов}
\justify
Механизм авторизованных участников может дать сбой в кризис. Лектор имеет 
радость владеть ETF-ом облигаций, который просел относительно net asset value 
на 6\% в марте 2020 г. Если бы в этот момент лектору срочно понадобились 
деньги, было бы очень обидно.

\justify
Чтобы заработать что-то большее, чем минимальная комиссия за управление, многие 
фонды одалживают ценные бумаги желающим продать их в короткую (margin lending). 
Гипотетически, возможен сценарий, когда фонду не вернут бумаги и он понесёт 
убытки.

\justify
Некоторые фонды (особенно фонды золота и биржевых товаров) являются 
<<синтетическими>>. Внутри --- не корзина товаров, а дериватив, total return 
swap с крупным банком. У инвестора фонда возникает кредитный риск на банк.
\end{frame}



\begin{frame}{Stock picking}
\justify
Индексное инвестирование --- это скучно. Почему бы не выбрать одну-две 
недооценённые акции (сыграть в т.н. stock picking) и купить следующий Google 
или Facebook?

\justify
Bessembinder (2018): 4\% лучших акций обеспечили весь рост рынка акций США 
сверх безрисковой ставки начиная с 1926 г. В среднем четыре акции из семи акций 
в течение жизни не обгоняют безрисковые облигации. Вы уверены, что сможете 
стабильно раз за разом выбирать одну <<правильную>> акцию из 25?

\justify
Но как же Уоррен Баффетт? Frazzini, Kabiller Pedersen (2018): <<альфа>> Баффета 
к CAPM положительная и статистически значимая. После добавления трёх других 
факторных портфелей статистическая значимость пропадает.
\end{frame}



\begin{frame}{Гипотеза эффективного рынка}
\justify
Гипотеза эффективного рынка (\en{efficient market hypothesis, EMH}) утверждает,
что рыночные цены отражают имеющуюся информацию.

\justify
Сильная (\en{strong}) форма: рыночные цены отражают всю имеющуюся публичную
и непубличную (инсайдерскую) информацию. Обогнать рынок невозможно в принципе.

\justify
Полусильная (\en{semi-strong}) форма: рыночные цены отражают всю имеющуюся 
публичную информацию. Нельзя обогнать рынок с помощью фундаментального анализа
(изучения публично доступной отчётности).

\justify
Слабая (\en{weak}) форма: рыночные цены отражают всю информацию, которую можно
извлечь из прошлых значений цен. Технический анализ не работает.
\end{frame}



\begin{frame}{Почему <<гипотеза>>?}
\justify
Почему гипотеза эффективного рынка --- именно гипотеза? Можно ли проверить
её статистическими методами?

\justify
Проблема совместного теста: мы будем одновременно проверять модель 
ценообразования активов и гипотезу эффективного рынка.

\justify
Чтобы проверить, что гипотеза выполняется, нужно убедиться, что цены активов
на рынке равны фундаментально обоснованным ценам. Но если бы мы знали 
фундаментально обоснованные цены активы, то нам не нужен бы был сам рынок.
\end{frame}



\begin{frame}{Эффективность рынка и космос}
\justify
28 января 1986 г. потерпел крушение космический шаттл <<Челленджер>>. Погибли
все 7 членов экипажа.

\justify
В то время на рынке были представлены акции четырёх крупных подрядчиков: 
\en{Lockheed}, \en{Martin Marietta}, \en{Morton Tiokhol}, \en{Rockwell}. Акции
\en{Morton Tiokhol}\ за день потеряли \$200 миллионов капитализации и потом
несколько месяцев не следовали за общим ростом рынка.

\justify
9 июня 1986 г. правительственная комиссия установила, что катастрофа произошла
из-за дефекта бокового твердотопливного ускорителя, который произвела 
\en{Morton Tiokhol}. Штрафы, компенсации и отменённые контракты обошлись 
компании в \$200 млн.
\end{frame}



\begin{frame}{Роль активного управления}
\justify
Могут ли все-все-все инвесторы инвестировать только в индекс? Вряд ли. Кто же 
будет оценивать отдельные акции относительно друг друга, покупать хорошие 
компании и продавать плохие?

\justify
На рынке должно быть достаточно неэффективностей (\en{inefficiencies}), чтобы 
компенсировать инвесторам затраты на поиск новой информации. Конкуренция среди
активных инвесторов приводит к тому, что в равновесии затраты на поиск 
информации будут равны прибыли, которую можно извлечь из неэффективности рынка.

\justify
Подробнее в Pedersen, Efficiently Inefficient (2015).
\end{frame}



\begin{frame}{Ставка против <<беты>> - 1}
\justify
Граница эффективности для инвесторов, которые не могут занимать под
безрисковую ставку (брать <<плечо>>).

\centering
\begin{tikzpicture}
\begin{axis}[
    width=\textwidth,
    height=\textheight-2.5cm,
    xlabel={\small Стандартное отклонение (волатильность), \%},
    ylabel={\small Ожидаемая доходность, \%},
    xlabel near ticks,
    ylabel near ticks,
    xmin=0, xmax=14,
    ymin=0, ymax=10
]

\addplot[line width=1pt, color=Set1-B, domain=0:4.83] {4 + x * (6.75 - 4) / 4.83};

\addplot[line width=1pt, color=Set1-B, domain=4.83:21, dashed] {4 + x * (6.75 - 4) / 4.83};

\drawPortfolioNodeFive{1.45}{4.83}{$0.7F + 0.3T$}{7.3}{20.9}{0.4}{1.4}{$70.0$}{north west}

\drawPortfolioNodeFive{7.25}{8.12}{$-0.5F + 1.5T$}{36.3}{104.2}{2.3}{7.2}{$-50.0$}{north west}

\drawAssetNode{0.0}{4.0}{F}{north west}

\drawPortfolioNode{4.83}{6.75}{Портфель T}{24.2}{69.5}{1.5}{4.8}{south east}

\addplot[line width=1pt, color=Set1-B, solid, restrict x to domain=4.83:21] table[x=std_dev, y=target_return, col sep=comma] {data/efficient_frontier_plot_data.csv};

\addplot[line width=1pt, color=Set1-B, dashed, restrict x to domain=0:4.83] table[x=std_dev, y=target_return, col sep=comma] {data/efficient_frontier_plot_data.csv};

\end{axis}
\end{tikzpicture}
\end{frame}



\begin{frame}{Ставка против <<беты>> - 2}
\justify
Многие инвесторы не могут или не хотят пользоваться <<плечом>> (торговать на
заёмные деньги).

\justify
Если таким инвесторам не достаточно доходности рыночного портфеля, то они 
вынуждены добавлять в портфель более рискованные активы (акции с высокой 
<<бетой>>) и игнорировать менее рискованные активы (акции с низкой <<бетой>>).

\justify
Повышенный спрос на акции с высокой <<бетой>> уменьшает их доходность, низкий 
спрос на акции с низкой <<бетой>> увеличивает их доходность.

\justify
Фонд, который может продавать в короткую акции с высокой <<бетой>> и покупать
с плечом акции низкой <<бетой>>, будет зарабатывать дополнительную  прибыль.
\end{frame}


\begin{frame}{Налоги - 1}
\justify
<<Неизбежны только смерть и налоги>> (Бенджамин Франклин)

\justify
Если вы --- налоговый резидент России и покупаете ценные бумаги на российских биржах через российского брокера:
\begin{itemize}
\justifying
\item НДФЛ 13\% (15\%) на дивиденды, купоны по облигациям, выплату номинала облигаций, прибыль от продажи ценных бумаг (цена продажи минус цена покупки).
\item Налоговая льгота на долгосрочное владение (ЛДВ) через 3 года.
\item Брокер --- налоговый агент (считает и удерживает налог за вас).
\item Индивидуальный инвестиционный счёт (ИИС) позволяет либо получить вычет на взносы до 400 тыс., либо не платить налог с прибыли.
\end{itemize}
\end{frame}



\begin{frame}{Налоги - 2}
\justify
Если вы --- налоговый резидент России и покупаете ценные бумаги через иностранного брокера, то не забудьте:
\begin{itemize}
\justifying
\item до 30 апреля подать налоговую декларацию,
\item до 31 мая подать справку о движении денег по счёту,
\item в течение месяца сообщить в налоговую о счетах, открытых после 01.01.2020. 
\end{itemize}

\justify
Ставка налога --- обычные 13\% (15\%). Можно зачесть налог на дивиденды, уплаченный за рубежом. Например, брокер в США удержал с дивидендов 10\% в пользу IRS. В России нужно будет доплатить 3\% (5\%).

\justify
При продаже бумаг налог нужно будет заплатить с рублёвой (не долларовой!) прибыли. Льготы на долгосрочное владение нет. 
\end{frame}



\begin{frame}{Налоги - 3}
\justify
Налог на наследство (estate tax) в США для нерезидентов: до 40\% на имущество от \$60 тыс., расположенное в США.

\justify
Акция американской компании, купленная на европейской бирже через российского брокера, подпадает под налог.

\justify
Фонд акций США, зарегистрированный в Ирландии, купленный на американской бирже через американского брокера, не подпадает под налог.

\justify
Как избежать estate tax:
\begin{itemize}
\justifying
\item Не умирать.
\item Не держать на счетах в США больше \$60 тыс. наличными.
\item Не владеть американскими ценными бумагами больше, чем на \$60 тыс. Покупать ETF'ы, зарегистрированные в Европе (например VWRL вместо VT).
\end{itemize}
\end{frame}



\begin{frame}{Выводы и рекомендации}
\begin{itemize}
\justifying
\item Диверсифицируйтесь! По классам активов, по валютам, по брокерам и банкам, по странам.
\item Всегда оценивайте, какой риск вы (или ваш управляющий) берёте на себя, вознаграждается ли этот риск рынком, и почему вы готовы его держать, а продавец актива --- нет.
\item Если вы не торгуете 8/5, то задумайтесь об индексном инвестировании через дешёвый ПИФ или ETF.
\item Выберите такой уровень риска (например, соотношение акции/облигации), при котором вы, с вашей личной чувствительностью к риску, сможете крепко спать по ночам.
\item Соизмеряйте горизонт инвестирования с риском. Акции хороши на длинном горизонте (15 лет и больше). На коротком лучше консервативные активы.
\item Не рассчитывайте разбогатеть с помощью инвестиций.
\end{itemize}
\end{frame}



\begin{frame}{Личный опыт}
\justify
Портфель <<Сам себе пенсионный фонд>>: Vanguard VT (FTSE Global All Cap), FinEx FXUS (MSCI USA), FXCN (MSCI China), FXDE (MSCI Germany). Автоматическое пополнение на 10\% от зарплаты и премий (когда они есть).

\justify
Портфель <<Когда нет индексного фонда>>: акции на Московской бирже. AFLT, AKRN, GCHE, GMKN, LKOH, MOEX, MTSS, TATNP, UPRO.

\justify
Портфель <<Облигации>>: инфляционные ОФЗ-ИН 52001.

\justify
Портфель <<Отток капитала>>: iShares IVV (S\&P\,500), iShares IGSB (1-5Y Corp Bonds), Xtrackers DX2X (STOXX 600), Xtrackers D5BG (EUR Corp Bonds).

\justify
Акции/облигации: 35/65. Рубли/доллары/евро: 33/33/33.
\end{frame}



\begin{frame}{Дальнейшее чтение}
На русском:
\begin{itemize}
\justifying
\item Горяев, Чумаченко. <<Финансовая грамота>> (2012).
\item Бернстайн. <<Манифест инвестора>> (2017).
\item Фарбер. <<Глобальное распределение активов>> (2021).
\item Статьи лектора на Хабре:

\url{https://habr.com/ru/users/abak/posts/}
\end{itemize}
На английском:
\begin{itemize}
\item Pedersen. Efficiently Inefficient (2015).
\item Ang. Asset Management (2014).
\item Welch. Investments (2009).
\item Bali, Engle, Murray. Empirical Asset Pricing (2016).
\item Cochrane. Asset Pricing: Revised Edition (2005).

\end{itemize}
\end{frame}


\insertdisclaimerframe

\end{document}


