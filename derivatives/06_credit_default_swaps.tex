\documentclass{beamer}

\usepackage{cmap}				% To be able to copy-paste russian text from pdf
\usepackage[T2A]{fontenc}
\usepackage[utf8]{inputenc}
\usepackage[russian]{babel}
\usepackage{textpos}
\usepackage{ragged2e}
\usepackage{amssymb}
\usepackage{ulem}
\usepackage{tikz}
\usepackage{pgfplots}
\usepackage{color}
\usepackage{cancel}
\usepackage{multirow}
\pgfplotsset{compat=1.17}
\usetikzlibrary{arrows,snakes,backgrounds,shapes}
\usepgfplotslibrary{groupplots,colorbrewer,dateplot,statistics}
\usepackage{animate}

\usepackage{amsfonts}
\usepackage{amsmath}
\usepackage{amssymb}
\usepackage{graphicx}
\usepackage{setspace}
\usepackage{tabularx}

\usepackage{enumitem}
	\setitemize{label=\usebeamerfont*{itemize item}%
	  \usebeamercolor[fg]{itemize item}
	  \usebeamertemplate{itemize item}}
	

\usepackage{eurosym}
\renewcommand{\EUR}[1]{\textup{\euro}#1}

\title{Кредитные дефолтные свопы}
\author{Артём Бакулин}
\date{8 апреля 2024 г.}

%\def\usebankstyle{1}

\ifdefined\usebankstyle

	% Remove navigation bar from the bottom-right corner
	\beamertemplatenavigationsymbolsempty
	
	% Use black dash rather than blue triangle for each \item
	\setbeamertemplate{itemize item}{\color{black}\textemdash}
	
	\definecolor{bank-light-blue-title}{RGB}{74, 201, 227}
	\definecolor{bank-light-blue}{RGB}{0, 163, 224}
	\definecolor{bank-dark-blue}{RGB}{0, 24, 168}

	% In new template slide title is black
	%\setbeamercolor{frametitle}{fg=bank-light-blue}
	\setbeamercolor{frametitle}{fg=black}
	
	\setbeamertemplate{frametitle}{
	    \begin{beamercolorbox}[ht=0.9cm]{frametitle}
	    \strut \insertframetitle \strut
	    \end{beamercolorbox}
	}
	
	\addtobeamertemplate{frametitle}{}{%
	\begin{textblock*}{50mm}(\textwidth-0.68cm,-0.68cm)
	\includegraphics[height=0.68cm,width=0.68cm]{../index-investing/img/1280px_bank_logo.png}
	\end{textblock*}}
	
	\setbeamertemplate{footline}{
	    \begin{beamercolorbox}[leftskip=1cm, rightskip=1cm]{footline}
	    		\vspace{0.2cm}
			\begin{minipage}[t]{\textwidth-2cm}
				\hrule
				\vspace{1pt}			
				{\color{bank-dark-blue} Deutsche Bank}
				\hfill
				\insertauthor, \insertdate
				\hfill
				\insertpagenumber
				\newline
				{\color{bank-light-blue} Technology Center}
			\end{minipage}
	    \end{beamercolorbox}
	}

	\newcommand{\inserttitleframe}{
		% Background image is defined inside these curly braces
		{
			\usebackgroundtemplate{\includegraphics[width=\paperwidth]{../index-investing/img/bank_background.jpg}}
			\begin{frame}[plain]
				
				\vspace{-0.9cm} \hspace{-0.6cm} {\tiny
					\begin{tabular}{l}
						 \color{white} Deutsche Bank \\
						 \color{bank-light-blue-title} Technology Center
					\end{tabular}
				}

				\vspace{3.4cm}
				
				\hspace{-0.4cm} {\footnotesize \color{white} Математические модели в инвестиционных банках}			
				\vspace{0.5cm}
				
				\hspace{-0.4cm} {\huge \color{white} \inserttitle}
				
				\vspace{0.2cm}
				
				\hspace{-0.4cm} {\scriptsize \color{black} \#PositiveImpact}
				
				\vspace{0.2cm}
				
				\hspace{-0.4cm} {\footnotesize \color{white} \insertauthor, \insertdate}
			\end{frame}
		}
	}
	
	\newcommand{\insertdisclaimerframe}{
		\begin{frame}{Disclaimer}
			\small
			\justify
			Данный материал не является предложением или предоставлением
			какой-либо услуги. Данный материал предназначен исключительно для
			информационных и иллюстративных целей и не предназначен для
			распространения в рекламных целях. Любой анализ третьих сторон не 		
			предполагает какого-либо одобрения или рекомендации. Мнения, 
			выраженные в	данном материале, являются актуальными на текущий момент,
			появляются только в этом материале и могут быть изменены без 
			предварительного уведомления. Эта информация предоставляется с 
			пониманием того, что в отношении материала, предоставленного здесь, вы
			будете принимать самостоятельное решение в отношении любых действий в
			связи с настоящим материалом, и это решение является основанным на 
			вашем	собственном суждении, и что вы способны понять и оценить 
			последствия этих действий. ООО <<ТехЦентр Дойче Банка>> не несет 
			никакой ответственности за любые убытки любого рода, относящихся к
			 этому материалу.
		\end{frame}
	}

\else
	\usetheme{Warsaw}
	\usecolortheme{beaver}

	\newcommand{\inserttitleframe}{
		\begin{frame}
		\titlepage
		\end{frame}
	}
	
	\newcommand{\insertdisclaimerframe}{
	}
	
	% remove navigation bar
	\setbeamertemplate{navigation symbols}{}

	% add page counter
	\setbeamertemplate{page number in head/foot}[totalframenumber] 

	% remove navigation bar
	\setbeamertemplate{navigation symbols}{}
	
	\setbeamertemplate{page number in head/foot}[totalframenumber] 

\fi


\newcommand{\ru}[1]{\begin{otherlanguage}{russian}#1\end{otherlanguage}}
\newcommand{\en}[1]{\begin{otherlanguage}{english}#1\end{otherlanguage}}
\newcommand{\ruen}[2]{#1 (\en{#2})}

% https://tex.stackexchange.com/questions/98003/filter-rows-from-a-table
\pgfplotsset{
    discard if not/.style 2 args={
        x filter/.code={
            \edef\tempa{\thisrow{#1}}
            \edef\tempb{#2}
            \ifx\tempa\tempb
            \else
                \def\pgfmathresult{inf}
            \fi
        }
    }
}



\begin{document}

\inserttitleframe

\begin{frame}{Кредитный дефолтный своп}
\justify
\alert{Кредитный дефолтный своп} (\en{credit default swap, CDS}) --- дериватив, который позволяет купить или продать страховку от дефолта по облигациям или другим долгам.

\justify
Важные параметры сделки:

\justify
1. Базовый заёмщик (\en{reference entity}). Например, ООО <<Рога и копыта>>. Эта компания не является стороной сделки (покупателем или продавцом).

\justify
2. Застрахованный номинал. Например, \$10\,000\,000.

\justify
3. Срок действия страховки. Например, начало сегодня, окончание через год.

\justify
4. Фиксированный купон (<<спред>>) --- процент от номинала, который покупатель платит продавцу. Например, 2\% годовых.

\justify
<<Купить>> CDS --- согласиться платить фиксированный купон в обмен на страховку.
\end{frame}



\newcommand{\swapPartyNode}[5]{

	\draw (#1, #2)
		node[
			rectangle,
			draw,
			rounded corners,
			anchor = south,
			minimum height = 0.8cm,
			minimum width = 2.5cm
		]
		{#5}
	--
	(#3, #4);
}

\newcommand{\swapBuyerPaymentEx}[7]{

	\draw [
		->,
		>=triangle 90
	] 
	(#1, #2)
	node[
		label = left:{#7}
	]{}
	-- (#3, #4)
	node[
		pos=0.5,
		anchor=south
	]
	{#5}
	node[
		pos=0.5,
		anchor=north
	]
	{#6};
}

\newcommand{\swapBuyerPayment}[6]{

	\swapBuyerPaymentEx{#1}{#2}{#3}{#4}{#5}{}{#6}
}

\begin{frame}{Кредитный дефолтный своп - 2}
\justify
Пример: CDS на 1 год, номинал \$10\,000\,000, купон $2.0\%$ годовых.

\justify
\centering
\begin{tikzpicture}[thick]

		\swapPartyNode{0}{0}{0}{-4}{Покупатель}
		
		\swapPartyNode{5.5}{0}{5.5}{-4}{Продавец}
		
		\swapBuyerPayment{0}{-1}{2.5}{-1}{\$50\,000}{3 мес.}
		\swapBuyerPayment{0}{-2}{2.5}{-2}{\$50\,000}{6 мес.}
		\swapBuyerPayment{0}{-3}{2.5}{-3}{\$50\,000}{9 мес.}
		\swapBuyerPayment{0}{-4}{2.5}{-4}{\$50\,000}{12 мес.}
\end{tikzpicture}
\end{frame}



\begin{frame}{Кредитный дефолтный своп - 3}
\justify
Референсный заёмщик объявил дефолт через 8 месяцев.

\justify
\centering
\begin{tikzpicture}[thick]
		\swapPartyNode{0}{0}{0}{-2.67}{Покупатель}
		
		\swapPartyNode{5.5}{0}{5.5}{-2.67}{Продавец}
		
		\swapBuyerPayment{0}{-1}{2.5}{-1}{\$50\,000}{3 мес.}
		\swapBuyerPayment{0}{-2}{2.5}{-2}{\$50\,000}{6 мес.}
		\swapBuyerPaymentEx{0}{-2.67}{2.5}{-2.67}{\$33\,000}{Облигация}{8 мес.}
		
		\swapBuyerPayment{5.5}{-2.67}{3}{-2.67}{\$10\,000\,000}{}
\end{tikzpicture}


%\justify
%Две разновидности CDS:
%\begin{itemize}
%\justifying
%\item Поставочный (облигация в обмен на номинал).
%\item Расчётный (продавец выплачивает разницу между номиналом и рыночной ценой облигации).
%\end{itemize}
\end{frame}



\begin{frame}{Поставочный и расчётный своп}
\justify
Предположим, что случился дефолт.

1. Застрахованный номинал \$10\,000\,000.

2. Рыночная цена облигаций после дефолта \$4\,000\,000.

\justify
\alert{Поставочный} (\en{deliverable}) кредитный своп:

1. Покупатель передаёт продавцу дефолтнувшие облигации.

2. Продавец выплачивает покупателю номинал застрахованных облигаций (\$10\,000\,000).

\justify
\alert{Расчётный} (\en{cash settled}) кредитный своп:

1. Продавец выплачивает разницу между застрахованным номиналом и остаточной ценой облигаций (\$6\,000\,000).

\end{frame}



\begin{frame}{Хеджирование кредитного риска}
\justify
Допустим, мы владеем корпоративной облигацией. Вышли негативные новости, и теперь мы опасаемся, что вероятность дефолта сильно выросла. Наши варианты:

1. Продать облигацию на рынке.

2. Купить страховку --- кредитный своп.

\justify
На сферическом ликвидном рынке в вакууме продать облигацию можно мгновенно и с нулевыми расходами. На практике поиск покупателя не самой ликвидной бумаги может занять недели.

\justify
Купить кредитный своп иногда проще и дешевле, чем продать облигацию. Когда (и если) рынок успокоится, кредитный своп можно будет продать.

\justify
Особенно удобно использовать CDS для страхования неторгуемых долгов, таких как банковские кредиты.
\end{frame}



\begin{frame}{Спекуляция на кредитном риске}

\justify
Допустим, мы верим, что кредитное качество референсного заёмщика будет ухудшаться. Наши варианты:

1. Продать в короткую (зашортить) рискованную облигацию и купить безрисковую государственную (чтобы убрать риск общего движения процентных ставок).

2. Купить кредитный дефолтный своп.

\justify
Нужно ли владеть облигацией, чтобы купить страховку на неё? Нет! Такая позиция называется \en{naked CDS*}.

\justify
Продавая кредитные свопы, небольшой банк может диверсифицировать свой кредитный 
портфель и <<выдать>> кредиты клиентам, с которыми он никогда не смог бы работать
напрямую.

\justify
* В Европе запрещены \en{naked CDS}\ на государственные облигации членов ЕС.
\end{frame}


\renewcommand{\swapPartyNode}[4]{

	\draw (#1, #2)
		node[
			rectangle,
			draw,
			rounded corners,
			anchor = south,
			minimum height = 0.8cm,
			minimum width = 2.8cm
		]
		(#4)
		{#3};
}

\newcommand{\paymentFlow}[4] {
	\draw [
		->,
		>=triangle 90
	]
	(#1) -- (#2)
	node[
		pos = 0.5,
		anchor = #4
	]
	{#3};
}

\begin{frame}{Спекуляция на кредитном риске - 2}
\justify
Если мы купили своп за 2\% у компании А, а потом смогли продать своп за 3\% компании B, то мы будем зарабатывать 1\% годовых с застрахованного номинала до тех пор, пока не случится дефолт.

\justify
\centering
\begin{tikzpicture}

	\swapPartyNode{0}{2}{Компания А}{A}
	\swapPartyNode{0}{0}{Спекулянт}{S}
	\swapPartyNode{0}{-2}{Компания B}{B}
	
	\paymentFlow{[xshift=-0.5cm] S.north}{[xshift=-0.5cm] A.south}{2\%}{east};
	\paymentFlow{[xshift=-0.5cm] B.north}{[xshift=-0.5cm] S.south}{3\%}{east};
	
	\paymentFlow{[xshift=+0.5cm] A.south}{[xshift=+0.5cm] S.north}{Страховка}{west};
	\paymentFlow{[xshift=+0.5cm] S.south}{[xshift=+0.5cm] B.north}{Страховка}{west};
	
	\swapPartyNode{6}{2}{Компания А}{A}
	\swapPartyNode{6}{0}{Спекулянт}{S}
	\swapPartyNode{6}{-2}{Компания B}{B}
	
	\paymentFlow{[xshift=-0.5cm] A.south}{[xshift=-0.5cm] S.north}{Номинал}{east};
	\paymentFlow{[xshift=-0.5cm] S.south}{[xshift=-0.5cm] B.north}{Номинал}{east};
	
	\paymentFlow{[xshift=+0.5cm] S.north}{[xshift=+0.5cm] A.south}{Облигация}{west};
	\paymentFlow{[xshift=+0.5cm] B.north}{[xshift=+0.5cm] S.south}{Облигация}{west};
\end{tikzpicture}
\end{frame}



\begin{frame}{Спекуляция на кредитном риске - 3}
\justify
Допустим, мы верим, что кредитное качество референсного заёмщика будет улучшаться. Наши варианты:

\justify
1. Купить рискованную облигацию и продать в короткую (зашортить) безрисковую государственную (чтобы убрать риск общего движения процентных ставок).

\justify
2. Продать кредитный дефолтный своп.

\justify
Если мы продали своп за 3\% компании А, а потом смогли купить своп за 2\% компании B, то мы будем зарабатывать 1\% годовых с застрахованного номинала до тех пор, пока не случится дефолт.
\end{frame}



\begin{frame}{Факторы, влияющие на цену CDS}
\justify
На честную цену CDS влияют следующие факторы:

\justify
1. Вероятность дефолта референсного заёмщика. Её можно вычислить либо из котировок других CDS, либо из рыночных цен облигаций этого заёмщика.

\justify
2. \en{Recovery rate}\ --- остаточная стоимость  облигации в случае дефолта. Обычно вычисляется на основе исторических данных о дефолтах заёмщиков в той же юрисдикции, в том же секторе экономики, с сопоставимым кредитным рейтингом.

\justify
3. Ставка дисконтирования будущих платежей.

\end{frame}



\begin{frame}{Recovery rate}
\justify
Moody's Corporate Default and Recovery Rates, 1920-2010.

\vspace{\baselineskip}
\begin{tabular}{l|l|l|l}
Position						& 2010		& 2009		& 1982-2010 \\
\hline
1st	Lien Bank Loan	 		& 72.30\%	& 56.30\% & 59.60\% \\
2nd Lien Bank Loan 		& 18.40\%	& 20.80\% & 27.90\% \\
Sr. Unsecured	Bank Loan	& n.a.		& 37.90\% & 39.90\% \\
Sr. Secured Bond			& 54.70\%	& 29.60\% & 49.10\%	 \\
\hline
Sr. Unsecured	Bond		& 63.80\%	& 35.50\% & 37.40\%	 \\
\hline
Sr. Subordinated Bond	& 39.40\%	& 18.00\% & 25.30\%	\\
Subordinated Bond		& 32.20\%	& 25.10\% & 24.20\%	\\
Jr. Subordinated Bond	 	& n.a.		& n.a.		&17.10\%	\\
\end{tabular}

\justify
Правило большого пальца: 40\%.
\end{frame}



\begin{frame}{Кредитный рейтинг}
\justify
Почему бы не посмотреть на кредитный рейтинг эмитента, чтобы оценить вероятность дефолта? Шкала \en{S\&P}\ (от лучшего к худшему):
\begin{align*}
\text{Инвестиционный рейтинг}
\small
&\begin{cases}
\small \text{AAA} \\
\small \text{AA+, AA, AA-} \\
\small \text{A+, A, A-} \\
\small \text{BBB+, BBB, BBB-,}
\end{cases} \\
\text{Спекулятивный рейтинг}
&\begin{cases}
\small \text{BB+, BB, BB-,} \\
\small \text{B+, B, B-,} \\
\small \text{CCC+, CCC, CCC-,} \\
\small \text{CC,C, D}
\end{cases}
\end{align*}
\end{frame}



\newcommand{\addDefaultRatePlot}[3] {

	\addplot[
		color = #2,
		mark = #3,
		thick
	]
	table[
		x = year,
		y = #1,
		col sep = comma
	]
	{sp_2020_global_corporate_default_rates.csv};
}

\begin{frame}{Кредитный рейтинг - 2}
\justify
Кредитные рейтинги отражают скорее относительную вероятность дефолта, чем абсолютную.

\justify
\centering
\begin{tikzpicture}
	\begin{axis}[
		width = \textwidth,
		height = \textheight - 2cm,
		ymin = 0,
		ymax = 13,
		xmin = 1982,
		xmax = 2023,
		grid = major,
		xticklabel = {\pgfmathprintnumber[precision=0, 1000 sep=]{\tick}},
		yticklabel = {\pgfmathprintnumber[precision=0]{\tick}\%},
		ylabel = {Частота дефолтов},
		legend entries = {
  	   		\small Спекулятивный,
      		\small Инвестиционный
  		},
  		legend cell align={left}
	]
		
		\addDefaultRatePlot{speculative_grade_default_rate}{Set1-A}{square}
		
		\addDefaultRatePlot{investment_grade_default_rate}{Set1-B}{*}
	\end{axis}
\end{tikzpicture}

\centering
\small Данные: \en{S\&P Global}
\end{frame}



\begin{frame}{Вероятность дефолта и цена облигации}
\justify
Государственная (безрисковая) бескупонная облигация с погашением через 1 год стоит $G=98\%$ номинала. Вложили \$980 сегодня --- получили \$1\,000 через год. Доходность инвестиций $\$1\,000 / \$980 - 1 \approx 2.04\%$.

\justify
Корпоративная облигация на 1 год стоит $C=95\%$ номинала. Вложили \$950 сегодня --- получили \$1\,000 через год, если не случится дефолта. Доходность $\$1\,000 / \$950 - 1 \approx 5.3\%$.

\justify
Почему участники рынка ценят корпоративную облигацию меньше (требуют более высокую доходность в хорошем случае)?

\justify
В случае дефолта recovery rate составит $R=40\%$. Вложили \$950 сегодня --- получили \$400 через год. Доходность составит $\$400 / \$950 - 1 = -57.9\%$.
\end{frame}



\begin{frame}{Вероятность дефолта и цена облигации - 2}
\justify
Государственная облигация стоит $G=98\%$ номинала, корпоративная стоит $C=95\%$, при дефолте recovery rate составит $R=40\%$.

\justify
%Предположим, что участники рынка риск-нейтральны.
При какой вероятности дефолта $p$ мат. ожидания доходностей государственной и корпоративной облигаций совпадут?

\centering
\begin{tabular}{l|c|c|c|c|c}
& & \multicolumn{2}{c|}{Дефолт ($p$)} & \multicolumn{2}{c}{Нет дефолта ($1-p$)} \\
Облиг. & Цена & Выплата & Дох-ть & Выплата & Дох-ть \\
\hline
Гос.  & 0.98 & 1     & 1/0.98 - 1 & 1 & 1/0.98 - 1\\
Корп. & 0.95 & 0.4 & 0.4/0.95 - 1 & 1 & 1/0.95 - 1
\end{tabular}

\justify
Математические ожидания доходностей должны совпадать:
\begin{align*}
\frac{1}{0.98} - 1 &= p\left(\frac{0.4}{0.95} - 1\right) + (1-p)\left(\frac{1}{0.95} - 1\right) \Rightarrow \\
p &= \frac{1 - C/G}{1-R} = \frac{1 - 0.95/0.98}{1-0.4} \approx 5.1\%
\end{align*}
\end{frame}



\begin{frame}{Вероятность дефолта и доходность облигации}
\justify
Формулу вероятности дефолта можно переписать в терминах доходностей безкупонных облигаций $r_g$ и $r_c$. Если $T$ --- количество лет до погашения, то
\begin{align*}
G = \frac{1}{(1 + r_g)^T}, C = \frac{1}{(1 + r_c)^T}
\end{align*}
Тогда
\begin{align*}
p = \frac{1 - \dfrac{C}{G}}{1-R} = \frac{1 - \dfrac{(1 + r_g)^T}{(1 + r_c)^T}}{1 - R}
\end{align*}
(В предположении, что при дефолте мы получим остаточную стоимость облигации в ту же дату, в которую должны были получить номинал)
\end{frame}



\begin{frame}{Вероятность дефолта и кредитный спред}
\justify
Если процентные ставки не очень высоки, то формулу можно упростить до:
\begin{equation*}
p = \frac{1 - \dfrac{(1 + r_g)^T}{(1 + r_c)^T}}{1 - R} \approx \frac{T(r_c - r_g)}{1 - R}
\end{equation*}

\justify
Например, доходность годовой облигации одного крупного немецкого инвестиционного банка из Франкфурта составляет 4.8\%. Доходность аналогичной облигации Германии составляет $3.4\%$. При банкротстве банка кредиторы могут рассчитывать на recovery rate 40\%.

\justify
Какова вероятность дефолта этого банка в течение года?
\begin{equation*}
p \approx \frac{r_{Bank} - r_{Germany}}{1-R} = \frac{4.8\% - 3.4\%}{1-40\%} = 2.33\%
\end{equation*}
\end{frame}



\begin{frame}{Интерполяция вероятности дефолта}
\centering
\begin{tabular}{l|r|r|r|r}
Срок & Гос. дох-ть & Корп. дох-ть & Recovery & Вер-ть дефолта \\ \hline
1Y & 1.00\% & 2.23\% & 40\% & 2.00\% \\
2Y & 1.50\% & 3.06\% & 40\% & 5.00\%
\end{tabular}

\justify
Какова вероятность того, что дефолт случится за $T^*=1.5$ года?

\justify
Пусть $\xi$ --- случайная величина, которая определяет время до дефолта в годах.

\justify
Обозначим $D(T)$ вероятность того, что дефолт произойдёт до времени $T$ (\en{D --- default}). $S(T)$ --- вероятность того, что референсный заёмщик проживёт $T$ лет (\en{S --- survival}).
\begin{align*}
D(T) &= \mathbb{P}(\xi < T) \\
S(T) &= \mathbb{P}(\xi \ge T) = 1 - D(T)
\end{align*}
\end{frame}
 


\begin{frame}{Напоминание: геометрическое распределение}
\justify
Представьте, что мы подбрасываем монетку. Вероятность выпадения решки равна $p$ (обычно примерно 50\%). Мы продолжаем подбрасывать монетку до тех пор, пока не выпадет решка (случится дефолт). 

\justify
Рассмотрим случайную величину $\xi$ --- количество орлов, которые успеют выпасть до первой решки (количество лет до дефолта).

\justify
С какой вероятностью первая решка выпадет позже, чем на $n$-ой попытке (то есть мы выбросим как минимум $n$ орлов подряд)?
\begin{align*}
\mathbb{P}(\xi \ge n) = S(n) = (1 - p)^{n}
\end{align*}
\justify
Говорят, что $\xi$ следует \alert{геометрическому} распределению.
\end{frame}



\begin{frame}{Напоминание: геометрическое распределение - 2}
\centering
\begin{tikzpicture}
\begin{axis}[
    width = \textwidth,
    height = \textheight - 1cm,
    xmin = 0, xmax = 10,
    ymin = 0, ymax = 1,
    xtick distance=1,
    ytick distance=0.1,
    grid=major
]

\addplot[color=Set1-A, thick, mark=*, domain=0:10, samples=11]{(1-0.5)^x)};
\end{axis}
\end{tikzpicture}
\end{frame}



\begin{frame}{Напоминание: экспоненциальное распределение}
\justify
Допустим, что дефолт может произойти когда угодно, а не только раз в год. Допустим, что в течение любого малого интервала времени $dt$ дефолт случается с вероятностью $\lambda \cdot dt$.

\justify
Параметр $\lambda$ называется \alert{интенсивностью дефолта} или  \en{hazard rate}. Обычно его измеряют в процентах годовых.

\justify
Пусть вероятность того, что референсный заёмщик проживёт хотя бы $T$ лет равна $S(T)$. Какова вероятность того, что он проживёт чуть-чуть дольше, чем $T$?
\begin{align*}
S(T+dt) = S(T)(1 - \lambda dt)
\end{align*}

\justify
Решим простенькое дифференциальное уравнение:
\begin{align*}
\frac{S(T+dt) - S(T)}{dt} = -\lambda S(T) \quad
\Rightarrow
\quad
S(T) = e^{-\lambda T}
\end{align*}
\justify
Это \alert{экспоненциальное} распределение, непрерывная версия геометрического распределения.
\end{frame}



\begin{frame}{Напоминание: экспоненциальное распределение - 2}
\centering
\begin{tikzpicture}
\begin{axis}[
    width = \textwidth,
    height = \textheight - 1cm,
    xmin = 0, xmax = 10,
    ymin = 0, ymax = 1,
    xtick distance=1,
    ytick distance=0.1,
    grid=major
]

\addplot[color=Set1-A, thick, domain=0:10]{exp(-x*ln(2))};
\end{axis}
\end{tikzpicture}
\end{frame}



\begin{frame}{Интерполяция вероятности дефолта - 2}
\centering
\begin{tabular}{l|r|r|r|r}
Срок & Гос. дох-ть & Корп. дох-ть & Recovery & Вер-ть дефолта \\ \hline
1Y & 1.00\% & 2.23\% & 40\% & 2.00\% \\
2Y & 1.50\% & 3.06\% & 40\% & 5.00\%
\end{tabular}

\justify
Какова вероятность того, что дефолт случится за $T^*=1.5$ года?

\justify
Предположим, что после того, как референсный заёмщик выживет в течение $T_1=1$ года, на него начинает действовать экспоненциальное распределение с \en{hazard rate}\ $\lambda$.

\justify
Формально, для любого времени $T^*$ между $T_1=1$ год и $T_2=2$ года справедливо, что
\begin{align*}
S(T^*) = S(T_1)e^{-\lambda(T^* - T_1)}
\end{align*}
\end{frame}



\newcommand{\nodeWithDropLines}[2]{
    \node[
        circle,
        fill,
        color=Set1-A,
        inner sep=2pt
    ]
    at (axis cs: #1, #2)
    {};

    \draw[
        dashed,
        thick
    ]
    (axis cs: 0, #2) -- (axis cs: #1, #2) -- (axis cs: #1, 0);
}

\begin{frame}{Интерполяция вероятности дефолта - 3}
\centering
\begin{tikzpicture}
\begin{axis}[
    width = \textwidth,
    height = \textheight - 1cm,
    xmin = 0, xmax = 3,
    ymin = 0, ymax = 1.2,
    xtick = {1, 2},
    xticklabels = {$T_1$, $T_2$},
    xtick pos = left,
    ytick = {0.404, 0.9, 1},
    yticklabels = {$S(T_2)$, $S(T_1)$, 1},
    ytick pos = left,
    axis lines = middle
]

\addplot[color=Set1-A, thick, domain=0:1]{exp(ln(0.9)*x)};
\addplot[color=Set1-A, thick, domain=1:2]{0.9 * exp(-0.8 * (x - 1))};
\addplot[color=Set1-A, thick, domain=2:3]{0.9*exp(-0.8)*exp(-2*(x-2))};

\nodeWithDropLines{1}{0.9}
\nodeWithDropLines{2}{0.404}

\node[anchor=west] at (axis cs: 1.5, 0.65) {$S(T_1)e^{-\lambda (T - T_1)}$};

\end{axis}
\end{tikzpicture}
\end{frame}




\begin{frame}{Интерполяция вероятности дефолта - 4}
\justify
Значение $\lambda$ можно вычислить из значений $D(T_1)$ и $D(T_2)$.
\begin{align*}
S(T_2) = S(T_1)e^{-\lambda(T_2-T_1)}
\end{align*}
Следовательно,
\begin{align*}
\lambda = \dfrac{\ln\left(\dfrac{S(T_1)}{S(T_2)}\right)}{T_2 - T_1} = \dfrac{\ln\left(\dfrac{1 - D(T_1)}{1 - D(T_2)}\right)}{T_2 - T_1}
\end{align*}
В нашем случае, $\lambda \approx 3.11\%$. Тогда
\begin{align*}
D(1.5) = 1 - S(1.5) = 1 - (1 - D(1))e^{-3.11\% \cdot (1.5 - 1)} \approx 3.51\%
\end{align*}
\end{frame}



\begin{frame}{Ценообразование CDS}
\justify
Предположим, что мы знаем вероятность дефолта, \en{recovery rate}\ и коэффициенты дисконтирования. Как посчитать честный купон в кредитном свопе?

\justify
Математические ожидания дисконтированных платежей продавца и покупателя должны совпадать. Для каждой даты в будущем необходимо рассчитать:

\justify
1. Вероятность, что эмитент доживёт до этой даты $S(T)$.

\justify
2. Вероятность, что эмитент разорится именно в эту дату $D(T) - D(T - 1 \text{ день})$.
\end{frame}



\begin{frame}{Ценообразование CDS - 2}
\justify
С вероятностью $S(T)$ покупатель выплачивает купон, если $T$ --- одна из дат выплат купона.
\begin{center}
\begin{tikzpicture}[thick, scale=0.6]
		\draw (0, 0) node[rectangle,draw,rounded corners,anchor=south,minimum height=1cm] {Покупатель} -- (0, -1.5);
		\draw (7.5, 0) node[rectangle,draw,rounded corners,anchor=south,minimum height=1cm] {Продавец} -- (7.5, -1.5);
		\draw [->,>=triangle 90] (0, -1.0) node[label=left:{T}]{} -- (3.5, -1.0) node[pos=0.5,anchor=south]{\$50\,000};
\end{tikzpicture}
\end{center}

\justify
С вероятностью $D(T) - D(T - 1 \text{ день})$ покупатель платит накопленный купон и остаточную стоимость облигации, а продавец выплачивает застрахованный номинал.
\begin{center}
\begin{tikzpicture}[thick, scale=0.6]
		\draw (0, 0) node[rectangle,draw,rounded corners,anchor=south,minimum height=1cm] {Покупатель} -- (0, -1.5);
		\draw (7.5, 0) node[rectangle,draw,rounded corners,anchor=south,minimum height=1cm] {Продавец} -- (7.5, -1.5);
		\draw [->,>=triangle 90] (0, -1.0) node[label=left:{T}]{} -- (3.5, -1.0) node[pos=0.5,anchor=south]{\$33\,333} node[pos=0.5,anchor=north] {Облигация};
		\draw [->,>=triangle 90] (7.5, -1.0) -- (4.5, -1.0) node[pos=0.5,anchor=south]{\$10\,000\,000};
\end{tikzpicture}
\end{center}
\end{frame}



\begin{frame}{Пример задачи}
\justify
Оцените справедливую стоимость (купон) кредитного свопа со следующими параметрами.

\justify
1. Срок 1 год.

\justify
2. Выплата купона раз в квартал (1/4 года).

\justify
3. Recovery Rate 40\%.

\justify
4. Вероятность дефолта задана экспоненциальным распределением с параметром $\lambda = 2\%$.

\justify
5. При дефолте выплата страховки произойдёт в дату следующего купона. Своп расчётный.

\justify
6. Дисконтированием пренебречь.

\end{frame}



\begin{frame}{Решение}
\justify
Пусть $x$ --- годовой купон в кредитном свопе с застрахованным номиналом $1$ и recovery rate $R$. Пусть $S(t)$ --- вероятность того, что референсный заёмщик доживёт до момента $t$ лет. Распишем платежи покупателя и продавца и вероятности каждого платежа.

\begin{center}
\begin{tikzpicture}[thick, scale=0.7]
		\draw (0, 0) node[rectangle,draw,rounded corners,anchor=south,minimum height=1cm] {Покупатель} -- (0, -5.5);
		\draw (10.5, 0) node[rectangle,draw,rounded corners,anchor=south,minimum height=1cm] {Продавец} -- (10.5, -5.5);
		\draw [->,>=triangle 90] (0, -1.0) node[label=left:{$\frac{1}{4}$ года}]{} -- (5.0, -1.0) node[pos=0.5,anchor=south]{$\frac{x}{4}; S(\frac{1}{4})$};
		\draw [->,>=triangle 90] (0, -2.5) node[label=left:{$\frac{1}{2}$ года}]{} -- (5.0, -2.5) node[pos=0.5,anchor=south]{$\frac{x}{4}; S(\frac{1}{2})$};
		\draw [->,>=triangle 90] (0, -4.0) node[label=left:{$\frac{3}{4}$ года}]{} -- (5.0, -4.0) node[pos=0.5,anchor=south]{$\frac{x}{4}; S(\frac{3}{4})$};
		\draw [->,>=triangle 90] (0, -5.5) node[label=left:{$1$ год}]{} -- (5.0, -5.5) node[pos=0.5,anchor=south]{$\frac{x}{4}; S(1)$};
    
    \draw [->,>=triangle 90] (10.5, -1.0) -- (5.5, -1.0) node[pos=0.5,anchor=south]{$1-R; 1 - S(\frac{1}{4})$};
    \draw [->,>=triangle 90] (10.5, -2.5) -- (5.5, -2.5) node[pos=0.5,anchor=south]{$1-R; S(\frac{1}{4}) - S(\frac{1}{2})$};
    \draw [->,>=triangle 90] (10.5, -4.0) -- (5.5, -4.0) node[pos=0.5,anchor=south]{$1-R; S(\frac{1}{2}) - S(\frac{3}{4})$};
    \draw [->,>=triangle 90] (10.5, -5.5) -- (5.5, -5.5) node[pos=0.5,anchor=south]{$1-R; S(\frac{3}{4}) - S(1)$};

\end{tikzpicture}
\end{center}
\end{frame}



\begin{frame}{Решение - 2}
\justify
Вероятность того, что референсный эмитент проживёт 1/4 года, равна $S(1/4) = 98\%$. Вероятность того, что он проживёт 1/2 года, равна $S(1/2) = 95\%$. 

\justify
С какой вероятностью эмитенты разорится в точности в промежутке между 1/4 и 1/2 года (в течение второго квартала)? 

\justify
Рассмотрим выборку из 100\,000 эмитентов, которые живы сегодня.

\justify
98\,000 эмитентов (98\% начальной популяции) выживут в течение первого квартала.

\justify
95\,000 (95\% начальной популяции) выживут в течение полугода.

\justify
В течение второго квартала разорятся 3\,000 эмитентов. Это $98\% - 95\% = 3\%$ начальной выборки. Эмитент, который жив сегодня, с вероятностью 3\% разорится в точности в течение второго квартала (не раньше и не позже).
\end{frame}



\begin{frame}{Решение - 3}
\justify
Математическое ожидание суммы платежей покупателя равно математическому ожиданию суммы платежей продавца.

\begin{equation*}
\frac{x}{4} \cdot \left( S\left(\frac{1}{4}\right) + S\left(\frac{1}{2}\right) + S\left(\frac{3}{4}\right) +S\left(1\right) \right) = (1 - R)(1 - S(1))
\end{equation*}

По условию задачи, $S(t)=e^{-\lambda t}$, поэтому

\begin{equation*}
x = \frac{4(1 - R)(1 - e^{-\lambda})}{e^{-\lambda / 4} + e^{-\lambda / 2} + e^{-3\lambda / 4} + e^{-\lambda}}
\end{equation*}

Подставляя $R = 0.4$ и $\lambda = 0.02$, получим:

\begin{equation*}
\frac{4(1 - 0.4)(1 - e^{-0.02})}{e^{-0.02 / 4} + e^{-0.02 / 2} + e^{-3\cdot0.02 / 4} + e^{-0.02}} \approx 1.203\%
\end{equation*}
\end{frame}



\begin{frame}{Решение (второй способ)}
\justify
Референсный эмитент выживет в течение 1 года с вероятностью $S(1) = e^{-\lambda \cdot 1} \approx 1 - \lambda = 0.98$.

\justify
Другими словами, дефолт случится в течение года с вероятностью $\approx 2\%$. Если дефолт произойдёт, то ваши потери составят $1 - R = 0.6$ или $60\%$.

\justify
Сколько вы готовы заплатить за страховку от события, которое произойдёт с вероятностью $2\%$ и причинит вам убыток в $60\%$ инвестированного капитала?

\justify
Конечно, $2\% \cdot 60\% = 1.2\%$! 
\end{frame}



\begin{frame}{Демонстрация}
\end{frame}



\begin{frame}{Риск-нейтральная и реальная вероятности}
\justify
Насколько точны рыночные оценки вероятности дефолта?

\justify
Не особо. Рынок почти всегда переоценивает вероятность дефолта. Компании разоряются реже, чем можно было бы предполагать, глядя на рыночные котировки.

\justify
В нашей модели все инвесторы \alert{риск-нейтральные}, то есть их интересует только математическое ожидание доходности, и не интересует риск (дисперсия).

\justify
Риск-нейтральному инвестору всё равно: получить \$100 наверняка, или с вероятностью 50/50 получить либо \$0, либо \$200.

\justify
Правда ли, что Homo Sapiens риск-нейтральны? Нет!
\end{frame}



\begin{frame}{Риск-нейтральность и страховка}
\justify
Новая BMW X5 стоит 10 миллионов рублей. Страховка КАСКО на год стоит 400 тысяч рублей. Вероятность того, что лектор разобьёт машину за год 2\%. После аварии машина стоит 0. Следует ли лектору застраховать машину?

\justify
1. Мат. ожидание со страховкой: $-400\,000 \cdot 100\% = -400\,000$.

\justify
2. Мат. ожидание без страховки: $-10\,000\,000 \cdot 2\% = -200\,000$.

Дополнительный фактор: серьёзный разговор с женой.

\justify
Лектор предпочтёт избежать риска и купить страховку. Никакая экономия в мат. ожидании не окупает
разговора с женой в плохом сценарии!
\end{frame}



\begin{frame}{Риск-нейтральная и реальная вероятности - 2}
\justify
Люди, принимающие инвестиционные решения, избегают риска (являются risk-averse), и это влияет на рынок облигаций и CDS.

\justify
Покупателей рискованных облигаций не устраивает математическое ожидание доходности, равное доходности безрисковых бумаг. Они требуют \alert{премию за риск} (\en{risk premium}), которая компенсирует дискомфорт от рискованной инвестиции. Рискованные облигации должны быть дешевле, а их доходность --- выше.

\justify
Покупатели CDS боятся потерь от дефолта, поэтому они готовы платить бОльшую премию за страховку, чем сферические риск-нейтральные инвесторы. Больше спрос --- выше цена CDS.
\end{frame}



\begin{frame}{Премия за кредитный риск - 1}
\justify
Премия за кредитный риск (\en{credit premium}) компенсирует инвесторам душевные страдания от потерь при дефолтах по облигациям. Рискованные облигации должны
давать доходность, которая не только компенсирует мат. ожидание потерь от
дефолтов, но и даёт премию за риск.

\justify
\centering
\begin{tabular}{r|r|r|r|r}
AAA    & AA     & A      & BBB    & CCC \\ \hline
0.32\% & 0.42\% & 0.43\% & 1.04\% & 0.86\%  
\end{tabular}

\justify
\centering
{\scriptsize Избыточная доходность корпоративных облигаций сверх гос. 
облигаций, 1987--2011. Данные: \en{Ang (2014)}.}

\justify
Следствие. Если вам предлагают еврооблигации ООО <<Рога и копыта>> с 
доходностью 10\% в долларах, то помните, что лишь 1\% --- это премия за риск,
которую вы заработаете в мат. ожидании, а 9\% компенсируют ожидаемые
потери при дефолте.
\end{frame}



\begin{frame}{Премия за кредитный риск - 2}
\centering
\begin{tikzpicture}
	\begin{axis}[
		width = \textwidth,
		height = \textheight - 1cm,
		date coordinates in = x,
		xticklabel = {\year},
		xtick = {1990-01-01, 1995-01-01, 2000-01-01, 2005-01-01, 2010-01-01, 2015-01-01, 2020-01-01, 2025-01-01},
		ylabel = {\small Рост \$1 начальных инвестиций},
		ylabel near ticks,
		ymode = log,
		log ticks with fixed point,
  		extra y ticks = {0.5, 2, 3, 4, 5, 20},
 		extra y tick labels = {0.5, 2, 3, 4, 5, 20},
		xmin = 1986-10-31,
		xmax = 2025-01-01,
		ymin = 0.9,
		ymax = 20,
		grid = both,
		legend entries = {
			BofA High Yield Index,
			BofA Corporate Index
		},
		legend pos=north west,
      legend style={font=\small},
      legend cell align={left}
	]
		\addplot[color=Set1-A, thick, mark=*, mark phase=9, mark repeat=30] table[x=month, y=high_yield_growth, col sep=comma] {bofa_bond_indices.csv};
		
		\addplot[color=Set1-B, thick, mark=square, mark phase=9, mark repeat=30] table[x=month, y=corp_growth, col sep=comma] {bofa_bond_indices.csv};
	\end{axis}
\end{tikzpicture}

\centering
\small Данные: Bank of America, St Louis Fed.
\end{frame}




\begin{frame}{Оценка деривативов через репликацию}
\justify
Вероятности дефолта и recovery rate нужны, чтобы подобрать правильную комбинацию 
ингредиентов, которые повторяют кредитный дефолтный своп:

\justify
1. Дать кому-то деньги в долг под залог корпоративной облигации.

\justify
2. Продать облигацию на рынке.

\justify
3. Купить безрисковую облигацию или заключить asset swap.

\justify
Если модель помогает подобрать реплицирующий портфель, то не так важно, что она исходит из <<неверной>> \en{implied}\ вероятности дефолта.
\end{frame}



\begin{frame}{Риск-нейтральная вероятность и букмекеры}
\justify
Котировки букмекеров на матч <<Реал>> --- <<Сити>> завтра:

\centering
\begin{tabular}{c|c|c}
<<Реал>> & Ничья & <<Сити>> \\
3.07 & 3.53 & 2.56
\end{tabular}

\justify
Если коэффициент равен $k$, то $1/k$ --- <<вероятность>> события.

\centering
\begin{tabular}{c|c|c}
<<Реал>> & Ничья & <<Сити>> \\
32.6\% & 28.3\% & 39.1\%
\end{tabular}

\justify
Допустим, что азартные болельщики поставили \$100 миллионов, из них 28.3 --- на ничью. При 
ничейном исходе победители заберут $\$28.3 \cdot 3.53 = \$100$ миллионов, то 
есть весь банк.

\justify
Нужно ли букмекеру стремиться к тому, чтобы коэффициенты отражали истинную вероятность из реального мира? Нет! Ему важно, чтобы при любом исходе ставок проигравших хватило на выплаты победителям. Больше ставок на исход --- ниже коэффициент.
\end{frame}



\begin{frame}{Реформа рынка в 2009 г.}
\justify
В начале 2009 года регуляторы провели реформу рынка кредитных деривативов, чтобы избежать повторения кризиса 2008 года.

\justify
1. Обязательный централизованный клиринг.

\justify
2. 4 стандартных даты выплаты купонов: 20 марта, 20 июня, 20 сентября, 20 декабря.

\justify
3. Фиксированные купоны либо по 1\% (облигации с высоким рейтингом), либо по 5\% (мусорные облигации).

\justify
Все по-прежнему котируют CDS в терминах спреда (например, 2.40\%/2.41\%), но реальные купоны всё равно будут либо 1\%, либо 5\%.

\justify
Одна из сторон сделки выплачивает другой сумму, рассчитанную по стандартной методологии, чтобы компенсировать разницу между купоном в котировке (2.4\%) и реальным фиксированным купоном (1.0\%).
\end{frame}



\begin{frame}{Базис CDS-Bond}
\justify
На идеальном сферическом рынке в вакууме не должно быть разницы между двумя стратегиями

\justify
1. Купить рискованную облигацию с доходностью 5\% и застраховать её CDS-ом с купоном 3\%.

\justify
2. Купить безрисковую облигацию с доходностью 2\%.

\justify
На практике наблюдается ненулевой CDS-Bond basis:
\begin{align*}
\text{CDS Basis} = \text{CDS} - \text{Z-spread}
\end{align*}

\justify
Некоторые факторы, которые влияют на базис:

1. Ликвидность.

2. Несовпадение купонов.

3. Накопленный купон на момент дефолта.

4. Встроенные в облигацию опционы.
\end{frame}



\begin{frame}{Wrong Way CDS}
\justify
Что бы Вы сказали о следующих сделках?

\justify
1. Купить CDS на Германию у Deutsche Bank.

\justify
2. Купить CDS на Deutsche Bank у сегодняшнего лектора.

\justify
Страховка надёжна ровно настолько, насколько надёжен её продавец, который тоже может разориться. Важно, чтобы дефолт референсного эмитента не имел корреляции с дефолтом продавца страховки.

\justify
Загадка: кто и у кого покупает CDS на гос. долг США?
\end{frame}



\begin{frame}{Индексные кредитные свопы}
\justify
Индексный кредитный своп:

1. 100 компаний-заемщиков.

2. Начальный номинал \$10\,000\,000.

3. При дефолте каждой компании номинал (и купон) уменьшается на \$100\,000.

\justify
Nth-to-default CDS:

1. 100 компаний-заёмщиков

2. Начальный номинал \$10\,000\,000.

3. После первых $N$ дефолтов выплачивается полный номинал.

\justify
Цены таких свопов зависят не только от вероятностей дефолта, но и от корреляций между дефолтами. Подробности в учебнике Hull.
\end{frame}



\begin{frame}{Зачем?}
\justify
Кредитные деривативы --- это не только огромный рынок, но и важный компонент цены других деривативов:

\justify
1. Credit valuation adjustment (CVA)

\justify
2 Debt valuation adjustment (DVA)

\end{frame}

\end{document}


