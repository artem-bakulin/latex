\documentclass{beamer}

\usepackage{cmap}				% To be able to copy-paste russian text from pdf
\usepackage[T2A]{fontenc}
\usepackage[utf8]{inputenc}
\usepackage[russian]{babel}
\usepackage{textpos}
\usepackage{ragged2e}
\usepackage{amssymb}
\usepackage{ulem}
\usepackage{tikz}
\usepackage{pgfplots}
\usepackage{color}
\usepackage{cancel}
\usepackage{multirow}
\pgfplotsset{compat=1.17}
\usetikzlibrary{arrows,snakes,backgrounds,shapes}
\usepgfplotslibrary{groupplots,colorbrewer,dateplot,statistics}
\usepackage{animate}

\usepackage{amsfonts}
\usepackage{amsmath}
\usepackage{amssymb}
\usepackage{graphicx}
\usepackage{setspace}
\usepackage{tabularx}

\usepackage{enumitem}
\setitemize{label=\usebeamerfont*{itemize item}%
  \usebeamercolor[fg]{itemize item}
  \usebeamertemplate{itemize item}}

% remove navigation bar
\setbeamertemplate{navigation symbols}{} 

\usepackage{eurosym}
\renewcommand{\EUR}[1]{\textup{\euro}#1}

\title{Лекция 5. Метод Монте-Карло}
\author{Артём Бакулин}
\date{11 ноября 2021 г.}

\usetheme{Warsaw}
\usecolortheme{beaver}

\newcommand{\ru}[1]{\begin{otherlanguage}{russian}#1\end{otherlanguage}}
\newcommand{\en}[1]{\begin{otherlanguage}{english}#1\end{otherlanguage}}
\newcommand{\ruen}[2]{#1 (\en{#2})}

\begin{document}



\begin{frame}
\titlepage
\end{frame}



\newcommand{\drawStockNode}[5]{

	\node (#5)
	[
		draw,
		rectangle,
		rounded corners,
		inner sep = 0pt,
		outer sep = 0pt,
		minimum width = 2.4cm,
		minimum height = 0.55cm,
		align = center
	]
	at (#3, #4)
	{
		\begin{tabular}{c|c}
		#1 & #2
		\end{tabular}
	};
}

\newcommand{\drawStockLink}[4]{

	\draw[
		->,
		>=triangle 90
	]
	(#1.east) -- (#2.west)
	node[
		pos = 0.5,
		anchor = #4
	]
	{#3};
}

\newcommand{\drawOneStepBinomialTree}{
	\drawStockNode{$S_0$} {?}{0}{ 0}{S0_node}
	\drawStockNode{$S_0u$}{$V_u$}{4}{ 1}{Su_node}
	\drawStockNode{$S_0d$}{$V_d$}{4}{-1}{Sd_node}
	
	\drawStockLink{S0_node}{Su_node}{$p$}{south east}	
	\drawStockLink{S0_node}{Sd_node}{$1-p$}{north east}
}

\begin{frame}{Напоминание: биномиальная модель}

\centering
\begin{tikzpicture}
\drawOneStepBinomialTree
\end{tikzpicture}

\justify
Текущая цена акции $S_0$. Акция не платит дивидендов.

\justify
Акция может либо вырасти в цене до $S_0u$ с вероятностью $p$, либо снизиться до $S_0d$ с вероятностью $1-p$.

\justify
Безрисковая процентная ставка $r$, длина одного периода $T$ лет, причём $d < 1 + rT < u$.

\justify
Дериватив на акцию (скажем, опцион) принесёт прибыль либо $V_u$, либо $V_d$.
\end{frame}



\begin{frame}{Напоминание: биномиальная модель}

\centering
\begin{tikzpicture}
\drawOneStepBinomialTree
\end{tikzpicture}

\justify
Рассмотрим портфель, который состоит из $\Delta$ акций и долга $L$. Подберём $\Delta$ и $L$ так, чтобы портфель реплицировал опцион.
\begin{align*}
&\begin{cases}
L(1+rT) + \Delta S_0u = V_u \\
L(1+rT) + \Delta S_0d = V_d
\end{cases}
\Rightarrow \\
&\begin{cases}
\Delta = \dfrac{V_u - V_d}{S_0(u - d)} \\
L = 	\dfrac{V_du - V_ud}{(1+rT)(u - d)}
\end{cases}
\end{align*}
\end{frame}



\begin{frame}{Напоминание: биномиальная модель}
\centering
\begin{tikzpicture}
\drawOneStepBinomialTree
\end{tikzpicture}

\justify
Сегодняшняя цена дериватива $V$ равна цене реплицирующего портфеля:
\begin{align*}
V &= \Delta S_0 + L = \\
&= \frac{V_u - V_d}{S_0(u - d)}S_0 + \frac{V_du - V_ud}{(1 + rT)(u - d)} = \\
&= \frac{qV_u + (1-q)V_d}{1+rT}
\end{align*}
Здесь
\begin{align*}
q = \frac{1 + rT - d}{u - d} \text{--- <<риск-нейтральная вероятность>>}
\end{align*}
\end{frame}



\begin{frame}{Напоминание: биномиальная модель}
\centering
\begin{tikzpicture}
	\drawStockNode{$\$100$} {?}{0}{ 0}{S0_node}
	\drawStockNode{$\$120$}{$\$20$}{4}{0.5}{Su_node}
	\drawStockNode{$\$80$}{$\$0$}{4}{-0.5}{Sd_node}
	
	\drawStockLink{S0_node}{Su_node}{$90\%$}{south}	
	\drawStockLink{S0_node}{Sd_node}{$10\%$}{north}
\end{tikzpicture}

\justify
Акция стоит $\$100$ и может либо вырасти до $\$120$ ($u=1.2$), либо упасть до $\$80$ ($d=0.8$). Безрисковая ставка $r=0\%$. Сколько стоит колл-опцион со страйком $\$100$?

\justify
Риск-нейтральная вероятность:
\begin{align*}
q = \frac{1 + rT - d}{u - d} = \frac{1 - 0.8}{1.2 - 0.8} = 0.5
\end{align*}

\justify
Цена опциона:
\begin{align*}
V = \frac{qV_u + (1-q)V_d}{1 + rT} = 0.5\cdot\$20 + 0.5\cdot\$0 = \$10
\end{align*}
\end{frame}



\renewcommand{\drawStockLink}[2]{

	\draw[
		->,
		>=triangle 45
	]
	(#1.east) -- (#2.west)
	{};
}

\renewcommand{\drawStockNode}[5]{

	\node (#5)
	[
		draw,
		rectangle,
		rounded corners,
		inner sep = 1pt,
		outer sep = 0pt,
		minimum width = 1.5cm
	]
	at (#3, #4)
	{
		\centering
		\begin{tabular}{c}
		#1 \\ \hline #2
		\end{tabular}
	};
}

\newcommand{\nodeVerticalStep}{0.7}
\newcommand{\nodeHorizontalStep}{2.75}

\begin{frame}{Напоминание: биномиальная модель}
\centering
\begin{tikzpicture}
\drawStockNode{$\$100$}{\only<1-7>{?}\only<8->{\$14.8}}{0}{0}{S0_node}

\drawStockNode{$\$120$}{\only<1-5>{?}\only<6->{\$25.8}}{\nodeHorizontalStep}{\nodeVerticalStep}{Su_node}
\drawStockNode{$\$80$}{\only<1-6>{?}\only<7->{\$3.8}}{\nodeHorizontalStep}{-\nodeVerticalStep}{Sd_node}

\drawStockNode{$\$144$}{\only<1-2>{?}\only<3->{\$44}}{2*\nodeHorizontalStep}{2*\nodeVerticalStep}{Suu_node}
\drawStockNode{$\$96$}{\only<1-3>{?}\only<4->{\$7.6}}{2*\nodeHorizontalStep}{0}{Sud_node}
\drawStockNode{$\$64$}{\only<1-4>{?}\only<5->{\$0}}{2*\nodeHorizontalStep}{-2*\nodeVerticalStep}{Sdd_node}

\drawStockNode{$\$172.8$}{\only<1>{?}\only<2->{\$72.8}}{3*\nodeHorizontalStep}{3*\nodeVerticalStep}{Suuu_node}
\drawStockNode{$\$115.2$}{\only<1>{?}\only<2->{\$15.2}}{3*\nodeHorizontalStep}{\nodeVerticalStep}{Suud_node}
\drawStockNode{$\$76.8$}{\only<1>{?}\only<2->{\$0}}{3*\nodeHorizontalStep}{-\nodeVerticalStep}{Sudd_node}
\drawStockNode{$\$51.2$}{\only<1>{?}\only<2->{\$0}}{3*\nodeHorizontalStep}{-3*\nodeVerticalStep}{Sddd_node}

\drawStockLink{S0_node}{Su_node}
\drawStockLink{S0_node}{Sd_node}

\drawStockLink{Su_node}{Suu_node}
\drawStockLink{Su_node}{Sud_node}

\drawStockLink{Sd_node}{Sud_node}
\drawStockLink{Sd_node}{Sdd_node}

\drawStockLink{Suu_node}{Suuu_node}
\drawStockLink{Suu_node}{Suud_node}

\drawStockLink{Sud_node}{Suud_node}
\drawStockLink{Sud_node}{Sudd_node}

\drawStockLink{Sdd_node}{Sudd_node}
\drawStockLink{Sdd_node}{Sddd_node}
\end{tikzpicture}

\justify
Пусть $u=1.2$, $d=0.8$, $r=0\%$, $S_0=\$100$. Сколько стоит колл-опцион со страйком $K=\$100$, истекающий через три шага?

\justify
Риск-нейтральная вероятность:
\begin{align*}
q = \frac{1 + rT - d}{u - d} = \frac{1 - 0.8}{1.2 - 0.8} = 0.5
\end{align*}
\end{frame}



\newcommand{\highlightStockLink}[6]{
	\draw[
		color=#4,
		very thick,
		->,
		>=triangle 45
	]
	(#1.east) -- (#2.west)
	node[
		pos=#5,
		anchor=#6
	]
	{#3};
}

\newcommand{\highlightStockLinkUp}[3]{
	\highlightStockLink{#1}{#2}{$q$}{#3}{0.5}{south}
}

\newcommand{\highlightStockLinkDown}[3]{
	\highlightStockLink{#1}{#2}{$1-q$}{#3}{0.15}{west}
}

\begin{frame}{Напоминание: биномиальная модель}
\centering
\begin{tikzpicture}
\drawStockNode{$S_0$}{?}{0}{0}{S0_node}

\drawStockNode{$S_0u$}{?}{\nodeHorizontalStep}{\nodeVerticalStep}{Su_node}
\drawStockNode{$S_0d$}{?}{\nodeHorizontalStep}{-\nodeVerticalStep}{Sd_node}

\drawStockNode{$S_0u^2$}{?}{2*\nodeHorizontalStep}{2*\nodeVerticalStep}{Suu_node}
\drawStockNode{$S_0ud$}{?}{2*\nodeHorizontalStep}{0}{Sud_node}
\drawStockNode{$S_0d^2$}{?}{2*\nodeHorizontalStep}{-2*\nodeVerticalStep}{Sdd_node}

\drawStockNode{$S_0u^3$}{$V_3$}{3*\nodeHorizontalStep}{3*\nodeVerticalStep}{Suuu_node}
\drawStockNode{$S_0u^2d$}{$V_2$}{3*\nodeHorizontalStep}{\nodeVerticalStep}{Suud_node}
\drawStockNode{$S_0ud^2$}{$V_1$}{3*\nodeHorizontalStep}{-\nodeVerticalStep}{Sudd_node}
\drawStockNode{$S_0d^3$}{$V_0$}{3*\nodeHorizontalStep}{-3*\nodeVerticalStep}{Sddd_node}

\only<1-2>{
	\drawStockLink{S0_node}{Su_node}
	\drawStockLink{S0_node}{Sd_node}

	\drawStockLink{Su_node}{Suu_node}
	\drawStockLink{Su_node}{Sud_node}

	\drawStockLink{Sd_node}{Sud_node}
	\drawStockLink{Sd_node}{Sdd_node}

	\drawStockLink{Suu_node}{Suuu_node}
	\drawStockLink{Suu_node}{Suud_node}

	\drawStockLink{Sud_node}{Suud_node}
	\drawStockLink{Sud_node}{Sudd_node}

	\drawStockLink{Sdd_node}{Sudd_node}
	\drawStockLink{Sdd_node}{Sddd_node}
}

\only<3>{
	\highlightStockLinkUp{S0_node}{Su_node}{Set1-A}
	\highlightStockLinkUp{Su_node}{Suu_node}{Set1-A}
	\highlightStockLinkDown{Suu_node}{Suud_node}{Set1-A}
}

\only<4>{
	\highlightStockLinkUp{S0_node}{Su_node}{Set1-A}
	\highlightStockLinkDown{Su_node}{Sud_node}{Set1-A}
	\highlightStockLinkUp{Sud_node}{Suud_node}{Set1-A}
}

\only<5>{
	\highlightStockLinkDown{S0_node}{Sd_node}{Set1-A}
	\highlightStockLinkUp{Sd_node}{Sud_node}{Set1-A}
	\highlightStockLinkUp{Sud_node}{Suud_node}{Set1-A}
}

\end{tikzpicture}

\justify
Риск-нейтральная вероятность: $q = \dfrac{1 + rT - d}{u - d}$.

\justify
Цена дериватива сегодня:
\begin{align*}
V = \frac{q^3V_3 + \only<1>{3q^2(1-q)}\only<2->{\alert{3q^2(1-q)}}V_2 + 3q(1-q)^2V_1 + (1-q)^3V_0}{(1+rT)^3}
\end{align*}
\end{frame}



\begin{frame}{Риск-нейтральная вероятность}
\justify
Если представить, что $q$ --- вероятность движения акции вверх, то $3q^2(1-q)$ --- вероятность того, что акция дважды вырастет и один раз упадёт (не важно, в каком порядке). Тогда акция будет стоить $S_0u^2d$, а дериватив принесёт прибыль $V_2$.

\justify
\centering
\begin{tabular}{l|l|l}
Цена акции & Прибыль & <<Вероятность>> \\ \hline
$S_0u^3$   & $V_3$   & $q^3$ \\
$S_0u^2d$  & $V_2$   & $q^2(1-q)$ \\
$S_0ud^2$  & $V_1$   & $q(1-q)^2$ \\ 
$S_0d^3$   & $V_0$   & $(1-q)^3$ 
\end{tabular}

\justify
Цена дериватива похожа на дисконтированное <<математическое ожидание>> прибыли.
\begin{align*}
V = \frac{q^3V_3 + 3q^2(1-q)V_2 + 3q(1-q)^2V_1 + (1-q)^3V_0}{(1+rT)^3}
\end{align*}
\end{frame}



\begin{frame}{Риск-нейтральная вероятность}
\justify
Рассмотрим акцию как дериватив с выплатой $V(S) = S$. Акция --- это колл-опцион на саму себя со страйком 0.
\begin{align*}
S &= \frac{q^3S_0u^3 + 3q^2(1-q)S_0u^2d + 3q(1-q)^2S_0ud^2 + (1-q)^3S_0d^3}{(1+rT)^3} = \\
&= S_0\frac{q^3u^3 + 3q^2u^2(1-q)d + 3qu(1-q)^2d^2 + (1-q)^3d^3}{(1+rT)^3} = \\
&= S_0\frac{\Big(qu + (1-q)d \Big)^3}{(1+rT)^3} 
= S_0\frac{\left(\dfrac{1+rT-d}{u-d}u + \dfrac{u-1-rT}{u-d}d \right)^3}{(1+rT)^3} = \\
&= S_0 \frac{\left(\dfrac{(1+rT)(u-d)}{u-d} \right)^3}{(1+rT)^3} = S_0
\end{align*}

\justify
Акция стоит своё <<математическое ожидание>>.
\end{frame}



\begin{frame}{Риск-нейтральная вероятность}
\justify
Рассмотрим безрисковую бескупонную облигацию, которая выплатит $\$1$ номинала на третьем шаге, то есть $V(S) = 1$.
\begin{align*}
B &= \frac{q^3 \cdot 1 + 3q^2(1-q) \cdot 1 + 3q(1-q)^2 \cdot 1 + (1-q)^3 \cdot 1}{(1+rT)^3} = \\
&= \frac{\Big(q + (1-q) \Big)^3}{(1+rT)^3} = \frac{1}{(1+rT)^3}
\end{align*}

\justify
Безрисковая облигация стоит своё <<математическое ожидание>>.
\end{frame}



\begin{frame}{Риск-нейтральная вероятность}
\justify
Чтобы вычислить коэффициенты $q$ и $1-q$, мы составили реплицирующий портфель и предположили отсутствие возможностей для арбитража. Значение $q$ механически вычисляется из параметров задачи $u$ и $d$ (возможный разброс цены акции вверх и вниз) и $r$ (безрисковая процентная ставка).

\justify
Оказалось, что цены всех трёх активов (дериватива, акции, безрисковой облигации) оказались равны дисконтированному <<математическому ожиданию>> будущих выплат, если вместо истинных вероятностей подставить <<риск-нейтральные вероятности>> $q$ и $1-q$.

\justify
Совпадение ли это?
\end{frame}



\begin{frame}{Фундаментальная теорема}
\justify
Справедлива фундаментальная теорема оценки активов (\en{Fundamental Theorem of Asset Pricing}):

\justify
На рынке с дискретным временем* нет возможностей для арбитража тогда и только тогда, когда существует риск-нейтральная вероятностная мера $\mathbb{Q}$, эквивалентная** вероятностной мере реального мира $\mathbb{P}$.

\justify
*Существует формулировка в непрерывном времени.

\justify
**Две вероятностные меры $\mathbb{P}$ и $\mathbb{Q}$ называются эквивалентными, если для любого события $A$ справедливо, что
\begin{align*}
\mathbb{P}(A)=0 \quad \Leftrightarrow \quad \mathbb{Q}(A)=0
\end{align*}
Меры $\mathbb{P}$ и $\mathbb{Q}$ согласны в том, какие события хотя бы гипотетически возможны, но могут присваивать им разную вероятность.
\end{frame}



\begin{frame}{Риск-нейтральная вероятность}
\justify
В риск-нейтральном мире все активы стоят своё дисконтированное математическое ожидание. Другими словами, все активы в среднем растут на безрисковую процентную ставку.

\justify
Это было бы возможно, если бы инвесторы интересовались лишь математическим ожиданием доходности, то есть были бы риск-нейтральными (\en{risk neutral}).

\justify
Верно ли, что реальный мир населён риск-нейтральными роботами? Нет! Среднестатистический инвестор на реальном рынке избегает риска (является \en{risk averse}).
\end{frame}



\begin{frame}{Тест на риск-нейтральность -- 1}
\justify
Вы должны вложить весь свой капитал на 1 год либо в безрисковые облигации, либо в рискованные акции. Какой вариант вы предпочтёте?

\justify
\centering
\begin{tabular}{l|r|r|r}
Сценарий & Вероятность & Облигации & Акции \\ \hline
Хороший  & 50\%   & $+2\%$    & $+22\%$  \\
Плохой   & 50\%   & $+2\%$    & $-18\%$  \\ \hline
\multicolumn{2}{l|}{Мат. ожидание} & $+2\%$ & $+2\%$
\end{tabular}

\pause
\justify
В среднем при одинаковой доходности люди предпочитают менее рискованные активы. Чтобы убедить избегающих риска инвесторов купить акцию, нужно дать им скидку, чтобы ожидаемая будущая доходность была чуть выше.
\end{frame}



\begin{frame}{Тест на риск-нейтральность -- 2}
\justify
Вы должны вложить весь свой капитал на 1 год либо в акцию А, либо в акцию B. Какой вариант вы предпочтёте?

\justify
\centering
\begin{tabular}{l|r|r|r}
Сценарий & Вероятность   & Акция А & Акция B \\ \hline
Кризис и потеря работы   & 50\%    & $+25\%$ & $+25\%$  \\
Премия 1\,000\,000 р.    & 50\%    & $-15\%$ & $-15\%$  \\ \hline
\multicolumn{2}{l|}{Мат. ожидание} & $+5\%$  & $+5\%$
\end{tabular}

\pause
\justify
Когда люди оценивают активы, они интересуются не только разбросом возможных доходностей, но и корреляцией со всем рынком. Активы, которые меньше падают в кризис, ценятся выше и имеют меньшую ожидаемую доходность.
\end{frame}



\begin{frame}{Ценообразование в риск-нейтральном мире}
\justify
Мы не можем залезть в головы к инвесторам, чтобы вычислить степень их риск-нейтральности. Поэтому мы не знаем, какая премия за риск заложена в текущие цены активов.

\justify
Хорошая новость: нам и не нужно этого знать. FATP гарантирует, что цена дериватива в реальном мире окажется такой же, как и в риск-нейтральном мире. Для цены базового актива это, разумеется, не верно.
\end{frame}



\begin{frame}{Ценообразование в риск-нейтральном мире}
\justify
FATP гарантирует, что любой дериватив можно оценить 	по следующему алгоритму:

\justify
1. Записать случайный процесс, которому следует цена базового актива в риск-нейтральном мире.

2. Выписать функцию выплаты по деривативу в зависимости от цены базового актива.

3. Вычислить математическое ожидание выплаты.

4. Умножить математическое ожидание на коэффициент дисконтирования.

\justify
Обычно самый сложный шаг --- шаг номер 3. Зачастую аналитического решения не существует.
\end{frame}



\begin{frame}{Пример: модель Блэка-Шоулза}
\justify
В модели Блэка-Шоулза цена базового актива в реальном мире $S(t)$ следует геометрическому броуновскому движению. Приращение цены $dS$ за малое время $dt$ равно
\begin{align*}
\frac{dS}{S} = \mu dt + \sigma \sqrt{dt} \xi, \quad \xi \sim \mathcal{N}(0,1) 
\end{align*}
Здесь $mu$ --- тренд, $\sigma$ --- волатильность.

\justify
В риск-нейтральном мире базовый актив растёт на безрисковую процентную ставку $r$ (поэтому цена дериватива не зависит от тренда $\mu$).
\begin{align*}
\frac{dS}{S} = r \cdot dt + \sigma \sqrt{dt} \xi, \quad \xi \sim \mathcal{N}(0,1) 
\end{align*}
\end{frame}



\begin{frame}{Пример: модель Блэка-Шоулза}
\justify
Если сложить все приращения за все шаги по $dt$, то на длинном интервале $(0, T)$ цена базового актива в риск-нейтральном мире изменяется по следующему закону:
\begin{align*}
S_T(\xi) = S_0\exp{\Big[\left(r - \sigma^2/2\right)T + \sigma\sqrt{T}\xi\Big]}, \quad \xi \sim \mathcal{N}(0, 1)
\end{align*}

\justify
Пусть $f(S_T) = \max(S_T - K, 0)$ --- функция выплаты ванильного европейского колл-опциона. Тогда цена опциона согласно FATP равна
\begin{align*}
C = e^{-rT} \int\limits_{-\infty}^{+\infty}f\Big(S_t(x)\Big)\phi(x)dx
\end{align*}
Здесь $\phi(x)$ --- функция плотности стандартного нормального распределения $\mathcal{N}(0,1)$. Если раскрыть интеграл, то получится формула Блэка-Шоулза.
\end{frame}



\begin{frame}{Численные методы}
\justify
Как быть, если случайный процесс и/или формула выплаты настолько сложны, что не получается вычислить риск-нейтральное математическое ожидание в явном виде?

\justify
На помощь приходят численные методы:
\begin{itemize}
\item Биномиальные деревья.
\item Разностные схемы.
\item Метод Монте-Карло.
\end{itemize}
\end{frame}



\begin{frame}{Метод Монте-Карло}
\justify
Как вычислить число $\pi$? Нужно наудачу бросить на квадрат $N$ точек и посмотреть, сколько из них попали во вписанный круг. 

\centering
\begin{tikzpicture}
	\begin{axis}[
			width = 6cm,
			height = 6cm,
			only marks,
			xmin = 0, xmax = 1,
			ymin = 0, ymax = 1,
			xtick = {\empty},
			ytick = {\empty}
		]
				
		\addplot[mark=*, mark size=1.5pt, color=Set1-A] table[x=x, y=y, col sep=comma] {monte_carlo_pi_in_circle.csv};
		
		\addplot[mark=x, mark size=1.5pt, color=Set1-B] table[x=x, y=y, col sep=comma] {monte_carlo_pi_not_in_circle.csv};
		
		\draw[very thick] (0.5, 0.5) circle (0.5);
		\draw[very thick] (0, 0) rectangle (1, 1);
	\end{axis}
\end{tikzpicture}
\justify
Если радиус круга $R$, то в круг должно попасть примерно $N\dfrac{\pi R^2}{4R^2}$ точек. В данном случае в круге $N_0=770$ точек из $N=1000$. 
\begin{align*}
\pi \approx \frac{4N_0}{N} = \frac{4 \cdot 770}{1000} = 3.08
\end{align*}
\end{frame}



\begin{frame}{Метод Монте-Карло}
\justify
Пусть цена базового актива а риск-нейтральном мире, а вслед за ней и выплата по деривативу, зависят от реализации некоторой случайной величины $\xi$ (возможно многомерной). Предположим, что мы знаем закон распределения $\xi$.

\justify
1. Выберем наудачу $n$ (достаточно много) реализаций случайной величины $\xi$: $\xi_1, \xi_2, ..., \xi_n$.

2. Для каждой реализации $\xi_i$ вычислим цену базового актива $S(\xi_i)$ и выплату по деривативу $f(S(\xi_i))$.

3. Оценим математическое ожидание выплаты как 
\begin{align*}
\hat{f} = \frac{1}{n}\sum\limits_{i=1}^{n}f\Big(S(\xi_i)\Big)
\end{align*}

4. Выполним дисконтирование.

\justify
Можно надеяться (почему?), что с ростом $n$ оценка мат. ожидания будет сходиться к истинному мат. ожиданию.
\end{frame}



\begin{frame}{Закон больших чисел}
\justify
Пусть $\xi_i$ --- независимые одинаково распределённые случайные величины, которые имеют конечное математическое ожидание $\mathbb{E}\xi_i=\mu$. Тогда с вероятностью 1 (почти наверное)
\begin{align*}
\lim_{n \to \infty} \frac{1}{n} \sum\limits_{i=1}^{n}\xi_i = \mu
\end{align*}

\justify
Это закон больших чисел (ЗБЧ). Если усреднить много-много реализаций одной и той же случайной величины, то получится число, достаточно близкое к истинному математическому ожиданию.

\justify
Что такое <<достаточно>> близко?
\end{frame}



\begin{frame}{Центральная предельная теорема}
\justify
Пусть $\xi_i$ --- независимые одинаково распределённые случайные величины, которые имеют конечное математическое ожидание $\mathbb{E}\xi_i=\mu$ и конечную дисперсию $\operatorname{Var}(\xi_i) = \sigma^2$. Тогда имеет место сходимость по распределению:
\begin{align*}
\lim_{n \to \infty} \sqrt{n}\frac{\dfrac{1}{n}\sum\limits_{i=1}^{n}\xi_i - \mu}{\sigma} = \mathcal{N}(0, 1)
\end{align*}

\justify
$\mathcal{N}(0, 1)$ --- стандартное нормальное распределение.

\justify
Это --- центральная предельная теорема (ЦПТ). Среднее арифметическое большого количества реализаций случайной величины следует нормальному распределению.
\end{frame}



\begin{frame}{Случайные числа}
\justify
Для метода Монте-Карло нужно много, очень много случайных чисел.

\justify
В большинстве современных процессоров есть встроенный аппаратный генератор случайных чисел. Например, инструкция x86 RDRAND позволяет получить 16, 32 или 64 случайных бита.

\justify
Недостатки:
\begin{itemize}
\item Быстродействие (450 тактов на Core i7-7700).
\item Невоспроизводимость результатов.
\end{itemize}

\justify
Нужен быстрый алгоритм получения <<почти>> случайных чисел.
\end{frame}



\begin{frame}{Псевдослучайные последовательности}
\justify
Простой генератор псевдослучайных чисел по методу Лемера, реализованный в std::minstd\_rand:
\begin{align*}
X_{k+1} = (48\,271 \cdot X_k) \ \operatorname{MOD} \ 2\,147\,483\,647
\end{align*}

\justify
Для каждого целого положительного начального значения $X_0$ (которое называется \en{seed}) получается новая <<почти случайная>> последовательность целых чисел от 0 до 2\,147\,483\,646.

\justify
Как получить случайные действительные числа из равномерного распределения $U(0, 1)$?
\begin{align*}
U_k = X_k / (2\,147\,483\,647 - 1)
\end{align*}
\end{frame}



\begin{frame}{Псевдослучайные последовательности}
\justify
Если мы умеем создавать случайные числа $U_k$ из равномерного распределения $U(0,1)$, то как получить случайные числа, например, из нормального распределения $\mathcal{N}(0, 1)$?

\justify
Пусть $N(x)$ --- функция распредления стандартного нормального распределения, а $N^{-1}(x)$ --- её обратная функция. Тогда величины
\begin{align*}
N_k = N^{-1}(U_k)
\end{align*}
будут иметь стандартное нормальное распределение:
\begin{align*}
\mathbb{P}(N_k < x) = \mathbb{P}(N^{-1}(U_k) < x) = \mathbb{P}(U_k < N(x)) = U(N(x)) = N(x)
\end{align*}
\end{frame}


\end{document}


