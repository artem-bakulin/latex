\documentclass{beamer}

\usepackage{cmap}				% To be able to copy-paste russian text from pdf
\usepackage[T2A]{fontenc}
\usepackage[utf8]{inputenc}
\usepackage[russian]{babel}
\usepackage{textpos}
\usepackage{ragged2e}
\usepackage{amssymb}
\usepackage{ulem}
\usepackage{tikz}
\usepackage{pgfplots}
\usepackage{color}
\usepackage{cancel}
\usepackage{multirow}
\pgfplotsset{compat=1.17}
\usetikzlibrary{arrows,snakes,backgrounds,shapes}
\usepgfplotslibrary{groupplots,colorbrewer,dateplot,statistics,fillbetween}
\usepackage{animate}

\usepackage{amsfonts}
\usepackage{amsmath}
\usepackage{amssymb}
\usepackage{graphicx}
\usepackage{setspace}
\usepackage{tabularx}

\usepackage{enumitem}
\setitemize{label=\usebeamerfont*{itemize item}%
  \usebeamercolor[fg]{itemize item}
  \usebeamertemplate{itemize item}}

% remove navigation bar
\setbeamertemplate{navigation symbols}{}

\setbeamertemplate{page number in head/foot}[totalframenumber] 

\usepackage{eurosym}
\renewcommand{\EUR}[1]{\textup{\euro}#1}

\title{Лекция 7. Value at Risk (VaR)}
\author{Артём Бакулин}
\date{25 ноября 2021 г.}

\usetheme{Warsaw}
\usecolortheme{beaver}

\newcommand{\ru}[1]{\begin{otherlanguage}{russian}#1\end{otherlanguage}}
\newcommand{\en}[1]{\begin{otherlanguage}{english}#1\end{otherlanguage}}
\newcommand{\ruen}[2]{#1 (\en{#2})}

% https://tex.stackexchange.com/questions/98003/filter-rows-from-a-table
\pgfplotsset{
    discard if not/.style 2 args={
        x filter/.code={
            \edef\tempa{\thisrow{#1}}
            \edef\tempb{#2}
            \ifx\tempa\tempb
            \else
                \def\pgfmathresult{inf}
            \fi
        }
    }
}



\begin{document}

\begin{frame}
\titlepage
\end{frame}



\begin{frame}{Рыночный риск}
\justify
\alert{Рыночный риск} (\en{market risk}) --- возможность того, что текущая стоимость (present value, PV)  
портфеля изменится, если изменятся цены рыночных инструментов. Например, возможность 
того, что купленная облигация подешевеет, а проданный опцион подорожает.

\justify
Трейдеру, его менеджерам, акционерам, регуляторам --- всем нужна численная оценка 
рыночного риска, который он взял на себя.

\justify
Стандартный ответ  --- оценить рыночный риск в 
терминах частных производных PV по рыночным переменным. Эти частные производные 
называются <<греки>> (Greeks) или чувствительности (sensitivities).
\end{frame}



\begin{frame}{Частные производные}
\justifying
На ваш взгляд, насколько рискованна такая позиция?
\begin{itemize}
\justifying
\item Дельта:  $+\$100$ на каждые $0.0001$ роста курса EURUSD.
\item Вега: $+\$4\,000$ на каждый процентный пункт роста ATMF 1Y волатильности валютной пары EURUSD.
\item CS01: $-\$50$ на каждый базисный пункт роста индексного кредитного свопа \en{iTraxx Europe Plus Index}.
\end{itemize}

\justify
Если вы не трейдер, который целыми днями занимается одним узким сегментом рынка, то, скорее всего, цифры частных производных не скажут вам ничего. Реальность крупного инвестиционного банка --- десятки тысяч частных производных первого, второго и третьего порядка.
\end{frame}



\begin{frame}{Value at Risk (VaR)}
\justify
\alert{Value at Risk (VaR)} --- размер потерь, который с заданной вероятностью (обычно $1\%$ или $5\%$) не будет превышен за заданное время (обычно 1 день или 10 дней). 

\justify
Например, если однодневный 5\% VaR равен $\$1\,000\,000$, то с вероятностью 95\% наш убыток в течение одного дня не превысит один миллион долларов.

\justify
VaR --- ответ на вопрос <<Насколько плохо могут пойти дела?>>, выраженный одним числом и в понятных единицах измерения (доллары, евро).

\justify
Кстати, почему бы не считать 0\% VaR?
\end{frame}



\begin{frame}{Напоминание: функция распределения}
\justify 
\alert{Функцией распределения} (\en{cumulative distribution function}) случайной величины $\xi$ называется функция $F(x)$, такая что
\begin{align*}
F(x) = \mathbb{P}(\xi \le x)
\end{align*}

\justify
\centering
\begin{tikzpicture}
	\begin{axis}[
		width = \textwidth,
		height = 5cm,
		xmin = -3, xmax = 3,
		ymin = 0, ymax = 1,
		grid = major
	]
		
		\addplot[color=Set1-A, thick] table[x=x, y=norm_cdf, col sep=comma] {var_dist_example.csv};
		
		\draw[dashed, thick] (-3, 0.05) -- (-1.645, 0.05) node[anchor=south] {\small (-1.645, 0.05)} -- (-1.645, 0);
		
		\node[fill=Set1-A, circle, inner sep=1.5pt] at (-1.645, 0.05) {};
	\end{axis}
\end{tikzpicture}

\justify
Если будущая прибыль (или убыток) --- случайная величина, то 5\% VaR --- значение обратной функции распределения этой случайной величины $F^{-1}(0.05)$, то есть квантиль.
\end{frame}



\begin{frame}{Напоминание: плотность распределения}
\justify
\alert{Функцией плотности распределения} (\en{probability density function}) абсолютно непрерывной случайной величины $\xi$ называется функция $f(x)$, такая что
\begin{align*}
\mathbb{P}(a \le \xi \le b) = \int\limits_a^b f(x)dx \quad \Rightarrow \quad f(x) = F'(x)
\end{align*}

\justify
\centering
\begin{tikzpicture}
	\begin{axis}[
		width = \textwidth,
		height = 5cm,
		xmin = -3, xmax = 3,
		ymin = 0, ymax = 0.4,
		grid = major
	]
	
		\addplot[color=Set1-A, thick, name path=pdf] table[x=x, y=norm_density, col sep=comma] {var_dist_example.csv};
		
		\path[name path=axis] (axis cs:-3, 0) -- (axis cs:-1.645, 0);
		
		\addplot[fill=Pastel1-A] fill between [of=pdf and axis, soft clip={domain=-3:-1.645}];
		
		\node[anchor=south, inner sep=1pt] at (-2, 0) {\small 5\%};
		
		\draw[dashed, thick] (-3, 0.103) -- (-1.645, 0.103) node[anchor=south] {\small (-1.645, 0.103)} -- (-1.645, 0);
		
		\node[fill=Set1-A, circle, inner sep=1.5pt] at (-1.645, 0.103) {};
	\end{axis}
\end{tikzpicture}
\end{frame}



\begin{frame}{VaR и толстые хвосты}
\justify
VaR не учитывает <<толстые хвосты>> --- маловероятные крупные убытки. На графике оба распределения имеют 5\% VaR $-1.645$. С точки зрения VaR, они несут одинаковый риск.

\justify
\centering
\begin{tikzpicture}
	\begin{axis}[
		width = \textwidth,
		height = 6.5cm,
		xmin = -4, xmax = 5,
		ymin = 0, ymax = 0.4,
		grid = major
	]
	
		\addplot[color=Set1-A, thick, name path=mixture_pdf] table[x=x, y=mixture_density, col sep=comma] {var_dist_example.csv};
		
		\addplot[color=Set1-B, thick, dashed, name path=norm_pdf] table[x=x, y=norm_density, col sep=comma] {var_dist_example.csv};
		
		\path[name path=axis] (axis cs:-4, 0) -- (axis cs:-1.645, 0);
		
		\addplot[fill=Pastel1-A] fill between [of=mixture_pdf and axis, soft clip={domain=-4:-1.645}];
		
		\addplot[fill=Pastel1-B] fill between [of=norm_pdf and axis, soft clip={domain=-4:-1.645}];
		
		%\node[anchor=south, inner sep=1pt] at (-2.5, 0) {\small 5\%};
		
		\node[fill=Set1-A, circle, inner sep=1.5pt] at (-1.645, 0.001) {};
		\node[fill=Set1-B, circle, inner sep=1.5pt] at (-1.645, 0.103) {};
	\end{axis}
\end{tikzpicture}
\end{frame}



\begin{frame}{Expected Shortfall}
\justify
\alert{Expected Shortfall (ES)} --- средний размер потерь при условии, что потери превзойдут соответствующий VaR.

\justify
Например, предположим, что наш однодневный 5\% ES равен $\$1\,000\,000$. В случае, если реализуется один из 5\% наихудших сценариев, мы потеряем в среднем $\$1\,000\,000$ (может быть больше, может быть меньше).

\justify
ES --- ответ на вопрос <<Если дела пойдут плохо, то сколько это будет нам стоить?>>. В отличие от VaR, ES оценивает не один квантиль распределения, а весь <<хвост>>.

\justify
Согласно стандарту Fundamental Review of Trading Book (FRTB), банки будут обязаны считать 1\% ES на несколько горизонтов: 10, 40, 60, 120 дней.
\end{frame}



\begin{frame}{Методы расчёта VaR и ES}
\justify
Вычисление VaR можно разбить на несколько этапов:
\begin{itemize}
\justifying
\item Определить рыночные переменные (риск-факторы), которые влияют на PV нашего портфеля: FX, ставки, волатильности и т.п.
\item Определить совместное распределение будущих изменений риск-факторов.
\item Определить распределение изменений PV портфеля.
\item Вычислить VaR и/или ES исходя из функции распределения PV портфеля.
\end{itemize}

\justify
На каждом этапе мы будем делать те или иные предположения и допущения, и итоговая оценка VaR будет зависеть от использованной модели.
\end{frame}



\begin{frame}{Напоминание: мат. ожидание}
\justify
\alert{Математическое ожидание} (\en{expected value}) случайной величины $\xi: \Omega \to \mathbb{R}$, заданной на вероятностном пространстве $(\Omega, \mathcal{F}, \mathbb{P})$ --- интеграл Лебега
\begin{align*}
\mathbb{E}(\xi) = \int\limits_\Omega\xi(\omega)d\mathbb{P}(\omega)
\end{align*}

\justify
Если у случайной величины $\xi$ есть плотность $f(x)$, то
\begin{align*}
\mathbb{E}(\xi) = \int\limits_{-\infty}^{+\infty}x\cdot f(x)dx
\end{align*}

\justify
Если дискретная случайная величина $\xi$ принимает значения $x_1,...,x_n$ с вероятностями $p_1,...,p_n$, то
\begin{align*}
\mathbb{E}(\xi) = \sum\limits_{i=1}^{n} p_i x_i
\end{align*}
\end{frame}



\begin{frame}{Напоминание: мат. ожидание}
\justify
Математическое ожидание линейно (даже если случайные величины не являются независимыми):
\begin{align*}
\mathbb{E}(a\xi + b\psi) = a\mathbb{E}(\xi) + b\mathbb{E}(\psi)
\end{align*}
для любых случайных величин $\xi$ и $\psi$, для которых существует мат. ожидание.

\justify
Пример: пусть $\xi$ --- результат броска игрального кубика, на котором грани от 1 до 6 выпадают с вероятностями $1/6$.
\begin{align*}
\mathbb{E}(\xi) = \frac{1}{6} \cdot 1 + \frac{1}{6} \cdot 2 + ... + \frac{1}{6} \cdot 6 = 3.5
\end{align*}

\justify
Если у нас два кубика (как в <<Монополии>>), то мат. ожидание суммы очков равно
\begin{align*}
\mathbb{E}(\xi + \xi) = 2\mathbb{E}(\xi) = 2 \cdot 3.5 = 7
\end{align*}
\end{frame}



\begin{frame}{Напоминание: дисперсия}
\justify
\alert{Дисперсия} (\en{variance}) случайной величины $\xi$ --- мера разброса случайной величины относительно мат. ожидания:
\begin{align*}
\operatorname{Var}(\xi) = \mathbb{E}[(\xi - \mathbb{E}(\xi))^2]
\end{align*}

\justify
\alert{Стандартное отклонение} (\en{standard deviation}) --- квадратный корень из дисперсии:
\begin{align*}
\sigma_{\xi} = \sqrt{\operatorname{Var}(\xi)}
\end{align*}

\justify
Если $\xi$ --- результат броска кубика, то
\begin{align*}
\operatorname{Var}(\xi) = \frac{1}{6}(1 - 3.5)^2 + \frac{1}{6}(2 - 3.5)^2 + ... + \frac{1}{6}(6-3.5)^2 \approx 2.92
\end{align*}
\end{frame}



\begin{frame}{Напоминание: нормальное распределение}
\justify
\alert{Нормальное распределение} (\en{normal distribution}) $\mathcal{N}(\mu, \sigma^2)$ с мат. ожиданием $\mu$ и дисперсией $\sigma^2$ имеет плотность
\begin{align*}
\phi(x) = \frac{1}{\sigma\sqrt{2\pi}}e^{-\frac{(x-\mu)^2}{2\sigma^2}}
\end{align*}

\justify
\centering
\begin{tikzpicture}
	\begin{axis}[
		width = \textwidth,
		height = 6cm,
		xmin = -6, xmax = 8,
		ymin = 0, ymax = 0.4,
		grid = major,
		legend entries = {
			{$\mu=0, \sigma=1$},
			{$\mu=1, \sigma=2$}
		}
	]
	
		\addplot[color=Set1-A, thick] table[x=x, y=norm_density, col sep=comma] {var_dist_example.csv};
		
		\addplot[color=Set1-B, thick, dashed] table[x=x, y=norm_density_one_two, col sep=comma] {var_dist_example.csv};

	\end{axis}
\end{tikzpicture}
\end{frame}



\begin{frame}{Напоминание: ковариация и корреляция}
\justify
\alert{Ковариация} (\en{covariance}) показывает линейную связь между случайными величинами:
\begin{align*}
\operatorname{Cov}(\xi, \psi) = \mathbb{E}[(\xi - \mathbb{E}(\xi))(\psi - \mathbb{E}(\psi))]
\end{align*}

\justify
\alert{Коэффициент корреляции Пирсона} $\rho_{\xi,\psi}$ (\en{Pearson correlation}):
\begin{align*}
\rho_{\xi, \psi} = \frac{\operatorname{Cov}(\xi, \psi)}{\sqrt{\operatorname{Var}(\xi)\operatorname{Var}(\psi)}}
\end{align*}

\justify
Если обозначить $\sigma_\xi$ и $\sigma_\psi$ стандартные отклонения, то
\begin{align*}
\operatorname{Cov}(\xi, \psi) = \rho_{\xi, \psi}\sigma_\xi\sigma_\psi
\end{align*}

Дисперсия линейной комбинации:
\begin{align*}
\operatorname{Var}(a\xi + b\psi) &= a^2\operatorname{Var}(\xi) + b^2\operatorname{Var}(\psi) + 2\operatorname{Cov}(\xi, \psi) = \\
&= a^2\sigma_\xi^2 + b^2\sigma_\psi^2 + 2ab\rho_{\xi,\psi}\sigma_{\xi}\sigma_{\psi}
\end{align*}
\end{frame}



    \newcommand{\addCovarianceSubplot}[3]{
        \nextgroupplot[
            title = {$\rho = #1$}
        ]
        \addplot[
            color = Set1-B,
            only marks
        ]
        table[
            x = #2,
            y = #3,
            col sep=comma
        ]
        {covariance_plot_random_samples.csv};
    }

\begin{frame}{Напоминание: корреляция}
\centering
\begin{tikzpicture}
\begin{groupplot}[
    width = \textwidth / 1.8,
    height = \textwidth / 1.8,
    group style = {group size = 2 by 1},
    xmin = -4, xmax=4,
    ymin = -4, ymax=4
]
    
    %\addCovarianceSubplot{-0.75}{X-75}{Y-75}
    
    \addCovarianceSubplot{0}{X0}{Y0}
    
    \addCovarianceSubplot{0.75}{X75}{Y75}
    
\end{groupplot}
\end{tikzpicture}
\end{frame}



\begin{frame}{Аналитическая формула}
\justify
DV01 (dollar value of 1 basis point) --- другое название грека <<Ро>>, чувствительность цены портфеля к изменению процентных ставок или доходностей облигаций. Как изменится $PV$, если ставки вырастут на 1 базисный пункт (0.01\%)?

\justify
Мы купили десятилетних облигаций на DV01 $-100$ и продали пятилетних на DV01 $+100$.

\justify
Предположим, что будущие однодневные изменения доходностей облигаций --- две случайные величины $\xi_5$ и $\xi_{10}$, которые имеют совместное нормальное распределение.

\justify
\centering
\begin{tabular}{l|c|c}
 & $\xi_5$ & $\xi_{10}$ \\ \hline
Мат. ожидание, б.п. & 0.20 & 0.25 \\
Стандартное отклонение, б.п. & 2.0 & 2.5 \\ \hline
Корреляция & \multicolumn{2}{c}{0.9}
\end{tabular}

\justify
Чему равен наш однодневный 5\% VaR?
\end{frame}



\begin{frame}{Аналитическая формула}
\centering
\begin{tikzpicture}
	\begin{axis}[
		height = \textheight - 1cm,
		grid = major,
		xmin = -6, xmax = 6,
		ymin = -6, ymax = 6,
		xlabel = {\small $\xi_5$ --- изменение дох-ти 5Y облигации, б.п.},
		ylabel = {\small $\xi_{10}$ --- изменение дох-ти 10Y облигации, б.п.}
	]
	
	\addplot[color=Set1-B, only marks] table[x=x, y=y, col sep=comma] {var_example_yields.csv};
	
	\draw[thick] (-10, 0) -- (10, 0);
	\draw[thick] (0, -10) -- (0, 10);
	
	\end{axis}
\end{tikzpicture}
\end{frame}



\begin{frame}{Аналитическая формула}
\centering
\begin{tabular}{l|c|c}
 & $\xi_5$ & $\xi_{10}$ \\ \hline
DV01, USD/б.п. & 100 & $-100$ \\
Мат. ожидание, б.п. & $0.20$ & $0.25$ \\
Стандартное отклонение, б.п. & $2.0$ & $2.5$ \\ \hline
Корреляция & \multicolumn{2}{c}{0.9}
\end{tabular}

\justify
Завтрашнее изменение PV портфеля --- случайная величина $\psi$:
\begin{align*}
\psi = 100\xi_5 - 100\xi_{10}
\end{align*}

\justify
Сумма двух нормальных величин --- тоже нормальная случайная величина с параметрами
\begin{align*}
\mu &= \mathbb{E}(100\xi_5 - 100\xi_{10}) = 100\mathbb{E}(\xi_5) - 100\mathbb{E}(\xi_{10}) = \\
&= 100 \cdot 0.20 - 100 \cdot 0.25 = \alert{-5.0} \\
\sigma^2 &= \operatorname{Var}(100\xi_5 - 100\xi_{10}) = \\
&= 100^2\operatorname{Var}(\xi_5) + 100^2\operatorname{Var}(\xi_{10}) + 2 \cdot 100 \cdot(-100) \cdot \operatorname{Cov}(\xi_5, \xi_{10}) = \\
&= 100^2 \cdot 2.0^2 + 100^2\cdot 2.5^2 - 2 \cdot 100^2 \cdot 0.9 \cdot 2.0 \cdot 2.5
\approx \alert{111.80^2}
\end{align*}
%Первый перцентиль такого распределения: $-\$265$.
\end{frame}



\begin{frame}{Аналитическая формула}
\centering
\begin{tikzpicture}
	\begin{axis}[
		width = \textwidth,
		height = 4cm,
		xmin = -400, xmax = 400,
		ymin = 0,
		grid = major,
		xlabel = {Однодневная прибыль или убыток, \$},
		ylabel = {Плотность},
		yticklabel = {\pgfmathprintnumber[fixed, fixed zerofill, precision=0]{\tick}}
	]
	
		\addplot[color=Set1-A, thick, name path=pdf] table[x=x, y=y, col sep=comma] {var_example_portfolio_density.csv};
		
		\path[name path=axis] (axis cs:-400, 0) -- (axis cs:-188.90, 0);
		
		\addplot[fill=Pastel1-A] fill between [of=pdf and axis, soft clip={domain=-400:-188.90}];
		
		\node[anchor=south, inner sep=1pt] at (-220, 0) {\small 5\%};
		
		\draw[dashed, thick] (-400, 0.00092) -- (-188.90, 0.00092) node[anchor=west] {\small (-188.90, 0.00092)} -- (-188.90, 0);
		
		\node[fill=Set1-A, circle, inner sep=1.5pt] at (-188.90, 0.00092) {};
	\end{axis}
\end{tikzpicture}

\justify
5-й перцентиль стандартного нормального распределения $\mathcal{N}(0, 1)$ равен $-1.645$. 5-й перцентиль нормального распределения с мат. ожиданием $\mu$ и дисперсией $\sigma^2$ равен
\begin{align*}
\mu - 1.645\sigma = -5 - 111.80\cdot 1.645 \approx -188.90
\end{align*}

\justify
Это и есть 5\% VaR: с вероятностью 95\% убыток будет не хуже, чем $-\$188.90$.
\end{frame}



\begin{frame}{Аналитическая формула}
\justify
Достоинства:
\begin{itemize}
\justifying
\item Очень легко считать.
\end{itemize}

\justify
Недостатки:
\begin{itemize}
\justifying
\item Постулирует нормальность всех распределений. Распределения из реального мира имеют <<толстые хвосты>>.
\item Оценка изменения PV по частным производным хорошо работает с <<простыми>> линейными деривативами (свопы, форварды) и не очень хорошо с опционами.
\end{itemize}

\justify
Не пытайтесь повторить это дома!

\justify
Кстати, как мы можем оценить распределение \alert{будущих} изменений рыночных переменных?
\end{frame}



\begin{frame}{Исторические симуляции}
\justify
Пусть нашем портфеле есть только деривативы на фондовый индекс S\&P 500. Прибыль или убыток зависят от трёх рыночных переменных (риск-факторов):
\begin{itemize}
\justifying
\item Базовый актив --- индекс S\&P 500.
\item Волатильность --- индекс \en{implied}\ волатильности VIX.
\item Безрисковая ставка --- трёхмесячный долларовый LIBOR.
\end{itemize}

\justify
* Дивидендную доходность индекса будем считать 0\%.
\end{frame}



\begin{frame}{Исторические симуляции: S\&P 500}
\centering
\begin{tikzpicture}
	\begin{axis}[
		width = \textwidth,
		height = \textheight - 1.75cm,
		date coordinates in=x,
		xmin = 2019-11-11, xmax = 2021-11-16,
		ymin = 2000, ymax = 4800,
		grid = major,
		xtick = {2020-01-01, 2020-04-01, 2020-07-01, 2020-10-01, 2021-01-01, 2021-04-01, 2021-07-01, 2021-10-01},
		xticklabel = {\month.\year},
		xticklabel style = {rotate=45, anchor=east},
	]
	
	\addplot[color=Set1-B, thick] table[x=date, y=sp500, col sep=comma] {var_sp500_data.csv};
	\end{axis}
\end{tikzpicture}

\centering
\small Данные: St. Louis Fed (SP500).
\end{frame}



\begin{frame}{Исторические симуляции: волатильность}
\centering
\begin{tikzpicture}
	\begin{axis}[
		width = \textwidth,
		height = \textheight - 1.75cm,
		date coordinates in=x,
		xmin = 2019-11-11, xmax = 2021-11-16,
		ymin = 0, ymax = 0.8,
		grid = major,
		xtick = {2020-01-01, 2020-04-01, 2020-07-01, 2020-10-01, 2021-01-01, 2021-04-01, 2021-07-01, 2021-10-01},
		xticklabel = {\month.\year},
		xticklabel style = {rotate=45, anchor=east},
		yticklabel = {\pgfmathparse{\tick*100}\pgfmathprintnumber[precision=0]{\pgfmathresult}\%}
	]
	
	\addplot[color=Set1-B, thick] table[x=date, y=vix, col sep=comma] {var_sp500_data.csv};
	\end{axis}
\end{tikzpicture}

\centering
\small Данные: St. Louis Fed (VXVCLS).
\end{frame}



\begin{frame}{Исторические симуляции: безрисковая ставка}
\centering
\begin{tikzpicture}
	\begin{axis}[
		width = \textwidth,
		height = \textheight - 1.75cm,
		date coordinates in=x,
		xmin = 2019-11-11, xmax = 2021-11-16,
		ymin = 0, ymax = 0.02,
		grid = major,
		xtick = {2020-01-01, 2020-04-01, 2020-07-01, 2020-10-01, 2021-01-01, 2021-04-01, 2021-07-01, 2021-10-01},
		xticklabel = {\month.\year},
		xticklabel style = {rotate=45, anchor=east},
		yticklabel = {\pgfmathparse{\tick*100}\pgfmathprintnumber[fixed, fixed zerofill, precision=1]{\pgfmathresult}\%},
		scaled y ticks = false
	]
	
	\addplot[color=Set1-B, thick] table[x=date, y=libor, col sep=comma] {var_sp500_data.csv};
	\end{axis}
\end{tikzpicture}

\centering
\small Данные: St. Louis Fed (USD3MTD156N).
\end{frame}



\begin{frame}{Исторические симуляции}

\justify
Предположим, что в будущем рыночные переменные будут следовать тому же распределению, что и в прошлом. Выберем некоторый период истории, например, последние 2 года, или стрессовый период 2008--2009. Построим выборку изменений рыночных переменных.
 
\justify
\centering
\begin{tabular}{l|r|r|r}
           & S\&P 500 & Вол-ть   & Ставка \\ \hline
02.09.2020 & 3\,581   & 32.7\%   & 0.251\% \\
03.09.2020 & 3\,455   & 37.6\%   & 0.250\% \\ \hline
Изменение  & $-3.5\%$ & +4.9\%   & $-0.001\%$ 
\end{tabular}

\justify
Если повторится 3 сентября 2020 года, то индекс S\&P 500 упадёт на 3.5\% относительного \alert{сегодняшнего} уровня, волатильность вырастет на 4.9 процентных пункта, процентная ставка упадёт на 0.001 процентных пункта.

\end{frame}



\begin{frame}{Исторические симуляции}
\justify
Предположим, что 3 сентября 2020 года повторится завтра (произойдут такие же изменения рыночных переменных).

\justify
\centering
\begin{tabular}{l|r|r|r}
                      & S\&P 500 & Вол-ть   & Ставка \\ \hline
Сегодня 25.11.2021    & 4\,701   & 20.9\%   & 0.160\% \\
Изменение             & $-3.5\%$ & +4.9\%   & $-0.001\%$ \\ \hline
<<Завтра 03.09.2020>> & 4\,536   & 25.8\%   & 0.159\%
\end{tabular}

\justify
Оценим наш портфель в этом сценарии и посчитаем нашу гипотетическую прибыль или убыток.

\justify
\centering
\begin{tabular}{l|r}
Сегодня               & \$19\,589 \\
<<Завтра 03.09.2020>> & \$23\,323 \\ \hline
Прибыль/убыток        & $+\$3\,734$
\end{tabular}
\end{frame}



\begin{frame}{Исторические симуляции}
\justify
Если у нас есть 2 года исторических наблюдений (примерно 500 рабочих дней), то мы сможем составить выборку из 500 сценариев. Убыток в 5-м наихудшем из них будет 1\% VaR.

\justify
Достоинства:
\begin{itemize}
\justifying
\item Не требует предположения о форме распределения.
\item Точно учитывает нелинейность продуктов.
\end{itemize}

\justify
Недостатки:
\begin{itemize}
\justifying
\item Вычислительно более сложен (требует переоценки каждой сделки в каждом сценарии).
\item Требует предположения о стационарности распределений и корреляций.
\item Если распределение имеет толстые хвосты, то выборки в 250--500 наблюдений может не хватить для надёжной оценки квантилей.
\end{itemize}
\end{frame}



\begin{frame}{Исторические симуляции}
\justifying
Демонстрация: портфель из опциона и базового актива.
\end{frame}



\begin{frame}{Метод Монте-Карло}
\justifying
Предположим, что рыночные переменные следуют некоторому аналитическому распределению с <<хорошими>> свойствами --- толстыми хвостами, скошенностью, и т.п.

\justify
Примеры распределений: Стьюдента, Коши, логистическое...

\justify
Оценим параметры этого распределения по исторической выборке. После этого сгенерируем много (десятки тысяч) случайных реализаций этого распределения. Переоценим наш портфель в каждом из сценариев, чтобы вычислить VaR.
\end{frame}



\begin{frame}{Дневные колебания S\&P 500}
\centering
\begin{tikzpicture}
	\begin{axis}[
		width = \textwidth,
		height = \textheight - 1.75cm,
		date coordinates in=x,
		xmin = 2019-11-11, xmax = 2021-11-16,
		ymin = -0.12, ymax = 0.1,
		grid = major,
		xtick = {2020-01-01, 2020-04-01, 2020-07-01, 2020-10-01, 2021-01-01, 2021-04-01, 2021-07-01, 2021-10-01},
		xticklabel = {\month.\year},
		xticklabel style = {rotate=45, anchor=east},
		yticklabel = {\pgfmathparse{\tick*100}\pgfmathprintnumber[fixed, fixed zerofill, precision=0]{\pgfmathresult}\%},
		scaled y ticks = false
	]
	
	\addplot[color=Set1-B, thick] table[x=date, y=sp500_chg, col sep=comma] {var_sp500_data.csv};
	\end{axis}
\end{tikzpicture}

\centering
\small Данные: St. Louis Fed (SP500).
\end{frame}



\begin{frame}{Дневные колебания S\&P 500}
\centering
\begin{tikzpicture}
	\begin{axis}[
		width = \textwidth,
		height = \textheight - 1.5cm,
		xmin = -0.15, xmax = 0.15,
		ymin = 0, ymax = 50,
		xticklabel = {\pgfmathparse{\tick*100}\pgfmathprintnumber[fixed, fixed zerofill, precision=0]{\pgfmathresult}\%},
		scaled x ticks = false,
		grid = major,
		xlabel = {\small Дневные изменения},
		ylabel = {Плотность распределения},
		legend entries = {
			S\&P 500,
			Норм. распр.
		}
	]
		\addplot[color=Set1-B, thick] table[x=x, y=sp500_density, col sep=comma] {var_sp500_changes_density.csv};
	
		\addplot[color=Set1-A, thick, dashed] table[x=x, y=norm_density, col sep=comma] {var_sp500_changes_density.csv};
	
	\end{axis}
\end{tikzpicture}
\end{frame}



\begin{frame}{Распределение Стьюдента}
\justify
Распределение Стьюдента ($t$-распределение) с $\nu$ степенями свободы часто встречается в статистике. Чем меньше степеней свободы, чем толще хвосты.

\justify
\centering
\begin{tikzpicture}
	\begin{axis}[
		width = \textwidth,
		height = \textheight - 2.5cm,
		xmin = -5, xmax = 5,
		ymin = 0, ymax = 0.4,
		%xticklabel = {\pgfmathparse{\tick*100}\pgfmathprintnumber[fixed, fixed zerofill, precision=0]{\pgfmathresult}\%},
		%scaled x ticks = false,
		grid = major,
		%xlabel = {\small Дневные изменения},
		%ylabel = {Плотность распределения},
		legend entries = {
			$\nu=100$,
			$\nu=5$,
			$\nu=2$
		}
	]
	
		\addplot[color=Set1-A, thick] table[x=x, y=t100, col sep=comma] {var_t_dist_example.csv};
	
		\addplot[color=Set1-B, thick] table[x=x, y=t5, col sep=comma] {var_t_dist_example.csv};
		
		\addplot[color=Set1-C, thick] table[x=x, y=t2, col sep=comma] {var_t_dist_example.csv};
	\end{axis}
\end{tikzpicture}
\end{frame}



\begin{frame}{Метод моментов}
\justify
Случайная величина $\xi$ --- дневное изменение индекса. Предположим, что она следует следует обобщённому распределению Стьюдента с $\nu$ степенями свободы:
\begin{align*}
\xi \sim \mu + \sigma T_{\nu}
\end{align*}

\justify
Моменты $\xi$:
\begin{align*}
\mathbb{E}(\xi) = \mu, \quad
\operatorname{Var}(\xi) = \sigma^2\frac{\nu}{\nu-2}, \quad
\operatorname{Kurt}(\xi) = \frac{6}{\nu - 4}
\end{align*}

\justify
Подгоним параметры $\mu$, $\sigma$ и $\nu$ так, чтобы выборочные моменты по исторической выборке совпадали с теоретическими.

\justify
Оценка максимального правдоподобия (\en{maximum likelihood estimation, MLE}) работает лучше, но выходит за рамки лекции.
\end{frame}



\begin{frame}{Дневные колебания S\&P 500}
\centering
\begin{tikzpicture}
	\begin{axis}[
		width = \textwidth,
		height = \textheight - 1.5cm,
		xmin = -0.15, xmax = 0.15,
		ymin = 0, ymax = 50,
		xticklabel = {\pgfmathparse{\tick*100}\pgfmathprintnumber[fixed, fixed zerofill, precision=0]{\pgfmathresult}\%},
		scaled x ticks = false,
		grid = major,
		xlabel = {\small Дневные изменения},
		ylabel = {Плотность распределения},
		legend entries = {
			S\&P 500,
			$t$-распр.
		}
	]
		\addplot[color=Set1-B, thick] table[x=x, y=sp500_density, col sep=comma] {var_sp500_changes_density.csv};
	
		\addplot[color=Set1-A, thick, dashed] table[x=x, y=t_density, col sep=comma] {var_sp500_changes_density.csv};
	
	\end{axis}
\end{tikzpicture}
\end{frame}

\begin{frame}{Метод Монте-Карло}
\justifying
Демонстрация: портфель из опциона и базового актива.
\end{frame}



\begin{frame}{Метод Монте-Карло}
\justify
Достоинства:
\begin{itemize}
\justifying
\item  Позволяет получить больше реализаций редких событий.
\end{itemize}

\justify
Недостатки:
\begin{itemize}
\justifying
\item Необходимо выбрать параметрическое семейство распределений. 
\item Требует большого количества симуляций для приемлемой сходимости.
\end{itemize}

\justify
\textit{``Prediction is very difficult, especially about the future.``}
(приписывается Нильсу Бору)
\end{frame}


\end{document}


