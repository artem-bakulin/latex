\documentclass{beamer}

\usepackage{cmap}				% To be able to copy-paste russian text from pdf
\usepackage[T2A]{fontenc}
\usepackage[utf8]{inputenc}
\usepackage[russian]{babel}
\usepackage{textpos}
\usepackage{ragged2e}
\usepackage{amssymb}
\usepackage{ulem}
\usepackage{tikz}
\usepackage{pgfplots}
\usepackage{color}
\usepackage{cancel}
\usepackage{multirow}
\pgfplotsset{compat=1.17}
\usetikzlibrary{arrows,snakes,backgrounds,shapes}
\usepgfplotslibrary{groupplots,colorbrewer,dateplot,statistics}
\usepackage{animate}

\usepackage{amsfonts}
\usepackage{amsmath}
\usepackage{amssymb}
\usepackage{graphicx}
\usepackage{setspace}
\usepackage{tabularx}

\usepackage{enumitem}
\setitemize{label=\usebeamerfont*{itemize item}%
  \usebeamercolor[fg]{itemize item}
  \usebeamertemplate{itemize item}}

% remove navigation bar
\setbeamertemplate{navigation symbols}{} 

\setbeamertemplate{page number in head/foot}[totalframenumber] 

\usepackage{eurosym}
\renewcommand{\EUR}[1]{\textup{\euro}#1}

\title{Monte-Carlo Method}
\author{Artem Bakulin}
\date{November 23, 2023}

\usetheme{Warsaw}
\usecolortheme{beaver}

\newcommand{\ru}[1]{\begin{otherlanguage}{russian}#1\end{otherlanguage}}
\newcommand{\en}[1]{\begin{otherlanguage}{english}#1\end{otherlanguage}}
\newcommand{\ruen}[2]{#1 (\en{#2})}

% https://tex.stackexchange.com/questions/98003/filter-rows-from-a-table
\pgfplotsset{
    discard if not/.style 2 args={
        x filter/.code={
            \edef\tempa{\thisrow{#1}}
            \edef\tempb{#2}
            \ifx\tempa\tempb
            \else
                \def\pgfmathresult{inf}
            \fi
        }
    }
}



\begin{document}



\begin{frame}
\titlepage
\end{frame}



\newcommand{\drawStockNode}[5]{

	\node (#5)
	[
		draw,
		rectangle,
		rounded corners,
		inner sep = 0pt,
		outer sep = 0pt,
		minimum width = 2.4cm,
		minimum height = 0.55cm,
		align = center
	]
	at (#3, #4)
	{
		\begin{tabular}{c|c}
		#1 & #2
		\end{tabular}
	};
}

\newcommand{\drawStockLink}[4]{

	\draw[
		->,
		>=triangle 90
	]
	(#1.east) -- (#2.west)
	node[
		pos = 0.5,
		anchor = #4
	]
	{#3};
}

\newcommand{\drawOneStepBinomialTree}{
	\drawStockNode{$S_0$} {?}{0}{ 0}{S0_node}
	\drawStockNode{$S_0u$}{$V_u$}{4}{ 1}{Su_node}
	\drawStockNode{$S_0d$}{$V_d$}{4}{-1}{Sd_node}
	
	\drawStockLink{S0_node}{Su_node}{$p$}{south east}	
	\drawStockLink{S0_node}{Sd_node}{$1-p$}{north east}
}



\begin{frame}{Recap: binomial model}
\centering
\begin{tikzpicture}
	\drawOneStepBinomialTree
\end{tikzpicture}

\justify
\begin{itemize}
\justifying
\item Current stock price is $S_0$.
\item Stock price can either increase to $S_0\cdot u$ (u>1) or drop to $S_0 \cdot d$ (d<1).
\item Single period of $\tau$ years, risk-free interest rate is $r$, such that $d < 1+r\tau < u$.
\item An option pays off (has the value of) either $V_u$ or $V_d$ depending on the stock price moving up or down.
\end{itemize}
\end{frame}



\begin{frame}{Recap: binomial model - 2}
\centering
\begin{tikzpicture}
	\drawOneStepBinomialTree
\end{tikzpicture}

\justify
Consider a portfolio of $\Delta$ stock and debt $L$. Pick $\Delta$ and $L$ so that this portfolio replicates the option
\begin{equation*}
\begin{cases}
L(1+r\tau) + \Delta S_0 u = V_u \\
L(1+r\tau) + \Delta S_0 d = V_d
\end{cases}
\end{equation*}

\begin{equation*}
\begin{cases}
\Delta = \dfrac{V_u - V_d}{S_0(u-d)} \\
L = \dfrac{V_du - V_ud}{(1+r\tau)(u-d)}
\end{cases}
\end{equation*}
\end{frame}



\begin{frame}{Recap: binomial model - 3}
\centering
\begin{tikzpicture}
\drawOneStepBinomialTree
\end{tikzpicture}

\justify
The option is worth the same as the replicating portfolio:
\begin{align*}
C &= \Delta S_0 +L = \\
 &= \dfrac{V_u-V_d}{(u-d)\cancel{S_0}}\cancel{S_0} + \dfrac{V_du -V_ud}{(1+r\tau)(u-d)} = \\
 &= \dfrac{qV_u +(1-q)V_d}{1+r\tau},
\end{align*}
where
\begin{equation*}
q = \dfrac{1+r\tau - d}{u-d}
\end{equation*}
This magic parameter $q$ is called the "risk-neutral probability".
\end{frame}



\renewcommand{\drawStockLink}[2]{

	\draw[
		->,
		>=triangle 45
	]
	(#1.east) -- (#2.west)
	{};
}

\renewcommand{\drawStockNode}[5]{

	\node (#5)
	[
		draw,
		rectangle,
		rounded corners,
		inner sep = 1pt,
		outer sep = 0pt,
		minimum width = 1.5cm
	]
	at (#3, #4)
	{
		\centering
		\begin{tabular}{c}
		#1 \\ \hline #2
		\end{tabular}
	};
}

\newcommand{\nodeVerticalStep}{0.7}
\newcommand{\nodeHorizontalStep}{2.75}

\begin{frame}{Recap: binomial model - 4}
\centering
\begin{tikzpicture}
\drawStockNode{$\$100$}{\only<1-7>{?}\only<8->{\$14.8}}{0}{0}{S0_node}

\drawStockNode{$\$120$}{\only<1-5>{?}\only<6->{\$25.8}}{\nodeHorizontalStep}{\nodeVerticalStep}{Su_node}
\drawStockNode{$\$80$}{\only<1-6>{?}\only<7->{\$3.8}}{\nodeHorizontalStep}{-\nodeVerticalStep}{Sd_node}

\drawStockNode{$\$144$}{\only<1-2>{?}\only<3->{\$44}}{2*\nodeHorizontalStep}{2*\nodeVerticalStep}{Suu_node}
\drawStockNode{$\$96$}{\only<1-3>{?}\only<4->{\$7.6}}{2*\nodeHorizontalStep}{0}{Sud_node}
\drawStockNode{$\$64$}{\only<1-4>{?}\only<5->{\$0}}{2*\nodeHorizontalStep}{-2*\nodeVerticalStep}{Sdd_node}

\drawStockNode{$\$172.8$}{\only<1>{?}\only<2->{\$72.8}}{3*\nodeHorizontalStep}{3*\nodeVerticalStep}{Suuu_node}
\drawStockNode{$\$115.2$}{\only<1>{?}\only<2->{\$15.2}}{3*\nodeHorizontalStep}{\nodeVerticalStep}{Suud_node}
\drawStockNode{$\$76.8$}{\only<1>{?}\only<2->{\$0}}{3*\nodeHorizontalStep}{-\nodeVerticalStep}{Sudd_node}
\drawStockNode{$\$51.2$}{\only<1>{?}\only<2->{\$0}}{3*\nodeHorizontalStep}{-3*\nodeVerticalStep}{Sddd_node}

\drawStockLink{S0_node}{Su_node}
\drawStockLink{S0_node}{Sd_node}

\drawStockLink{Su_node}{Suu_node}
\drawStockLink{Su_node}{Sud_node}

\drawStockLink{Sd_node}{Sud_node}
\drawStockLink{Sd_node}{Sdd_node}

\drawStockLink{Suu_node}{Suuu_node}
\drawStockLink{Suu_node}{Suud_node}

\drawStockLink{Sud_node}{Suud_node}
\drawStockLink{Sud_node}{Sudd_node}

\drawStockLink{Sdd_node}{Sudd_node}
\drawStockLink{Sdd_node}{Sddd_node}
\end{tikzpicture}

\justify
Suppose that $u=1.2$, $d=0.8$, $S_0=\$100$, $r=0\%$. What is fair value of a call option at strike $K=100$?

\justify
The "risk-neutral probability":
\begin{align*}
q = \dfrac{1+r\tau - d}{u - d} = \dfrac{1 - 0.8}{1.2 - 0.8} = 0.5
\end{align*}
\end{frame}





\newcommand{\highlightStockLink}[6]{
	\draw[
		color=#4,
		very thick,
		->,
		>=triangle 45
	]
	(#1.east) -- (#2.west)
	node[
		pos=#5,
		anchor=#6
	]
	{#3};
}

\newcommand{\highlightStockLinkUp}[3]{
	\highlightStockLink{#1}{#2}{$q$}{#3}{0.5}{south}
}

\newcommand{\highlightStockLinkDown}[3]{
	\highlightStockLink{#1}{#2}{$1-q$}{#3}{0.15}{west}
}

\begin{frame}{Recap: risk-neutral probability}
\centering
\begin{tikzpicture}
\drawStockNode{$S_0$}{?}{0}{0}{S0_node}

\drawStockNode{$S_0u$}{?}{\nodeHorizontalStep}{\nodeVerticalStep}{Su_node}
\drawStockNode{$S_0d$}{?}{\nodeHorizontalStep}{-\nodeVerticalStep}{Sd_node}

\drawStockNode{$S_0u^2$}{?}{2*\nodeHorizontalStep}{2*\nodeVerticalStep}{Suu_node}
\drawStockNode{$S_0ud$}{?}{2*\nodeHorizontalStep}{0}{Sud_node}
\drawStockNode{$S_0d^2$}{?}{2*\nodeHorizontalStep}{-2*\nodeVerticalStep}{Sdd_node}

\drawStockNode{$S_0u^3$}{$V_3$}{3*\nodeHorizontalStep}{3*\nodeVerticalStep}{Suuu_node}
\drawStockNode{$S_0u^2d$}{$V_2$}{3*\nodeHorizontalStep}{\nodeVerticalStep}{Suud_node}
\drawStockNode{$S_0ud^2$}{$V_1$}{3*\nodeHorizontalStep}{-\nodeVerticalStep}{Sudd_node}
\drawStockNode{$S_0d^3$}{$V_0$}{3*\nodeHorizontalStep}{-3*\nodeVerticalStep}{Sddd_node}

\only<1-2>{
	\drawStockLink{S0_node}{Su_node}
	\drawStockLink{S0_node}{Sd_node}

	\drawStockLink{Su_node}{Suu_node}
	\drawStockLink{Su_node}{Sud_node}

	\drawStockLink{Sd_node}{Sud_node}
	\drawStockLink{Sd_node}{Sdd_node}

	\drawStockLink{Suu_node}{Suuu_node}
	\drawStockLink{Suu_node}{Suud_node}

	\drawStockLink{Sud_node}{Suud_node}
	\drawStockLink{Sud_node}{Sudd_node}

	\drawStockLink{Sdd_node}{Sudd_node}
	\drawStockLink{Sdd_node}{Sddd_node}
}

\only<3>{
	\highlightStockLinkUp{S0_node}{Su_node}{Set1-A}
	\highlightStockLinkUp{Su_node}{Suu_node}{Set1-A}
	\highlightStockLinkDown{Suu_node}{Suud_node}{Set1-A}
}

\only<4>{
	\highlightStockLinkUp{S0_node}{Su_node}{Set1-A}
	\highlightStockLinkDown{Su_node}{Sud_node}{Set1-A}
	\highlightStockLinkUp{Sud_node}{Suud_node}{Set1-A}
}

\only<5>{
	\highlightStockLinkDown{S0_node}{Sd_node}{Set1-A}
	\highlightStockLinkUp{Sd_node}{Sud_node}{Set1-A}
	\highlightStockLinkUp{Sud_node}{Suud_node}{Set1-A}
}

\end{tikzpicture}

\justify
Risk-neutral probability: $q = \dfrac{1 + rT - d}{u - d}$.

\justify
Fair value of an option today:
\begin{align*}
V = \frac{q^3V_3 + \only<1>{3q^2(1-q)}\only<2->{\alert{3q^2(1-q)}}V_2 + 3q(1-q)^2V_1 + (1-q)^3V_0}{(1+rT)^3}
\end{align*}
\end{frame}


\begin{frame}{Recap: risk-neutral probability - 2}
\justify
Close your eyes and pretend that $q$ is the probability of the sock price moving up (in reality it is not). Then $3q^2(1-q)$ is probability of the stock moving up twice and moving down once. In this scenario the stock is worth  $S_0u^2d$, and the option pays off $V_2$.

\justify
\centering
\begin{tabular}{l|l|l}
Stock price & Payoff & "Probability" \\ \hline
$S_0u^3$   & $V_3$   & $q^3$ \\
$S_0u^2d$  & $V_2$   & $3q^2(1-q)$ \\
$S_0ud^2$  & $V_1$   & $3q(1-q)^2$ \\ 
$S_0d^3$   & $V_0$   & $(1-q)^3$ 
\end{tabular}

\justify
Fair value of an option looks similar to the discounted "expected"\ payoff.
\begin{align*}
V = \frac{q^3V_3 + 3q^2(1-q)V_2 + 3q(1-q)^2V_1 + (1-q)^3V_0}{(1+rT)^3}
\end{align*}
\end{frame}



\begin{frame}{Stock price}
\justify
A stock is equivalent to a call option at strike 0. This option's terminal payoff is  $V(S) = \max(S-0, 0) = S$.
\begin{align*}
S &= \frac{q^3S_0u^3 + 3q^2(1-q)S_0u^2d + 3q(1-q)^2S_0ud^2 + (1-q)^3S_0d^3}{(1+rT)^3} = \\
&= S_0\frac{q^3u^3 + 3q^2u^2(1-q)d + 3qu(1-q)^2d^2 + (1-q)^3d^3}{(1+rT)^3} = \\
&= S_0\frac{\Big(qu + (1-q)d \Big)^3}{(1+rT)^3} 
= S_0\frac{\left(\dfrac{1+rT-d}{u-d}u + \dfrac{u-1-rT}{u-d}d \right)^3}{(1+rT)^3} = \\
&= S_0 \frac{\left(\dfrac{(1+rT)(u-d)}{u-d} \right)^3}{(1+rT)^3} = S_0
\end{align*}

\justify
A stock is worth its discounted "expected value".
\end{frame}



\begin{frame}{Bond price}
\justify
Consider a risk-free bond that will pay \$1 at the third step. A bond is a cbombination of a digital call and a digital put. This derivative has a payoff of $V(S) = 1$.
\begin{align*}
B &= \frac{q^3 \cdot 1 + 3q^2(1-q) \cdot 1 + 3q(1-q)^2 \cdot 1 + (1-q)^3 \cdot 1}{(1+rT)^3} = \\
&= \frac{\Big(q + (1-q) \Big)^3}{(1+rT)^3} = \frac{1}{(1+rT)^3}
\end{align*}

\justify
A risk-free bond is worth its discounted "expected value".
\end{frame}



\begin{frame}{Risk-neutral probability}
\justify
To calculate the coefficients $q$ and $1-q$, we constructed a replicating portfolio and assumed the absence of arbitrage opportunities.

\justify
The value of $q$ follows mechanically from the parameters of the problem, $u$ and $d$ (potential upward and downward price changes) and $r$ (risk-free interest rate).

\justify
It turned out that the prices of all three assets (a derivative, stock, risk-free bond) are equal to the discounted "expected value"\ of future payouts if we substitute "risk-neutral probabilities" $q$ and $1-q$ instead of true probabilities.

\justify
Is this a coincidence?
\end{frame}



\begin{frame}{Fundamental theorem}
\justify
The following \alert{Fundamental Theorem of Asset Pricing (FATP)} holds true:

\justify
In a market with discrete time*, there are no arbitrage opportunities if and only if there exists a risk-neutral probability measure $\mathbb{Q}$ that is equivalent** to the real-world probability measure $\mathbb{P}$.

\justify
*There is a similar theorem in continuous time.

\justify
**Two probability measures, $\mathbb{P}$ and $\mathbb{Q}$, are equivalent if it holds true for any event $A$ that 
\begin{align*}
\mathbb{P}(A)=0 \quad \Leftrightarrow \quad \mathbb{Q}(A)=0
\end{align*}
Measures $\mathbb{P}$ and $\mathbb{Q}$ agree on which events are hypothetically possible but may assign different probabilities to these events.
\end{frame}



\begin{frame}{Risk-neutral measure}
\justify
What makes a probability measure $\mathbb{Q}$ risk-neutral?

\justify
In a risk-neutral world all assets are priced at their discounted expected values. In other words, on average, all assets grow at the risk-free interest rate (less dividend yield when applicable).

\justify
This would be possible if investors were only interested in the expected value of returns, meaning they were \alert{risk-neutral}.

\justify
Is it true that the real world is populated by risk-neutral robots? No! The average investor in the real market is \alert{risk-averse}.
\end{frame}



\begin{frame}{Test your risk-neutrality}
\justify
You must invest all your savings for 1 year. You may choose between risk-free bonds and risky stocks. What would you choose?

\justify
\centering
\begin{tabular}{l|r|r|r}
Scenario & Probability & Bonds & Stocks \\ \hline
Good & 50\%   & $+2\%$    & $+22\%$  \\
Bad   & 50\%   & $+2\%$    & $-18\%$  \\ \hline
\multicolumn{2}{l|}{Expected return} & $+2\%$ & $+2\%$
\end{tabular}

\pause
\justify
On average, given equal returns, people prefer those assets that are less risky. To persuade risk-averse investors to buy stocks, one must offer them a discount so that the expected future returns are slightly higher.
\end{frame}



\begin{frame}{Test your risk-neutrality - 2}
\justify
You must invest all your savings for 1 year into stock A or stock B. Which option would you choose?

\justify
\centering
\begin{tabular}{l|r|r|r}
Scenario & Probability & Stock А & Stock B \\ \hline
Crisis and losing a job  & 50\%    & $+25\%$ & $-25\%$  \\
12 months salary bonus    & 50\%    & $-15\%$ & $+15\%$  \\ \hline
\multicolumn{2}{l|}{Expected return} & $+5\%$  & $+5\%$
\end{tabular}

\pause
\justify
When people evaluate assets, they are not only concerned with the variance of potential returns but also with the correlation to the overall market. Assets that fall less during crises are priced higher and have lower expected returns.
\end{frame}



\begin{frame}{Pricing derivatives in risk-neutral world}
\justify
We cannot delve into the minds of investors to calculate the extent of their risk-neutrality. Therefore, we do not know what risk premium is embedded into current asset prices.

\justify
Good news: we don't need to know this. FATP ensures that the price of a derivative in the real world will be the same as in the risk-neutral world. Of course, this is not true for the price of the underlying asset.
\end{frame}



\begin{frame}{Pricing derivatives in risk-neutral world - 2}
\justify
FATP ensures that we can price any derivative using the following algorithm:

\justify
1. Specify a stochastic process for the underlying asset price in risk-neutral world.

2. Define a payoff function that depends on the underlying asset price.

3. Compute expected value of the derivative payoff.

4. Multiply the expected value by discount factor.

\justify
Typically, step 3 is the most challenging of all. Quite often, there is no analytical solution.
\end{frame}



\begin{frame}{Example: Black-Scholes model}
\justify
In the Black-Scholes model, the underlying asset price $S(t)$ follows geometric Brownian motion in real world. Price increment $dS$ over small time interval $dt$ is
\begin{align*}
\frac{dS}{S} = \mu dt + \sigma  \xi \sqrt{dt}, \quad \xi \sim \mathcal{N}(0,1) 
\end{align*}
Here $\mu$ is the trend, and $\sigma$ is volatility.

\justify
In risk-neutral world, the asset grows at the risk-free rate $r$ less dividend yield $q$. Hence price of a derivative does not depend on real-world trend $\mu$.
\begin{align*}
\frac{dS}{S} = (r - q)dt + \sigma \xi \sqrt{dt} , \quad \xi \sim \mathcal{N}(0,1) 
\end{align*}
\end{frame}



\begin{frame}{Example: Black-Scholes model - 2}
\justify
Add up all the small price increments over all the $dt$ time steps. Then over a long time interval $(0, T)$ the underlying asset price follows the following process in risk-neutral world:
\begin{align*}
S_T(\xi) = S_0\exp{\left[\left(r - q - \frac{\sigma^2}{2}\right)T + \sigma\xi\sqrt{T}\right]}, \quad \xi \sim \mathcal{N}(0, 1)
\end{align*}

\justify
Define $f(S_T(\xi)) = \max(S_T(\xi) - K, 0)$ a payoff function of a vanilla European call option. According to FATP fair price of this option is equal to
\begin{align*}
C = e^{-rT} \int\limits_{-\infty}^{+\infty}f\Big(S_T(x)\Big)\phi(x)dx
\end{align*}
Here $\phi(x)$ is probability density function of the standard normal distribution $\mathcal{N}(0,1)$. If we solve the integral we get the Black-Scholes formula.
\end{frame}



\begin{frame}{Numerical procedures}
\justify
What if the stochastic process and/or the payoff function are so complex that it is not feasible to compute the risk-neutral expected value explicitly?

\justify
Numerical procedures come to rescue
\begin{itemize}
\item Binomial trees.
\item Finite difference methods.
\item Monte-Carlo method.
\end{itemize}
\end{frame}



\begin{frame}{Monte-Carlo method}
\justify
How do we compute $\pi$? Randomly drop $N$ points at a square. Count how many points hit the inscribed circle.

\centering
\begin{tikzpicture}
	\begin{axis}[
			width = 5.5cm,
			height = 5.5cm,
			only marks,
			xmin = 0, xmax = 1,
			ymin = 0, ymax = 1,
			xtick = {\empty},
			ytick = {\empty}
		]
				
		\addplot[mark=*, mark size=1.5pt, color=Set1-A] table[x=x, y=y, col sep=comma] {monte_carlo_pi_in_circle.csv};
		
		\addplot[mark=x, mark size=1.5pt, color=Set1-B] table[x=x, y=y, col sep=comma] {monte_carlo_pi_not_in_circle.csv};
		
		\draw[very thick] (0.5, 0.5) circle (0.5);
		\draw[very thick] (0, 0) rectangle (1, 1);
	\end{axis}
\end{tikzpicture}
\justify
Denote circle radius $R$. Approximately $N\dfrac{\pi R^2}{4R^2}$ points should hit the circle. In this example $N_0=770$ points out of $N=1000$ hit the circle. 
\begin{align*}
\pi \approx \frac{4N_0}{N} = \frac{4 \cdot 770}{1000} = 3.08
\end{align*}
\end{frame}



\begin{frame}{Метод Монте-Карло}
\justify
Пусть цена базового актива в риск-нейтральном мире, а вслед за ней и выплата по деривативу, зависят от реализации некоторой случайной величины $\xi$ (возможно многомерной). Предположим, что мы знаем закон распределения $\xi$.

\justify
1. Выберем наудачу $n$ (достаточно много) реализаций случайной величины $\xi$: $\xi_1, \xi_2, ..., \xi_n$.

2. Для каждой реализации $\xi_i$ вычислим цену базового актива $S_T(\xi_i)$ и выплату по деривативу $f(S_T(\xi_i))$.

3. Оценим математическое ожидание выплаты как 
\begin{align*}
\hat{f} = \frac{1}{n}\sum\limits_{i=1}^{n}f\Big(S_T(\xi_i)\Big)
\end{align*}

4. Выполним дисконтирование.

\justify
Можно надеяться (почему?), что с ростом $n$ оценка мат. ожидания будет сходиться к истинному мат. ожиданию.
\end{frame}



\begin{frame}{Закон больших чисел}
\justify
Пусть $\xi_i$ --- независимые одинаково распределённые случайные величины, которые имеют конечное математическое ожидание $\mathbb{E}\xi_i=\mu$. Тогда с вероятностью 1 (почти наверное)
\begin{align*}
\lim_{n \to \infty} \frac{1}{n} \sum\limits_{i=1}^{n}\xi_i = \mu
\end{align*}

\justify
Это закон больших чисел (ЗБЧ). Если усреднить много-много реализаций одной и той же случайной величины, то получится число, достаточно близкое к истинному математическому ожиданию.

\justify
Что такое <<достаточно>> близко?
\end{frame}



\begin{frame}{Центральная предельная теорема}
\justify
Пусть $\xi_i$ --- независимые одинаково распределённые случайные величины, которые имеют конечное математическое ожидание $\mathbb{E}\xi_i=\mu$ и конечную дисперсию $\operatorname{Var}(\xi_i) = \sigma^2$. Тогда имеет место сходимость по распределению:
\begin{align*}
\lim_{n \to \infty} \sqrt{n}\frac{\dfrac{1}{n}\sum\limits_{i=1}^{n}\xi_i - \mu}{\sigma} = \mathcal{N}(0, 1)
\end{align*}

\justify
$\mathcal{N}(0, 1)$ --- стандартное нормальное распределение.

\justify
Это --- центральная предельная теорема (ЦПТ). Среднее арифметическое большого количества реализаций случайной величины следует нормальному распределению.
\end{frame}



\begin{frame}{Случайные числа}
\justify
Для метода Монте-Карло нужно много, очень много случайных чисел.

\justify
В большинстве современных процессоров есть встроенный аппаратный генератор случайных чисел. Например, инструкция x86 RDRAND позволяет получить 16, 32 или 64 случайных бита.

\justify
Недостатки:
\begin{itemize}
\item Быстродействие (450 тактов на Core i7-7700).
\item Невоспроизводимость результатов.
\end{itemize}

\justify
Нужен быстрый алгоритм получения <<почти>> случайных чисел.
\end{frame}



\begin{frame}{Псевдослучайные последовательности}
\justify
Простой генератор псевдослучайных чисел по методу Лемера, реализованный в std::minstd\_rand:
\begin{align*}
X_{k+1} = (48\,271 \cdot X_k) \ \operatorname{MOD} \ 2\,147\,483\,647
\end{align*}

\justify
Для каждого целого положительного начального значения $X_0$ (которое называется \en{seed}) получается новая <<почти случайная>> последовательность целых чисел от 0 до 2\,147\,483\,646.

\justify
Как получить случайные действительные числа из равномерного распределения $U(0, 1)$?
\begin{align*}
U_k = X_k / (2\,147\,483\,647 - 1)
\end{align*}
\end{frame}



\begin{frame}{Псевдослучайные последовательности}
\justify
Если мы умеем создавать случайные числа $U_k$ из равномерного распределения $U(0,1)$, то как получить случайные числа, например, из нормального распределения $\mathcal{N}(0, 1)$?

\justify
Пусть $N(x)$ --- функция распределения стандартного нормального распределения, а $N^{-1}(x)$ --- её обратная функция. Тогда величины
\begin{align*}
N_k = N^{-1}(U_k)
\end{align*}
будут иметь стандартное нормальное распределение:
\begin{align*}
\mathbb{P}(N_k < x) = \mathbb{P}(N^{-1}(U_k) < x) = \mathbb{P}(U_k < N(x)) = U(N(x)) = N(x)
\end{align*}
\end{frame}



\begin{frame}{Псевдослучайные последовательности}
\centering
\begin{tikzpicture}
	\begin{axis}[
			width = \textheight,
			height = \textheight,
			xmin = -2.25, xmax = 2.25,
			ymin = -2.25, ymax = 2.25,
			axis x line = center,
			axis y line = center,
			legend entries = {
				$N(x)$,
				$N^{-1}(x)$
			},
			legend style = {
				at = {(0.03,0.97)},
				anchor = north west
			}
		]

		\addplot[color=Set1-A, thick] table[x=x, y=y, col sep=comma] {monte_carlo_norm_cdf.csv};
		
		\addplot[color=Set1-B, thick] table[x=y, y=x, col sep=comma] {monte_carlo_norm_cdf.csv};
		
		\draw[dashed] (axis cs: -5, 1) -- (axis cs: 5, 1);
		\draw[dashed] (axis cs: 1, -5) -- (axis cs: 1, 5);
		
		\draw[dashed, thick, color = Set1-B] (0.25, 0) -- (0.25, -0.6745) -- (0, -0.6745);
		\draw[dashed, thick, color = Set1-A] (0, 0.25) -- (-0.6745, 0.25) -- (-0.6745, 0);
		
		\node[anchor=south] at (0.25, 0) {\small 0.25};
		\node[anchor=east] at (0, -0.6745) {\small -0.67};
		\node[circle, fill, inner sep=1.5pt, color=Set1-B] at (0.25, 0) {};
		\node[circle, fill, inner sep=1.5pt, color=Set1-B] at (0, -0.6745) {};
		\node[circle, fill, inner sep=1.5pt, color=Set1-B] at (0.25, -0.6745) {};
	\end{axis}
\end{tikzpicture}
\end{frame}



\begin{frame}{Демонстрация}
\justify
Пример: 10\,000 нормально распределённых случайных величин в Excel.
\end{frame}



\begin{frame}{Ванильный колл-опцион}
\justify
Рассмотрим европейский колл-опцион со страйком $K$ и сроком исполнения $T$ лет. Начальная цена базового актива $S_0$. В риск-нейтральном мире базовый актив следует геометрическому броуновскому движению со средним $r$ (безрисковая ставка) и волатильностью $\sigma$:
\begin{align*}
S_T(\xi) = S_0 \exp\left[\left(r - \frac{\sigma^2}{2}\right)T + \sigma\xi\sqrt{T}\right], \quad \xi \sim \mathcal{N}(0, 1)
\end{align*}

Выплата по колл-опциону:
\begin{align*}
f(S_T(\xi)) = \max(S_T(\xi) - K, 0)
\end{align*}
\end{frame}



\begin{frame}{Ванильный колл-опцион}
\justify
1. Сгенерируем $n$ реализаций случайной величины $\xi$: $\xi_1, \xi_2, ..., \xi_n$. 

\justify
2. Посчитаем $n$ выплат по колл-опциону:
\begin{align*}
f_i = f(S_T(\xi_i)) = \max\left\{S_0 \exp\left[\left(r - \frac{\sigma^2}{2}\right)T + \sigma\xi_i\sqrt{T}\right] - K, 0\right\}
\end{align*}

\justify
3. Посчитаем среднюю выплату и дисконтируем её --- это будет сегодняшняя цена опциона.
\begin{align*}
\hat{C} = e^{-rT} \cdot \frac{1}{n}\sum\limits_{i=1}^{n}f(S_T(\xi_i))
\end{align*}
\end{frame}



\begin{frame}{Демонстрация}
\justify
Пример: ванильный колл-опцион.
\end{frame}



\begin{frame}{Оценка погрешности}
\justify
Пусть $\hat{C}$ --- дисконтированная средняя выплата 	опциона, которую мы получили в результате $n$ симуляций, а $\hat{s}$ --- её выборочное стандартное отклонение.

\justify
Согласно ЦПТ, при больших $n$ ошибка (разность $\hat{C}$ и истинного математического ожидания $C$) --- нормально распределённая случайна величина:
\begin{align*}
\sqrt{n}\frac{\hat{C} - C}{\hat{s}} \sim \mathcal{N}(0, 1)
\end{align*}

\justify
Мы можем построить доверительный интервал для $C$. Например, известно, что $N(1.96) \approx 0.975$. Поэтому 95\% доверительный интервал $C$ равен
\begin{align*}
\left(\hat{C} - 1.96\frac{\hat{s}}{\sqrt{n}}; \hat{C} + 1.96\frac{\hat{s}}{\sqrt{n}} \right)
\end{align*}  
\end{frame}



\begin{frame}{Seed variance}
\justify
\en{Seed variance}\ --- оценка того, насколько результат метода Монте-Карло зависит от значения seed, которым мы инициализировали генератор случайных чисел.

\justify
Если генератор псевдослучайных чисел достаточно хороший, а в алгоритме нет ошибок, то seed variance будет близка к теоретической погрешности.

\justify
Например, мы провели $k$ вычислений для $k$ разных значений seed и получили $k$ цен опциона $\hat{C}_1, \hat{C}_2,...,\hat{C}_k$. Тогда seed variance ($\hat{s})$
 --- выборочное стандартное отклонение $\hat{C}_i$:
\begin{align*}
\bar{C} = \frac{1}{k}\sum\limits_{i=1}^{k}\hat{C}_i \quad \hat{s} = \sqrt{\frac{1}{n-1}\sum\limits_{i=1}^{k}(\bar{C} - \hat{C}_i)^2} 
\end{align*}

\justify
Проверка seed variance --- стандартный приём при тестировании нового алгоритма. Она позволяет а) найти ошибки и б) оценить количество симуляций, которые дают приемлемую точность.
\end{frame}



\begin{frame}{Азиатский опцион}
\justify
В азиатском колл-опционе выплата зависит от средней цены базового актива за время жизни опциона.

\justify
Стороны договариваются о $k$ датах наблюдений (<<фиксингов>>) $0 \le t_1 < t_2 < ... < t_k \le T$. Выплата по колл-опциону со страйком $K$ равна
\begin{align*}
f = \max\left(\frac{S(t_1) + S(t_2) + ... + S(t_k)}{n} - K, 0\right)
\end{align*}

\justify
Например, азиатский опцион может платить разность между средним курсом USDRUB ЦБ РФ за месяц и страйком $80$. 

\justify
Выплата по азиатскому опциону зависит от всей траектории цены, а не только от конечной точки (является \en{path-dependent}). Аналитической формулы цены такого опциона нет. 
\end{frame}



\begin{frame}{Азиатский опцион}
\justify
Чтобы оценить азиатский опцион, нужно сгенерировать случайные траектории цены базового актива в риск-нейтральном мире.

\justify
Например, если в опционе $k$ дат фиксингов $t_1, t_2, ... t_k$, то один такой путь $i$ мог бы выглядеть как 
\begin{align*}
S_0 \to S_{i,1} \to S_{i,2} \to ... \to S_{i,k}
\end{align*}
причём
\begin{align*}
S_{i,0} &= S_0, \quad t_0 = 0 \\
S_{i,j} &= S_{i,j-1} \exp\left[\left(r - \frac{\sigma^2}{2}\right)(t_j - t_{j-1}) + \sigma\xi_{i,j}\sqrt{t_j - t_{j-1}}\right] \\
\xi_{i,j} &\sim \mathcal{N}(0, 1)
\end{align*}
\end{frame}



\newcommand{\plotStockPath}[2] {
	
	\addplot[
		mark = *,
		color = #2,
		thick
	]
	table[
		x = t,
		y = stock_price,
		col sep = comma,
		discard if not={path}{#1}
	]
	{monte_carlo_paths.csv};
}

\begin{frame}{Азиатский опцион}
\centering
\begin{tikzpicture}
	\begin{axis}[
		width = \textwidth - 0.5cm,
		height = \textheight - 1cm,
		xlabel = {Время (годы)},
		ylabel = {Цена базового актива},
		xmin=0, xmax=0.5,
		xticklabel = {\pgfmathprintnumber[fixed, precision=2]{\tick}}
	]
	
		\plotStockPath{1}{Set1-A}
		\plotStockPath{2}{Set1-B}
		\plotStockPath{3}{Set1-C}
		\plotStockPath{4}{Set1-D}
		\plotStockPath{5}{Set1-E}
		\plotStockPath{6}{Set1-F}
		\plotStockPath{7}{Set1-G}
		\plotStockPath{8}{Set1-H}
		\plotStockPath{9}{Set1-I}
	\end{axis}
\end{tikzpicture}
\end{frame}



\begin{frame}{Демонстрация}
\justify
Пример: азиатский колл-опцион.
\end{frame}



\begin{frame}{Радужный опцион}
\justify
Фиксируем четыре валютные пары и их текущие курсы:

\centering
\begin{tabular}{l|l}
EURUSD & 1.1050 \\
GBPUSD & 1.2530 \\
JPYUSD & 0.00754 \\
CADUSD & 0.7490
\end{tabular}

\justify
Через три месяца ($T=0.25$) посмотрим, какая валюта сильнее всего укрепилась к доллару, и выберем её для определения выплаты:
\begin{align*}
M = \max\left(\frac{S_{eurusd}(T)}{S_{eurusd}(0)}, \frac{S_{gbpusd}(T)}{S_{gbpusd}(0)}, \frac{S_{jpyusd}(T)}{S_{jpyusd}(0)}, \frac{S_{cadusd}(T)}{S_{cadusd}(0)} \right)
\end{align*}

\justify 
Радужный (\en{rainbow}) опцион с номиналом $N=\$1000$ и страйком $K=100\%$ имеет выплату
\begin{align*}
\max(M - K, 0) \cdot N = \max(M-1, 0) \cdot \$1000
\end{align*}
\end{frame}



\begin{frame}{Корреляции}
\justify
Цена радужного опциона зависит не только от волатильностей валютных пар, но и от корреляций между ними. Если какие-то корреляции отрицательные (например, если EURUSD дешевеет, то GBPUSD обычно дорожает), то цена опциона должна быть выше.

\justify
В методе Монте-Карло мы симулируем движение базовых активов в риск-нейтральном мире. Следовательно, нам нужны риск-нейтральные корреляции, а не корреляции из реального мира (например, исторические).

\justify
Какие рыночные инструменты могут показать <<ожидаемую рынком>> корреляцию?
\end{frame}



\begin{frame}{Корреляции}
\justify
Помимо <<мейджоров>> EURUSD и GBPUSD, на валютном рынке есть и опционы на <<кросс>> EURGBP. Implied волатильность кросса EURGBP может подсказать ожидаемую рынком корреляцию.

\justify
Как вычислить курс EURGBP?
\begin{align*}
S_{eurgbp} = \frac{S_{eurusd}}{S_{gbpusd}}
\end{align*}

\justify
Если корреляция высокая (когда евро растёт, то и фунт растёт), то числитель и знаменатель дроби чаще растут вместе, и сама дробь изменяется не так сильно. Если корреляция отрицательная, то наоборот, вся дробь изменяется сильнее, чем числитель и знаменатель по отдельности.
\end{frame}


\newcommand{\Var}{\operatorname{Var}}
\newcommand{\Cov}{\operatorname{Cov}}

\begin{frame}{Корреляции}
\justify
Перейдём от курсов к логарифмам:
\begin{align*}
S_{eurgbp} = \frac{S_{eurusd}}{S_{gbpusd}} \quad \Rightarrow \quad \ln S_{eurgbp} = \ln S_{eurusd} - \ln S_{gbpusd} 
\end{align*}

\justify
В риск-нейтральном мире Блэка-Шоулза цены следуют логнормальному распределению, а логарифмы цен --- нормальному распределению.
\begin{align*}
\Var(\ln S_{eurgbp}) &= \Var(\ln S_{eurusd} - \ln S_{gbpusd}) = \\
&= \Var(\ln S_{eurusd}) + \Var(\ln S_{gbpusd}) - \\
&- 2\Cov(\ln S_{eurusd}, \ln S_{gbpusd})
\end{align*}
В терминах волатильностей:
\begin{align*}
\sigma_{eurgbp}^2 &= \sigma_{eurusd}^2 + \sigma_{gbpusd}^2 - 2\rho\sigma_{eurusd}\sigma_{gbpusd} \Rightarrow \\
\rho &= \frac{\sigma_{eurusd}^2 + \sigma_{gbpusd}^2 - \sigma_{eurgbp}^2}{2\sigma_{eurusd}\sigma_{gbpusd}}
\end{align*}
\end{frame}



\begin{frame}{Корреляции}
\justify
Как сгенерировать $n$ случайных величин с заданной матрицей корреляции $Q$?

\justify
Для любой положительно определённой матрицы $Q$ существует разложение Холецкого (\en{Cholesky}) на нижнюю треугольную матрицу $L$ и верхнюю треугольную матрицу $L^T$:
\begin{align*}
Q = L \cdot L^T
\end{align*}

\justify
Если в случайном векторе $x=(x_1,...,x_n)^T$ все величины независимые, то в векторе $Lx$ все компоненты будут связаны матрицей корреляции $Q$.
\end{frame}



\begin{frame}{Радужный опцион}
\justify
Алгоритм оценки радужного опциона методом Монте-Карло:

1. Оценить матрицу риск-нейтральных корреляций $Q$ из волатильностей четырёх валютных пар и их кроссов.

2. Выполнить разложение Холецкого: $Q = L \cdot L^T$.

3. Сгенерировать четыре независимые стандартные нормальные величины $\xi = (\xi_1,...,\xi_4)^T$.

4. Получить вектор скоррелированных величин $(\psi_1,...,\psi_4)^T = L\xi$.

5. Посчитать финальный курс каждой валюты ($r$ --- безрисковая ставка в долларах, $q_i$ --- в валюте):
\begin{align*}
S_{T,i}(\psi_i) = S_{0,i}\exp\left[\left(r - q_i - \frac{\sigma_i^2}{2}\right)T + \sigma_i\psi_i\sqrt{T}\right]
\end{align*}

6. Вычислить выплату по радужному опциону.

7. Повторить 10\,000 раз.
\end{frame}



\begin{frame}{Variance reduction}
\justify
Наша псевдослучайная последовательность $\mathcal{N}(0,1)$ не является идеальной и может иметь перекос влево или вправо относительно истинного математического ожидания 0.

\justify
Anti-thetic trials --- метод борьбы со смещением выборки. Каждый раз, когда генератор случайных чисел выдаёт реализацию нормальной величины, например 0.42, можно добавить в выборку противоположное число, например $-0.42$. По построению новая выборка будет иметь среднее 0.

\justify
Другие более сложные методы снижения дисперсии метода Монте-Карло:
\begin{itemize}
\item Stratified sampling, importance sampling
\item Orthogonal array sampling
\item Sobol numbers
\end{itemize}
\end{frame}



\begin{frame}{Демонстрация}
\justify
Пример: радужный опцион с \en{anti-thetic trials}.
\end{frame}

\end{document}


