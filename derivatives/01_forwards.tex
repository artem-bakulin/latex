\documentclass{beamer}

\usepackage{cmap}				% To be able to copy-paste russian text from pdf
\usepackage[T2A]{fontenc}
\usepackage[utf8]{inputenc}
\usepackage[russian]{babel}
\usepackage{textpos}
\usepackage{ragged2e}
\usepackage{amssymb}
\usepackage{ulem}
\usepackage{tikz}
\usepackage{pgfplots}
\usepackage{color}
\usepackage{cancel}
\usepackage{multirow}
\pgfplotsset{compat=1.17}
\usetikzlibrary{arrows,snakes,backgrounds,shapes}
\usepgfplotslibrary{groupplots,colorbrewer,dateplot,statistics}
\usepackage{animate}

\usepackage{amsfonts}
\usepackage{amsmath}
\usepackage{amssymb}
\usepackage{graphicx}
\usepackage{setspace}

\usepackage{enumitem}
\setitemize{label=\usebeamerfont*{itemize item}%
  \usebeamercolor[fg]{itemize item}
  \usebeamertemplate{itemize item}}

% remove navigation bar
\setbeamertemplate{navigation symbols}{}

\setbeamertemplate{page number in head/foot}[totalframenumber] 

\usepackage{eurosym}
\renewcommand{\EUR}[1]{\textup{\euro}#1}

\title{Процентные ставки. Валютные форварды}
\author{Артём Бакулин}
\date{14 марта 2022 г.}

\usetheme{Warsaw}
\usecolortheme{beaver}

\newcommand{\ru}[1]{\begin{otherlanguage}{russian}#1\end{otherlanguage}}
\newcommand{\en}[1]{\begin{otherlanguage}{english}#1\end{otherlanguage}}
\newcommand{\ruen}[2]{#1 (\en{#2})}

\begin{document}



\begin{frame}
\titlepage
\end{frame}



\begin{frame}{Чего не будет в этом курсе}
\begin{itemize}
\justifying
\item Сигма-алгебры, фильтрации и прочая теория.
\item Стохастические дифференциальные уравнения.
\item Модели со стохастической локальной волатильностью.
\item Как разбогатеть на деривативах быстро и без риска.
\end{itemize}
\end{frame}



\begin{frame}{Процентные ставки}
\justify
14.03.2022 я положил в банк 1\,000\,000 рублей под 12\% годовых на 1 год до 
14.03.2023. Сколько денег мне вернут в конце срока?

\begin{enumerate}[label=\Alph*]
\item 1\,120\,000 руб. \only<2>{--- без капитализации, ACT/365}
\item 1\,121\,764 руб. \only<2>{--- без капитализации, ACT/360}
\item 1\,126\,825 руб. \only<2>{--- ежемесячная капитализация, ACT/365}
\item 1\,128\,695 руб. \only<2>{--- ежемесячная капитализация, ACT/360}
\end{enumerate}

\only<2>{Правильный ответ зависит от того, что написано мелким шрифтом в 
договоре.}
\end{frame}


\begin{frame}{Капитализация процентов}
\justify
<<Сложные проценты --- самая могущественная формула сила во Вселенной>> 
(ошибочно приписывается Альберту Эйнштейну).

\justify
Пусть депозит под ставку (\en{rate}) $r$ действует (\en{term}) $T$ лет с 
частотой капитализации (\en{frequency})\ $f$ раз в год. Начальная сумма вклада 
$N_0$. В конце срока мы получим
\begin{align*}
N &= N_0\underbrace{\left(1 + \frac{r}{f}\right) \cdot \left(1 + \frac{r}{f}\right) \cdot ... \cdot \left(1 + \frac{r}{f}\right)}_{fT \text{\ множителей}} = 
 \left(1 + \frac{r}{f}\right)^{fT}
\end{align*}

Например, депозит на 1\,000\,000 под 6\% на 1 год с ежемесячной капитализацией:
\begin{align*}
1\,000\,000 \cdot \left(1 + \frac{0.12}{12}\right)^{12} \approx 1\,126\,825
\end{align*}
\end{frame}



\begin{frame}{Общеупотребительные методы капитализации}
\centering
\begin{tabular}{l|l|l}
Частота ($f$) & Русское название & Английское название \\ \hline 
0 & Без капитализации* & Zero \\
1 & Ежегодно & Annually \\
2 & Раз в полгода & Semi-annually \\
4 & Ежеквартально & Quarterly \\
12 & Ежемесячно & Monthly \\
365 & Ежедневно & Daily \\
$\infty$ & Непрерывно* & Continuously 
\end{tabular}

\justify
Простые проценты без капитализации (\en{zero compounding}), $f=0$:
\begin{align*}
N = N_0 (1 + rT)
\end{align*}

Непрерывная капитализация (\en{continuous compounding}), $f=\infty$:
\begin{align*}
N = N_0e^{rT}
\end{align*}
\end{frame}



\begin{frame}{Непрерывная капитализация процентов}
\justify
Предположим, что совсем-совсем небольшой процент прибавляется ко вкладу каждый 
час, каждую минуту, каждую миллисекунду. Частота капитализации $f$ стремится к 
бесконечности.
\begin{align*}
N &= N_0 \lim_{f \to +\infty} \left(1 + \frac{r}{f}\right)^{fT} = 
\begin{Bmatrix}x= f/r \\ f=xr \\ x \to +\infty \\ f \to +\infty\end{Bmatrix} = \\
&= N_0\lim_{x \to +\infty} \left(1 + \frac{r}{xr}\right) ^ {xrT} =
N_0{\underbrace{\left(\lim_{x \to +\infty} \left( 1 + \frac{1}{x} \right) ^ x\right)}_{e \approx 2.71828}}  ^ {rT} = \\
&= N_0e^{rT} \approx N_0(1 + rT)
\end{align*}
<<Непрерывные>> проценты не встречаются в реальной жизни, но заметно упрощают 
формулы в теоретических моделях.
\end{frame}



\begin{frame}{Капитализация процентов}
\justify
Рассмотрим депозиты на 1\,000\,000 рублей под ставку 12\% на срок 1 год и 5 лет. 
Как конечная сумма вклада зависит от капитализации процентов?


\begin{table}
\centering
\begin{tabular}{l|r|r|r}
Начисление процентов & $f$ & 1 год & 5 лет \\ \hline
Без капитализации & 0         & 1\,120\,000 & 1\,600\,000 \\
Ежегодно          & 1         & 1\,120\,000 & 1\,762\,342 \\
Раз в полгода     & 2         & 1\,123\,600 & 1\,790\,848 \\
Ежеквартально     & 4         & 1\,125\,509 & 1\,806\,111 \\
Ежемесячно        & 12        & 1\,126\,825 & 1\,816\,697 \\
Ежедневно         & 365       & 1\,127\,475 & 1\,821\,939 \\
Непрерывно        & $+\infty$ & 1\,127\,497 & 1\,822\,119
\end{tabular}
\end{table}

\justifying
\vspace{\baselineskip}
\begin{itemize}
\justifying
\item Чем выше ставка и дольше срок, тем важнее роль капитализации.
\item Ежедневная капитализация процентов достаточно близка к непрерывной.
\end{itemize}
\end{frame}



\begin{frame}{Капитализация процентов}

	\centering
	\begin{tikzpicture}
		\begin{axis}[
			xlabel=\text{Время, месяцы},
			ylabel=\text{Сумма вклада, рубли},
			xmin=0,
			xmax=24.5,
			width=\textwidth,
			height=\textheight - 1.5cm,
			ymin=100,
			ymax=130,
			legend style={at={(0.01, 0.99)}, anchor=north west}
		]
	

		\addplot[const plot, samples at={0,0.1,...,60}, very thick, color=Set1-A] {100 * (1 + 0.12/4) ^ (4*3*floor(x/3)/12.0)};
		\addlegendentry{$f=4$}
	
		\addplot[const plot, samples at={0,0.5,...,60}, very thick, color=Set1-B] {100 * (1 + 0.12/12) ^ (12*floor(x)/12.0)};
		\addlegendentry{$f=12$}
		
		\addplot[domain=0:60, color=Set1-C, very thick] {100 * exp(0.12*x/12)};
		\addlegendentry{$f=+\infty$}
		\end{axis}
	\end{tikzpicture}
	\scriptsize{Рост вклада с течением времени, $N_0=100$, $r=12\%$}
\end{frame}



\begin{frame}{Соглашение о подсчёте дней}
\justify
Сколько лет между 1 марта 2022 года и 1 апреля 2022 года?
\begin{enumerate}[label=\Alph*]
\item $31/365 \approx 0.08493$ \only<2>{(ACT/365)}
\item $31/360 \approx 0.08611$ \only<2>{(ACT/360)}
\item $30/360 \approx 0.08333$ \only<2>{(30/360)}
\item $23/252 \approx 0.09127$ \only<2>{(BUS/252 по календарю Нью-Йорка)}
\end{enumerate}

\vspace{\baselineskip}
\only<2>{
ACT --- считаем календарные дни. 30 --- считаем, что в месяце 30 дней. BUS --- считаем только рабочие дни.

\justify
Выбор способа начисления процентов --- рыночная конвенция. Правило большого пальца: в странах Содружества ACT/365, в других странах --- ACT/360, в Бразилии --- BUS/252. Читайте мелкий шрифт! 
}
\end{frame}



\begin{frame}{Конвенция 30/360}
\justify
Сколько лет между датами $D_1.M_1.Y_1$ и $D_2.M_2.Y_2$?
\begin{align*}
T = \frac{360(Y_2-Y_1) + 30(M_2-M_1) + (D_2-D_1)}{360}
\end{align*}

\justify
Конвенция 30/360 возникла в докомпьютерную эпоху и позволяет быстро прикинуть результат в уме. Она по-прежнему живее всех живых (например, на рынке облигаций).
\begin{itemize}
\item Обратная совместимость
\item Размер платежей по кредитам не меняется в течение года
\end{itemize}

\vspace{\baselineskip}
\justify
*Несколько вариантов (30/360 Bond, 30/360 US, 30E/360, 30E/360 ISDA) отличаются тем, как они обрабатывают февраль и месяцы с 31 днями. Скажем спасибо библиотекам финансовых вычислений за то, что прячут от нас эти детали!
\end{frame}



\begin{frame}{Переход между способами капитализации}
\justify
Если вам нужна <<непрерывная>> ставка $r^*$, а есть только <<обычная>> $r$ (или наоборот), то всегда можно выразить одну через другую.
\begin{align*}
e^{r^*T} &= \left(1 + \frac{r}{f}\right)^{fT} \\
r^* &= f\ln \left(1 + \frac{r}{f}\right) \\
r &= f\left(e^{r^*/f} - 1\right)
\end{align*}

\justify
То же самое для <<простых>> процентов без капитализации:
\begin{align*}
e^{r^*T} &= 1+rT \\
r^* &= \frac{\ln(1+rT)}{T} \\
r &= \frac{e^{r^*T}-1}{T}
\end{align*}

\end{frame}



\begin{frame}{Текущая стоимость}
\justify
\alert{Текущая стоимость} (\en{present value}) выплаты в $N$ рублей через $T$ лет --- сумма, 
которую участники рынка готовы заплатить сегодня за право получить эту выплату.

\justify
Сколько рублей вы готовы заплатить сегодня (\en{PV}) за право получить 1\,120\,000 (\en{future value, FV}) рублей через год? Предположим, что на рынке есть идеальный безрисковый депозит под 12\% с простой капитализацией.

\begin{align*}
PV = \frac{FV}{1+rT} = \frac{1\,120\,000}{1 + 12\%} = 1\,000\,000
\end{align*}

\justify
Никто не захочет платить больше, чем 1\,000\,000 сегодняшних рублей за 1\,120\,000 будущих рублей, потому что выгоднее будет вложить 1\,000\,000 под 12\% на год. Никто не захочет продавать 1\,120\,000 будущих рублей дешевле, чем за 1\,000\,000 сегодняшних, потому что выгоднее будет взять в кредит 1\,000\,000 под 12\% и погасить его через год.
\end{frame}



\begin{frame}{Коэффициент дисконтирования}
\justify
\alert{Коэффициент дисконтирования} (\en{discount factor}) для будущей даты $T$ --- текущая стоимость права получить 1 единицу валюты в день $T$. Другая формулировка: сколько рублей нужно иметь сегодня, чтобы сделать из них 1 рубль к моменту $T$?

\justify
Ответ зависит от того, какая безрисковая ставка $r$ нам доступна:
\begin{align*}
\delta_T &= \frac{1}{1 + rT} \quad \text{(простые проценты)} \\
\delta_T &= \frac{1}{e^{r^*T}} = e^{-r^*T} \quad \text{(непрерывные проценты)}
\end{align*}

\justify
Например, если безрисковая процентная ставка равна 12\%, то коэффициент дисконтирования на 1 год равен
\begin{align*}
\delta_1 = \frac{1}{1.12} \approx 0.8929
\end{align*}
\end{frame}



\begin{frame}{Отложенный депозит}
\justify
Предположим, что на рынке доступны два депозита с <<нулевой>> капитализацией:
\begin{itemize}
\item На $T_1=1$ год под ставку $r_1=12\%$ 
\item На $T_2=2$ года под ставку $r_2=8\%$ 
\end{itemize}

Банк предлагает <<отложенный>> депозит: вы обязаны принести деньги через год на 1 год, а банк обязан зафиксировать ставку по депозиту $x$ уже сегодня. На какую ставку вы согласитесь?

\centering
\begin{tikzpicture}
		\draw [->,>=triangle 90] (0, 0) -- (8.5, 0);

		\draw [->,>=triangle 45] (1,0) node[anchor=north east]{$0$} .. controls (1.5, 1) and (3.5, 1) .. (4,0) node[anchor=north]{$T_1=1$} node[pos=0.5,anchor=south]{$r_1=12\%$};

		\draw [->,>=triangle 45] (4,0) .. controls (4.5, 1) and (6.5, 1) .. (7,0) node[anchor=north west]{$T_2=2$} node[pos=0.5,anchor=south]{$x$};

		\draw [->,>=triangle 45] (1,0) .. controls (1.5, -1) and (6.5, -1) .. (7,0) node[pos=0.5,anchor=north]{$r_2=8\%$};
	\end{tikzpicture}
\end{frame}



\begin{frame}{Отложенный депозит}
\centering
\begin{tikzpicture}
		\draw [->,>=triangle 90] (0, 0) -- (8.5, 0);

		\draw [->,>=triangle 45] (1,0) node[anchor=north east]{$0$} .. controls (1.5, 1) and (3.5, 1) .. (4,0) node[anchor=north]{$T_1=1$} node[pos=0.5,anchor=south]{$r_1=12\%$};

		\draw [->,>=triangle 45] (4,0) .. controls (4.5, 1) and (6.5, 1) .. (7,0) node[anchor=north west]{$T_2=2$} node[pos=0.5,anchor=south]{$x$};

		\draw [->,>=triangle 45] (1,0) .. controls (1.5, -1) and (6.5, -1) .. (7,0) node[pos=0.5,anchor=north]{$r_2=8\%$};
	\end{tikzpicture}
	
\justifying
Не должно быть разницы между депозитом на два года и цепочкой из двух депозитов:
\begin{align*}
(1 + r_1T_1)\Big(1+x(T_2-T_1)\Big) = 1 + r_2T_2
\end{align*}
\begin{align*}
x = \frac{\dfrac{1+r_2T_2}{1+r_1T_1} - 1}{T_2-T_1} = \frac{\dfrac{1 + 0.08 \cdot 2}{1 + 0.12} - 1}{2-1} \approx 3.57\%
\end{align*}

При любой другой ставке <<отложенный>> депозит будет не выгоден либо вкладчикам (и они не будут им пользоваться), либо банку (и он не будет его предлагать).
\end{frame}



\begin{frame}{Непрерывная капитализация процентов}
\justify
Пересчитаем простые проценты в непрерывные:
\begin{align*}
r_1^* &= \frac{\ln(1+r_1T_1)}{T_1} = \frac{\ln 1.12}{1} \approx 11.333\% \\
r_2^* &= \frac{\ln(1+r_2T_2)}{T_2} = \frac{\ln 1.14}{2} \approx 7.421\%
\end{align*}

\centering
\begin{tikzpicture}
		\draw [->,>=triangle 90] (0, 0) -- (8.5, 0);

		\draw [->,>=triangle 45] (1,0) node[anchor=north east]{$0$} .. controls (1.5, 1) and (3.5, 1) .. (4,0) node[anchor=north]{$T_1=1$} node[pos=0.5,anchor=south]{$r_1^*=11.333\%$};

		\draw [->,>=triangle 45] (4,0) .. controls (4.5, 1) and (6.5, 1) .. (7,0) node[anchor=north west]{$T_2=2$} node[pos=0.5,anchor=south]{$x^*$};

		\draw [->,>=triangle 45] (1,0) .. controls (1.5, -1) and (6.5, -1) .. (7,0) node[pos=0.5,anchor=north]{$r_2^*=7.421\%$};
	\end{tikzpicture}
\end{frame}



\begin{frame}{Непрерывная капитализация процентов}
\centering
\begin{tikzpicture}
		\draw [->,>=triangle 90] (0, 0) -- (8.5, 0);

		\draw [->,>=triangle 45] (1,0) node[anchor=north east]{$0$} .. controls (1.5, 1) and (3.5, 1) .. (4,0) node[anchor=north]{$T_1=1$} node[pos=0.5,anchor=south]{$r_1^*=11.333\%$};

		\draw [->,>=triangle 45] (4,0) .. controls (4.5, 1) and (6.5, 1) .. (7,0) node[anchor=north west]{$T_2=2$} node[pos=0.5,anchor=south]{$x^*$};

		\draw [->,>=triangle 45] (1,0) .. controls (1.5, -1) and (6.5, -1) .. (7,0) node[pos=0.5,anchor=north]{$r_2^*=7.421\%$};
	\end{tikzpicture}

\justifying
Не должно быть разницы между цепочкой из <<непрерывных>> годовых депозитов и двухгодовым депозитом:
\begin{align*}
e^{r_1^*T_1}e^{x^*(T_2-T_1)} = e^{r_2^*T_2}
\end{align*}
\begin{align*}
x^* = \frac{r_2^*T_2-r_1^*T_1}{T_2-T_1} = \frac{7.421\%\cdot2 - 11.333\%}{2 - 1} \approx 3.509\%
\end{align*}

Пересчитаем <<непрерывные>> обратно в простые:
\begin{align*}
x = \frac{e^{x^*(T_2-T_1)} - 1}{T_2 - T_1} = e^{0.03509} - 1 \approx 3.57\% 
\end{align*}
\end{frame}



\begin{frame}{Непрерывные проценты}
\justify
Наблюдаемая действительность --- то, что люди готовы вложить $100$ рублей, чтобы получить $112$ рублей через год. Это можно интерпретировать и как <<простую>> ставку 12\% , и как <<непрерывную>> ставку 11.33\%, и как коэффициент дисконтирования $1/1.12 = 0.8929$. Это разные точки зрения (или разные единицы измерения) на одно и то же явление.

\justify 
В зависимости от контекста может быть удобнее пользоваться разными абстракциями. Можно показывать пользователям <<простые>> проценты, математику будущих ставок программировать в терминах <<непрерывных>> ставок, а \en{present value} будущих выплат вычислять с помощью коэффициентов дисконтирования.
\end{frame}



\begin{frame}{Форвардные ставки}
\justify 
Ставка $x$ по <<отложенному>> депозиту называется также \alert{ожидаемой или вменённой форвардной ставкой} (\en{implied forward rate}). Является ли она математическим ожиданием будущей ставки?

\justify
В среднем, на дистанции, форвардные ставки чаще оказываются завышенными. Когда подойдёт срок открытия <<отложенного>> депозита, текущая рыночная ставка по депозитам с большей вероятностью окажется ниже, чем та, о которой удалось договориться заранее. 

\justify
На этом наблюдении основана стратегия кэрри-трейдинга на ставках. Можно занимать деньги на короткий срок (на день) и вкладывать на длинный (на год). Средняя ставка дневных кредитов с большой вероятностью окажется ниже, чем ставка по длинном депозиту. Минус: если случится кризис и хотя бы один раз дневная ставка выстрелит вверх, будет очень больно.
\end{frame}



\begin{frame}{Финансовые деривативы}
\justify
\alert{Финансовый дериватив} (\en{financial derivative}) --- контракт, доход по которому зависит 
от (выводится из) поведения другого финансового актива, который называется базовым 
(\en{underlying}).

\justify
Примеры базовых активов: валютные курсы, процентные ставки, ценные бумаги, пшеница, нефть, погода.

\justify
Примеры деривативов: форварды и фьючерсы, опционы, процентные свопы, кредитные дефолтные свопы.

\justify
Деривативы можно использовать для управления риском (хэджирование) и для того, чтобы заработать деньги на движении рынка (спекуляция). 
\end{frame}



\begin{frame}{Валютный спот}
\justify
\alert{Валютная спот-сделка} (\en{foreign exchange spot, FX spot}) --- сделка, в которой стороны меняют одну валюту на другую <<почти сразу>> (\en{on the spot}), обычно на второй рабочий день.

\justify
Купить валютную пару, например EURUSD, означает согласиться получить первую валюту (EUR) и отдать вторую (USD). Например, купить спот 1\,000\,000 EURUSD по курсу 1.091 означает следующие платежи:

\justify
\centering
\begin{tabular}{l|r|r}
Дата                          & EUR & USD \\ \hline
Пн 14.03.2022 (сегодня)  & 0   & 0   \\
Ср 16.03.2022 (спот)     & +1\,000\,000 & $-1\,091\,000$
\end{tabular}
\end{frame}



\begin{frame}{Валютный форвард}
\justify
\alert{Валютный форвард} (\en{FX forward, outright forward}) --- сделка, в которой стороны берут 
на себя обязательства произвести обмен валюты в заданную дату в будущем по 
фиксированному курсу и в фиксированном объёме. 

\justify
Пример: покупка форварда EURUSD сроком 1 год по курсу 1.112 в размере 1\,000\,000:

\centering
\begin{tabular}{l|r|r}
Дата                          & EUR & USD \\ \hline
Пн 14.03.2022 (сегодня)  & 0   & 0   \\
Ср 16.03.2022 (спот) (спот) & 0   & 0   \\
Чт 16.03.2023 (1Y)   & 1\,000\,000 & $-1\,112\,000$
\end{tabular}

\justify
Сам по себе форвард обычно нисколько не стоит (сегодня ничего платить не надо). <<Купить>> форвард --- взять на себя обязательство получить первую валюту и заплатить вторую по фиксированному курсу. Главное договориться об этому курсе обмена.
\end{frame}



\begin{frame}{Форвардный курс и форвардные пункты}
\justify
Важнейший параметр, о котором должны сторговаться участники сделки --- будущий курс обмена, который называется \alert{форвардный курс} (\en{forward rate}). Его ещё можно назвать <<ценой>> форварда.

\justify
Иногда используют не сам форвардный курс, а разность между форвардом и спотом --- \alert{форвардные пункты} (\en{forward points}). Это удобно, потому что спот-курс меняется каждую миллисекунду и тянет за собой форвард, а разность между ними меняется реже и не так сильно.

\vspace{\baselineskip}
\centering
\begin{tabular}{l|l}
Спот-курс & 1.091 \\
Форвардный курс & 1.112 \\
\hline
Форвардные пункты & $1.112 - 1.091 = 0.021 = 210$ <<пипов>>
\end{tabular}

\justify
* 1 <<пип>> в паре EURUSD --- 0.0001.
\end{frame}



\begin{frame}{Валютный риск}
\justify
Предположим, что мы --- российский экспортёр рогов и копыт. Мы заключили контракт на поставку рогов в Европу по рогопроводу <<Северный олень 2>>. По контракту мы должны через год поставить 100 кубометров рогов, и тогда же через год нам заплатят 1\,000\,000 евро. Допустим, себестоимость партии рогов 80\,000\,000 рублей. Текущий курс EURRUB 85.0.

\justify
\centering
\begin{tabular}{r|r|r|r|r|r}
EURRUB      & Выручка & \multicolumn{2}{c|}{Спот} & Расходы & Прибыль \\
\cline{3-4}
через год   & EUR     & EUR    & RUB              & RUB     & RUB   \\ \hline
75.0        & +1 млн  & -1 млн & +75 млн          & -80 млн & -5 млн \\
85.0        & +1 млн  & -1 млн & +85 млн          & -80 млн & +5 млн \\
95.0        & +1 млн  & -1 млн & +95 млн          & -80 млн & +15 млн
\end{tabular}

\justify
Это \alert{валютный риск} (\en{FX risk}) --- риск того, что наша прибыль изменится из-за колебаний валютного курса.
\end{frame}



\begin{frame}{Хэджирование валютного риска}
\justify
Чтобы избежать риска, можно заключить форвард. Например, крупный немецкий инвестбанк из 
Франкфурта может купить у нас годовой форвард EURRUB по курсу 90.

\justify
\centering
\begin{tabular}{r|r|r|r|r|r}
EURRUB      & Выручка & \multicolumn{2}{c|}{Форвард} & Расходы & Прибыль \\
\cline{3-4}
через год   & EUR     & EUR    & RUB              & RUB     & RUB   \\ \hline
75.0        & +1 млн  & -1 млн & +90 млн          & -80 млн & +10 млн \\
85.0        & +1 млн  & -1 млн & +90 млн          & -80 млн & +10 млн \\
95.0        & +1 млн  & -1 млн & +90 млн          & -80 млн & +10 млн
\end{tabular}

\justify
Форвард --- обязательный к исполнению контракт. Даже если через год евро укрепится до 
105 рублей, мы будем обязаны продать евро по 90. Форвард защищает не только от риска 
потерь (если евро подешевеет), но и от риска неожиданно высокой прибыли (если евро 
укрепится).

\justify
Нужно ли иметь 1\,000\,000 евро сегодня, чтобы продать форвард EURRUB? Нет!
\end{frame}



\begin{frame}{Спекуляция на валютном риске}
\justify
Помечтаем, что мы -- экстрасенсы и точно знаем, что за год евро сильно подешевеет. Как мы можем на этом заработать? Простой путь: продать 1 миллион евро сейчас (<<на споте>>). Но что, если под рукой нет свободного миллиона евро?

\justify
Решение: продадим кому-нибудь годовой форвард EURRUB по 90. Через год нам принесут 90\,000\,000 рублей, а мы купим на них евро по текущему спот-курсу.

\justify
\centering
\begin{tabular}{r|r|r|r|r|r}
EURRUB      & \multicolumn{2}{c|}{Форвард} & \multicolumn{2}{c|}{Спот} & Прибыль \\
\cline{2-5}
через год & EUR     & RUB     & EUR     & RUB      & RUB \\ \hline
80.0      & -1 млн & +90 млн  & +1 млн  & -80 млн  & +10 млн\\
90.0      & -1 млн & +90 млн  & +1 млн  & -90 млн  & 0 \\
100.0     & -1 млн & +90 млн  & +1 млн  & -100 млн & -10 млн \\
\end{tabular}
 
\justify
* Банк, которому мы продадим форвард, скорее всего потребует от нас залог на случай, если наша ставка не сыграет. Но залог будет меньше, чем миллион евро.
\end{frame}



\begin{frame}{Честный форвардный курс}
\justify
От чего зависит <<справедливый>> форвардный курс, при котором продавец и покупатель будут рады заключить контракт? 

\justify
Представьте, что вы --- инвестор из Европы. Сейчас у вас есть 1 миллион евро, но через год вам понадобятся рубли для поездки в Сочи. Как вы можете зафиксировать сумму рублей, которой вы будете владеть через год? 
\end{frame}



\begin{frame}{Честный форвардный курс}
\justify
Текущий спот-курс EURRUB $S_{eurrub}=85$, безрисковый депозит в евро 
приносит $r_{eur}=0.25\%$, а в рублях --- $r_{rub}=6\%$. 
На рынке форвардов можно заключить контракт по курсу $F_{eurrub}=89.0$.

\justify
\centering
\begin{tabular}{l|l}
Стратегия 1 & Стратегия 2 \\ \hline
1) Продать спот:    & 1) Депозит под 0.25\%: \\
-1\,000\,000 EUR    & -1\,000\,000 EUR \\
+85\,000\,000 RUB   & 2) Продать форвард по 89.0 \\ \cline{2-2}
2) Депозит под 6\%: & 3) Закрыть депозит: \\
-85\,000\,000 RUB   & +1\,002\,500 EUR \\ \cline{1-1}
3) Закрыть депозит: & 4) Поставка по форварду: \\
+90\,100\,000 RUB   & -1\,002\,500 EUR \\
                    & +89\,222\,500 RUB \\ \hline
$1\,000\,000 \cdot S_{eurrub} \cdot (1+r_{rub})$ & $1\,000\,000 \cdot (1+r_{eur}) \cdot F_{eurrub}$
\end{tabular}
\end{frame}



\begin{frame}{Честный форвардный курс}
\justify
Могут ли две стратегии приводить к разным результатам?

\justify
\centering
\begin{tabular}{l|l}
Стратегия 1 & Стратегия 2 \\ \hline
+90\,100\,000 RUB  & +89\,222\,500 RUB \\
$1\,000\,000 \cdot S_{eurrub} \cdot (1+r_{rub})$ & $1\,000\,000 \cdot (1+r_{eur}) \cdot F_{eurrub}$
\end{tabular}

\justify
Стратегия 1 явно лучше, и все будут пользоваться ей. 

1) Больше желающих продать евро --- курс $S_{eurrub}$ снижается.

2) Выше спрос на депозиты в рублях --- ставка $r_{rub}$ снижается.

3) Никому не нужны депозиты в евро --- ставка $r_{eur}$ растёт. 

4) Никто не продаёт форварды --- цена форварда $F_{eurrub}$ растёт.
\end{frame}



\begin{frame}{Честный форвардный курс}
\justify
Что, если форвардный курс выше, например $F_{eurrub}=91$?


\justify
\centering
\begin{tabular}{l|l}
Стратегия 1 & Стратегия 2 \\ \hline
+90\,100\,000 RUB  & +91\,227\,500 RUB \\
$1\,000\,000 \cdot S_{eurrub} \cdot (1+r_{rub})$ & $1\,000\,000 \cdot (1+r_{eur}) \cdot F_{eurrub}$
\end{tabular}

\justify
Теперь стратегия 2 лучше. 

1) Выше спрос на депозиты в евро --- ставка $r_{eur}$ снижается.

2) Больше продавцов форвардов --- курс $F_{eurrub}$ снижается.

3) Никто не продаёт евро на споте --- курс $S_{eurrub}$ растёт. 

4) Депозиты в рублях не нужны --- ставка $r_{rub}$ растёт.
\end{frame}



\begin{frame}{Честный форвардный курс}
\justify
При каком форвардном курсе $F_{eurrub}$ рынок будет в равновесии, а стратегии 1 и 2 будут приводить к одинаковым результатам?

\begin{align*}
&1\,000\,000 \cdot S_{eurrub} \cdot (1 + r_{rub}) = 1\,000\,000 \cdot (1+r_{eur}) \cdot F_{eurrub} \Rightarrow \\
&F_{eurrub} = S_{eurrub} \frac{1 + r_{rub}}{1 + r_{eur}} = 85 \cdot \frac{1 + 0.06}{1 + 0.0025} \approx 89.87531
\end{align*}

\justify
Вывод: честная цена форварда зависит только от текущего спот-курса и соотношения процентных ставок. Не нужно предсказывать будущее, чтобы торговать форвардами!
\end{frame}



\begin{frame}{Арбитраж}
\justify
Где гарантия, что рынок сойдётся к равновесию за конечное время? Вдруг он может пребывать в неравновесном состоянии сколь угодно долго?

\justify
Когда клиент открывает вклад в банке под безрисковую ставку, то банк де-факто берёт у клиента кредит под эту ставку. Что изменится в нашей модели, если допустить, что некоторые участники умеют не только открывать вклады, но и брать кредиты?
\end{frame}



\begin{frame}{Арбитраж}
\justify
Допустим, что цена форварда ниже равновесной. $S_{eurrub}=85$, $r_{eur}=0.25\%$, $r_{rub}=6\%$, $F_{eurrub}=89$. 

\justify
Хитрый план:

1) Взять 1\,000\,000 евро в кредит под 0.25\%.

2) Продать евро на споте, получить 85\,000\,000 рублей.

3) Вложить рубли под 6\%.

4) Купить форвард по 89 (согласиться отдать рубли, получить евро).

5) Через год снять $85\,000\,000 \cdot (1+0.06) = 90\,100\,000$ рублей.

6) По форварду обменять рубли на $90\,100\,000 / 89 = 1\,012\,360$ евро.

7) Отдать по кредиту $1\,000\,000 \cdot (1 + 0.0025) = 1\,002\,500$ евро.

8) PROFIT: 9\,860 евро при нулевых начальных инвестициях.
\end{frame}



\begin{frame}{Арбитраж}
\justify
Пусть цена форварда выше равновесной. $S_{eurrub}=85$, $r_{eur}=0.25\%$, $r_{rub}=6\%$, $F_{eurrub}=91$. 

\justify
Второй хитрый план:

1) Взять 85\,000\,000 рублей в кредит под 6\%.

2) Продать рубли на споте, получить 1\,000\,000 евро.

3) Вложить евро под 0.25\%.

4) Продать форвард по 91 (согласиться отдать евро, получить рубли).

5) Через год снять $1\,000\,000 \cdot (1+0.0025) = 1\,002\,500$ евро.

6) По форварду обменять евро на $1\,002\,500 \cdot 91 = 91\,227\,500$ рублей.

7) Отдать по кредиту $85\,000\,000 \cdot (1 + 0.06) = 90\,100\,000$ рублей.

8) PROFIT: 1\,127\,500 рублей. Снова деньги из воздуха!
\end{frame}



\begin{frame}{Арбитраж}
\justify
Совокупность сделок, в результате которой участник рынка может ничем не рискуя заработать ненулевую прибыль при нулевых начальных вложениях, называется \alert{арбитражем} (\en{arbitrage}). Самого этого участника называют арбитражёром (\en{arbitrageur}).

\justify
Толпы алчных арбитражёров толпами рыскают по рынку и выискивают малейшие возможности для арбитража. Если рыночная цена форварда отклонится от равновесной хотя бы на мгновение, арбитражёры налетят коршунами и помогут невидимой руке рынка исправить ошибку (\en{mispricing}). Каждый хочет сделать деньги из воздуха! 

\justify
Мы будем оценивать форварды и другие деривативы так, чтобы цена дериватива не оставляла возможностей для арбитража.
\end{frame}



\begin{frame}{Репликация форварда}
\justify
Следующие две стратегии приводят к одинаковому результату:

1. Купить форвард EURRUB по 89.87531 на 1\,002\,500 евро.

2. Взять 85\,000\,000 рублей в кредит под 6\%. Купить евро на споте по 85. Положить евро на депозит под 0.25\%.

\justify
\centering
\begin{tabular}{l|l}
Стратегия 1       & Стратегия 2 \\ \hline
+1\,002\,500 EUR  & +1\,002\,500 EUR \\
-90\,100\,000 RUB & -90\,100\,000 RUB
\end{tabular}

\justify
Стратегия 2 полностью \alert{реплицирует} стратегию 1. Представьте, что стратегии спрятали в два чёрных ящика. Всё, что вы видите --- что иногда из чёрных ящиков вылетают евро, а в ящики залетают рубли. Не заглядывая внутрь, вы никогда не угадаете, в каком ящике настоящий форвард, а в каком --- синтетическая стратегия из кредита, депозита и спот-сделки.
\end{frame}



\begin{frame}{Оценка деривативов через репликацию}
\justify
Булочка стоит 30 р., котлетка 200 р., салатик 100 р., майонезик 50 р. Сколько должен стоить бургер на идеальном эффективном рынке? 380 р. плюс стоимость сборки, которая на финансовых рынках близка к нулю.

\justify
Цена дериватива выводится (derived) из цены базовых инструментов. Оценка дериватива
не абсолютная (<<какова фундаментально обоснованная цена бургера?>>), а относительная (<<если стоимость ингредиентов X, то сколько стоит их комбинация?>>).

\justify
Если мы продали клиенту бургер за 381 р., а сами собрали его из ингредиентов за 380 р., то мы зафиксируем прибыль 1 р., даже если рынок сошёл с ума и фундаментально обоснованная цена бургера 1000 р. или 100 р.

\justify
Ещё лучше --- купить бургер у одного клиента за 379 р., и через минуту продать
другом клиенту за 381 р.
\end{frame}



\begin{frame}{Безарбитражная цена форварда}
\justify
На рынке не будет возможностей для арбитража, если форвардный курс будет следовать за спот-курсом и процентными ставками:

\begin{align*}
F_{xxxyyy} &= S_{xxxyyy} \frac{1 + r_{yyy}T_{yyy}}{1 + r_{xxx}T_{xxx}} \\
%F_{xxxyyy} &= S_{xxxyyy} e^{r_{yyy}^*T_{yyy} - r_{xxx}^*T_{xxx}} \\
F_{xxxyyy} &= S_{xxxyyy} \frac{\delta_{xxx}(T)}{\delta_{yyy}(T)}
\end{align*}

\justify
Предсказывает ли форвардный курс будущий спот курс? Нет!
\end{frame}



\begin{frame}{Биржевые фьючерсы}
\justify
\alert{Валютный фьючерс} (\en{FX future}) --- биржевой контракт, по которому можно обменять одну валюту на другую по фиксированному курсу в фиксированную дату в будущем.

\justify
Отличия фьючерса от форварда:
\begin{itemize}
\justifying
\item Фьючерс торгуется на бирже, форвард --- на внебиржевом рынке.
\item Фьючерс --- стандартизованный контракт (фиксированный размер лота, стандартные даты поставки --- третья среда каждого третьего месяца). В форвардах стороны могут договориться о чём угодно.
\item Чтобы торговать фьючерсами, нужно обязательно внести гарантийное обеспечение. В форвардном контракте --- как решат стороны.
\end{itemize}

\justify
Справедливый фьючерсный курс вычисляется так же, как и форвардный.
\end{frame}



\begin{frame}{Фьючерсы и предсказание будущего курса}
\center
\begin{tikzpicture}
\begin{axis}[
  width=\textwidth,
  height=\textheight - 1cm,
  date coordinates in=x,
  date ZERO=2014-01-01,
  xtick={2014-02-01,2014-04-01, 2014-06-01, 2014-08-01, 2014-10-01, 2014-12-01},
  xticklabel={\day.\month.14},
  xmin=2014-01-01,
  xmax=2014-12-31,
  ymin=30,
  ymax=62,
  grid=major,
  ylabel={\small{Курс USDRUB}},
  xlabel near ticks,
  ylabel near ticks,
  legend entries = {
      Спот-курс ЦБ РФ,
      Цена фьючерса Si-12.14
  },
  legend pos=north west,
  %legend style={font=\tiny},
  legend cell align={left}
]
\addplot[color=Set1-A, mark=none, thick] table[x=date, y=cbr_spot_rate, col sep=comma]{Si-12.14.csv};
\addplot[color=Set1-B, mark=none, thick] table[x=date, y=futures_price, col sep=comma]{Si-12.14.csv};
\end{axis}
\end{tikzpicture}

\scriptsize Данные: Московская Биржа.
\end{frame}



\begin{frame}{Беспоставочные (расчётные) форварды}
\justify
Многие валюты развивающихся стран не являются свободно-конвертируемыми. Кроме того, запрещены деривативы, такие как форварды.

\justify
Примеры:
\begin{itemize}
\item Китай.
\item Индия.
\item Корея.
\item Тайвань.
\item Бразилия.
\item ...
\end{itemize}
\end{frame}



\begin{frame}{Беспоставочные (расчётные) форварды}
\justify
Невозможная троица международных финансов (\en{international finance trilemma}): можно выбрать любые два пункта из трёх.
\begin{itemize}
\item Фиксированный курс национальной валюты.
\item Свободное движение капитала через границу.
\item Независимая кредитно-денежная политика.
\end{itemize}

\justify
Например, мы хотим фиксированный курс 30 рублей за доллар.
\begin{itemize}
\justifying
\item ФРС повышает ставки --- все бегут в доллары
\item Чтобы удержать курс, ЦБ РФ продаёт доллары
\item Резервы большие, но конечные. Всё равно придётся либо отпустить курс, либо повысить ставки вслед за ФРС.
\end{itemize}
\end{frame}



\begin{frame}{Пример: Китай}
\begin{itemize}
\justifying
\item Движение капитала (покупка долга или акций) ограничено и требует предварительного одобрения правительства Китая.

\item Текущие операции (оплата товаров и услуг) не ограничиваются. 

\item Предоставив подтверждающие документы, можно купить или продать юани на бирже CFETS (China Foreign Exchange Trading System)

\item Народный банк Китая выступает контрагентом в 70\% сделок на CFETS и имеет неограниченные возможности для манипулирования курсом.

\item Никаких деривативов!
\end{itemize}
\end{frame}



\begin{frame}{Беспоставочный форвард}
\justify
Как управлять валютным риском, если форварды запрещены? Нужен \alert{беспоставочный форвард} (\en{non-deliverable forward, NDF}), он же расчётный форвард.

\justify
Беспоставочный форвард между Citi и Deutsche:

\justify
\centering
\begin{tabular}{l|l}
	Тип контракта 		   & Non-deliverable forward		\\
	Deutsche <<продаёт>>  & 1\,000\,000 долларов (USD)	\\
	Deutsche <<покупает>> & 6\,500\,000 юаней (CNY)		\\
	Курс		 		      & 6.50 						\\
	Дата поставки		   & 18 апреля 2022 г. (пн) \\
	Референсный курс	   & Курс Народного банка Китая	\\
	Дата фиксинга		   & 14 апреля 2022 г. (чт) \\
	Сегодня (справочно)	& 14 марта 2022 г. (пн) \\
	Спот-дата (справочно) & 16 марта 2022 г. (ср)
\end{tabular}
\end{frame}



\begin{frame}{Механика беспоставочного форварда}
\justify
При заключении контракта (14 марта) никто никому ничего не платит.

\justify
В дату фиксинга (14 апреля) Народный банк Китая публикует официальный курс, например 6.40.

\justify
Deutsche угадал, что курс снизится, поэтому:
\begin{itemize}
\justifying
\item Deutsche <<отдаёт>> 1\,000\,000 USD.
\item Deutsche <<получает>> $6\,500\,000 \text{CNY} = \dfrac{6\,500\,000}{6.40} = 1\,015\,625 \text{USD}$.
\end{itemize}

\justify
В дату поставки (12 ноября) Citi переведёт Deutsche выигрыш: 15\,625 USD.

\justify Банки не обязаны сообщать о сделке правительству Китая. Беспоставочный форвард --- мечта спекулянта.
\end{frame}



\begin{frame}{Хэджирование валютного риска}
\justify
Одна американская фруктовая компания должна заплатить поставщику 6\,500\,000 юаней через месяц.
\begin{itemize}
\item Продаём форвард 1\,000\,000 USDCNY по 6.50.
\item В дату фиксинга покупаем 6\,500\,000 CNY по спот-курсу $S$.
\item В дату поставки получаем и выплату по форварду, и юани по спот-сделке.
\end{itemize}

\justify
\centering
\begin{tabular}{l|l|l}
		Платёж 				& $S=6.40$ 			& $S=6.60$ \\
		\hline
		Беспоставочный форвард 	& $+15\,625$ USD 		& $-15\,151$ USD \\
		\hline
		\multirow{2}{*}{Спот-сделка} & $+6\,500\,000$ CNY 	& $+6\,500\,000$ CNY \\
				   			& $-1\,015\,625$ USD		& $-984\,849$ USD \\
		\hline
		Оплата поставщику 		& $-6\,500\,000$ CNY 	& $-6\,500\,000$ CNY \\ 
		 \hline
		Итого				& $-1\,000\,000$ USD 	& $-1\,000\,000$ USD
\end{tabular}
\end{frame}



\begin{frame}{Ценообразование беспоставочного форварда}
\justify
Аргументы про рыночное равновесие и отсутствие арбитража не работают, если есть ограничения на движение капитала. Цена беспоставочного форварда может далеко отклониться от <<теоретической>>, и никто не сможет её заарбитражить.

\justify
Беспоставочный форвард на 1 месяц --- это скорее базовый актив, а не дериватив. Невидимая рука рынка ищет такую цену, при которой спрос (желающие заплатить поставщикам в Китае или поспекулировать на укреплении доллара) и предложение (желающие вывести прибыль из Китая или поспекулировать на укреплении юаня) уравновешиваются.
\end{frame}

\end{document}