\documentclass{beamer}

\usepackage{cmap}				% To be able to copy-paste russian text from pdf
\usepackage[T2A]{fontenc}
\usepackage[utf8]{inputenc}
\usepackage[russian]{babel}
\usepackage{textpos}
\usepackage{ragged2e}
\usepackage{amssymb}
\usepackage{ulem}
\usepackage{tikz}
\usepackage{pgfplots}
\usepackage{color}
\usepackage{cancel}
\usepackage{multirow}
\pgfplotsset{compat=1.17}
\usetikzlibrary{arrows,snakes,backgrounds,shapes}
\usepgfplotslibrary{groupplots,colorbrewer,dateplot,statistics}
\usepackage{animate}

\usepackage{amsfonts}
\usepackage{amsmath}
\usepackage{amssymb}
\usepackage{graphicx}
\usepackage{setspace}

\usepackage{enumitem}
\setitemize{label=\usebeamerfont*{itemize item}%
  \usebeamercolor[fg]{itemize item}
  \usebeamertemplate{itemize item}}

% remove navigation bar
\setbeamertemplate{navigation symbols}{} 

\usepackage{eurosym}
\renewcommand{\EUR}[1]{\textup{\euro}#1}

\title{Interest Rate Swaps}
\author{Artem Bakulin}
\date{October 17, 2023}

\usetheme{Warsaw}
\usecolortheme{beaver}

% remove navigation bar
\setbeamertemplate{navigation symbols}{}

\setbeamertemplate{page number in head/foot}[totalframenumber] 

\newcommand{\ru}[1]{\begin{otherlanguage}{russian}#1\end{otherlanguage}}
\newcommand{\en}[1]{\begin{otherlanguage}{english}#1\end{otherlanguage}}
\newcommand{\ruen}[2]{#1 (\en{#2})}

\begin{document}



\begin{frame}
\titlepage
\end{frame}

\begin{frame}{Compound interest}
\justify
"Compound interest is the most powerful force in the universe" (attribution to Albert Einstein is wrong).

\justify
Consider a deposit at interest rate $r$ for $T$ years. Interest compounding frequency is $f$ times a year. Initial amount is $N_0$. At the end an investor will receive
\begin{align*}
N &= N_0\underbrace{\left(1 + \frac{r}{f}\right) \cdot \left(1 + \frac{r}{f}\right) \cdot ... \cdot \left(1 + \frac{r}{f}\right)}_{fT \text{terms}} = 
N_0\left(1 + \frac{r}{f}\right)^{fT}
\end{align*}

Example: a deposit of 1\,000\,000 at 12\% for 1 year with monthly compounding:
\begin{align*}
1\,000\,000 \cdot \left(1 + \frac{0.12}{12}\right)^{12} \approx 1\,126\,825
\end{align*}
\end{frame}



\begin{frame}{Common capitalization frequencies}
\centering
\begin{tabular}{l|l}
Frequency ($f$) & Name \\ \hline 
0 & Zero (no compounding) \\
1 & Annually \\
2 & Semi-annually \\
4 & Quarterly \\
12 & Monthly \\
365 & Daily \\
$\infty$ & Continuously 
\end{tabular}

\justify
Simple interest (also known as zero compounding), $f=0$:
\begin{align*}
N = N_0 (1 + rT)
\end{align*}

Continuous compounding, $f=\infty$:
\begin{align*}
N = N_0e^{rT}
\end{align*}
\end{frame}



\begin{frame}{Continuously compounded interest rate}
\justify
Suppose that we add a tiny interest to the deposit every hour, every minute, every second. Compounding frequency $f$ goes to infinity.

\begin{align*}
N &= N_0 \lim_{f \to +\infty} \left(1 + \frac{r}{f}\right)^{fT} = 
\begin{Bmatrix}x= f/r \\ f=xr \\ x \to +\infty \\ f \to +\infty\end{Bmatrix} = \\
&= N_0\lim_{x \to +\infty} \left(1 + \frac{r}{xr}\right) ^ {xrT} =
N_0{\underbrace{\left(\lim_{x \to +\infty} \left( 1 + \frac{1}{x} \right) ^ x\right)}_{e \approx 2.71828}}  ^ {rT} = \\
&= N_0e^{rT} \approx N_0(1 + rT)
\end{align*}
Continuously compounded interest rates are very uncommon in real life, but they help us make theoretical formulas a bit more concise.
\end{frame}



\begin{frame}{Compound interest  - 1}
\justify
Consider deposits of 1\,000\,000 at the interest rate of 12\% with terms of 1 and 5 years. How does terminal amount on the account depend on compounding frequency?


\begin{table}
\centering
\begin{tabular}{l|r|r|r}
Compounding frequency & $f$ & 1 год & 5 лет \\ \hline
Zero & 0         & 1\,120\,000 & 1\,600\,000 \\
Annually          & 1         & 1\,120\,000 & 1\,762\,342 \\
Semi-annually     & 2         & 1\,123\,600 & 1\,790\,848 \\
Quarterly     & 4         & 1\,125\,509 & 1\,806\,111 \\
Monthly        & 12        & 1\,126\,825 & 1\,816\,697 \\
Daily         & 365       & 1\,127\,475 & 1\,821\,939 \\
Continuously        & $+\infty$ & 1\,127\,497 & 1\,822\,119
\end{tabular}
\end{table}

\justifying
The higher the interest rate and longer the term, the more important becomes compounding frequency. Daily compounding is a reasonably good  approximation of continuous compounding.
\end{frame}



\begin{frame}{Compound interest - 2}

	\centering
	\begin{tikzpicture}
		\begin{axis}[
			xlabel=\text{Time, months},
			ylabel=\text{Deposit amount},
			xmin=0,
			xmax=24.5,
			width=\textwidth,
			height=\textheight - 1.5cm,
			ymin=100,
			ymax=130,
			legend style={at={(0.01, 0.99)}, anchor=north west}
		]
	

		\addplot[const plot, samples at={0,0.1,...,60}, very thick, color=Set1-A] {100 * (1 + 0.12/4) ^ (4*3*floor(x/3)/12.0)};
		\addlegendentry{$f=4$}
	
		\addplot[const plot, samples at={0,0.5,...,60}, very thick, color=Set1-B] {100 * (1 + 0.12/12) ^ (12*floor(x)/12.0)};
		\addlegendentry{$f=12$}
		
		\addplot[domain=0:60, color=Set1-C, very thick] {100 * exp(0.12*x/12)};
		\addlegendentry{$f=+\infty$}
		\end{axis}
	\end{tikzpicture}
	\scriptsize{Growth of the deposit over time, $N_0=100$, $r=12\%$}
\end{frame}



\begin{frame}{Day count conventions}
\justify
How many years elapse between March 1, 2023 and April 1, 2023?

\justify
\centering
\begin{tabular}{l|l}
Convention & Year fraction \\ \hline
ACT/365 & $31/365 \approx 0.08493$ \\
ACT/360 & $30/360 \approx 0.08611$ \\
30/360  & $30/360 \approx 0.08333$ \\
BUS/252 & $23/252 \approx 0.09127$ 
\end{tabular}

\justify
ACT --- count calendar days. 30 --- assume there are 30 days in each month. BUS --- count only business days.

\justify
Which method should you choose? Usually there is a market convention. Rule of thumb: ACT/365 in the British Commonwealth, ACT/360 in most other countries, BUS/252 in Brazil. 
\end{frame}



\begin{frame}{30/360 convention}
\justify
How many years are there between dates $D_1.M_1.Y_1$ and $D_2.M_2.Y_2$?
\begin{align*}
T = \frac{360(Y_2-Y_1) + 30(M_2-M_1) + (D_2-D_1)}{360}
\end{align*}

\justify
The 30/360 convention originated in pre-computer era, because it allows for quick mental math. It is still being used, for example, on the bond market.
\begin{itemize}
\item Backward compatibility
\item The size of interest payments does not fluctuate during the year
\end{itemize}

\vspace{\baselineskip}
\justify
* There are some variations: 30/360 Bond, 30/360 US, 30E/360, 30E/360 ISDA. They differ in handling February and 31-days months. Kudos to financial math libraries which hide these details from us!
\end{frame}



\begin{frame}{Converting between methods of compounding}
\justify
Suppose that you know a continuously compounded interest rate $r^*$, and you need a rate $r$ at frequency $f$ or vice versa. Then you can convert between conventions.
\begin{align*}
e^{r^*T} &= \left(1 + \frac{r}{f}\right)^{fT} \\
r^* &= f\ln \left(1 + \frac{r}{f}\right) \\
r &= f\left(e^{r^*/f} - 1\right)
\end{align*}

\justify
The same approach is also valid for zero rates without compounding.
\begin{align*}
e^{r^*T} &= 1+rT \\
r^* &= \frac{\ln(1+rT)}{T} \\
r &= \frac{e^{r^*T}-1}{T}
\end{align*}

\end{frame}



\begin{frame}{Present value}
\justify
\alert{Present value} of a cashflow of $N$ units of currency in $T$ years is the price which market participants would be willing to pay today to be entitled for this future cashflow.

\justify
How much USD you would be willing to pay today (present value, PV) to be entitled to receive 1\,050\,000 USD in one year (future value, FV)? Suppose that the market offers and ideal risk-free deposit at 5\% with zero compounding.

\begin{align*}
PV = \frac{FV}{1+rT} = \frac{1\,050\,000}{1 + 5\%} = 1\,000\,000
\end{align*}

\justify
Nobody will be willing to pay more than 1\,000\,000 current dollars for 1\,050\,000 future dollars. It would be more profitable to deposit current dollars at 5\%.

\justify
There is no sense in selling 1\,050\,000 future dollars cheaper than 1\,000\,000 today, because one could borrow dollars today at 5\% and pay the debt in 1 year.
\end{frame}



\begin{frame}{Discount factor}
\justify
\alert{Discount factor} for a future date $T$ is present value of a cashflow of 1 unit of a currency on date $T$. How many units of a currency one needs to posses today in order to make 1 unit of a currency by time $T$?

\justify
The answer depends on risk-free interest rate:
\begin{align*}
\delta_T &= \frac{1}{1 + rT} \quad \text{(simple interest)} \\
\delta_T &= \frac{1}{e^{r^*T}} = e^{-r^*T} \quad \text{(continuously compounded interest)}
\end{align*}

\justify
For example, suppose that risk-free interest rate is 5\%.  Discount factor for 1 year us then
\begin{align*}
\delta_1 = \frac{1}{1.05} \approx 0.9524
\end{align*}
\end{frame}



\begin{frame}{Deferred deposit}
\justify
The market offers two deposits (simple interest)
\begin{itemize}
\item Term $T_1=1$ year, interest rate $r_1=5\%$ 
\item Term $T_2=2$ years, interest rate $r_2=4\%$ 
\end{itemize}

A bank offers a  "deferred" deposit. You are obliged to deposit the cash in 1 year for 1 year. The bank is obliged to tell you the interest rate $x$ today. What would be fair rate for this deposit?

\centering
\begin{tikzpicture}
		\draw [->,>=triangle 90] (0, 0) -- (8.5, 0);

		\draw [->,>=triangle 45] (1,0) node[anchor=north east]{$0$} .. controls (1.5, 1) and (3.5, 1) .. (4,0) node[anchor=north]{$T_1=1$} node[pos=0.5,anchor=south]{$r_1=5\%$};

		\draw [->,>=triangle 45] (4,0) .. controls (4.5, 1) and (6.5, 1) .. (7,0) node[anchor=north west]{$T_2=2$} node[pos=0.5,anchor=south]{$x$};

		\draw [->,>=triangle 45] (1,0) .. controls (1.5, -1) and (6.5, -1) .. (7,0) node[pos=0.5,anchor=north]{$r_2=4\%$};
	\end{tikzpicture}
\end{frame}



\begin{frame}{Deferred deposit - 2}
\centering
\begin{tikzpicture}
		\draw [->,>=triangle 90] (0, 0) -- (8.5, 0);

		\draw [->,>=triangle 45] (1,0) node[anchor=north east]{$0$} .. controls (1.5, 1) and (3.5, 1) .. (4,0) node[anchor=north]{$T_1=1$} node[pos=0.5,anchor=south]{$r_1=5\%$};

		\draw [->,>=triangle 45] (4,0) .. controls (4.5, 1) and (6.5, 1) .. (7,0) node[anchor=north west]{$T_2=2$} node[pos=0.5,anchor=south]{$x$};

		\draw [->,>=triangle 45] (1,0) .. controls (1.5, -1) and (6.5, -1) .. (7,0) node[pos=0.5,anchor=north]{$r_2=4\%$};
	\end{tikzpicture}
	
\justifying
The should be no difference between a longer deposit and a chain of two shorter deposits.
\begin{align*}
(1 + r_1T_1)\Big(1+x(T_2-T_1)\Big) = 1 + r_2T_2
\end{align*}
\begin{align*}
x = \frac{\dfrac{1+r_2T_2}{1+r_1T_1} - 1}{T_2-T_1} = \frac{\dfrac{1 + 0.04 \cdot 2}{1 + 0.05} - 1}{2-1} \approx 2.86\%
\end{align*}

At any other "deferred" rate $x$ the deferred deposit would be unfavorable for the bank or for clients.
\end{frame}



\begin{frame}{Continuous compounding}
\justify
Convert simple interest rate into continuously compounded rates:
\begin{align*}
r_1^* &= \frac{\ln(1+r_1T_1)}{T_1} = \frac{\ln 1.05}{1} \approx 4.879\% \\
r_2^* &= \frac{\ln(1+r_2T_2)}{T_2} = \frac{\ln 1.08}{2} \approx 3.848\%
\end{align*}

\centering
\begin{tikzpicture}
		\draw [->,>=triangle 90] (0, 0) -- (8.5, 0);

		\draw [->,>=triangle 45] (1,0) node[anchor=north east]{$0$} .. controls (1.5, 1) and (3.5, 1) .. (4,0) node[anchor=north]{$T_1=1$} node[pos=0.5,anchor=south]{$r_1^*=4.879\%$};

		\draw [->,>=triangle 45] (4,0) .. controls (4.5, 1) and (6.5, 1) .. (7,0) node[anchor=north west]{$T_2=2$} node[pos=0.5,anchor=south]{$x^*$};

		\draw [->,>=triangle 45] (1,0) .. controls (1.5, -1) and (6.5, -1) .. (7,0) node[pos=0.5,anchor=north]{$r_2^*=3.848\%$};
	\end{tikzpicture}
\end{frame}



\begin{frame}{Continuous compounding - 2}
\centering
\begin{tikzpicture}
		\draw [->,>=triangle 90] (0, 0) -- (8.5, 0);

		\draw [->,>=triangle 45] (1,0) node[anchor=north east]{$0$} .. controls (1.5, 1) and (3.5, 1) .. (4,0) node[anchor=north]{$T_1=1$} node[pos=0.5,anchor=south]{$r_1^*=4.879\%$};

		\draw [->,>=triangle 45] (4,0) .. controls (4.5, 1) and (6.5, 1) .. (7,0) node[anchor=north west]{$T_2=2$} node[pos=0.5,anchor=south]{$x^*$};

		\draw [->,>=triangle 45] (1,0) .. controls (1.5, -1) and (6.5, -1) .. (7,0) node[pos=0.5,anchor=north]{$r_2^*=3.848\%$};
	\end{tikzpicture}

\justifying
There should be no difference between a 2 year deposit and a combination of two shorter deposits.
\begin{align*}
e^{r_1^*T_1}e^{x^*(T_2-T_1)} = e^{r_2^*T_2}
\end{align*}
\begin{align*}
x^* = \frac{r_2^*T_2-r_1^*T_1}{T_2-T_1} = \frac{3.848\%\cdot2 - 4.879\%}{2 - 1} \approx 2.817\%
\end{align*}

Covert continuously compounded rate back into simple interest rate:
\begin{align*}
x = \frac{e^{x^*(T_2-T_1)} - 1}{T_2 - T_1} = e^{0.02817} - 1 \approx 2.86\% 
\end{align*}
\end{frame}



\begin{frame}{Continuous compounding - 3}
\justify
People are willing to invest $100$ dollars in order to receive $105$ in one year. This is observable reality. We can interpret this as "simple" interest rate 5\%, as continuously compounded interest rate 4.879\%, as discount factor of 0.9524. These are all equivalent views (using different units of measurement) on one and the same situation.

\justify 
Depending on the context it may be more convenient to prefer one abstraction over the other. Human beings like simple interest. Continuously compounded rates simplify mathematics and source code. Discount factors help in case we know future cashflows and need to know their value in todays money.
\end{frame}



\begin{frame}{European Interbank Offered Rate}
\justify
\alert{EURIBOR} is benchmark interest rate for unsecured Euro loans among European bank. It is published by European Money Market Institute (EMMI).

\justify
At which rate a large bank could borrow Euro on the market (from another bank or financial organization)?

\justify
\centering
\begin{tabular}{l|r}
Term (tenor)     & EURIBOR 13.10.2023 \\ \hline
1W (1 week)    & $3.874\%$ \\
1M (1 month)     & $3.860\%$ \\
3M (3 months)    & $3.985\%$ \\
6M (6 months)   & $4.121\%$ \\
12M (12 months) & $4.171\%$ 
\end{tabular}

\justify
*EURIBOR is simple rate (no compounding) in ACT/360 convention.
\end{frame}



\begin{frame}{Computing EURIBOR}
\justify
19 banks from the Eurozone and the UK take part in determining the EURIBOR. Every day each bank submits its interest rate for each of 5 terms (from 1 week to 1 year). The banks follow a methodology which defines 3 "levels":

\justify 
1) Weighted average interest rate of actual transactions which the bank executed on this day on this term. Each transaction should not be smaller than 20mm.

\justify
2) Interpolation of actual transactions from different terms (e.g. interpolate 1M from 1W and 3M) or extrapolating older transactions of the same term.

\justify
3) Interest rates from similar instruments, transactions smaller than 20mm, derivatives, deals with non-financial parties, expert judgment. 

\justify
15\% highest quotes and 15\% lowest quotes are truncated. Arithmetic average of the remaining 70\%, rounded to the third decical digit, is proclaimed the EURIBOR.
\end{frame}



\begin{frame}{Computing EURIBOR - 2}
\centering
\begin{tikzpicture}
		\draw [->,>=triangle 90] (0, 0) -- (9.5, 0);

		\draw [dashed] (0.5,0) node[anchor=north]{$T-1$} .. controls (1.0, 0.5) and (2.5, 0.5) .. (3,0) node[anchor=north]{$T+1$} node[pos=0.5,anchor=south]{spot lag};

		\draw [->,>=triangle 45] (3,0) .. controls (4, 1) and (7, 1) .. (8,0) node[anchor=north]{$T+1+3M$} node[pos=0.5,anchor=south]{$EURIBOR_{T}$};

		\node[anchor=north] at (1.75, 0) {$T$};
		
		\node[circle, fill, inner sep=1.5pt] at (0.5, 0) {};
		\node[circle, fill, inner sep=1.5pt] at (1.75, 0) {};
		\node[circle, fill, inner sep=1.5pt] at (3.0, 0) {};
		\node[circle, fill, inner sep=1.5pt] at (8.0, 0) {};
	\end{tikzpicture}
	
\justify
T--1 (yesterday):  banks were borrowing Euro and gathering data.

T (today): at 11:00 the EMMI announces the average of yesterdays transactions --- 3 months EURIBOR on date $T$.

T+1 (tomorrow): the transactions will be settled (borrowers will receive their Euros).

T+1+3M (in 3 months): the loans will expire (lenders will receive their money back).

\justify
EURIBOR is published on every TARGET2 business day (this is the ECB's payment system). Three months term is computed under "modified following business day" convention.
\end{frame}



\begin{frame}{Date rolling convention}
\justify
A 3-months loan began on Friday March 31, 2023. When is it going to end: on Friday June 30 or on Monday July 3?

\justify
Modified following / business month end convention:

\justify
1) In case the start date is the last business day of a month, then the end date is also last business day of the corresponding month. Example: 31.03.2022 (Friday) --- 30.06.2023 (Friday).

\justify
2) In case end date has fallen on a non-business day, then it is rolled forwards, unless the nearest business day is in the following month. Example: 17.03.2023 (Friday) --- 19.06.2023 (Monday).

\justify
3) Otherwise the end date is rolled backwards. Example: 29.11.2022 (Tuesday) --- 28.02.2023 (Tuesday).
\end{frame}



\begin{frame}{Euro short-term rate}
\justify
\alert{ESTER} is bencmark interest rate for unsecured overnight loans among banks of the Eurozone. It is published by the ECB.

\justify
\centering
\begin{tabular}{l|r}
Term  & ESTER 12.10.2023 \\ \hline
1 day & $3.901\%$
\end{tabular}

\justify
A panel of 48 banks determine ESTER. Each bank informs the ECB about its transactions when it borrows Euro on the market. The ECB discard the lower 25\%  and the 25\% higher values and then computes the weighed average.

\justify
ESTER is simple interest rate (no compounding) in ACT/360 convention.
\end{frame}



\begin{frame}{Computing ESTER}
\justify
\centering
\begin{tikzpicture}
		\draw [->,>=triangle 90] (0, 0) -- (4.0, 0);

		\draw [->, >= triangle 45] (0.5,0) node[anchor=north]{$T$} .. controls (1.0, 0.75) and (2.5, 0.75) .. (3,0) node[anchor=north]{$T+1$} node[pos=0.5,anchor=south]{$ESTER_T$};
\end{tikzpicture}

\justify
T (today): banks borrow Euro overnight and communicate their transactions to the ECB.

T+1 (tomorrow): the ECB announces the weighted average rate --- ESTER for the date T. The loans expire and banks pay Euro back to the lenders.

\justify
Similarly to EURIBOR, ESTER is computed on TARGET2 business days.
\end{frame}



\newcommand{\plotBenchmarkRate}[2] {
	
	\addplot[
		color = #2,
		mark = none,
		thick
	]
	table[
		x=date,
		y=#1,
		col sep=comma
	]
	{euro_benchmark.csv};
}



\begin{frame}{ESTER and EURIBOR}
\centering
\begin{tikzpicture}
\begin{axis}[
  width=\textwidth,
  height=\textheight - 1cm,
  date coordinates in=x,
  date ZERO=2012-01-01,
  xtick={2012-01-01, 2014-01-01, 2016-01-01, 2018-01-01, 2020-01-01, 2022-01-01, 2024-01-01},
  minor xtick={2013-01-01, 2015-01-01, 2017-01-01, 2019-01-01, 2021-01-01, 2023-01-01},
  ytick={-1, 0, 1, 2, 3, 4},
  minor ytick={-0.5, 0.5, 1.5, 2.5, 3.5},
  xticklabel={\year},
  xmin=2012-01-01,
  xmax=2024-01-01,
  ymin=-1,
  ymax=4.5,
  grid=both,
  yticklabel={\pgfmathprintnumber{\tick}\%},
%  ylabel={\small{Курс USDRUB}},
  xlabel near ticks,
  ylabel near ticks,
  legend entries = {
  	   EURIBOR 6M,
      EURIBOR 3M,
      ESTER
  },
  legend cell align={left},
  legend style={at={(0.03,0.97)},anchor=north west}
]

	\plotBenchmarkRate{6m}{Set1-A}
	\plotBenchmarkRate{3m}{Set1-B}
	\plotBenchmarkRate{ester}{Set1-C}
	
	\draw[thick, color=black] (axis cs: 2012-01-01, 0) -- (axis cs: 2024-01-01, 0);
\end{axis}
\end{tikzpicture}

\scriptsize Data: Bundesbank.
\end{frame}



\begin{frame}{EURIBOR futures}
\justify
ICE Futures Europe exchange (former LIFFE) offers three months EURIBOR futures. For examples, \alert{FEIZ3} is December 2023 futures.

\justify
\centering
\begin{tabular}{r|r}
Bid & Offer \\ \hline
95.99 & 96.00
\end{tabular}

\justify
Suppose that we blindly click "buy 10 lots at 96.00".
\begin{itemize}
\justifying
\item No initial cost to be paid today.
\item On December 18 2023** the exchange will set final price for the futures at $F = 100 - X$, where 
$X$ is the EURIOBOR which will be announced on that day.
\item We will earn $10 \cdot (F - 96.00) \cdot \EUR{2\,500}$, where \EUR{2\,500} is a constant from the futures' specification.
\end{itemize}

\justify
*FEI is contract code. Z is month code (H for March, M for June, U for September, Z for December). 3 is last digit of the year.

\justify
**Two business days prior to the third Wednesday of the month.
\end{frame}



\begin{frame}{EURIBOR futures - 2}
\justify
I know in advance that in December I will become richer by \EUR{1\,000\,000}. I plan to deposit the money for three months, collect it in March and go on holiday. I face \alert{interest rate risk}: I do not know how  much money I will have by March.
\justify
Solution: buy one lot of EURIBOR December futures at 96.00.

\justify
Suppose that by December 18 the EURIBOR will drop to $3.50\%$.
\begin{itemize}
\justifying
\item The exchange will set final price at $100 - 3.50 = 96.50$.
\item Gain on the futures is $(96.50-96.00)\cdot\EUR{2\,500} = \EUR{1\,250}$.
\item Three months deposit at EURIBOR $3.50\%$ will earn

$\EUR{1\,000\,000} \cdot 0.035 / 4 = \EUR{8\,750}$.
\item In total, I will earn $\EUR{1\,250} + \EUR{8\,750} = \EUR{10\,000}$.
\end{itemize} 

\justify
I would have earned the same amount if EURBOR was $4.00\%$:

$\EUR{1\,000\,000} \cdot 0.04/4 =\EUR{10\,000}$.
\end{frame}



\begin{frame}{EURIBOR futures - 3}
\justify
We bought the December futures in October at 96.00. In December we deposited cash at the prevailing EURIBOR. How much money are we going to make?\
\justify
\centering
\begin{tabular}{r|r|r|r|r}
EURIBOR   & Final price & Futures         & Deposit & Total \\ \hline
$2.50\%$ & 97.50        & $\EUR{3\,750}$      &   $\EUR{6\,250}$  & $\EUR{10\,000}$ \\
$3.00\%$ & 97.00        & $\EUR{2\,500}$      &   $\EUR{7\,500}$  & $\EUR{10\,000}$ \\
$3.50\%$ & 96.50        & $\EUR{1\,250}$      & $\EUR{8\,750}$  & $\EUR{10\,000}$ \\
$4.00\%$ & 96.00        & $\EUR{0}$                & $\EUR{10\,000}$  & $\EUR{10\,000}$ \\
$4.50\%$ & 95.50        & $-\EUR{1\,250}$    & $\EUR{11\,250}$ & $\EUR{10\,000}$ \\
$5.00\%$ & 95.00        & $-\EUR{2\,500}$    & $\EUR{12\,500}$ & $\EUR{10\,000}$
\end{tabular}

\justify
When we buy 1 futures lot at $100-X$ we secure interest rate for a future deposit of \EUR{1\,000\,000} at EURIBOR. When we sell we secure a future loan of \EUR{1\,000\,000} at EURIBOR.
\end{frame}



\begin{frame}{Interest rate swap}
\justify
\alert{Interest rate swap} is a derivative in which two parties exchange interest payments at a fixed rate and a floating (reference) rate.

\justify
Parameters of a swap:
\begin{itemize}
\justifying
\item Term or maturity. For example, 1 year.
\item Notional amount or principal amount --- an amount that accrues interest. For example, \EUR{1} million.
\item Coupon or swap rate --- fixed interest rate which is paid by one f the parties. For example, $3.0\%$ per annum.
\item Reference rate --- particular interest rate benchmark which is paid by the other party. For example, 3M EURIBOR.
\end{itemize}

\justify
Buying a swap means receiving the floating rate and paying the fixed rate. Selling a swap means paying the floating rate and receiving the fixed rate..
\end{frame}



\begin{frame}{Interest rate swap - 2}
\centering
\begin{tikzpicture}[thick, scale=0.75]
		\draw (0, 0) node[rectangle,draw,rounded corners,anchor=south,minimum height=1cm] {Company A} -- (0, -7);
		\draw (9.5, 0) node[rectangle,draw,rounded corners,anchor=south,minimum height=1cm] {Company B} -- (9.5, -7);

		\draw [dashed,->,>=triangle 90] (0, -1) node[label=left:{$T_0$}]{} -- (4.5, -1) node[pos=0.5,anchor=south]{$\EUR{1\,000\,000}$};

		\draw [dashed,->,>=triangle 90] (9.5, -1) node[label=right:{\text{$E_1$ fixing}}]{} -- (5, -1) node[pos=0.5,anchor=south]{\euro 1\,000\,000};

		\draw [->,>=triangle 90] (0, -2.5) node[label=left:{$T_1 = \text{3M}$}]{} -- (4.5, -2.5) node[pos=0.5,anchor=south]{$\EUR{1\,000\,000} \cdot \dfrac{3.0\%}{4}$};

		\draw [snake=snake,->,>=triangle 90] (9.5, -2.5)  node[label=right:{\text{$E_2$ fixing}}]{} -- (5, -2.5) node[pos=0.5,anchor=south] {$\EUR{1\,000\,000} \cdot \dfrac{E_1}{4}$};

		\draw [->,>=triangle 90] (0, -4) node[label=left:{$T_2 = \text{6M}$}]{} -- (4.5, -4) node[pos=0.5,anchor=south]{$\EUR{1\,000\,000} \cdot \dfrac{3.0\%}{4}$};

		\draw [snake=snake,->,>=triangle 90] (9.5, -4)  node[label=right:{\text{$E_3$ fixing}}]{} -- (5, -4) node[pos=0.5,anchor=south] {$\EUR{1\,000\,000} \cdot \dfrac{E_2}{4}$};

		\draw [->,>=triangle 90] (0, -5.5) node[label=left:{$T_3 = \text{9M}$}]{} -- (4.5, -5.5) node[pos=0.5,anchor=south]{$\EUR{1\,000\,000} \cdot \dfrac{3.0\%}{4}$};

		\draw [snake=snake,->,>=triangle 90] (9.5, -5.5)  node[label=right:{\text{$E_4$} fixing}]{} -- (5, -5.5) node[pos=0.5,anchor=south] {$\EUR{1\,000\,000} \cdot \dfrac{E_3}{4}$};
		\draw [->,>=triangle 90] (0, -7) node[label=left:{$T_4 = \text{12M}$}]{} -- (4.5, -7) node[pos=0.5,anchor=south]{$\EUR{1\,000\,000} \cdot \dfrac{3.0\%}{4}$};

		\draw [snake=snake,->,>=triangle 90] (9.5, -7) -- (5, -7) node[pos=0.5,anchor=south] {$\EUR{1\,000\,000} \cdot \dfrac{E_4}{4}$};
\end{tikzpicture}
\end{frame}



\begin{frame}{Managing interest rate risk}
\justify
An interest rate swap turns a loan at floating rate (for example EURIBOR+1\%) into a loan at fixed rate (for example, 4.0\%), and vice versa..

\justify
\centering
	\begin{tikzpicture}[thick]

		\tikzstyle{company}=[rectangle,draw,rounded corners,minimum height=1.2cm,minimum width=3cm];
		\tikzstyle{fixed}=[->,>=triangle 90];
		\tikzstyle{floating}=[snake=snake,->,>=triangle 90];

		\node (A) at (0, 0) [company] {Company A};
		\node (B) at (7, 0) [company] {Company B};
		\node (Deutsche) at (0, -3) [company] {Deutsche};
		\node (Citi) at (7, -3) [company] {Citi};

		\draw [fixed] ([yshift=0.2cm]A.east) -- ([yshift=0.2cm]B.west) node[pos=0.5,anchor=south]{3.0\%};

		\draw [floating] ([yshift=-0.2cm]B.west) -- ([yshift=-0.2cm]A.east) node[pos=0.5,anchor=north]{EURIBOR};

		\draw [floating] (A.south) -- (Deutsche.north) node[pos=0.5,anchor=west]{EURIBOR + 1\%};

		\draw [fixed] (B.south) -- (Citi.north) node[pos=0.5,anchor=east]{$3.5\%$};
	\end{tikzpicture}

\justify
Company A pays 4.0\%, company B pays EURIBOR + 0.5\%.
\end{frame}



\begin{frame}{Loans at floating rate}
\justify
Suppose that we are a small company with low credit rating. We need a loan for 5 years.

\begin{itemize}
\justifying
\item It costs EURIBOR + 2\% to borrow the money for 3 months. 
\item It costs 7.0\% to borrow them money for 5 years from a bank.
\item 5 years swap rate is 4.0\%.
\end{itemize}

\justify
We can make a bet we will be able to borrow at EURIBOR+2\% in the future just like today. Then we can start a chain of 3 months loans (we borrow at time $T$ to repay the debt which we borrowed at time $T-1$). Then buy a swap to hedge interest rate risk.

\begin{itemize}
\justifying
\item Loans: pay EURIBOR+2\%.
\item Bought swap: pay 4\%, receive EURIBOR.
\item Net: pay 6\%.
\end{itemize}

\justify
Is there a catch? Our credit quality may deteriorate in future. One day we may have to renew the loan at EURIBOR+10\% and the strategy will blow up.
\end{frame}



\begin{frame}{Fair swap rate}
\centering
\begin{tikzpicture}[thick, scale=0.7]
		\draw (0, 0) node[rectangle,draw,rounded corners,anchor=south,minimum height=1cm] {Company A} -- (0, -3);
		\draw (7.5, 0) node[rectangle,draw,rounded corners,anchor=south,minimum height=1cm] {Company B} -- (7.5, -3);

		\draw [->,>=triangle 90] (0, -1) node[label=left:{$T_1$}]{} -- (3.5, -1) node[pos=0.5,anchor=south]{$N \cdot x \cdot t_1$};

		\draw [snake=snake,->,>=triangle 90] (7.5, -1)  -- (4, -1) node[pos=0.5,anchor=south] {$N \cdot E_1 \cdot t_1$};
		
		\draw [->,>=triangle 90] (0, -2.5) node[label=left:{$T_2$}]{} -- (3.5, -2.5) node[pos=0.5,anchor=south]{$N \cdot x \cdot t_2$};

		\draw [snake=snake,->,>=triangle 90] (7.5, -2.5) -- (4, -2.5) node[pos=0.5,anchor=south] {$N \cdot E_2 \cdot t_2$};

		\draw (0, -3.5) -- (0, -4.5);				
		\draw (7.5, -3.5) -- (7.5, -4.5);

		\draw [->,>=triangle 90] (0, -4) node[label=left:{$T_i$}]{} -- (3.5, -4) node[pos=0.5,anchor=south]{$N \cdot x \cdot t_i$};

		\draw [snake=snake,->,>=triangle 90] (7.5, -4) -- (4, -4) node[pos=0.5,anchor=south] {$N \cdot E_i \cdot t_i$};
\end{tikzpicture}

\justify
$N$ --- notional amount.

$T_i$ --- number of years till the $i$-th payment.

$E_i$ --- expected EURIBOR on date $T_{i-1}$.

$t_i = T_i - T_{i-1}$ --- year fractions for interest accrual.

$\delta_i$ --- discount factor from date $T_i$ to today.

\justify
What should be fair swap rate (fair coupon) $x$?
\end{frame}



\begin{frame}{Fair swap rate - 2}
\justify
Present values of company A's payments and company B's payments should match. Otherwise one of the parties is just giving the money away.

\justify
\begin{align*}
\sum\limits_{i=1}^{n} N \cdot x \cdot t_i \cdot \delta_i 
= \sum\limits_{i=1}^{n} N \cdot E_i \cdot t_i \cdot \delta_i
\end{align*}
\begin{align*}
x = \frac{\sum\limits_{i=1}^{n} E_i t_i \delta_i}{\sum\limits_{i=1}^{n} t_i \delta_i}
\end{align*}

\justify
If someone tells us future values of the EURIBOR $E_i$ and discount factors $\delta_i$ we will be able to compute fair coupon for any interest rate swap.
\end{frame}



\begin{frame}{Fair swap rate - 3}
\justify
Suppose that the discount rate is close to zero, and all discount factors are close to zero. $\delta_i \approx1$.

\justify
Lets ignore peculiarities of the Gregorian calendar and assume that all intervals between interest rate payments are equal. For example, $t_i = t = 1/4$ in a swap with quarterly payments.

\begin{align*}
x = \frac{\sum\limits_{i=1}^{n} E_i t_i \delta_i}{\sum\limits_{i=1}^{n} t_i \delta_i}
\approx
\frac{\sum\limits_{i=1}^{n} E_i t}{\sum\limits_{i=1}^{n} t}
=
\frac{\sum\limits_{i=1}^{n} E_i}{n}
\end{align*}

\justify
Captain Obvious is reporting: swap rate is average of EURIBOR's during the swap's lifetime.
\end{frame}



\begin{frame}{Computing expected EURIBOR values}
\justify
Bloomberg and Reuters terminals broadcast the following quotes from the market for three months EURIBOR swaps. This is our observable reality.

\justify
\centering
\begin{tabular}{l|r}
Instrument        & Swap rate (coupon) \\ \hline
Last fixing & 3.00\% \\
6M swap           & 3.40\% \\
9M swap           & 3.60\% \\
1Y swap           & 3.70\% \\
18M swap          & 3.75\% \\
2Y swap           & 3.60\%
\end{tabular}

\end{frame}



\begin{frame}{Computing expected EURIBOR values - 2}
\centering
\begin{tikzpicture}[thick, scale=0.7]
		\draw (0, 0) node[rectangle,draw,rounded corners,anchor=south,minimum height=1cm] {Company A} -- (0, -2.5);
		\draw (7.5, 0) node[rectangle,draw,rounded corners,anchor=south,minimum height=1cm] {Company B} -- (7.5, -2.5);

		\draw [->,>=triangle 90] (0, -1) node[label=left:{3M}]{} -- (3.5, -1) node[pos=0.5,anchor=south]{$3.40\%$};

		\draw [snake=snake,->,>=triangle 90] (7.5, -1)  -- (4, -1) node[pos=0.5,anchor=south] {$3.00\%$};
		
		\draw [->,>=triangle 90] (0, -2.5) node[label=left:{6M}]{} -- (3.5, -2.5) node[pos=0.5,anchor=south]{$3.40\%$};

		\draw [snake=snake,->,>=triangle 90] (7.5, -2.5) -- (4, -2.5) node[pos=0.5,anchor=south] {$E_2$};
\end{tikzpicture}

\justify
We know 6 months swap rate ($3.40\%$). We also know the most recent EURIBOR fxing ($3.00\%$) which defines the very first floating payment. We can derive "implied"\ value of $E_2$.
\begin{align*}
3.40\% + 3.40\% &= 3.00\% + E_2 \Rightarrow \\
E_2 &= 3.80\%
\end{align*}

\end{frame}



\begin{frame}{Musings on "implied"\ values}
\justify
Can we hope that the implied value $E_2=3.80\%$ is correct? No, we cannot. Most probably, the prevailing EURIBOR in three months will be different. On average, interest rates tend to be lower than "implied"\ rates which one could derive from the market swap rates.

\justify
Implied interest rate is just a rate that corresponds to current market swap rates. These are two equivalent views on the same reality:

1) The latest fixing is $3.00\%$, 6 months swap rate is $3.40\%$.

2) The latest fixing is $3.00\%$, EURIBOR in 3 months is $3.80\%$.
\end{frame}



\begin{frame}{Musings on "implied"\ values - 2}
\justify
It does not matter whether the market is correct when it estimates the future value of $E_2$. It is important that the market offers liquid instruments which allow us hedge interest rate risk.

\justify
Suppose we will have to pay interest of $E_2$ on  a floating-rate loan. Then lets buy a 6 months swap.

\centering
\begin{tabular}{l|r|r|r}
    & \multicolumn{2}{c|}{Swap} & Loan \\ \hline
Today &          0  &  0            & 0 \\
3M      & $-3.40\%$  &  $+3.00\%$   & 0 \\
6M      & $-3.40\%$  &  $+E_2$       & $-E_2$ \\ \hline
PV       & \multicolumn{3}{c}{-3.80\%}
\end{tabular}

\justify
With a swap we can eliminate all the uncertainty and live in a world where the future EURIBOR $E_2$ is always 3.80\%.
\end{frame}



\begin{frame}{Computing expected EURIBOR values - 3}
\centering
\begin{tikzpicture}[thick, scale=0.7]
		\draw (0, 0) node[rectangle,draw,rounded corners,anchor=south,minimum height=1cm] {Company A} -- (0, -4);
		\draw (7.5, 0) node[rectangle,draw,rounded corners,anchor=south,minimum height=1cm] {Company B} -- (7.5, -4);

		\draw [->,>=triangle 90] (0, -1) node[label=left:{3M}]{} -- (3.5, -1) node[pos=0.5,anchor=south]{$3.60\%$};

		\draw [snake=snake,->,>=triangle 90] (7.5, -1)  -- (4, -1) node[pos=0.5,anchor=south] {$3.00\%$};
		
		\draw [->,>=triangle 90] (0, -2.5) node[label=left:{6M}]{} -- (3.5, -2.5) node[pos=0.5,anchor=south]{$3.60\%$};

		\draw [snake=snake,->,>=triangle 90] (7.5, -2.5) -- (4, -2.5) node[pos=0.5,anchor=south] {$3.80\%$};
		
		\draw [->,>=triangle 90] (0, -4) node[label=left:{9M}]{} -- (3.5, -4) node[pos=0.5,anchor=south]{$3.60\%$};

		\draw [snake=snake,->,>=triangle 90] (7.5, -4) -- (4, -4) node[pos=0.5,anchor=south] {$E_3$};
\end{tikzpicture}

\justify
We know 9 months swap rate (3.60\%) and the first two EURIBORs (3.00\% и 3.80\%). We can compute implied value of $E_3$.
\begin{align*}
3.60\% + 3.60\% + 3.60\% &= 3.0\% + 3.80\% + E_3 \Rightarrow \\
E_3 &= 4.00\%
\end{align*}
\end{frame}



\begin{frame}{Computing expected EURIBOR values - 4}
\centering
\begin{tikzpicture}[thick, scale=0.7]
		\draw (0, 0) node[rectangle,draw,rounded corners,anchor=south,minimum height=1cm] {Company A} -- (0, -5.5);
		\draw (7.5, 0) node[rectangle,draw,rounded corners,anchor=south,minimum height=1cm] {Company B} -- (7.5, -5.5);

		\draw [->,>=triangle 90] (0, -1) node[label=left:{3M}]{} -- (3.5, -1) node[pos=0.5,anchor=south]{$3.70\%$};

		\draw [snake=snake,->,>=triangle 90] (7.5, -1)  -- (4, -1) node[pos=0.5,anchor=south] {$3.00\%$};
		
		\draw [->,>=triangle 90] (0, -2.5) node[label=left:{6M}]{} -- (3.5, -2.5) node[pos=0.5,anchor=south]{$3.70\%$};

		\draw [snake=snake,->,>=triangle 90] (7.5, -2.5) -- (4, -2.5) node[pos=0.5,anchor=south] {$3.80\%$};
		
		\draw [->,>=triangle 90] (0, -4) node[label=left:{9M}]{} -- (3.5, -4) node[pos=0.5,anchor=south]{$3.70\%$};

		\draw [snake=snake,->,>=triangle 90] (7.5, -4) -- (4, -4) node[pos=0.5,anchor=south] {$4.00\%$};
		
		\draw [->,>=triangle 90] (0, -5.5) node[label=left:{12M}]{} -- (3.5, -5.5) node[pos=0.5,anchor=south]{$3.70\%$};

		\draw [snake=snake,->,>=triangle 90] (7.5, -5.5) -- (4, -5.5) node[pos=0.5,anchor=south] {$E_4$};
\end{tikzpicture}

\justify
We know 12 months swap rate (3.70\%) and the first three EURIBORs (3.00\%, 3.80\% и 4.00\%). We can compute implied value of $E_4$.
\begin{align*}
3.70\%+3.70\%+3.70\%+3.70\% &= 3.00\% + 3.80\% + 4.00\% + E_4 \Rightarrow \\
E_4 &= 4.00\%
\end{align*}
\end{frame}



\begin{frame}{Computing expected EURIBOR values - 5}
\centering
\begin{tikzpicture}[thick, scale=0.7]
		\draw (0, 0) node[rectangle,draw,rounded corners,anchor=south,minimum height=1cm] {Company A} -- (0, -8.5);
		\draw (7.5, 0) node[rectangle,draw,rounded corners,anchor=south,minimum height=1cm] {Company B} -- (7.5, -8.5);

		\draw [->,>=triangle 90] (0, -1) node[label=left:{3M}]{} -- (3.5, -1) node[pos=0.5,anchor=south]{$3.75\%$};

		\draw [snake=snake,->,>=triangle 90] (7.5, -1)  -- (4, -1) node[pos=0.5,anchor=south] {$3.00\%$};
		
		\draw [->,>=triangle 90] (0, -2.5) node[label=left:{6M}]{} -- (3.5, -2.5) node[pos=0.5,anchor=south]{$3.75\%$};

		\draw [snake=snake,->,>=triangle 90] (7.5, -2.5) -- (4, -2.5) node[pos=0.5,anchor=south] {$3.80\%$};
		
		\draw [->,>=triangle 90] (0, -4) node[label=left:{9M}]{} -- (3.5, -4) node[pos=0.5,anchor=south]{$3.75\%$};

		\draw [snake=snake,->,>=triangle 90] (7.5, -4) -- (4, -4) node[pos=0.5,anchor=south] {$4.00\%$};
		
		\draw [->,>=triangle 90] (0, -5.5) node[label=left:{12M}]{} -- (3.5, -5.5) node[pos=0.5,anchor=south]{$3.75\%$};

		\draw [snake=snake,->,>=triangle 90] (7.5, -5.5) -- (4, -5.5) node[pos=0.5,anchor=south] {$4.00\%$};
		
				\draw [->,>=triangle 90] (0, -7) node[label=left:{15M}]{} -- (3.5, -7) node[pos=0.5,anchor=south]{$3.75\%$};

		\draw [snake=snake,->,>=triangle 90] (7.5, -7) -- (4, -7) node[pos=0.5,anchor=south] {$E_5$};
		
				\draw [->,>=triangle 90] (0, -8.5) node[label=left:{18M}]{} -- (3.5, -8.5) node[pos=0.5,anchor=south]{$3.75\%$};

		\draw [snake=snake,->,>=triangle 90] (7.5, -8.5) -- (4, -8.5) node[pos=0.5,anchor=south] {$E_6$};
\end{tikzpicture}
\end{frame}



\begin{frame}{Computing expected EURIBOR values - 6}
\justify
There is no observable quote for a 15 months swap, we only have 18 months swap rate. This leaves us with single equation and two unknown EURIBORs $E_5$ и $E_6$.
\begin{align*}
3.00\%+3.80\%+4.00\%+4.00\%+E_5+E_6 = 6\cdot3.75\%
\end{align*}

\justify
Suppose that there is linear dependence between interest rates. $E_5$ is higher than $E_4$ by the same amount as $E_6$ is higher than $E_5$.
\begin{align*}
\begin{cases}
3.00\%+3.80\%+4.00\%+4.00\%+E_5+E_6 = 6\cdot3.75\% \\
E_5 - 4.00\% = E_6 - E_5
\end{cases}
\end{align*}

Therefore, if we assume linear interpolation, we can compute
\begin{align*}
\begin{cases}
E_5 = 3.90\% \\
E_6 = 3.80\%
\end{cases}
\end{align*}
\end{frame}



\begin{frame}{Вычисление ожидаемых значений EURIBOR}
\centering
\begin{tikzpicture}[thick, scale=0.7]
		\draw (0, 0) node[rectangle,draw,rounded corners,anchor=south,minimum height=1cm] {Компания A} -- (0, -2.5);
		\draw (7.5, 0) node[rectangle,draw,rounded corners,anchor=south,minimum height=1cm] {Компания B} -- (7.5, -2.5);

		\draw [->,>=triangle 90] (0, -1) node[label=left:{3M}]{} -- (3.5, -1) node[pos=0.5,anchor=south]{$S_{6M}t_1\alert{\delta_1}$};

		\draw [snake=snake,->,>=triangle 90] (7.5, -1)  -- (4, -1) node[pos=0.5,anchor=south] {$E_1t_1\alert{\delta_1}$};
		
		\draw [->,>=triangle 90] (0, -2.5) node[label=left:{6M}]{} -- (3.5, -2.5) node[pos=0.5,anchor=south]{$S_{6M}t_2\alert{\delta_2}$};

		\draw [snake=snake,->,>=triangle 90] (7.5, -2.5) -- (4, -2.5) node[pos=0.5,anchor=south] {$\alert{E_2}t_2\alert{\delta_2}$};
\end{tikzpicture}

\justify
Вернёмся в реальность и добавим в модель коэффициенты дисконтирования на 3 и 6 месяцев $\delta_1$ и $\delta_2$. Обычно они не известны заранее, поэтому у нас получается одно уравнение с тремя неизвестными.
\begin{align*}
S_{6M}t_1\alert{\delta_1} + S_{6M}t_2\alert{\delta_2} = E_1t_1\delta_1 + \alert{E_2}t_2\alert{\delta_2}
\end{align*}

\justify
Как быть?
\end{frame}



\begin{frame}{Вычисление ожидаемых значений EURIBOR}
\justify
Предположим, что ставки трёхмесячного EURIBOR являются безрисковыми. Это разумное предположение, потому что о проблемах у банка обычно становится известно заранее.

\justify
Мы можем построить цепочку из безрисковых депозитов под EURIBOR:

1) Выбрать надёжный банк и вложить деньги на три месяца под EURIBOR.

2) Подождать три месяца.

3) Снять деньги и вернуться к пункту 1.

\justify
\centering
\begin{tikzpicture}
		\draw [->,>=triangle 90] (0.5, 0) -- (11, 0);

		\draw [->,>=triangle 45] (0.5,0) node[anchor=north east]{0} .. controls (1, 0.5) and (2, 0.5) .. (2.5,0) node[anchor=north]{$T_1$} node[pos=0.5,anchor=south]{$E_1$};

		\draw [->,>=triangle 45] (2.5,0) .. controls (3, 0.5) and (4, 0.5) .. (4.5,0) node[anchor=north]{$T_2$} node[pos=0.5,anchor=south]{$E_2$};

		\draw [->,>=triangle 45] (4.5,0) .. controls (5, 0.5) and (6, 0.5) .. (6.5,0) node[anchor=north]{$T_3$} node[pos=0.5,anchor=south]{$E_3$};
		
		\draw [dashed] (6.5,0) .. controls (7, 0.5) and (8, 0.5) .. (8.5,0);
		
		\draw [->,>=triangle 45] (8.5,0) node[anchor=north]{$T_{n-1}$} .. controls (9, 0.5) and (10, 0.5) .. (10.5,0) node[anchor=north]{$T_n$} node[pos=0.5,anchor=south]{$E_n$};
	\end{tikzpicture}
%\justify
%Каждый раз на шаге 1 мы можем (но не обязаны) выбирать новый банк, если кредитный рейтинг предыдущего банка нам больше не нравится.
\end{frame}



\begin{frame}{Вычисление ожидаемых значений EURIBOR}
\centering
\begin{tikzpicture}
		\draw [->,>=triangle 90] (0.5, 0) -- (11, 0);

		\draw [->,>=triangle 45] (0.5,0) node[anchor=north]{$T_0$} .. controls (1, 0.5) and (2, 0.5) .. (2.5,0) node[anchor=north]{$T_1$} node[pos=0.5,anchor=south]{$E_1$};

		\draw [->,>=triangle 45] (2.5,0) .. controls (3, 0.5) and (4, 0.5) .. (4.5,0) node[anchor=north]{$T_2$} node[pos=0.5,anchor=south]{$E_2$};

		\draw [->,>=triangle 45] (4.5,0) .. controls (5, 0.5) and (6, 0.5) .. (6.5,0) node[anchor=north]{$T_3$} node[pos=0.5,anchor=south]{$E_3$};
		
		\draw [dashed] (6.5,0) .. controls (7, 0.5) and (8, 0.5) .. (8.5,0);
		
		\draw [->,>=triangle 45] (8.5,0) node[anchor=north]{$T_{n-1}$} .. controls (9, 0.5) and (10, 0.5) .. (10.5,0) node[anchor=north]{$T_n$} node[pos=0.5,anchor=south]{$E_n$};
		
		\node [anchor=north] at (1.5, 0) {$t_1$};
		\node [anchor=north] at (3.5, 0) {$t_2$};
		\node [anchor=north] at (5.5, 0) {$t_3$};
		\node [anchor=north] at (9.5, 0) {$t_n$};
	\end{tikzpicture}
	
\justify
Если мы вложили $N_0$ евро в цепочку депозитов, то сколько евро будет на счету к моменту времени $T_n$?
\begin{align*}
N_n = N_0(1 + E_1t_1)(1 + E_2t_2) ... (1 + E_nt_n)
\end{align*}

\justify
Чему равен коэффициент дисконтирования на дату окончания цепочки $T_n$?
\begin{align*}
\delta_n = \frac{1}{(1 + E_1t_1)(1 + E_2t_2) ... (1 + E_nt_n)}
\end{align*}
\end{frame}



\begin{frame}{Вычисление ожидаемых значений EURIBOR}
\centering
\begin{tikzpicture}[thick, scale=0.7]
		\draw (0, 0) node[rectangle,draw,rounded corners,anchor=south,minimum height=1cm] {Компания A} -- (0, -2.5);
		\draw (7.5, 0) node[rectangle,draw,rounded corners,anchor=south,minimum height=1cm] {Компания B} -- (7.5, -2.5);

		\draw [->,>=triangle 90] (0, -1) node[label=left:{3M}]{} -- (3.5, -1) node[pos=0.5,anchor=south]{$S_{6M}t_1\alert{\delta_1}$};

		\draw [snake=snake,->,>=triangle 90] (7.5, -1)  -- (4, -1) node[pos=0.5,anchor=south] {$E_1t_1\alert{\delta_1}$};
		
		\draw [->,>=triangle 90] (0, -2.5) node[label=left:{6M}]{} -- (3.5, -2.5) node[pos=0.5,anchor=south]{$S_{6M}t_2\alert{\delta_2}$};

		\draw [snake=snake,->,>=triangle 90] (7.5, -2.5) -- (4, -2.5) node[pos=0.5,anchor=south] {$\alert{E_2}t_2\alert{\delta_2}$};
\end{tikzpicture}

\justify
Мы только что связали будущие значения EURIBOR $E_i$ с коэффициентами дисконтирования $\delta_i$. Теперь у нас столько уже уравнений, сколько и неизвестных.
\begin{align*}
\begin{cases}
S_{6M}t_1\alert{\delta_1} + S_{6M}t_2\alert{\delta_2} = E_1t_1\delta_1 + \alert{E_2}t_2\alert{\delta_2} \\
\alert{\delta_1} = \dfrac{1}{1+E_1t_1} \\
\alert{\delta_2} = \dfrac{1}{(1+E_1t_1)(1+\alert{E_2}t_2)}
\end{cases}
\Rightarrow E_2 \approx 3.804\%
\end{align*}
\end{frame}



\begin{frame}{Метод bootstrap}
\justify
Если рассматривать EURIBOR как безрисковую ставку, то удобно представлять кривую будущих значений как совокупность коэффициентов дисконтирования $\delta_i$. Будущее значение EURIBOR для депозита, который начнётся через $T_1$ лет и закончится через $T_2 = T_1 + 0.25$ лет равно
\begin{align*}
E(T_1, T_2) = \frac{\dfrac{\delta_{T_1}}{\delta_{T_2}} - 1}{T_2 - T_1}
\end{align*}

\justify
Алгоритм пошагового расчёта коэффициентов дисконтирования и будущих ставок из котировок рыночных инструментов называется \alert{\en{bootstrap}}. Он реализован во многих финансовых библиотеках, таких как QuantLib и tf-quant-finance. 
\end{frame}



\begin{frame}{Пример: bootstrap}
\justify
Рассмотрим котировки свопов на трёхмесячный EURIBOR: 

\justify
\centering
\begin{tabular}{l|r|l|r}
Срок & Купон & Срок & Купон \\ \hline
3m   & 3.00\% & 4y   & 3.25\% \\
6m   & 3.40\% & 5y   & 3.15\% \\
9m   & 3.60\% & 6y   & 3.10\% \\
1y   & 3.70\% & 7y   & 3.05\% \\
18m  & 3.75\% & 8y   & 3.00\% \\
2y   & 3.60\% & 9y   & 3.00\% \\
3y   & 3.40\% & 10y  & 3.00\% 
\end{tabular}
\end{frame}



\begin{frame}{Пример: интерполяция}
\begin{tikzpicture}
\begin{axis}[
  width=\textwidth,
  height=\textheight - 1cm,
  date coordinates in=x,
  date ZERO=2012-01-01,
  xtick={2023-01-01, 2024-01-01, 2025-01-01, 2026-01-01, 2027-01-01, 2028-01-01, 2029-01-01, 2030-01-01},
  %minor xtick={2013-01-01, 2015-01-01, 2017-01-01, 2019-01-01, 2021-01-01},
  %ytick={-0.5, 0, 0.5, 1.0, 1.5},
  %minor ytick={-0.75, -0.25, 0.25, 0.75, 1.25},
  xticklabel={\year},
  xmin=2023-03-13,
  xmax=2030-01-01,
  %ymin=-0.75,
  %ymax=1.5,
  grid=major,
  ylabel={\small{EURIBOR, б.п.}},
  xlabel near ticks,
  ylabel near ticks,
  legend entries = {	
  	   Линейная,
  	   Сплайн
  },
  legend cell align={left},
  legend pos = north east
]

	\addplot[
		color = Set1-A,
		mark = none,
		very thick
	]
	table[
		x=date,
		y=linear,
		col sep=comma
	]
	{euribor_bootstrap.csv};
	
	\addplot[
		color = Set1-B,
		mark = none,
		very thick
	]
	table[
		x=date,
		y=spline,
		col sep=comma
	]
	{euribor_bootstrap.csv};
	
	\draw[thick, color=black] (axis cs: 2012-01-01, 0) -- (axis cs: 2030-01-01, 0);
\end{axis}
\end{tikzpicture}
\end{frame}



\begin{frame}{Пример: нестандартный своп}
\justify
Рассмотрим нестандартный своп, который начнётся через 2.5 года и закончится чрез 5 лет 
(30M--5Y).

\justify
Поскольку мы знаем будущие значения EURIBOR и коэффициентов дисконтирования, мы можем 
подобрать такой купон, что PV свопа было равно 0.

\justify
\centering
\begin{tabular}{l|l|l}
& Receive & Pay \\ \hline
Effective date & 15.09.2025 & 15.09.2025 \\ 
Maturity date  & 15.03.2028 & 15.03.2028 \\
Notional       & \EUR{1\,000\,000} & \EUR{1\,000\,000} \\
Frquency       & Semi-annually     & Quarterly \\
Rate           & \alert{2.7646\%}          & EURIBOR-3M   \\
Basis          & ACT/360           & ACT/360
\end{tabular}
\end{frame}



\begin{frame}{Пример: риск}
\justify
PV01 (present value of 1 bp) или DV01 (dollar value of 1 bp): на сколько изменится
PV, если процентная ставка изменится на 1 базисный пункт.

\justify
\centering
\begin{tabular}{l|r|r}
Срок  & PV01 & Размер хэджа \\ \hline
18m   & $-23$  евро/б.п. & $+\EUR{158\,000}$ \\
2y    & $+129$ евро/б.п. & $-\EUR{668\,000}$ \\
3y    & $+155$ евро/б.п. & $-\EUR{545\,000}$ \\
4y    & $-23$  евро/б.п. & $+\EUR{62\,000}$ \\
5y    & $-455$ евро/б.п. & $+\EUR{989\,000}$ \\ \hline
Итого & $-217$ евро/б.п. & $-\EUR{4\,000}$
\end{tabular}

\justify
Правило большого пальца для PV01: размер * срок / 10\,000.
\end{frame}



\begin{frame}{Кругом обман}
\centering
\makebox[\textwidth]{\includegraphics[width=\textwidth]{we_need_to_go_deeper.jpg}}
\end{frame}



\begin{frame}{Дисконтирование по EURIBOR}
\justify
Мы использовали трёхмесячный EURIBOR для дисконтирования будущих платежей. Но есть же, например, шестимесячный EURIBOR и свопы на шестимесячный EURIBOR. По аналогии, чтобы забутсрапить кривую шестимесячного EURIBOR мы должны были бы дисконтировать платежи по этой ставке (притвориться, что строим цепочку из шестимесячных депозитов).

\justify
Ещё хуже: есть базисные свопы (\en{tenor basis swap}), в которых одна сторона платит шестимесячный EURIBOR, а другая --- трёхмесячный с доплатой (базисом). По какой ставке  дисконтировать платежи в этих свопах? Хотелось бы, чтобы коэффициент дисконтирования на дату $T$ не зависел от того, в какой задаче мы его используем.

\justify
Кроме того, в определении EURIBOR участвуют банки с кредитным рейтингом BBB. Действительно ли это безрисковая ставка?
\end{frame}



\begin{frame}{Базисный процентный своп}
\justify
\begin{tikzpicture}[thick, scale=0.75]
		\draw (0, 0) node[rectangle,draw,rounded corners,anchor=south,minimum height=1cm] {Компания A} -- (0, -7);
		\draw (10, 0) node[rectangle,draw,rounded corners,anchor=south,minimum height=1cm] {Компания B} -- (10, -7);

		\draw [dashed,->,>=triangle 90] (0, -1) node[label=left:{$T_0$}]{} -- (4, -1) node[pos=0.5,anchor=south]{$\EUR{1\,000\,000}$};

		\draw [dashed,->,>=triangle 90] (10, -1) node[label=right:{\setlength\tabcolsep{1pt}\begin{tabular}{l}фиксинг\\$E_1^{3M}$ и $E_1^{6M}$\end{tabular}}]{} -- (4.5, -1) node[pos=0.5,anchor=south]{\euro 1\,000\,000};

		\node [label=left:{$3M$}] at (0, -2.5) {};

		\draw [snake=snake,->,>=triangle 90] (10, -2.5)  node[label=right:{\text{фиксинг $E_2^{3M}$}}]{} -- (4.5, -2.5) node[pos=0.5,anchor=south] {$\EUR{1\,000\,000} \cdot \dfrac{E_1^{3M} + 0.1\%}{4}$};

		\draw [snake=snake,->,>=triangle 90] (0, -4) node[label=left:{$6M$}]{} -- (4, -4) node[pos=0.5,anchor=south]{$\EUR{1\,000\,000} \cdot \dfrac{E_1^{6M}}{2}$};

		\draw [snake=snake,->,>=triangle 90] (10, -4)  node[label=right:{\setlength\tabcolsep{1pt}\begin{tabular}{l}фиксинг\\$E_3^{3M}$ и $E_2^{6M}$\end{tabular}}]{} -- (4.5, -4) node[pos=0.5,anchor=south] {$\EUR{1\,000\,000} \cdot \dfrac{E_2^{3M} + 0.1\%}{4}$};

		\node [label=left:{$9M$}] at (0, -5.5) {};
		
		\draw [snake=snake,->,>=triangle 90] (10, -5.5)  node[label=right:{\text{фиксинг $E_4^{3M}$}}]{} -- (4.5, -5.5) node[pos=0.5,anchor=south] {$\EUR{1\,000\,000} \cdot \dfrac{E_3^{3M} + 0.1\%}{4}$};
		\draw [snake=snake,->,>=triangle 90] (0, -7) node[label=left:{$12M$}]{} -- (4, -7) node[pos=0.5,anchor=south]{$\EUR{1\,000\,000} \cdot \dfrac{E_2^{6M}}{2}$};

		\draw [snake=snake,->,>=triangle 90] (10, -7) -- (4.5, -7) node[pos=0.5,anchor=south] {$\EUR{1\,000\,000} \cdot \dfrac{E_4^{3M} + 0.1\%}{4}$};
\end{tikzpicture}

\justify
$0.1\%$ --- базис или цена свопа.
\end{frame}



\begin{frame}{Ставки овернайт}
\justify
Если даже депозит в банке по EURIBOR на 3 или 6 месяцев --- слишком рискованная инвестиция, как можно уменьшить риск? Можно размещать деньги не на квартал, а на день (точнее, на ночь). Придётся каждое утро следить за новостями и выбирать наиболее надёжный банк на следующие сутки, зато получится очень надёжно.

\justify
Под какую ставку мы сможем вкладывать деньги на день? Скорее всего --- близко к ESTER, средней ставке по депозитам овернайт в крупных банках.

\justify
Осталось только разобраться с процентным риском. Начиная цепочку депозитов сейчас, при ESTER $2.40\%$, мы не знаем, по какой ставке будем открывать вклады через год. Вдруг ставки упадут ещё ниже?
\end{frame}



\begin{frame}{Overnight index swap}
\justify
Своп на индекс овернайт (\en{overnight index swap, OIS}) --- контракт, в котором одна сторона платит фиксированную ставку, а другая --- <<среднюю>> ставку овернайт (например, ESTER) за период.

\justify
\centering
\begin{tikzpicture}[thick, scale=0.75]
		\draw (0, 0) node[rectangle,draw,rounded corners,anchor=south,minimum height=1cm] {Компания A} -- (0, -2.5);
		\draw (10.5, 0) node[rectangle,draw,rounded corners,anchor=south,minimum height=1cm] {Компания B} -- (10.5, -2.5);

		\draw [->,>=triangle 90] (0, -1) node[label=left:{1Y}]{} -- (5, -1) node[pos=0.5,anchor=south]{$\EUR{1\,000\,000} \cdot (2.55\%)$};

		\draw [snake=snake,->,>=triangle 90] (10.5, -1) -- (5.5, -1) node[pos=0.5,anchor=south] {$\EUR{1\,000\,000} \cdot \overline{E_1}$};


		\draw [->,>=triangle 90] (0, -2.5) node[label=left:{2Y}]{} -- (5, -2.5) node[pos=0.5,anchor=south]{$\EUR{1\,000\,000} \cdot (2.55\%)$};

		\draw [snake=snake,->,>=triangle 90] (10.5, -2.5) -- (5.5, -2.5) node[pos=0.5,anchor=south] {$\EUR{1\,000\,000} \cdot \overline{E_2}$};
\end{tikzpicture}

\justify
$\overline{E_1}$ и $\overline{E_2}$ --- средние ставки ESTER ($e_i$) за первый и второй год:
\begin{align*}
\overline{E_1} &= \left(1 + \dfrac{e_1}{360}\right)
\left(1 + \dfrac{e_2}{360}\right)
...
\left(1 + \dfrac{e_{365}}{360}\right) - 1 \\
\overline{E_2} &= \left(1 + \dfrac{e_{366}}{360}\right)
\left(1 + \dfrac{e_{367}}{360}\right)...
\left(1 + \dfrac{e_{730}}{360}\right) - 1
\end{align*}

\justify
Фиксированная ставка $2.55\%$ --- цена свопа.
\end{frame}



\begin{frame}{Дисконтирование OIS}
\justify
Цепочка из однодневных депозитов под ESTER, в которой процентный риск захеджирован OIS свопом --- лучшее приближение теоретической безрисковой процентной ставки. Текущий рыночный консенсус --- использовать дисконтирование по OIS для оценки деривативов.

\justify
Дело в шляпе! У нас есть котировки OIS свопов, из которых методом bootstrap можно вычислить и будущие значения ESTER, и связанные с ними коэффициенты дисконтирования. Потом эти коэффициенты дисконтирования можно использовать для вычисления значений EURIBOR из свопов на EURIBOR.
\end{frame}



\begin{frame}{Дисконтирование OIS}
\justify
К сожалению, рынок OIS ликвиден только для свопов на сроки до одного-двух лет. Дальше участники предпочитают свопы EURIBOR-OIS (одна сторона платит трёхмесячный EURIBOR, другая --- среднюю ESTER плюс базис). Нужно одновременно вычислять и значения EURIBOR, и значения ESTER, и коэффициенты дисконтирования.

\justify
<<До кучи>> можно добавить в модель и шестимесячный EURIBOR. Его тоже обычно восстанавливают не из простых свопов, а из базисных свопов 3M-6M.

\justify
Получается одна огромная задача оптимизации, которая одновременно подбирает будущие значения ESTER, трёхмесячного EURIBOR и шестимесячного EURIBOR.
\end{frame}



%\begin{frame}{Демонстрация}
%\begin{itemize}
%\item Калибрация кривых ESTER и EURIBOR.
%\item Вычисление цены свопа на шестимесячный EURIBOR.
%\end{itemize}
%\end{frame}



\begin{frame}{Кругом обман}
\centering
\makebox[\textwidth]{\includegraphics[width=\textwidth]{we_need_to_go_deeper.jpg}}
\end{frame}



\begin{frame}{Если всего этого мало}
\justify
Коэффициенты дисконтирования, которые нужно применять к платежам по сделке, зависят от соглашения о гарантийном обеспечении (\en{credit support annex, CSA}). Начисление процентов на гарантийное обеспечение по ставке овернайт --- стандартная рыночная практика. Поэтому дисконтирование по OIS работает довольно часто, не не всегда.

\justify
Кроме того, кривые OIS в двух валютах не воспроизводят цены форвардов, потому что не являются идеальными безрисковыми ставками. Существует валютный базис (\en{cross-currency basis}), который тоже нужно учитывать при оценке деривативов.

\justify
Подробности в \en{Marc Henrard, Interest Rate Modelling in the Multi-Curve Framework (2014)}.
\end{frame}

\end{document}