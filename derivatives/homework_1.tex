\documentclass[a4paper,14pt]{extarticle}
\usepackage{cmap}				% To be able to copy-paste russian text from pdf			
\usepackage[utf8]{inputenc}
\usepackage[T2A]{fontenc}
\usepackage[margin=1in]{geometry}
\usepackage[english, russian]{babel}

\usepackage{amsmath}
\usepackage{amsfonts}

\usepackage[hyphens]{url}
\urlstyle{same}
\usepackage{hyperref}

\newcommand{\ru}[1]{\begin{otherlanguage}{russian}#1\end{otherlanguage}}
\newcommand{\en}[1]{\begin{otherlanguage}{english}#1\end{otherlanguage}}

\begin{document}

\section*{Фьючерсы на ставку ФРС}

Напишите на любом языке программирования код, который вычисляет вероятные будущие значения ставки ФРС исходя из котировок 30-дневных фьючерсов на ставку фед. резерва (\en{30-days Fed Fund Futures}) на Чикагской товарной бирже (\en{Chicago Mercantile Exchange, CME}).

На сайте биржи есть инструмент, который делает именно это вычисление:
\href{https://www.cmegroup.com/trading/interest-rates/countdown-to-fomc.html}{FedWatch 
Tool}. Методология также описана на сайте биржи: \href{https://www.cmegroup.com/education/
demos-and-tutorials/fed-funds-futures-probability-tree-calculator.html}{CME Fed Watch Tool 
Methodology}. В материалах на my.NES вы найдёте файл Excel, который повторяет эти вычисления.

\subsection*{Ставка ФРС и фьючерсы}

Каждый рабочий день ФРС вычисляет эффективную ставку фед. резерва 
(\en{effective federal funds rate, EFFR}). Для этого она собирает с банков информацию о 
сделках по однодневным беззалоговым кредитам, которые банки дают друг другу. 
Средневзвешенная по размерам ставка кредитов и будет EFFR. 
Например, если ФРС объявила, что 14 марта 2022 года EFFR была
8 базисных пунктов, то это означает, что в среднем по палате банки могли привлекать 
однодневные кредиты под 0.08\% годовых.

Когда вы слышите в новостях <<ФРС повышает ставки>>, то речь идёт именно об EFFR. Если 
точнее, то ФРС устанавливает не одну ставку, а целевой диапазон (target federal funds 
rate). Например, на момент написания диапазон равен 0--25 базисных пунктов. Если ФРС 
примет решение повысить ставку, то она установит целевой диапазон на уровне 25--50 
базисных пунктов.

Фьючерсы на ставку ФРС на CME имеют вид ZQH3, где ZQ --- это 
код типа контракта (30-дневные фьючерсы на ставку ФРС), H3 --- обозначение календарного 
месяца. Буква H обозначает март. Цифра 3 обозначает 2023 год. Все коды месяцев 
перечислены в таблице \ref{month_table}. Не спрашивайте меня, какие инопланетяне их 
придумали.

\begin{table}[h]
\centering
\begin{tabular}{l|l|l|l}
Код & Месяц   & Код & Месяц \\ \hline
F   & Январь  & N   & Июль \\
G   & Февраль & Q   & Август \\
H   & Март    & U   & Сентябрь \\
J   & Апрель  & V   & Октябрь \\
K   & Май     & X   & Ноябрь \\
M   & Июнь    & Z   & Декабрь \\
\end{tabular}
\caption{Обозначения месяцев в контрактах CME.}
\label{month_table}
\end{table}

Что такое фьючерс ZQM2? Это фьючерс на среднюю EFFR, которая будет зафиксирована в июне 
2022 года. Цена фьючерса --- это величина $100 - \overline{r}$, где $\overline{r}$ --- 
средняя EFFR за все 30 дней июня. Например, если $r_1,...,r_{30}$ --- 30 значений EFFR в 
процентных пунктах, которые будут зафиксированы с 1 июня по 30 июня 2022 года, то биржа 
установит финальную цену июньского фьючерса равной
\begin{align*}
F = 100 - \dfrac{r_1+...+r_{30}}{30}
\end{align*}

По-простому, если сегодня июньский фьючерс торгуется на уровне 
99.05, то участники рынка ожидают, что средняя ставка ФРС за июнь составит 0.95\% или 95 
базисных пунктов.

\subsection*{Фьючерсы и вероятность}

Чтобы вычислить вероятность повышения ставок из котировок фьючерсов, нужно сделать 
несколько предположений.

Во-первых, ФРС обычно не меняет ставку спонтанно. Чтобы повысить ставку, нужно собрать 
заседание (\en{meeting}) комитета по операциям на открытом рынке (\en{Federal Open Market 
Committee, FOMC}). Даты заседаний комитета обычно известны заранее и опубликованы на 
сайте ФРС. Как правило, FOMC заседает 8 раз в год, один раз в 6--8 недель.

Во-вторых, ФРС обычно меняет ставки с шагом 0.25\% годовых или 25 базисных пунктов. То 
есть мы можем увидеть повышение целевого диапазона с 0--25 до 25--50 или 50--75, но вряд
ли ФРС решит установить диапазон 37--62.

Рассмотрим некоторый календарный месяц, в котором запланирован митинг FOMC, например, май 2022 года. Митингов всего 8 за год, поэтому наверняка митинга не будет либо в следующем 
месяце (июне), либо в предыдущем (апреле). Так и есть: ближайшие митинги --- 16 марта, 
4 мая и 15 июня. То есть перед майским митингом будет целый месяц (апрель) без митингов. 

Допустим, прямо сейчас участники рынка продают и покупают апрельские фьючерсы по 99.6375 , то есть ожидают среднюю EFFR в апреле 36.25 б.п. Майские фьючерсы стоят
99.2950, то есть средняя EFFR, по мнению рынка, будет 70.50 б.п. Но митингов в апреле 
нет, поэтому можно считать, что весь апрель ставка будет одинаковой --- 36.25 б.п. И 
первые три дня мая (с 1 по 3 число) ставка будет такой же. Только начиная с 4 мая ставка 
может измениться.

Внимание, школьная арифметическая задача. Если первые 3 дня мая ставка равна 36.25, 
то какая ставка должна действовать 28 дней, с 4 по 31 число, чтобы средняя ставка за май
оказалась равной 70.50? Очевидно, $(31\cdot70.50 - 3\cdot36.25) / 28 = 74.17$. Таким образом, 
рынок прогнозирует рост ставки на 37.92 б.п. на заседании ФРС 4 мая.

Случай, когда в предыдущем месяце тоже был митинг, но его не будет в следующем, 
симметричен. Попробуйте разобрать его сами. Если что, он описан в документе CME с 
методологией как случай Type 1.

Допустим, мы вычислили, что рынок ожидает рост ставки на $x$ б.п. Что мы можем сказать о 
вероятности этого решения? Если ФРС может либо оставить ставку без изменения, либо 
повысить на 25 б.п., то какой должна быть вероятность повышения $p$, чтобы матожидание изменения
ставки было равно $x$? Наверное, $x / 25$. Например, для майского митинга мы можем 
вычислить вероятность повышения $p = 37.92 / 25 \approx 151\%$.

Хьюстон, у нас проблема: вероятность больше 100\%. CME предлагает трактовать это как 
возможность повышения сразу на два шага, то есть на 50 б.п. Тогда вероятность повышения 
на 25 б.п. равна 49\%, а на 50 б.п. --- 51\%.

Бывает, что рынок может закладываться на риск понижения ставок (если следующий фьючерс 
ниже предыдущего). Тогда мы будем вычислять вероятность понижения на 25 б.п. Впрочем,
прямо сейчас кривая цен фьючерсов монотонно растёт, поэтому можно об этом не думать.

По такому алгоритму мы можем вычислить вероятности повышения ставки на каждом из 
предстоящих митингов. Эти вероятности не зависят друг от друга, потому что для каждого митинга 
мы смотрим только на фьючерс на месяц митинга и либо на следующий, либо на предыдущий.

\section*{Наиболее вероятные ставки}

С помощью вероятностей изменения ставки на каждом митинге мы можем вычислить наиболее 
вероятный целевой диапазон для каждого момента в будущем. Для этого нам нужен рекурсивный 
алгоритм.

Текущий диапазон мы знаем (на момент написания --- 0--25 б.п.), он имеет вероятность 100\%.

Для каждого следующего митинга и целевого диапазона мы можем посмотреть, какими 
способами мы можем его достигнуть. Например, мы рассматриваем только возможности 
повышения ставки на 25 и 50 и неизменности ставки. Тогда:
\begin{align*}
\mathbb{P}(\text{75--100 сейчас}) &=
\mathbb{P}(\text{75--100 на прошлом митинге}) \cdot \mathbb{P}(\text{+0 сейчас}) + \\
&+ \mathbb{P}(\text{50--75 на прошлом митинге}) \cdot \mathbb{P}(\text{+25 сейчас}) + \\
&+ \mathbb{P}(\text{25--50 на прошлом митинге}) \cdot \mathbb{P}(\text{+50 сейчас})
\end{align*}

В результате у вас получится табличка с вероятными целевыми диапазонами для каждого
будущего митинга.

\begin{table}[h]
\centering
\begin{tabular}{l|r|r|r|r|r|r|r}
Митинг     &  0--25 & 25--50 & 50--75 & 75--100 & 100--125 & 125--150 & ... \\ \hline
16.03.2022 &	 0.0\%	 & \textbf{96.4\%} &  3.6\% &   0.0\% &	   0.0\% &    0.0\% & ... \\ 
04.05.2022 &	 0.0\%	 &  0.0\% & 46.6\% &  \textbf{51.6\%} &	   1.9\% &    0.0\% & ... \\
15.06.2022 &	 0.0\%	 &  0.0\% &  0.0\% &  23.3\%	 &   \textbf{49.1\%} &   26.7\%  & ... \\
27.07.2022 &	 0.0\%	 &  0.0\% &  0.0\% &   2.2\%	 &   25.7\% &   \textbf{47.0\%}  & ...\\
21.09.2022 &	 0.0\%	 &  0.0\% &  0.0\% &   0.5\%	 &    7.4\% &   30.4\% & ... \\
...        & ...    & ...    &  ...   &   ...   &    ...   &    ...   & ... 
\end{tabular}
\end{table}

Поздравляю! Теперь в ваших руках инструмент прогнозирования будущей политики ФРС, 
который лишь немного уступает по точности экспертному мнению Олега Шибанова.

\end{document}