\documentclass{beamer}

\usepackage{cmap}				% To be able to copy-paste russian text from pdf
\usepackage[T2A]{fontenc}
\usepackage[utf8]{inputenc}
\usepackage[russian]{babel}
\usepackage{textpos}
\usepackage{ragged2e}
\usepackage{amssymb}
\usepackage{ulem}
\usepackage{tikz}
\usepackage{pgfplots}
\usepackage{color}
\usepackage{cancel}
\usepackage{multirow}
\pgfplotsset{compat=1.17}
\usetikzlibrary{arrows,snakes,backgrounds,shapes}
\usepgfplotslibrary{groupplots,colorbrewer,dateplot,statistics}
\usepackage{animate}

\usepackage{amsfonts}
\usepackage{amsmath}
\usepackage{amssymb}
\usepackage{graphicx}
\usepackage{setspace}

\usepackage{enumitem}
\setitemize{label=\usebeamerfont*{itemize item}%
  \usebeamercolor[fg]{itemize item}
  \usebeamertemplate{itemize item}}

% remove navigation bar
\setbeamertemplate{navigation symbols}{} 

\usepackage{eurosym}
\renewcommand{\EUR}[1]{\textup{\euro}#1}

\title{Текущая стоимость, риск и греки}
\author{Артём Бакулин}
\date{1 апреля 2024 г.}

\usetheme{Warsaw}
\usecolortheme{beaver}

\newcommand{\ru}[1]{\begin{otherlanguage}{russian}#1\end{otherlanguage}}
\newcommand{\en}[1]{\begin{otherlanguage}{english}#1\end{otherlanguage}}
\newcommand{\ruen}[2]{#1 (\en{#2})}

\setbeamertemplate{page number in head/foot}[totalframenumber] 

\begin{document}



\begin{frame}
\titlepage
\end{frame}



\begin{frame}{Текущая стоимость}
\justify
\alert{Текущая стоимость} (\en{present value, PV}) выплаты в $N$ рублей через $T$ лет 
--- сумма, которую участники рынка готовы заплатить сегодня за право получить эту 
выплату.

\justify
Текущая стоимость дериватива или портфеля деривативов --- сумма, которую участники рынка были бы гипотетически готовы заплатить за эти сделки или портфель. Сколько нам заплатят, если мы вычеркнем из всех контрактов своё имя и впишем имя покупателя?

\end{frame}



\begin{frame}{Деривативы и МСФО}
\justify
Из отчётности одного крупного немецкого инвестиционного банка из Франкфурта:

\justify
Assets (billions):

\en{Positive market values from derivative financial instruments}: \EUR{251.9}

\justify
Liabilities (billions):

\en{Negative market values from derivative financial instruments}: \EUR{238.3}

\justify
Согласно МСФО, изменение PV портфеля деривативов должно быть учтено в прибыли (или убытках) текущего года, даже если выплаты по деривативам случатся через 100 лет.
\end{frame}



\begin{frame}{Текущая стоимость спот-сделки}
\justify
Мы купили на споте $N=1\,000\,000$ EURRUB по курсу $S_0 = 100$. Текущий рыночный курс $S=101$. Каково наше PV?

\justify
\centering
\begin{tabular}{c|c|c}
Дата & EUR  & RUB \\ \hline
Спот & $+N$ & $-NS_0$
\end{tabular}

\justify
$N$ евро стоят $N$ евро (пренебрегая дисконтированием на один-два дня).

\justify
$NS_0$ рублей стоят $NS_0/S$ евро.

\begin{align*}
\text{PV} = N - N\frac{S_0}{S} = N\frac{S - S_0}{S} = \EUR{1\,000\,000} \cdot \frac{101 - 100}{101} = \EUR{9\,901}
\end{align*}

\justify
Вычисление PV из рыночных цен (в данном случае, спот-курса) называется \alert{приведением к рынку} (\en{mark-to-market}).
\end{frame}



\begin{frame}{Текущая стоимость спот-сделки - 2}
\justify
Мы купили на споте $N=1\,000\,000$ EURRUB по курсу $S_0 = 100$. Текущий рыночный курс $S=101$. Каково наше PV?

\justify
\centering
\begin{tabular}{r|r|r}
         & EUR               & RUB \\ \hline
Сделка   & $+1\,000\,000$ & $-100\,000\,000$ \\
Закрытие &    $-990\,099$ & $+100\,000\,000$ \\ \hline
Итого    &      $+9\,901$ & $0$
\end{tabular}

\justify
Если бы мы решили закрыть позицию на рынке, то мы смогли бы обнулить все платежи в рублях и остаться с гарантированным фиксированным платежом \EUR{9\,901}. Это и есть PV сделки.
\end{frame}



\begin{frame}{Текущая стоимость и рыночный спред}
\justify
На реальном рынке есть спред между покупкой и продажей. Допустим, что евро можно продать за 99 руб. и купить за 101. Две компании договорились друг с другом о сделке на 1 миллион евро по курсу 100.

\justify
Компания-покупатель:

\centering
\begin{tabular}{r|r|r}
                & EUR                   & RUB \\ \hline
Сделка по 100   & $+1\,000\,000$ & $-100\,000\,000$ \\
Закрытие по 99 & $-1\,010\,101$ & $+100\,000\,000$ \\ \hline
Итого (PV)      &      $-10\,101$ & $0$
\end{tabular}

\justify
Компания-продавец:

\centering
\begin{tabular}{r|r|r}
                 & EUR                   & RUB \\ \hline
Сделка по 100     & $-1\,000\,000$ & $+100\,000\,000$ \\
Закрытие по 101   &    $+990\,099$ & $-100\,000\,000$ \\ \hline
Итого (PV)       &      $-9\,901$ & $0$
\end{tabular} 
\end{frame}



\begin{frame}{Текущая стоимость и рыночный спред - 2}
\justify
Можно ли обязать всех участников рынка вычислять PV и прибыль с учётом спреда между покупкой и продажей? Нет, потому что тогда вновь заключённая сделка окажется убыточной для обеих сторон.

\justify
Обычно PV вычисляется либо из <<справедливой>> (\en{fair}) средней цены между ценами покупки и продажи, либо из цены последней сделки на рынке. 

\justify
Следствие: если сделка прошла по средней цене между покупкой и продажей, то её PV --- ноль.
\end{frame}



\begin{frame}{Текущая стоимость}
\justify
Мы продали клиенту колл-опцион за премию 100 евро. На рынке такие опционы стоят 90 евро.  
Мы знаем это, потому что либо это ликвидный опцион, либо такую цену нам сказала модель, 
откалиброванная к рыночным ценам других опционов.

\justify
PV нашей позиции +10 евро.
\begin{itemize}
\item Проданный опцион: $-90$ евро.
\item Полученная премия: 100 евро.
\end{itemize}

\justify
На идеальном ликвидном рынке остальные участники будут готовы заплатить за наш портфель 
10 евро. Если цена будет ниже, например $9.90$, то кто-то предложит $9.95$, рассчитывая 
заработать 0.05 евро.

\justify
Другая интерпретация: мы можем выйти на рынок и купить опцион за 90 евро. Тогда у нас 
останется 10 евро, а позиция в опционе исчезнет. 
\end{frame}



\begin{frame}{Риск}
\justify
\alert{Риск} --- возможность того, что что-то пойдёт не так, как мы планировали, и 	прибыль окажется не такой, как мы рассчитывали.

\justify
Нельзя заработать что-то большее, чем безрисковая процентная ставка, если ничем не рисковать. Финансовые компании осознанно берут на себя множество рисков:
\begin{itemize}
\item Рыночный
\item Кредитный
\item Ликвидности
\item Регуляторный
\item Технический
\item Юридический
\item Политический
\end{itemize}
\end{frame}



\begin{frame}{Рыночный риск}
\justify
\alert{Рыночный риск} (\en{market risk}) --- возможность того, что наше PV изменится как вверх, так и вниз, из-за изменения рыночных цен.

\justify
Как измерить рыночный риск? Стандартный подход --- посчитать чувствительности (\en{sensitivities}) PV к изменениям рыночных переменных. На сколько изменится PV портфеля, если цена фьючерса Brent вырастет на \$1?

\justify
Стандартные обозначения --- греки (greeks):
\begin{itemize}
\item Дельта ($\Delta$) --- чувствительность к цене базового актива.
\item Вега ($\mathcal{V}$) --- чувствительность к implied волатильности.
\item Ро ($\rho$) --- чувствительность к процентным ставкам.
\item Тета ($\Theta$) --- чувствительность к течению времени.
\item Гамма ($\Gamma$) --- чувствительность дельты к цене базового актива.
\end{itemize}
\end{frame}



\begin{frame}{Дельта}
\justify
\alert{Дельта} --- чувствительность текущей стоимости портфеля $V$ к изменениям цены базового актива (например, спот-курса валюты) $S$. 
\begin{align*}
\Delta = \frac{\partial V}{\partial S}
\end{align*}

\justify
При небольшом изменении цены базового актива $dS$ текущая стоимость портфеля меняется на
\begin{align*}
dV \approx \Delta\cdot dS
\end{align*}

\justify
Это следует из определения производной:
\begin{align*}
F'(x_0) = \lim_{dx \to 0} \frac{F(x_0+dx) - F(x_0)}{dx} = \lim_{dx \to 0} \frac{dF}{dx}
\end{align*}
Следовательно, при малых $dx$
\begin{align*}
F'(x_0) \approx \frac{dF}{dx} \quad \Leftrightarrow \quad dF \approx F'(x_0)dx
\end{align*}
\end{frame}



\begin{frame}{Дельта спот-сделки}
\justify
Спот-сделка: покупка $N=1\,000\,000$ евро по справедливому курсу $S_0=100$.
\begin{align*}
V = N - N\frac{S_0}{S} \quad \Rightarrow \quad \Delta(S) = \frac{\partial V}{\partial S} = N\frac{S_0}{S^2}
\end{align*}

\justify
Сразу после сделки $S_0=S$:
\begin{align*}
\Delta(S_0) = N\frac{S_0}{S_0^2} = \frac{N}{S_0}
\end{align*}

\justify
На каждые $dS=0.01$ (1 копейка) изменения курса PV изменяется на
\begin{align*}
dV \approx \Delta \cdot dS = \frac{N}{S_0}dS = \frac{\EUR{1\,000\,000}}{100}\cdot 0.01 = \EUR{100}
\end{align*}
\end{frame}



\begin{frame}{Дельта ванильного колла}
\justify
В модели Блэка-Шоулза дельта ванильного колл-опциона со страйком $K$ и сроком экспирации $T$ лет равна
\begin{align*}
\Delta = \frac{\partial C(S, K, T, \sigma, r, q)}{\partial S} = e^{-qT}N(d_1)
\end{align*}

\justify
Если у нас сложная модель цены опциона, для которой нельзя выписать аналитическую формулу, то дельту можно оценить численно как разностную производную.
\begin{align*}
\Delta \approx \frac{C(S+\delta,...) - C(S,...)}{\delta} \approx \frac{C(S+\delta,...) - C(S-\delta,...)}{2\delta}
\end{align*}

\justify
Дельта опциона --- количество единиц базового актива, которые нужно купить, чтобы реплицировать опцион на следующий малый промежуток времени.
\end{frame}



\begin{frame}{Дельта ванильного колла - 2}
\justify
Пример: колл-опцион со страйком $K=100$, сроком $T=0.25$ лет, $r=5\%$, $\sigma=10\%$.

\centering
\begin{tikzpicture}
\begin{axis}[
			domain=94:106,
			xtick={94,96,...,106},
			ytick={0,1,2,...,10},
			xmin=94, xmax=106,
			ymin=0, ymax=6,
			grid = major,
			xlabel={Курс сегодня ($S_0$)},
			ylabel={Цена опциона}
]
	
	\addplot[color = Set1-A, mark = none, thick]
	table[
		x=S,
		y=C_10,
		col sep=comma
	]
	{call_price.csv};
	
	%\addplot[Set1-D, very thick, dashed] {(\x >= 72)*(\x - 72) + 0.05};
	
	\node[inner sep=2pt, circle, fill, color=Set1-A] at (axis cs: 100, 2.6648) {};
	
	\addplot[color=Set1-A, dashed, thick] {0.6083 * \x - 58.1936};
	
	\node[anchor=south west] at (axis cs: 96.5, 0) {$\Delta = 0.61$};
\end{axis}
\end{tikzpicture}
\end{frame}



\begin{frame}{Дельта ванильного пута}
\justify
Паритет опционов колл и пут, когда актив не платит дивидендов:
\begin{align*}
C - P = S - Ke^{-rT}
\end{align*}

\justify
Возьмём производные от обеих частей:
\begin{align*}
\frac{\partial C}{\partial S} - \frac{\partial P}{\partial S} = \frac{\partial S}{\partial S} \quad \Leftrightarrow \quad \frac{\partial P}{\partial S} = \frac{\partial C}{\partial S} - 1
\end{align*}

\justify
Это свойство не зависит от предположений о характере движения базового актива (то есть
справедливо не только в мире Блэка-Шоулза).

\justify
Если базовый актив имеет дивидендную доходность $q$:
\begin{align*}
\frac{\partial P}{\partial S} = \frac{\partial C}{\partial S} - e^{-qT}
\end{align*}
\end{frame}


\begin{frame}{Дельта ванильных опционов}
\centering
\begin{tikzpicture}
\begin{axis}[
			xtick={0.5},
			xticklabel={$K$},
			ytick={-1,-0.75,...,1},
			xmin=0, xmax=1,
			ymin=-1, ymax=1,
			grid = major,
			xlabel={Цена базового актива ($S$)},
			ylabel={Дельта опциона},
			legend entries = {
  	   Колл,
  	   Пут
  },
  legend cell align={left},
  legend style={at={(0.03,0.97)},anchor=north west}
]
	
	\addplot[color = Set1-A, mark = none, thick]
	table[
		x=S,
		y=call_delta,
		col sep=comma
	]
	{black_scholes_delta.csv};
	
	\addplot[color = Set1-B, mark = none, thick]
	table[
		x=S,
		y=put_delta,
		col sep=comma
	]
	{black_scholes_delta.csv};

   \draw[thick, color=black] (axis cs: 0, 0) -- (axis cs: 1, 0);
\end{axis}
\end{tikzpicture}
\end{frame}



\begin{frame}{Дельта и реплицирующий портфель}
\justify
Сколько актива и долга нужно для репликации опциона?

$C(S_0) \approx \Delta \cdot S_0 + L$

\justify
\centering
\begin{tikzpicture}
\begin{axis}[
						domain=92:108,
			xtick=\empty,
			ytick={-2.23, 0},
			yticklabels={$L$, 0},
			xmin=92, xmax=108,
			ymin=-3, ymax=6,
			xlabel={Курс сегодня ($S_0$)},
			ylabel={Цена опциона ($C(S_0)$)}
]
	
	\addplot[color = Set1-A, mark = none, thick]
	table[
		x=S,
		y=C_10,
		col sep=comma
	]
	{call_price.csv};
	
	%\addplot[Set1-D, very thick, dashed] {(\x >= 72)*(\x - 72) + 0.05};
	
	\node[inner sep=2pt, circle, fill, color=Set1-A] at (axis cs: 100, 2.6648) {};
	
	\addplot[color=Set1-A, dashed, thick] {0.6083 * \x - 58.1936};
	
   \draw[thick, color=black] (axis cs: 0, 0) -- (axis cs: 200, 0);
	
	\node[anchor=south west] at (axis cs: 96.5, 0) {$\Delta$};
\end{axis}
\end{tikzpicture}
\end{frame}



\begin{frame}{Котирование опционов}
\justify
Несмотря на то, что модель Блэка-Шоулза не работает (не описывает реальность), все продолжают её использовать, чтобы договариваться о ценах опционов. В разговоре удобнее оперировать волатильностью, чем премиями, чтобы понимать, что дорого, а что дёшево.

\justify
Вместо такого диалога:

--- Сколько стоит колл со страйком 110 на 3 месяца на \$1\,000?

--- 800 рублей.

\justify
Получается такой:

--- Сколько стоит колл со страйком 110 на 3 месяца на \$1\,000?

--- 45\%.

\justify
Предполагается, что каждый может подставить волатильность в формулу Блэка-Шоулза и вычислить премию.
\end{frame}



\begin{frame}{Дельта и страйки}
\justify
Что происходит с улыбкой волатильности, когда цена базового актива растёт? Обычно она смещается в ту же сторону.

\centering
\begin{tikzpicture}
\begin{axis}[
			width = \textwidth,
			height = \textheight - 2cm,
			domain=90:114,
			xtick={90,92,...,114},
			ytick={0.05,0.1,...,0.5},
			xmin=90, xmax=114,
			ymin=0, ymax=0.50,
			yticklabel={\pgfmathparse{\tick*100}\pgfmathprintnumber[precision=0]{\pgfmathresult}\%},
			grid = major,
			xlabel={Страйк ($K$)},
			ylabel={Волатильность ($\sigma$)},
			legend entries = {
  	   $S=98$,
  	   $S=104$
  },
  legend cell align={left},
  legend style={at={(0.97,0.03)},anchor=south east}
]

	\addplot[color = Set1-B, mark = none, thick, dashed] {0.002*(\x - 98)^2 + 0.2};
	
	\addplot[color = Set1-B, mark = none, thick] {0.002*(\x - 104)^2 + 0.2};
	
	\draw[->, >= triangle 45, thick] (98, 0.175) -- (104, 0.175);
\end{axis}
\end{tikzpicture}
\end{frame}



\begin{frame}{Дельта и страйки - 2}
\justify
Так как улыбка волатильности следует за движениями спота, нужно очень быстро договариваться о сделках. Сейчас страйку 68 соответствует волатильность 10\%, а через полминуты из-за движения спота --- уже 11\%.

\justify
На внебиржевом рынке принято оперировать следующим страйками, которые привязаны к дельтам:
\begin{itemize}
\item DN (delta neutral) --- такой страйк $K_{DN}$, что стрэддл из колла и пута имеет дельту 0.
\item 25C --- такой страйк $K_{25C}$, что колл имеет дельту 0.25.
\item 25P --- такой страйк $K_{25P}$, что пут имеет дельту $-0.25$.
\item 10C, 10P --- аналогично 25C и 25P.
\end{itemize}

\justify
Эмпирически, implied волатильность таких страйков более стабильная (эффект sticky delta).
\end{frame}



\newcommand{\drawVolNode}[4] {

	\node[
		circle,
		color=Set1-B,
		fill,
		inner sep=2pt
	] at (#1, #2) {};
	
	\node[anchor=#4] at (#1, #2) {$\sigma_{#3}$};
}

\begin{frame}{Дельта и страйки - 3}
\centering
\begin{tikzpicture}
\begin{axis}[
			width = \textwidth,
			height = \textheight - 2cm,
			xtick={0.10, 0.25, 0.5, 0.75, 0.90},
			xticklabels={10P, 25P, DN, 25C, 10C},
			ytick={0,0.05,0.1,...,0.4},
			xmin=0, xmax=1,
			ymin=0, ymax=0.4,
			yticklabel={\pgfmathparse{\tick*100}\pgfmathprintnumber[precision=0]{\pgfmathresult}\%},
			grid = major,
			xlabel={Страйк ($K$)},
			ylabel={Волатильность ($\sigma$)}
]

	\addplot[color = Set1-B, mark = none, thick, samples at={0,0.01,...,1}] {0.75*(\x - 0.5)^2 + 0.2};
	
	\drawVolNode{0.10}{0.320}{10P}{south west}
	\drawVolNode{0.25}{0.247}{25P}{south west}
	\drawVolNode{0.50}{0.200}{DN }{south}
	\drawVolNode{0.75}{0.247}{25C}{south east}
	\drawVolNode{0.90}{0.320}{10C}{south east}
\end{axis}
\end{tikzpicture}
\end{frame}



\begin{frame}{Параметризация улыбки волатильности}
\justify
Волатильности опционов со страйками 10P, 25P, DN, 25C и 10C хорошо описывают форму улыбки волатильности, но с ними тоже есть неудобство. Когда волатильность на рынке повышается, более-менее равномерно повышаются все пять волатильностей (<<прилив поднимает все лодки>>).

\justify
Риск-реверсал (\en{risk-reversal, risky}): RR25 --- купленный колл 25C и проданный пут 25P.
\begin{align*}
\sigma_{RR25} = \sigma_{25C} - \sigma_{25P}
\end{align*}

\justify
Бабочка (\en{butterfly, fly}): FLY25 --- купленные колл 25C и пут 25P, проданные колл и пут $DN$
\begin{align*}
\sigma_{FLY25} = 0.5\cdot(\sigma_{25P} - 2\sigma_{DN} + \sigma_{25C})
\end{align*}

\justify
Формулы RR и FLY похожи на первую и вторую разностную производную.
\end{frame}



\begin{frame}{Параметризация улыбки волатильности - 2}
\justify
Волатильность стрэддла DN задаёт общий уровень улыбки. Выше $\sigma_{DN}$ --- выше дисперсия распределения цены базового актива.

\centering
\begin{tikzpicture}
\begin{axis}[
			width = \textwidth,
			height = \textheight - 2cm,
			xtick={0.10, 0.25, 0.5, 0.75, 0.90},
			xticklabels={10P, 25P, DN, 25C, 10C},
			ytick={0,0.02,0.04,...,0.24},
			xmin=0, xmax=1,
			ymin=0, ymax=0.24,
			yticklabel={\pgfmathparse{\tick*100}\pgfmathprintnumber[precision=0]{\pgfmathresult}\%},
			grid = major,
			xlabel={Страйк ($K$)},
			ylabel={Волатильность ($\sigma$)},
			legend entries = {
  	   $\sigma_{DN}=6\%$,
  	   $\sigma_{DN}=10\%$
  },
  legend cell align={left},
  legend style={at={(0.97,0.03)},anchor=south east}
]

	\addplot[color = Set1-B, mark = none, thick, dashed, samples at={0,0.01,...,1}] {0.5*(\x - 0.5)^2 + 0.06};
	

	\addplot[color = Set1-B, mark = none, thick, samples at={0,0.01,...,1}] {0.5*(\x - 0.5)^2 + 0.1};
	
	\draw[<->, >= triangle 45, thick] (0.5, 0.0) -- (0.5, 0.1);
	
	\drawVolNode{0.5}{0.1}{DN}{south}
\end{axis}
\end{tikzpicture}
\end{frame}



\begin{frame}{Параметризация улыбки волатильности - 3}
\justify
Риск-реверсал задаёт наклон улыбки, или скошенность распределения базового актива влево или вправо.

\centering
\begin{tikzpicture}
\begin{axis}[
			width = \textwidth,
			height = \textheight - 2cm,
			xtick={0.10, 0.25, 0.5, 0.75, 0.90},
			xticklabels={10P, 25P, DN, 25C, 10C},
			ytick={0,0.02,0.04,...,0.24},
			xmin=0, xmax=1,
			ymin=0, ymax=0.24,
			yticklabel={\pgfmathparse{\tick*100}\pgfmathprintnumber[precision=0]{\pgfmathresult}\%},
			grid = major,
			xlabel={Страйк ($K$)},
			ylabel={Волатильность ($\sigma$)},
			legend entries = {
  	   $\sigma_{RR25}=0\%$,
  	   $\sigma_{RR25}=4\%$
  },
  legend cell align={left},
  legend style={at={(0.97,0.03)},anchor=south east}
]

	\addplot[color = Set1-B, mark = none, thick, dashed, samples at={0,0.01,...,1}] {0.5*(\x - 0.5)^2 + 0.1};
	

	\addplot[color = Set1-B, mark = none, thick, samples at={0,0.01,...,0.5}] {0.32*(\x - 0.5)^2 + 0.1};
	
	\addplot[color = Set1-B, mark = none, thick, samples at={0.5,0.51,...,1}] {0.96*(\x - 0.5)^2 + 0.1};
	
	
	\drawVolNode{0.25}{0.12}{25P}{north east}
	
	\drawVolNode{0.75}{0.16}{25C}{south east}
	
	\draw (0.25, 0.12) -- (0.75, 0.16);
	
	\draw (0.25, 0.12) -- (0.9, 0.12);
	\draw (0.75, 0.16) -- (0.9, 0.16);
	
	\draw[<->, >= triangle 45, thick] (0.87, 0.16) -- (0.87, 0.12) node[pos=0.5, anchor=west] {$\sigma_{RR25}$};
\end{axis}
\end{tikzpicture}
\end{frame}



\begin{frame}{Параметризация улыбки волатильности - 4}
\justify
Бабочка задаёт вогнутость улыбки, или толщину хвостов распределения базового актива.

\centering
\begin{tikzpicture}
\begin{axis}[
			width = \textwidth,
			height = \textheight - 2cm,
			xtick={0.10, 0.25, 0.5, 0.75, 0.90},
			xticklabels={10P, 25P, DN, 25C, 10C},
			ytick={0,0.02,0.04,...,0.24},
			xmin=0, xmax=1,
			ymin=0, ymax=0.24,
			yticklabel={\pgfmathparse{\tick*100}\pgfmathprintnumber[precision=0]{\pgfmathresult}\%},
			grid = major,
			xlabel={Страйк ($K$)},
			ylabel={Волатильность ($\sigma$)},
			legend entries = {
  	   $\sigma_{FLY25}=1\%$,
  	   $\sigma_{FLY25}=4\%$
  },
  legend cell align={left},
  legend style={at={(0.97,0.03)},anchor=south east}
]

	\addplot[color = Set1-B, mark = none, thick, dashed, samples at={0,0.01,...,1}] {0.16*(\x - 0.5)^2 + 0.1};

	\addplot[color = Set1-B, mark = none, thick, samples at={0.0,0.01,...,1}] {0.64*(\x - 0.5)^2 + 0.1};
	
	
	\drawVolNode{0.25}{0.14}{25P}{south west}
	
	\drawVolNode{0.75}{0.14}{25C}{south east}
	
	\drawVolNode{0.50}{0.10}{DN}{north}
	
	\draw (0.25, 0.14) -- (0.75, 0.14);

	\draw[<->, >= triangle 45, thick] (0.5, 0.10) -- (0.5, 0.14) node[pos=0.5, anchor=west] {$\sigma_{FLY25}$};
\end{axis}
\end{tikzpicture}
\end{frame}



\begin{frame}{Поверхность волатильности}
\justify
Рыночная поверхность волатильности --- котировки DN, RR и FLY для разных сроков экспирации.

\justify
\centering
\begin{tabular}{l|r|r|r|r|r}
Срок & DN     & RR25   & RR10   & FLY25  & FLY10  \\ \hline
1M   & 6.00\% & 0.50\% & 0.90\% & 0.25\% & 0.70\% \\
2M   & 6.40\% & 0.60\% & 1.10\% & 0.30\% & 0.70\% \\
3M   & 6.50\% & 0.65\% & 1.15\% & 0.30\% & 0.75\% 
\end{tabular}

\justify
Это --- наблюдаемая действительность. Дальше мы должны откалибровать модель, например SABR, то есть подобрать внутренние параметры, при которых модель будет выдавать такие же цены на DN, RR и FLY.

\justify
Если нам это удалось, то можно надеяться, что модель будет выдавать разумные цены не только на ликвидные ванильные опционы, но и на более сложные и менее ликвидные продукты, цены которых мы не видим.
\end{frame}



\begin{frame}{Вега}
\justify
Предположим, что мы умеем магическим образом подбирать вектор параметров SABR функцией $f$:
\begin{align*}
(\alpha, \beta, \rho) = f(\sigma_{1M,DN}, \sigma_{1M,RR25}, ..., \sigma_{3M,FLY10})
\end{align*}

\justify
Также у нас есть численный метод, который умеет при помощи SABR оценивать PV нашего портфеля $V$ --- функция $g$.
\begin{align*}
V = g(\alpha, \beta, \rho) = g\Big(f(\sigma_{1M,DN}, \sigma_{1M,RR25}, ..., \sigma_{3M,FLY10})\Big)
\end{align*}

\justify
Резонно задать вопрос: что будет с портфелем, если рыночные котировки изменятся? На этот вопрос отвечает производная PV портфеля по волатильности --- \alert{Вега}.
\begin{align*}
\mathcal{V}_{1M,RR25} &= \frac{\partial V}{\partial \sigma_{1M,RR25}} \approx \\
&\approx \frac{g\Big(f(...,\sigma_{1M,RR25}+\delta,...)\Big) - g\Big(f(...,\sigma_{1M,RR25},...)\Big)}{\delta}
\end{align*}
\end{frame}



\begin{frame}{Вега ванильного опциона}
\justify
В модели Блэка-Шоулза волатильность $\sigma$ --- константа, поэтому странно считать производную. Тем не менее, и для колл-опциона, и для пут-опциона вега равна
\begin{align*}
\mathcal{V} &= \frac{\partial C}{\partial \sigma} = \frac{\partial P}{\partial \sigma} = Se^{-qT}\sqrt{T}N'(d_1) \\
N'(x) &= \frac{1}{\sqrt{2\pi}}e^{-\frac{x^2}{2}}
\end{align*}

\justify
Обычные единицы измерения --- доллары или евро на процентный пункт изменения волатильности.

\justify
Вега положительная и для колла, и для пута. <<Опционы любят волатильность>>.
\end{frame}



\begin{frame}{Вега опциона}
\centering
\begin{tikzpicture}
\begin{axis}[
			xtick={0.5},
			xticklabel={$K$},
			ytick={\empty},
			xmin=0, xmax=1,
			ymin=0, ymax=0.25,
			grid = none,
			xlabel={Цена базового актива ($S$)},
			ylabel={Вега опциона}
]
	
	\addplot[color = Set1-B, mark = none, thick]
	table[
		x=S,
		y=vega,
		col sep=comma
	]
	{black_scholes_vega.csv};
\end{axis}
\end{tikzpicture}
\end{frame}



\begin{frame}{Тета}
\justify
\alert{Тета} --- чувствительность PV к течению времени. На сколько подешевеет купленный опцион, когда он станет на один день ближе к экспирации?

\begin{align*}
\Theta = -\frac{\partial V}{\partial T}
\end{align*}
Обычные единицы измерения --- доллары или евро на календарный или торговый день.

\justify
Тета колл-опциона в модели Блэка-Шоулза:
\begin{align*}
\Theta = -\frac{\partial C}{\partial T} = - \frac{S\sigma e^{-qT}N'(d_1)}{2\sqrt{T}} -rKe^{-rT}N(d_2) + qSe^{-qT}N(d_1)
\end{align*}
\end{frame}



\begin{frame}{Тета колл-опциона}
\centering
\begin{tikzpicture}
\begin{axis}[
			xtick={0.5},
			xticklabels={$K$},
			ytick={0},
			scaled y ticks = false,
			xmin=0, xmax=1,
			ymax=0,
			grid = none,
			xlabel={Цена базового актива ($S$)},
			ylabel={Тета опциона}
]
	
	\addplot[color = Set1-B, mark = none, thick]
	table[
		x=S,
		y=theta,
		col sep=comma
	]
	{black_scholes_theta.csv};
\end{axis}
\end{tikzpicture}
\end{frame}



\begin{frame}{Тета колл-опциона - 2}
\centering
\begin{tikzpicture}
\begin{axis}[
			xtick={\empty},
			ytick={0},
			xmin=0, xmax=1,
			ymin=-0.15, ymax=0,
			grid = major,
			xlabel={Срок экспирации ($T$)},
			ylabel={Тета опциона},
			legend entries = {
  	   Out of the money,
  	   In the money,
  	   At the money
  },
  legend cell align={left},
  legend style={at={(0.97,0.03)},anchor=south east}
]
	
	\addplot[color = Set1-A, mark = none, thick]
	table[
		x=T,
		y=theta_otm,
		col sep=comma
	]
	{black_scholes_theta_by_money.csv};
	
	\addplot[color = Set1-B, mark = none, thick]
	table[
		x=T,
		y=theta_itm,
		col sep=comma
	]
	{black_scholes_theta_by_money.csv};


	\addplot[color = Set1-C, mark = none, thick]
	table[
		x=T,
		y=theta_atm,
		col sep=comma
	]
	{black_scholes_theta_by_money.csv};
\end{axis}
\end{tikzpicture}
\end{frame}



\begin{frame}{Гамма}
\justify
\alert{Гамма} --- чувствительность дельты к изменению цены базового актива. Другое определение --- вторая производная PV по цене базового актива.
\begin{align*}
\Gamma = \frac{\partial V^2}{\partial^2 S} = \frac{\partial \Delta}{\partial S}
\end{align*}

\justify
Чем больше по модулю гамма, тем хуже дериватив приближается линейной комбинацией базового актива и долга, и тем чаще нужно ребалансировать реплицирующий портфель.

\justify
В модели Блэка-Шоулза гамма колла равна гамме пута:
\begin{align*}
\Gamma = \frac{\partial C^2}{\partial^2 S} = \frac{\partial P^2}{\partial^2 S} =
\frac{e^{-qT}N'(d_1)}{S\sigma\sqrt{T}} 
\end{align*} 
\end{frame}



\begin{frame}{Гамма и дельта-хеджирование}
\justify
Гамма отражает степень вогнутости функции. Чем более вогнута функция, тем больше потери на неточном хеджировании.

\centering
\begin{tikzpicture}
\begin{axis}[
			domain=92:110,
			xtick distance=2,
			ytick distance=1,
			xmin=94, xmax=110,
			ymin=0, ymax=10,
			grid = major,
			xlabel={Курс сегодня ($S$)},
			ylabel={Цена опциона}
]
	
	\addplot[color = Set1-A, mark = none, thick]
	table[
		x=S,
		y=C_10,
		col sep=comma
	]
	{call_price.csv};
	
	%\addplot[Set1-D, very thick, dashed] {(\x >= 72)*(\x - 72) + 0.05};
	
	\node[inner sep=2pt, circle, fill, color=Set1-A] at (axis cs: 100, 2.6648) {};
	
	\node[inner sep=2pt, circle, fill, color=Set1-A] at (axis cs: 108, 9.3179) {};
	
	\addplot[color=Set1-A, dashed, thick] {0.6083 * \x - 58.1936};
	
	\node[anchor=south west] at (axis cs: 96.5, 0) {$\Delta = 0.61$};
	
	\draw[<->, >= triangle 45, thick] (108, 9.3179) -- (108, 7.5028);
\end{axis}
\end{tikzpicture}
\end{frame}



\begin{frame}{Гамма опциона}
\centering
\begin{tikzpicture}
\begin{axis}[
			xtick={0.5},
			xticklabels={$K$},
			ytick={0},
			scaled y ticks = false,
			xmin=0, xmax=1,
			ymin=0,
			grid = none,
			xlabel={Цена базового актива ($S$)},
			ylabel={Гамма опциона}
]
	
	\addplot[color = Set1-B, mark = none, thick]
	table[
		x=S,
		y=gamma,
		col sep=comma
	]
	{black_scholes_gamma.csv};
\end{axis}
\end{tikzpicture}
\end{frame}



\begin{frame}{Гамма опциона}
\centering
\begin{tikzpicture}
\begin{axis}[
			xtick={\empty},
			ytick={0},
			xmin=0, xmax=1,
			ymin=0, ymax=15,
			grid = major,
			xlabel={Срок экспирации ($T$)},
			ylabel={Гамма опциона},
			legend entries = {
		At the money,
  	   Out of the money,
  	   In the money
  },
  legend cell align={left},
  legend style={at={(0.97,0.97)},anchor=north east}
]
	
	\addplot[color = Set1-C, mark = none, thick]
	table[
		x=T,
		y=gamma_atm,
		col sep=comma
	]
	{black_scholes_gamma_by_money.csv};
	
	\addplot[color = Set1-A, mark = none, thick]
	table[
		x=T,
		y=gamma_otm,
		col sep=comma
	]
	{black_scholes_gamma_by_money.csv};
	
	\addplot[color = Set1-B, mark = none, thick]
	table[
		x=T,
		y=gamma_itm,
		col sep=comma
	]
	{black_scholes_gamma_by_money.csv};

\end{axis}
\end{tikzpicture}
\end{frame}



\begin{frame}{Гамма и тета}
\justify
Рассмотрим портфель из базового актива и опциона. В модели Блэка-Шоулза выполняется условие на цену $V$ этого портфеля:
\begin{align*}
\frac{\partial V}{\partial t} + rS\frac{\partial V}{\partial S} + \frac{1}{2}\sigma^2S^2\frac{\partial^2 V}{\partial S^2} = rV
\end{align*}
\begin{align*}
-\Theta + rS\Delta + \frac{1}{2}\sigma^2S^2\Gamma = rV
\end{align*}

Если мы регулярно дельта-хеджируем портфель, то $\Delta=0$.
\begin{align*}
-\Theta + \frac{1}{2}\sigma^2S^2\Gamma = rV
\end{align*}

\justify
Большим значениям $\Gamma$ обычно соответствуют большие значения $\Theta$. Если мы купили опцион ($\Gamma > 0$), то мы зарабатываем на больших движениях базового актива, но теряем тету с течением времени (time value опциона уменьшается). Если мы продали опцион ($\Gamma < 0$), то мы теряем деньги из-за неточности хеджирования, но зарабатываем тету.
\end{frame}



\begin{frame}{Гамма и тета: пример}
\justify
Предположим, что в мире Блэка-Шоулза $S=\$100$, $r=0\%$, $q=0\%$, $\sigma=20\%$. Мы 
\alert{продали} клиенту колл-опцион со страйком $K=\$100$ и сроком исполнения $T=0.25$ 
лет.

\justify
Справедливая премия за опцион $\$3.99$, дельта 0.52. Мы захеджировали риск: купили
0.52 акции и взяли \$48.01 в долг. PV нашей позиции 0, дельта 0. 

\justify
Прошло время $dt = 1/252$ (один рабочий день). Цена акции изменилась. Сколько мы потеряли или заработали на нашей дельта-нейтральной позиции?
\end{frame}



\begin{frame}{Гамма и тета: пример - 2}
\centering
\begin{tikzpicture}
\begin{axis}[
	height = \textheight - 0.5cm,
	xmin = 96, xmax = 104,
	ymin = -0.3, ymax = 0.05,
	grid = major,
	xlabel = {Цена акции завтра ($S$)},
	ylabel = {PV позиции, доллары},
	y tick label style={
        /pgf/number format/.cd,
            fixed,
            fixed zerofill,
            precision=2,
        /tikz/.cd}
]

	\addplot[color=Set1-A, thick] table[x=S, y=full_pnl, col sep=comma] {gamma_pnl_example.csv};
	
	\draw[<->, >= triangle 45, thick] (100, 0.0) -- (100, 0.032) node[pos=0.5, anchor=west] {$\theta$};
	
	\draw[<->, >= triangle 45, thick] (103, 0.0) -- (103, -0.145) node[pos=0.5, anchor=west] {$\Gamma$};
	
   \draw[thick, color=black] (axis cs: 95, 0) -- (axis cs: 105, 0);
\end{axis}
\end{tikzpicture}
\end{frame}



\begin{frame}{Гамма и вега}
\justify
В модели Блэка-Шоулза гамма и вега отличаются на константный множитель.
\begin{align*}
\Gamma &= \frac{e^{-qT}N'(d_1)}{S\sigma\sqrt{T}} \\
\mathcal{V} &= Se^{-qT}\sqrt{T}N'(d_1)
\end{align*}

\justify
В некотором смысле обе производные показывают нашу зависимость от разброса будущих значений $S$.

\justify
Гамма --- главный риск для коротких опционов ($\sqrt{T}$ в знаменателе), вега --- для длинных ($\sqrt{T}$ в числителе). 
\end{frame}



\begin{frame}{Дельта-хеджирование волатильность}

\justify
Рассмотрим акцию, которая не платит дивидендов, стоит $S_0=\$100$ и следует 
геометрическому броуновскому движению с трендом $\mu=5\%$ и волатильностью
$\sigma=20\%$. Безрисковая ставка $r=0\%$.

\justify
Рассмотрим европейский колл-опцион со страйком $K=100$, сроком исполнения $T=1$ год. 
Представим, что нам не удалось его купить на рынке и мы занимаемся динамической 
репликацией.

\justify
Предположим, что сейчас у нас на счету есть $\$7.97$, и мы можем 
перебалансировать портфель $N=4\cdot365$ раз в год.

\end{frame}




\begin{frame}{Дельта-хеджирование и волатильность - 2}
\centering
\begin{tikzpicture}
\begin{axis}[
	width = \textwidth,
	height = \textheight - 0.5cm,
	xlabel = {Цена базового актива через $T=1$ год ($S_T$)},
	ylabel = {Цена портфеля},
	xmin = 70, xmax = 130,
	ymin = -5, ymax = 30,
	grid = major,
	ytick = {-10, -5, ..., 50}
]	

   \addplot[thick, mark = o, only marks, color=Set1-A]
     table [col sep=comma, x=S, y=balance]
     {black_scholes_hedging_1420.csv};    
    
    \addplot[black, very thick, solid, domain = 0:150] {0};
\end{axis}
\end{tikzpicture}
\end{frame}



\begin{frame}{Дельта-хеджирование и волатильность - 3}
\justify
Как мы считаем $\Delta$ для хеджа? Из формулы Блэка-Шоулза. Но в ней участвует 
\alert{implied} волатильность! Вдруг реализованная волатильность окажется выше?

\justify
Предположим, что на каждом шаге волатильность броуновском движении оказывается $\sigma_{realized}=25\%$. Мы не знаем этого заранее, поэтому подбираем размер хеджа,
исходя из $\sigma_{implied}=20\%$. 
\end{frame}



\begin{frame}{Дельта-хеджирование и волатильность - 4}
\centering
\begin{tikzpicture}
\begin{axis}[
	width = \textwidth,
	height = \textheight - 0.5cm,
	xlabel = {Цена базового актива через $T=1$ год ($S_T$)},
	ylabel = {Цена портфеля},
	xmin = 70, xmax = 130,
	ymin = -5, ymax = 30,
	grid = major,
	ytick = {-10, -5, ..., 50}
]	

	\addplot[very thick, dashed, color=Set1-A, domain=70:130] {(\x > 100) * (\x - 100) + 0.1};

   \addplot[thick, mark = o, only marks, color=Set1-A]
     table [col sep=comma, x=S, y=balance]
     {black_scholes_hedging_realized_vol.csv};    
    
    
    \addplot[black, very thick, solid, domain = 0:150] {0};
\end{axis}
\end{tikzpicture}
\end{frame}



\begin{frame}{Заключение}
\justify
Можно считать, что премия за опцион --- ожидаемая рынком прибыль по гамме (сколько в 
сумме заработает владелец опциона, если будет его дельта-хеджировать). С другой 
стороны, премия за опцион --- компенсация потерь продавца (он будет терять деньги на дельта-хеджировании).

\justify
Если реализовавшаяся волатильность окажется выше, чем \en{implied}, то продавец опциона
потеряет деньги на хеджировании, а покупатель --- заработает.

\justify
В среднем, на длинной дистанции, выгоднее скорее продавать гамму (продавать опционы, то
есть продавать остальным страховку от кризиса). Иногда такая стратегия даёт сбой: 
движение рынка такое сильное, что потери на гамме превосходят полученную премию. 
\end{frame}



\begin{frame}{Бонус: индекс GAMMA}
\justify
Чикагская опционная биржа (\en{CBOE}) публикует индекс \en{GAMMA}, который отслеживает доходность следующей торговой стратегии:

\justify
1. Купить \en{at-the-money}\ стрэдлы на индекс \en{S\&P\,500}\ с экспирациями в ближайшие пять ближайших торговых дней. Например, если сегодня понедельник, то стратегия покупает пары \en{ATM}\ колллов и путов с экспирациями со вторника по пятницу и в следующий понедельник.

\justify
2. Захеджировать дельту ближайшим фьючерсом.

\justify
3. На следующий день, после экспирации ближайшего стрэддла, купить новый стрэддл с экспирацией через неделю и снова выполнить дельта-хеджирование.

\justify
Стратегия зарабатывает деньги на резких колебаниях рынка, когда реализованная волатильность оказывается выше, чем \en{implied}\ волатильность, которую ожидали продавцы стрэддлов.
\end{frame}



\begin{frame}{Бонус: индекс GAMMA - 2}
\centering
\begin{tikzpicture}
\begin{axis}[
 	width=\textwidth,
	height=\textheight - 1cm,
	date coordinates in=x,
	date ZERO=2012-01-01,
  	xtick={2013-01-01, 2015-01-01, 2017-01-01, 2019-01-01, 2021-01-01, 2023-01-01, 2025-01-01},
  	minor xtick={2014-01-01, 2016-01-01, 2018-01-01, 2020-01-01, 2022-01-01,2024-01-01},
  	xticklabel={\year},
  	xmin=2012-05-31,
  	xmax=2025-01-01,
	 ymode = log,
 	grid=both,
   	log ticks with fixed point,
  	ylabel={\small{Рост \$1 инвестиций}},
  	xlabel near ticks,
  	ylabel near ticks,
	legend entries = {
		S\&P\,500 Total Return,
		CBOE Gamma (inverted),
		T-bills
	},
	legend pos=north east,
	legend style={font=\tiny},
	legend cell align={left}
]

\addplot[color = Set1-A, mark = none, very thick]
	table[
		x=month,
		y=sp500_growth,
		col sep=comma
	]
	{cboe_gamma.csv};

\addplot[color = Set1-B, mark = none, very thick]
	table[
		x=month,
		y=inv_gamma_growth,
		col sep=comma
	]
	{cboe_gamma.csv};

\addplot[color = Set1-C, mark = none, very thick, dashed]
	table[
		x=month,
		y=rf_growth,
		col sep=comma
	]
	{cboe_gamma.csv};

   \draw[thick, color=black] (axis cs: 1985-01-01, 1) -- (axis cs: 2025-01-01, 1);
\end{axis}
\end{tikzpicture}

\small Данные: CBOE, Yahoo Finance, Kenneth French.
\end{frame}



\begin{frame}{Бонус: индекс GAMMA - 3}
\centering
\begin{tabular}{l|r|r|r}
& Inv. GAMMA & S\&P\,500 & T-bills \\ \hline
Геом. среднее & 12.8\% & 13.7\% & 1.1\% \\
Избыт. дох-ть & 11.6\% & 12.5\% & 0.0\% \\
Ст. откл. & 60.1\% & 16.0\% & 0.0\% \\
Отн. Шарпа & 0.19 & 0.79 & 0.0 \\
Мес. бета & 0.94 & 1.00 & 0.00
\end{tabular}

\small{Период: 2013--2023. Все метрики, кроме беты, в годовом выражении.}

\justify
Индекс \en{GAMMA}\ показывал отрицательную среднюю доходность. Положительную доходность показывала обратная стратегия --- продавать (а не покупать) стрэддлы.

\justify
Продавать гамму (продавать остальным страховку от кризиса) --- прибыльная на дистанции стратегия.
\end{frame}
\end{document}


