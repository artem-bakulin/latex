\documentclass{beamer}

\usepackage{cmap}				% To be able to copy-paste russian text from pdf
\usepackage[T2A]{fontenc}
\usepackage[utf8]{inputenc}
\usepackage[russian]{babel}
\usepackage{textpos}
\usepackage{ragged2e}
\usepackage{amssymb}
\usepackage{ulem}
\usepackage{tikz}
\usepackage{pgfplots}
\usepackage{color}
\usepackage{cancel}
\usepackage{multirow}
\pgfplotsset{compat=1.17}
\usetikzlibrary{arrows,snakes,backgrounds,shapes}
\usepgfplotslibrary{groupplots,colorbrewer,dateplot,statistics}
\usepackage{animate}

\usepackage{amsfonts}
\usepackage{amsmath}
\usepackage{amssymb}
\usepackage{graphicx}
\usepackage{setspace}
\usepackage{tabularx}

\usepackage{enumitem}
	\setitemize{label=\usebeamerfont*{itemize item}%
	  \usebeamercolor[fg]{itemize item}
	  \usebeamertemplate{itemize item}}
	

\usepackage{eurosym}
\renewcommand{\EUR}[1]{\textup{\euro}#1}

\title{Credit Default Swaps}
\author{Artem Bakulin}
\date{November 16, 2023}


	\usetheme{Warsaw}
	\usecolortheme{beaver}

	\newcommand{\inserttitleframe}{
		\begin{frame}
		\titlepage
		\end{frame}
	}
	
	\newcommand{\insertdisclaimerframe}{
	}
	
	% remove navigation bar
	\setbeamertemplate{navigation symbols}{}

	% add page counter
	\setbeamertemplate{page number in head/foot}[totalframenumber] 

	% remove navigation bar
	\setbeamertemplate{navigation symbols}{}
	
	\setbeamertemplate{page number in head/foot}[totalframenumber] 


\newcommand{\ru}[1]{\begin{otherlanguage}{russian}#1\end{otherlanguage}}
\newcommand{\en}[1]{\begin{otherlanguage}{english}#1\end{otherlanguage}}
\newcommand{\ruen}[2]{#1 (\en{#2})}

% https://tex.stackexchange.com/questions/98003/filter-rows-from-a-table
\pgfplotsset{
    discard if not/.style 2 args={
        x filter/.code={
            \edef\tempa{\thisrow{#1}}
            \edef\tempb{#2}
            \ifx\tempa\tempb
            \else
                \def\pgfmathresult{inf}
            \fi
        }
    }
}



\begin{document}

\inserttitleframe

\begin{frame}{Credit default swap}
\justify
A \alert{credit default swap (CDS)} is a financial derivative that allows buying or selling insurance against default on bonds or other debts.

\justify
Key parameters of a swap:
\begin{itemize}
\justifying
\item Reference entity: the entity whose default is covered by the insurance. E.g. FooBar LLC. It is neither buyer nor seller of the swap, it does not take part in the contract
\item Notional amount or insured amount. E.g. \$10\,000\,000.
\item Coverage period. E.g. starting today and ending in a year.
\item Fixed coupon (spread): percentage of the notional amount which the buyer pays to the seller. E.g. 2\% annually.
\end{itemize}

\justify
Buying a CDS means agreeing to pay fixed coupons in exchange for insurance. 
\end{frame}



\newcommand{\swapPartyNode}[5]{

	\draw (#1, #2)
		node[
			rectangle,
			draw,
			rounded corners,
			anchor = south,
			minimum height = 0.8cm,
			minimum width = 2.5cm
		]
		{#5}
	--
	(#3, #4);
}

\newcommand{\swapBuyerPaymentEx}[7]{

	\draw [
		->,
		>=triangle 90
	] 
	(#1, #2)
	node[
		label = left:{#7}
	]{}
	-- (#3, #4)
	node[
		pos=0.5,
		anchor=south
	]
	{#5}
	node[
		pos=0.5,
		anchor=north
	]
	{#6};
}

\newcommand{\swapBuyerPayment}[6]{

	\swapBuyerPaymentEx{#1}{#2}{#3}{#4}{#5}{}{#6}
}

\begin{frame}{Credit default swap - 2}
\justify
CDS term is 1 year, notional amount is \$10\,000\,000, coupon is $2.0\%$.

\justify
\centering
\begin{tikzpicture}[thick]

		\swapPartyNode{0}{0}{0}{-4}{Buyer}
		
		\swapPartyNode{5.5}{0}{5.5}{-4}{Seller}
		
		\swapBuyerPayment{0}{-1}{2.5}{-1}{\$50\,000}{3 months}
		\swapBuyerPayment{0}{-2}{2.5}{-2}{\$50\,000}{6 months}
		\swapBuyerPayment{0}{-3}{2.5}{-3}{\$50\,000}{9 months}
		\swapBuyerPayment{0}{-4}{2.5}{-4}{\$50\,000}{12 months}
\end{tikzpicture}
\end{frame}



\begin{frame}{Credit default swaps - 3}
\justify
The reference entity announced bankruptcy in 8 months.

\justify
\centering
\begin{tikzpicture}[thick]
		\swapPartyNode{0}{0}{0}{-2.67}{Buyer}
		
		\swapPartyNode{5.5}{0}{5.5}{-2.67}{Seller}
		
		\swapBuyerPayment{0}{-1}{2.5}{-1}{\$50\,000}{3 months}
		\swapBuyerPayment{0}{-2}{2.5}{-2}{\$50\,000}{6 months}
		\swapBuyerPaymentEx{0}{-2.67}{2.5}{-2.67}{\$33\,000}{Bond}{8 months}
		
		\swapBuyerPayment{5.5}{-2.67}{3}{-2.67}{\$10\,000\,000}{}
\end{tikzpicture}
\end{frame}



\begin{frame}{Deliverable and cash-settled swaps}
\justify
Suppose that a default has happened
\begin{itemize}
\justifying
\item Insured notional amount \$10\,000\,000.
\item Market price of the defaulted bond is \$4\,000\,000.
\end{itemize}

\justify
\alert{Deliverable} credit default swap:
\begin{itemize}
\justifying
\item The buyer hands over defaulted bonds to the seller.
\item The seller pays the buyer full notional amount (\$10\,000\,000).
\end{itemize}

\justify
\alert{Cash-settled} credit default swap:
\begin{itemize}
\justifying
\item The seller pays the buyer the difference between notional amount and market value of the bonds  (\$6\,000\,000).
\end{itemize}

\end{frame}



\begin{frame}{Hedging credit risk}
\justify
Suppose we own a corporate bond. Negative news has been announced, and now we are concerned that the probability of default has increased. We have two options:
\begin{itemize}
\item Sell the bond on the market.
\item Buy insurance, i.e. buy a credit default swap.
\end{itemize}

\justify
In theory, if there was an ideal perfectly liquid market for bonds, we would sell the bond immediately at zero bid/ask spread. In practice, it may take weeks to find a buyer for an illiquid security..

\justify
Sometimes buying a credit default swap may be simpler and more cost-effective than selling an illiquid bond. When (and if) the market stabilizes we sill sell the swap.

\justify
Credit default swaps are particularly useful for hedging non-traded debts, such as bank loans.
\end{frame}



\begin{frame}{Speculating on credit risk}

\justify
We have a view that creditworthiness of a reference entity will deteriorate in future. Our options are:
\begin{itemize}
\justifying
\item Sell short the risky bond and buy a risk-free government bond (to eliminate the general interest rate risk).
\item Buy a credit default swap.
\end{itemize}

\justify
Do you have to own a bond  to buy a CDS? No! This position would be called a naked CDS.

\justify
A small bank could be selling credit default swaps to diversify its credit portfolio. It could "make loans"\ to companies (reference entities) that would never speak to this bank otherwise.

\justify
The EU regulation prohibits naked CDS in case reference entity is one of the EU governments.
\end{frame}


\renewcommand{\swapPartyNode}[4]{

	\draw (#1, #2)
		node[
			rectangle,
			draw,
			rounded corners,
			anchor = south,
			minimum height = 0.8cm,
			minimum width = 2.8cm
		]
		(#4)
		{#3};
}

\newcommand{\paymentFlow}[4] {
	\draw [
		->,
		>=triangle 90
	]
	(#1) -- (#2)
	node[
		pos = 0.5,
		anchor = #4
	]
	{#3};
}

\begin{frame}{Speculating on credit risk - 2}
\justify
We have bought a swap from company A at 2\%, and managed to a sell similar swap to company B at 3\%. We will be making 1\% of the notional amount per annum until the default happens or until maturity of the swaps.

\justify
\centering
\begin{tikzpicture}

	\swapPartyNode{0}{2}{Company A}{A}
	\swapPartyNode{0}{0}{Speculator}{S}
	\swapPartyNode{0}{-2}{Company B}{B}
	
	\paymentFlow{[xshift=-0.5cm] S.north}{[xshift=-0.5cm] A.south}{2\%}{east};
	\paymentFlow{[xshift=-0.5cm] B.north}{[xshift=-0.5cm] S.south}{3\%}{east};
	
	\paymentFlow{[xshift=+0.5cm] A.south}{[xshift=+0.5cm] S.north}{Insurance}{west};
	\paymentFlow{[xshift=+0.5cm] S.south}{[xshift=+0.5cm] B.north}{Insurance}{west};
	
	\swapPartyNode{6}{2}{Company А}{A}
	\swapPartyNode{6}{0}{Speculator}{S}
	\swapPartyNode{6}{-2}{Company B}{B}
	
	\paymentFlow{[xshift=-0.5cm] A.south}{[xshift=-0.5cm] S.north}{Notional amount}{east};
	\paymentFlow{[xshift=-0.5cm] S.south}{[xshift=-0.5cm] B.north}{Notional amount}{east};
	
	\paymentFlow{[xshift=+0.5cm] S.north}{[xshift=+0.5cm] A.south}{Bond}{west};
	\paymentFlow{[xshift=+0.5cm] B.north}{[xshift=+0.5cm] S.south}{Bond}{west};
\end{tikzpicture}
\end{frame}



\begin{frame}{Speculating on credit risk - 3}
\justify
We have a view that creditworthiness of a reference entity will improve in future. We have two options: 
\begin{itemize}
\justifying
\item Buy the risky bond and sell short a risk-free government bond (to eliminate the general interest rate risk).
\item Sell a credit default swap.
\end{itemize}

\justify
If we sold a swap to company A at 3\%, and bought s similar swap from company B at 2\%, then we will be making 1\% per annum until a default happens or until maturity of the swaps.
\end{frame}



\begin{frame}{Factors affecting a CDS price}
\justify
A few factors affect fair price of a CDS.
\begin{itemize}
\justifying
\item Probability of default. It can be estimated from other CDS quotes or from market prices for this entity's bonds.
\item Recovery rate is residual value of a bond in case of default. Usually it is estimated from historical data for particular jurisdiction, economy sector, credit rating.
\item Discounting interest rate for future cash flows.
\end{itemize}
\end{frame}



\begin{frame}{Recovery rate}
\justify
Moody's Corporate Default and Recovery Rates, 1920--2010.

\vspace{\baselineskip}
\begin{tabular}{l|l|l|l}
Position						& 2010		& 2009		& 1982--2010 \\
\hline
1st	Lien Bank Loan	 		& 72.30\%	& 56.30\% & 59.60\% \\
2nd Lien Bank Loan 		& 18.40\%	& 20.80\% & 27.90\% \\
Sr. Unsecured	Bank Loan	& n.a.		& 37.90\% & 39.90\% \\
Sr. Secured Bond			& 54.70\%	& 29.60\% & 49.10\%	 \\
\hline
Sr. Unsecured	Bond		& 63.80\%	& 35.50\% & 37.40\%	 \\
\hline
Sr. Subordinated Bond	& 39.40\%	& 18.00\% & 25.30\%	\\
Subordinated Bond		& 32.20\%	& 25.10\% & 24.20\%	\\
Jr. Subordinated Bond	 	& n.a.		& n.a.		&17.10\%	\\
\end{tabular}

\justify
Rule of thumb: 40\%.
\end{frame}



\begin{frame}{Credit rating}
\justify
Why don't we check credit rating to estimate probability of default? This is S\&P\,500 scale (best to worst):
\begin{align*}
\text{Investment-grade}
\small
&\begin{cases}
\small \text{AAA} \\
\small \text{AA+, AA, AA-} \\
\small \text{A+, A, A-} \\
\small \text{BBB+, BBB, BBB-,}
\end{cases} \\
\text{Speculative-grade}
&\begin{cases}
\small \text{BB+, BB, BB-,} \\
\small \text{B+, B, B-,} \\
\small \text{CCC+, CCC, CCC-,} \\
\small \text{CC,C, D}
\end{cases}
\end{align*}
\end{frame}



\newcommand{\addDefaultRatePlot}[3] {

	\addplot[
		color = #2,
		mark = #3,
		thick
	]
	table[
		x = year,
		y = #1,
		col sep = comma
	]
	{sp_2020_global_corporate_default_rates.csv};
}

\begin{frame}{Credit-rating - 2}
\justify
Credit ratings reflect relative riskiness rather than absolute probability.

\justify
\centering
\begin{tikzpicture}
	\begin{axis}[
		width = \textwidth,
		height = \textheight - 1.25cm,
		ymin = 0,
		ymax = 13,
		xmin = 1981,
		xmax = 2021,
		grid = major,
		xticklabel = {\pgfmathprintnumber[precision=0, 1000 sep=]{\tick}},
		yticklabel = {\pgfmathprintnumber[precision=0]{\tick}\%},
		ylabel = {Default frequency},
		legend entries = {
  	   		\small Speculative-grade,
      		\small Investment-grade
  		},
  		legend cell align={left}
	]
		
		\addDefaultRatePlot{speculative_grade_default_rate}{Set1-A}{square}
		
		\addDefaultRatePlot{investment_grade_default_rate}{Set1-B}{*}
	\end{axis}
\end{tikzpicture}

\centering
\small Data: \en{S\&P Global}
\end{frame}



\begin{frame}{Bond price and probability of default}
\justify
A risk-free zero-coupon government bond that matures in 1 year is priced at $G=98\%$ of its face value. If you invest \$980 today you receive \$1\,000 in a year. Rate of return is $\$1\,000 / \$980 - 1 \approx 2.04\%$.

\justify
A corporate bond of the same maturity is priced at $C=95\%$ of its face value. If you invest\$950 today, you receive \$1\,000 in a year (in case there is no default). Rate of return is $\$1\,000 / \$950 - 1 \approx 5.3\%$.

\justify
Why do market participants value the corporate bond less, i.e. demand higher rate of return?

\justify
Suppose that the recovery rate in case of default is $R=40\%$. You invest \$950 today and receive \$400 in a year. Rate of return in case of default is $\$400 / \$950 - 1 = -57.9\%$.
\end{frame}



\begin{frame}{Bond price and probability of default - 2}
\justify
Government bond is worth $G=98\%$ of its face value, corporate bond is worth $C=95\%$, recovery rate in case of default is $R=40\%$.

\justify
Which default probability $p$ equalizes the two expected rates of return?

\justify
\centering
\begin{tabular}{l|c|c|c|c|c}
& & \multicolumn{2}{c|}{Default ($p$)} & \multicolumn{2}{c}{No default ($1-p$)} \\
Bond & Price & Payoff & Return & Payoff & Return \\
\hline
Govt.  & 0.98 & 1     & 1/0.98 - 1 & 1 & 1/0.98 - 1\\
Corp. & 0.95 & 0.4 & 0.4/0.95 - 1 & 1 & 1/0.95 - 1
\end{tabular}

\justify
Expected rates of return should be equal:
\begin{align*}
\frac{1}{0.98} - 1 &= p\left(\frac{0.4}{0.95} - 1\right) + (1-p)\left(\frac{1}{0.95} - 1\right) \Rightarrow \\
p &= \frac{1 - C/G}{1-R} = \frac{1 - 0.95/0.98}{1-0.4} \approx 5.1\%
\end{align*}
\end{frame}



\begin{frame}{Bond price and probability of default - 3}
\justify
We can re-write the formula in terms of bond yields $r_g$ и $r_c$. Suppose that two bonds mature in $T$ years.
\begin{align*}
G = \frac{1}{(1 + r_g)^T}, C = \frac{1}{(1 + r_c)^T}
\end{align*}
Then
\begin{align*}
p = \frac{1 - \dfrac{C}{G}}{1-R} = \frac{1 - \dfrac{(1 + r_g)^T}{(1 + r_c)^T}}{1 - R}
\end{align*}

\justify
(We assume that in case of default we will recover the residual value on maturity date)
\end{frame}



\begin{frame}{Probability of default and credit spread}
\justify
In case interest rates are not too high, we can simplify the formula:
\begin{equation*}
p = \frac{1 - \dfrac{(1 + r_g)^T}{(1 + r_c)^T}}{1 - R} \approx \frac{T(r_c - r_g)}{1 - R}
\end{equation*}

\justify
For example, a one-year bond of a large German Frankfurt-based investment bank yields 3.5\%. Similar German government bond yields $3.0\%$. Recovery rate in case of bankruptcy is 40\%.

\justify
What is probability that the bank will default during a year?
\begin{equation*}
p \approx \frac{r_{Bank} - r_{Germany}}{1-R} = \frac{3.5\% - 3.0\%}{1-40\%} = 0.83\%
\end{equation*}
\end{frame}



\begin{frame}{Interpolating the probability of default}
\centering
\begin{tabular}{l|r|r|r|r}
Term & Govt. Yield & Corp. Yield & Recovery & PD \\ \hline
1Y & 1.00\% & 2.23\% & 40\% & 2.00\% \\
2Y & 1.50\% & 3.06\% & 40\% & 5.00\%
\end{tabular}

\justify
What is probability of the default happening between now and $T^*=1.5$ years from now?

\justify
Suppose that $\xi$ is a random variable that denotes the time till default in years.

\justify
$D(T)$ is probability that the default will happen until time $T$. $S(T)$ is probability that the reference entity will survive till time $T$.
\begin{align*}
D(T) &= \mathbb{P}(\xi < T) \\
S(T) &= \mathbb{P}(\xi \ge T) = 1 - D(T)
\end{align*}
\end{frame}
 


\begin{frame}{Reminder: geometric distribution}
\justify
Suppose we are flipping a coin. Probability of Tails outcome is $p$ (usually it is close to 50\%). We continue flipping the coin until the first Tails outcome (a default). We are interested in a random variable $\xi$ which is the number or Heads until the first Tails (years without a default).

\justify
What is probability that that the first Tail will occur after the $n$-th trial (i.e. we get at least $n$ Heads in a row)?
\begin{align*}
\mathbb{P}(\xi \ge n) = S(n) = (1 - p)^{n}
\end{align*}
\justify
$\xi$ is said to follow geometric distribution.
\end{frame}



\begin{frame}{Reminder: geometric distribution - 2}
\centering
\begin{tikzpicture}
\begin{axis}[
    width = \textwidth,
    height = \textheight - 1cm,
    xmin = 0, xmax = 10,
    ymin = 0, ymax = 1,
    xtick distance=1,
    ytick distance=0.1,
    grid=major
]

\addplot[color=Set1-A, thick, mark=*, domain=0:10, samples=11]{(1-0.5)^x)};
\end{axis}
\end{tikzpicture}
\end{frame}



\begin{frame}{Reminder: exponential distribution}
\justify
Suppose that the default may happen at any time, not just once a year. Suppose that during any tiny time interval $dt$ a default may happen with probability of $\lambda \cdot dt$. $\lambda$ is called \alert{hazard rate} and it is usually expressed in percent per annum.

\justify
Probability that a reference entity survives till time $T$ is $S(T)$. What is probability that it will survive a bit longer?
\begin{align*}
S(T+dt) = S(T)(1 - \lambda dt)
\end{align*}

\justify
Rearranging the terms and solving a tiny differential equation:
\begin{align*}
\frac{S(T+dt) - S(T)}{dt} = -\lambda S(T) \quad
\Rightarrow
\quad
S(T) = e^{-\lambda T}
\end{align*}
\justify
This is an exponential distribution, a continuous version of discrete geometric distribution.
\end{frame}



\begin{frame}{Reminder: exponential distribution - 2}
\centering
\begin{tikzpicture}
\begin{axis}[
    width = \textwidth,
    height = \textheight - 1cm,
    xmin = 0, xmax = 10,
    ymin = 0, ymax = 1,
    xtick distance=1,
    ytick distance=0.1,
    grid=major
]

\addplot[color=Set1-A, thick, domain=0:10]{exp(-x*ln(2))};
\end{axis}
\end{tikzpicture}
\end{frame}



\begin{frame}{Interpolating the probability of default - 2}
\centering
\begin{tabular}{l|r|r|r|r}
Term & Govt. Yield & Corp. Yield & Recovery & PD \\ \hline
1Y & 1.00\% & 2.23\% & 40\% & 2.00\% \\
2Y & 1.50\% & 3.06\% & 40\% & 5.00\%
\end{tabular}

\justify
What is probability of the default happening between now and $T^*=1.5$ years from now?

\justify
Suppose that after surviving through $T_1=1$ year, the reference entity finds itself under hazard rate $\lambda$ in an exponential distribution.

\justify
Formally, for any $T^*$ between $T_1=1$ and $T_2=2$ years it holds that
\begin{align*}
S(T^*) = S(T_1)e^{-\lambda(T^* - T_1)}
\end{align*}
\end{frame}





\newcommand{\nodeWithDropLines}[2]{
    \node[
        circle,
        fill,
        color=Set1-A,
        inner sep=2pt
    ]
    at (axis cs: #1, #2)
    {};

    \draw[
        dashed,
        thick
    ]
    (axis cs: 0, #2) -- (axis cs: #1, #2) -- (axis cs: #1, 0);
}

\begin{frame}{Interpolating the probability of default - 3}
\centering
\begin{tikzpicture}
\begin{axis}[
    width = \textwidth,
    height = \textheight - 1cm,
    xmin = 0, xmax = 3,
    ymin = 0, ymax = 1.2,
    xtick = {1, 2},
    xticklabels = {$T_1$, $T_2$},
    xtick pos = left,
    ytick = {0.404, 0.9, 1},
    yticklabels = {$S(T_2)$, $S(T_1)$, 1},
    ytick pos = left,
    axis lines = middle
]

\addplot[color=Set1-A, thick, domain=0:1]{exp(ln(0.9)*x)};
\addplot[color=Set1-A, thick, domain=1:2]{0.9 * exp(-0.8 * (x - 1))};
\addplot[color=Set1-A, thick, domain=2:3]{0.9*exp(-0.8)*exp(-2*(x-2))};

\nodeWithDropLines{1}{0.9}
\nodeWithDropLines{2}{0.404}

\node[anchor=west] at (axis cs: 1.5, 0.65) {$S(T_1)e^{-\lambda (T - T_1)}$};

\end{axis}
\end{tikzpicture}
\end{frame}




\begin{frame}{Interpolating the probability of default - 4}
\justify
We can derive the value of $\lambda$ from $S(T_1)$ and $S(T_2)$.
\begin{align*}
S(T_2) = S(T_1)e^{-\lambda(T_2-T_1)}
\end{align*}
Consequently
\begin{align*}
\lambda = \dfrac{\ln\left(\dfrac{S(T_1)}{S(T_2)}\right)}{T_2 - T_1} = \dfrac{\ln\left(\dfrac{1 - D(T_1)}{1 - D(T_2)}\right)}{T_2 - T_1}
\end{align*}
In our particular case, $\lambda \approx 3.11\%$. So we can interpolate the value of $D(1.5)$:
\begin{align*}
D(1.5) = 1 - S(1.5) = 1 - (1 - D(1))e^{-3.11\% \cdot (1.5 - 1)} \approx 3.51\%
\end{align*}
\end{frame}



\begin{frame}{Pricing a credit default swap}
\justify
Suppose that we know the probability of default, recovery rate, and discount factors for future cashflows. How to we compute fair coupon in a credit default swap?

\justify
Expected discounted cashflows of the buyer and the seller should match.

\justify
We compute two values for each future date.
\begin{itemize}
\item Probability of the reference entity surviving till this date, $S(T)$.
\item Probability of the reference entity defaulting precisely on this day, $D(T) - D(T - 1 \text{ day})$.
\end{itemize}
\end{frame}



\begin{frame}{Pricing a credit default swap - 2}
\justify
With probability $S(T)$, the buyer pays a coupon in case $T$  is one of the coupon dates.
\begin{center}
\begin{tikzpicture}[thick, scale=0.6]
		\draw (0, 0) node[rectangle,draw,rounded corners,anchor=south,minimum height=1cm] {Swap Buyer} -- (0, -1.5);
		\draw (7.5, 0) node[rectangle,draw,rounded corners,anchor=south,minimum height=1cm] {Swap Seller} -- (7.5, -1.5);
		\draw [->,>=triangle 90] (0, -1.0) node[label=left:{T}]{} -- (3.5, -1.0) node[pos=0.5,anchor=south]{\$50\,000};
\end{tikzpicture}
\end{center}

\justify
With probability $D(T) - D(T - 1 \text{day})$, the buyer pays the accrued interest and delivers the defaulted bond, the seller pays out the swap notional amount.
\begin{center}
\begin{tikzpicture}[thick, scale=0.6]
		\draw (0, 0) node[rectangle,draw,rounded corners,anchor=south,minimum height=1cm] {Swap Buyer} -- (0, -1.5);
		\draw (7.5, 0) node[rectangle,draw,rounded corners,anchor=south,minimum height=1cm] {Swap Seller} -- (7.5, -1.5);
		\draw [->,>=triangle 90] (0, -1.0) node[label=left:{T}]{} -- (3.5, -1.0) node[pos=0.5,anchor=south]{\$33\,333} node[pos=0.5,anchor=north] {Bond};
		\draw [->,>=triangle 90] (7.5, -1.0) -- (4.5, -1.0) node[pos=0.5,anchor=south]{\$10\,000\,000};
\end{tikzpicture}
\end{center}
\end{frame}



\begin{frame}{Sample problem}
\justify
Find fair price (coupon) for a credit default swap under following assumptions.

\begin{itemize}
\justifying
\item Term is 1 year.
\item Quarterly coupons (every 1/4th of a year).
\item Recovery rate is 40\%.
\item Probability of default is given by a exponential distribution with hazard rate of $\lambda = 2\%$.
\item In case of a default, the notional amount will be paid on the following coupon date. The swap is cash-settled.
\item Neglect the discounting.
\end{itemize}
\end{frame}



\begin{frame}{Solution}
\justify
Let $x$ be an annual coupon rate. Swap notional is $\$1$, recovery rate is $R$. Let $S(t)$ be probability of the reference entity surviving for $t$ years. Write down all payments of the seller and the buyer, along with probabilities of these payments.

\begin{center}
\begin{tikzpicture}[thick, scale=0.7]
		\draw (0, 0) node[rectangle,draw,rounded corners,anchor=south,minimum height=1cm] {Swap Buyer} -- (0, -5.5);
		\draw (10.5, 0) node[rectangle,draw,rounded corners,anchor=south,minimum height=1cm] {Swap Seller} -- (10.5, -5.5);
		\draw [->,>=triangle 90] (0, -1.0) node[label=left:{$\frac{1}{4}$ years}]{} -- (5.0, -1.0) node[pos=0.5,anchor=south]{$\frac{x}{4}; S(\frac{1}{4})$};
		\draw [->,>=triangle 90] (0, -2.5) node[label=left:{$\frac{1}{2}$ years}]{} -- (5.0, -2.5) node[pos=0.5,anchor=south]{$\frac{x}{4}; S(\frac{1}{2})$};
		\draw [->,>=triangle 90] (0, -4.0) node[label=left:{$\frac{3}{4}$ years}]{} -- (5.0, -4.0) node[pos=0.5,anchor=south]{$\frac{x}{4}; S(\frac{3}{4})$};
		\draw [->,>=triangle 90] (0, -5.5) node[label=left:{$1$ year}]{} -- (5.0, -5.5) node[pos=0.5,anchor=south]{$\frac{x}{4}; S(1)$};
    
    \draw [->,>=triangle 90] (10.5, -1.0) -- (5.5, -1.0) node[pos=0.5,anchor=south]{$1-R; 1 - S(\frac{1}{4})$};
    \draw [->,>=triangle 90] (10.5, -2.5) -- (5.5, -2.5) node[pos=0.5,anchor=south]{$1-R; S(\frac{1}{4}) - S(\frac{1}{2})$};
    \draw [->,>=triangle 90] (10.5, -4.0) -- (5.5, -4.0) node[pos=0.5,anchor=south]{$1-R; S(\frac{1}{2}) - S(\frac{3}{4})$};
    \draw [->,>=triangle 90] (10.5, -5.5) -- (5.5, -5.5) node[pos=0.5,anchor=south]{$1-R; S(\frac{3}{4}) - S(1)$};

\end{tikzpicture}
\end{center}
\end{frame}



\begin{frame}{Solution - 2}
\justify
Probability of the reference entity surviving during 1/4 of a year is $S(1/4) = 98\%$. Probability of the reference entity surviving during 1/2 of a year is $S(1/2) = 95\%$. 

\justify
What is probability that the reference entity will default at any time between 1/4 and 1/2 (during the second quarter)? 

\justify
Consider a sample of 100\,000 entities that are alive today.

\justify
98\,000 (98\% of the initial population) of these will survive during the first quarter. 

\justify
95\,000 (95\% of the initial population) will survive for half a year.

\justify
3\,000 will default during the second quarter. This is $98\% - 95\% = 3\%$ of the initial population. An entity that is alive today has a probability of 3\% of defaulting precisely during the second quarter (not earlier, not later)
\end{frame}



\begin{frame}{Solution - 3}
\justify
Expected value of the buyer's payments should be equal to expected value of the seller's payments

\begin{equation*}
\frac{x}{4} \cdot \left( S\left(\frac{1}{4}\right) + S\left(\frac{1}{2}\right) + S\left(\frac{3}{4}\right) +S\left(1\right) \right) = (1 - R)(1 - S(1))
\end{equation*}

According to the problem statement, $S(t)=e^{-\lambda t}$, so

\begin{equation*}
x = \frac{4(1 - R)(1 - e^{-\lambda})}{e^{-\lambda / 4} + e^{-\lambda / 2} + e^{-3\lambda / 4} + e^{-\lambda}}
\end{equation*}

Substitute $R = 0.4$ and $\lambda = 0.02$:

\begin{equation*}
\frac{4(1 - 0.4)(1 - e^{-0.02})}{e^{-0.02 / 4} + e^{-0.02 / 2} + e^{-3\cdot0.02 / 4} + e^{-0.02}} \approx 1.203\%
\end{equation*}
\end{frame}



\begin{frame}{Solution (a shortcut)}
\justify
Probability that the reference entity will survive for 1 year is $S(1) = e^{-\lambda \cdot 1} \approx 1 - \lambda = 0.98$.

\justify
Probability of the default during 1 year is $\approx 2\%$. If the default happens you lose $1 - R = 0.6$ or $60\%$.

\justify
How much would you pay for insurance, if you know that the credit event will happen with probability of 2\% and will cost you 60\% of the invested capital? Certainly, $2\% \cdot 60\% = 1.2\%$! 
\end{frame}



\begin{frame}{Demo}
\end{frame}



\begin{frame}{Risk-neutral and real-world probabilities}
\justify
How accurate are market estimates of default probability?

\justify
Not quite. Usually the market overestimates the probability of default. On average, companies go bankrupt less frequently than market quotes imply.

\justify
All investors are \alert{risk-neutral} in our model. Risk-neutral investors care only about the expected rate of return and do not care about the risk (variance).

\justify
A risk-neutral investor is indifferent between gaining \$100 for sure, or having a 50/50 chance to gain \$0 or \$200.

\justify
Are Homo Sapiens risk-neutral? No, they are not!
\end{frame}



\begin{frame}{Risk-neutrality and insurance}
\justify
New BMW X5 costs \EUR{100\,000}. Comprehensive insurance for a year costs \EUR{4\,000}. Probability that your lecturer will crash the car within a year is 2\%. After an incident the car is worth zero. Should the lecturer insure the car?

\justify
Expected value with insurance: $-\EUR{4\,000} \cdot 100\% = -\EUR{4\,000}$.

\justify
Expected value without insurance: $-\EUR{100\,000} \cdot 2\% = -\EUR{2\,000}$.

Additional factor: a serious conversation with the wife.

\justify
The lecturer will prefer to be risk-averse and to buy the insurance. Hypothetical expected savings can never justify a difficult conversation with the wife.
\end{frame}



\begin{frame}{Риск-нейтральная и реальная вероятности}
\justify
Люди, принимающие инвестиционные решения, избегают риска (являются risk-averse), и это влияет на рынок облигаций и CDS.

\justify
Покупателей рискованных облигаций не устраивает математическое ожидание доходности, равное доходности безрисковых бумаг. Они требуют премию за риск (risk premium), которая компенсирует дискомфорт от рискованной инвестиции. Выше доходность --- ниже цена облигаций.

\justify
Покупатели CDS боятся потерь от дефолта, поэтому они готовы платить бОльшую премию за страховку, чем сферические риск-нейтральные инвесторы. Больше спрос --- выше цена CDS.
\end{frame}



\begin{frame}{Премия за кредитный риск - 1}
\justify
Премия за кредитный риск (\en{credit premium}) компенсирует инвесторам душевные страдания от потерь при дефолтах по облигациям. Рискованные облигации должны
давать доходность, которая не только компенсирует мат. ожидание потерь от
дефолтов, но и даёт премию за риск.

\justify
\centering
\begin{tabular}{r|r|r|r|r}
AAA    & AA     & A      & BBB    & CCC \\ \hline
0.32\% & 0.42\% & 0.43\% & 1.04\% & 0.86\%  
\end{tabular}

\justify
\centering
{\scriptsize Избыточная доходность корпоративных облигаций сверх гос. 
облигаций, 1987--2011. Данные: \en{Ang (2014)}.}

\justify
Следствие. Если вам предлагают еврооблигации ООО <<Рога и копыта>> с 
доходностью 10\% в долларах, то помните, что лишь 1\% --- это премия за риск,
которую вы заработаете в мат. ожидании, а 9\% компенсируют ожидаемые
потери при дефолте.
\end{frame}



\begin{frame}{Премия за кредитный риск - 2}
\centering
\begin{tikzpicture}
	\begin{axis}[
		width = \textwidth,
		height = \textheight - 1cm,
		date coordinates in = x,
		xticklabel = {\year},
		xtick = {1990-01-01, 1995-01-01, 2000-01-01, 2005-01-01, 2010-01-01, 2015-01-01, 2020-01-01},
		ylabel = {\small Рост \$1 начальных инвестиций},
		ylabel near ticks,
		ymode = log,
		log ticks with fixed point,
  		extra y ticks = {0.5, 2, 3, 4, 5},
 		extra y tick labels = {0.5, 2, 3, 4, 5},
		xmin = 1986-10-31,
		xmax = 2023-12-31,
		ymin = 0.9,
		ymax = 16,
		grid = both,
		legend entries = {
			BofA High Yield Index,
			BofA Corporate Index
		},
		legend pos=north west,
      legend style={font=\small},
      legend cell align={left}
	]
		\addplot[color=Set1-A, thick, mark=*, mark phase=9, mark repeat=30] table[x=month, y=high_yield_growth, col sep=comma] {bofa_bond_indices.csv};
		
		\addplot[color=Set1-B, thick, mark=square, mark phase=9, mark repeat=30] table[x=month, y=corp_growth, col sep=comma] {bofa_bond_indices.csv};
	\end{axis}
\end{tikzpicture}

\centering
\small Данные: Bank of America, St Louis Fed.
\end{frame}



\begin{frame}{Риск-нейтральная и реальная вероятности}
\justify
Кроме того, мы не знаем, какую recovery rate участники рынка закладывают в цены. Не существует ликвидных деривативов (каких-нибудь recovery rate swap), которые позволяли бы разделить recovery rate и вероятность дефолта.

\justify
Означает ли это, что нельзя использовать наши вероятности для оценки CDS? Нет!

\justify
Во-первых, если у нас есть ликвидные рыночные инструменты, мы всегда можем захеджироваться и устранить неопределённость относительно вероятности дефолта.

\justify
Во-вторых, в формулу цены CDS входит только произведение (1-RecoveryRate) и вероятности, и нам почти никогда не нужно знать их по отдельности.
\end{frame}



\begin{frame}{Оценка деривативов через репликацию}
\justify
Вероятности дефолта и recovery rate нужны, чтобы подобрать правильную комбинацию 
ингредиентов, которые повторяют кредитный дефолтный своп:

\justify
\begin{itemize}
\justifying
\item Дать кому-то деньги в долг под залог корпоративной облигации.
\item Продать облигацию на рынке.
\item Купить безрисковую облигацию или заключить asset swap.
\end{itemize}

\justify
Если модель помогает подобрать реплицирующий портфель, то не так важно, что она исходит из неверной оценки вероятности дефолта.
\end{frame}



\begin{frame}{Риск-нейтральная вероятность и букмекеры}
\justify
Котировки букмекеров на матч <<Сити>> --- <<Бавария>> завтра:

\centering
\begin{tabular}{c|c|c}
<<Сити>> & Ничья & <<Бавария>> \\
1.88 & 4.22 & 4.32
\end{tabular}

\justify
Если коэффициент равен $k$, то $1/k$ --- <<вероятность>> события.

\centering
\begin{tabular}{c|c|c}
<<Сити>> & Ничья & <<Бавария>> \\
53.1\% & 23.7\% & 23.2\%
\end{tabular}

\justify
Допустим, всего азартные болельщики поставили \$100 миллионов, из них 23.7 --- на ничью. При 
ничейном исходе победители заберут $\$23.7 \cdot 4.22 = \$100$ миллионов, то 
есть весь банк.

\justify
Нужно ли букмекеру стремиться к тому, чтобы коэффициенты отражали истинную вероятность реального мира? Нет! Ему важно, чтобы при любом исходе ставок проигравших хватило на выплаты победителям. Больше ставок на исход --- ниже коэффициент.
\end{frame}



\begin{frame}{Реформа рынка в 2009 г.}
\justify
В начале 2009 года регуляторы провели реформу рынка кредитных деривативов, чтобы избежать повторения кризиса 2008 года.

\begin{itemize}
\justifying
\item Обязательный централизованный клиринг.
\item 4 стандартных даты выплаты купонов: 20 марта, 20 июня, 20 сентября, 20 декабря.
\item Фиксированные купоны либо по 1\% (облигации с высоким рейтингом), либо по 5\% (мусорные облигации).
\end{itemize}

\justify
Все по-прежнему котируют CDS в терминах спреда (например, 2.40\%/2.41\%), но реальные купоны всё равно будут либо 1\%, либо 5\%. Одна из сторон сделки выплачивает другой сумму, рассчитанную по стандартной методологии, чтобы компенсировать разницу в купоне.
\end{frame}



\begin{frame}{Базис CDS-Bond}
\justify
На идеальном сферическом рынке в вакууме не должно быть разницы между двумя стратегиями
\begin{itemize}
\justifying
\item Купить рискованную облигацию с доходностью 5\% и CDS с купоном 3\%.
\item Купить безрисковую облигацию с доходностью 2\%.
\end{itemize}

\justify
На практике наблюдается ненулевой CDS-Bond basis:
\begin{align*}
\text{CDS Basis} = \text{CDS} - \text{Z-spread}
\end{align*}

\justify
Некоторые факторы, которые влияют на базис:
\begin{itemize}
\justifying
\item Ликвидность.
\item Несовпадение купонов.
\item Накопленный купон на момент дефолта.
\item Встроенные в облигацию опционы.
\end{itemize}
\end{frame}



\begin{frame}{Wrong Way CDS}
\justify
Что бы Вы сказали о следующих сделках?
\begin{itemize}
\item Купить CDS на Германию у Deutsche Bank.
\item Купить CDS на Deutsche Bank у сегодняшнего лектора.
\end{itemize}

\justify
Страховка надёжна ровно настолько, насколько надёжен её продавец, который тоже может разориться. Важно, чтобы дефолт референсного эмитента не имел корреляции с дефолтом продавца страховки.

\justify
Загадка: кто и у кого покупает CDS на гос. долг США?
\end{frame}



\begin{frame}{Если всего этого мало}
\justify
Индексный кредитный своп:
\begin{itemize}
\item 100 компаний-заемщиков.
\item Начальный номинал \$10\,000\,000.
\item При дефолте каждой компании номинал (и купон) уменьшается на \$100\,000.
\end{itemize}

Nth-to-default CDS:
\begin{itemize}
\item 100 компаний-заёмщиков
\item Начальный номинал \$10\,000\,000.
\item После первых $N$ дефолтов выплачивается полный номинал.
\end{itemize}
Оценка зависит не только от вероятностей дефолта, но и от корреляций между дефолтами. Подробности в учебнике Hull.
\end{frame}



\begin{frame}{Зачем?}
\justify
Кредитные деривативы --- это не только огромный рынок, но и важный компонент цены других деривативов:
\begin{itemize}
\item Credit valuation adjustment (CVA)
\item Debt valuation adjustment (DVA)
\item Capital valuation adjustment (KVA)
\end{itemize}
\end{frame}


\insertdisclaimerframe

\end{document}


