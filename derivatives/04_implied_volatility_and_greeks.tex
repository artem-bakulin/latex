\documentclass{beamer}

\usepackage{cmap}				% To be able to copy-paste russian text from pdf
\usepackage[T2A]{fontenc}
\usepackage[utf8]{inputenc}
\usepackage[russian]{babel}
\usepackage{textpos}
\usepackage{ragged2e}
\usepackage{amssymb}
\usepackage{ulem}
\usepackage{tikz}
\usepackage{pgfplots}
\usepackage{color}
\usepackage{cancel}
\usepackage{multirow}
\pgfplotsset{compat=1.17}
\usetikzlibrary{arrows,snakes,backgrounds,shapes}
\usepgfplotslibrary{groupplots,colorbrewer,dateplot,statistics}
\usepackage{animate}

\usepackage{amsfonts}
\usepackage{amsmath}
\usepackage{amssymb}
\usepackage{graphicx}
\usepackage{setspace}

\usepackage{enumitem}
\setitemize{label=\usebeamerfont*{itemize item}%
  \usebeamercolor[fg]{itemize item}
  \usebeamertemplate{itemize item}}

% remove navigation bar
\setbeamertemplate{navigation symbols}{} 

\usepackage{eurosym}
\renewcommand{\EUR}[1]{\textup{\euro}#1}

\title{Лекция 4. Улыбка волатильности. Греки}
\author{Артём Бакулин}
\date{28 октября 2021 г.}

\usetheme{Warsaw}
\usecolortheme{beaver}

\newcommand{\ru}[1]{\begin{otherlanguage}{russian}#1\end{otherlanguage}}
\newcommand{\en}[1]{\begin{otherlanguage}{english}#1\end{otherlanguage}}
\newcommand{\ruen}[2]{#1 (\en{#2})}

\begin{document}



\begin{frame}
\titlepage
\end{frame}



\begin{frame}{Напоминание: модель Блэка-Шоулза}
\justify
В модели Блэка-Шоулза процентное изменение цены $dS/S$ за каждый малый период $dt$ имеет нормальное распределение с трендом $\mu$ и волатильностью $\sigma$:
\begin{align*}
\frac{dS}{S} = \mu dt + \sigma\varepsilon\sqrt{dt}, \quad \varepsilon \sim \mathcal{N}(0, 1)
\end{align*}

\justify
Если сложить все малые колебания за интервал длиной $T$, то цена базового актива имеет лог-нормальное распределение:
\begin{align*}
S(T) &= S_0 \exp\left[\left(r - \dfrac{\sigma^2}{2}\right)T + \sigma\varepsilon\sqrt{T}\right], \quad \varepsilon \sim \mathcal{N}(0, 1) \\
\ln\left(S(T)\right) &= \ln\left(S_0\right) + \left(r - \dfrac{\sigma^2}{2}\right)T + \sigma\varepsilon\sqrt{T}
\end{align*}
\end{frame}



\begin{frame}{Напоминание: модель Блэка-Шоулза}
\justify
Цена колл-опциона со страйком $K$ и сроком погашения $T$ зависит от текущей цены базового актива $S_0$, его волатильности $\sigma$ и безрисковой процентной ставки $r$:
\begin{align*}
C &= S_0N(d_1) - Ke^{-rT}N(d_2)
\end{align*}
где
\begin{align*}
d_1 &= \dfrac{1}{\sigma\sqrt{T}}\left( \ln\left(\dfrac{S_0}{K}\right) + \left(r + \dfrac{\sigma^2}{2}\right)T\right) \\
d_2 &= \dfrac{1}{\sigma\sqrt{T}}\left( \ln\left(\dfrac{S_0}{K}\right) + \left(r - \dfrac{\sigma^2}{2}\right)T\right) \\
N(x) &= \dfrac{1}{\sqrt{2\pi}}\int\limits_{-\infty}^x e^{-\dfrac{t^2}{2}}dt
\end{align*}
\end{frame}



\begin{frame}{Историческая волатильность}
\justify
Если мы живём в мире Блэка-Шоулза, то нам достаточно знать волатильность базового актива, чтобы посчитать справедливую цену любого опциона. Почему бы не посмотреть на историю?

\justify
Рассмотрим временной ряд исторических цен базового актива $S_0,S_1,...,S_n$, взятых с фиксированным шагом $\Delta t$. Например, если у нас есть дневные данные, то $\Delta t = 1/250$ (по количеству рабочих дней в году). 

\justify
Посчитаем логарифмические доходности (\en{log returns}), которые близки к простым процентным доходностям:
\begin{align*}
r_i &= \ln \left( \dfrac{S_{i}}{S_{i-1}} \right) \\
\ln \left(\dfrac{S_{i}}{S_{i-1}} \right) &= \ln \left[1 + \left(\dfrac{S_{i}}{S_{i-1}} - 1 \right) \right] \approx \dfrac{S_{i}}{S_{i-1}} - 1
\end{align*}
\end{frame}



\begin{frame}{Историческая волатильность}
\justify
В модели Блэка-Шоулза все лог-доходности $r_i$ --- н.о.р.с.в. Их выборочное стандартное отклонение --- историческая (\en{historical}) или реализованная (\en{realized}) волатильность.

\begin{align*}
\hat{r} &= \frac{1}{n}\sum\limits_{i=1}^{n}r_i \\
\hat{\sigma} &= \sqrt{\frac{1}{n-1}\sum\limits_{i=1}^{n}(r_i - \hat{r})^2}
\end{align*}

\justify
Чтобы перевести <<дневную>> волатильность в проценты годовых, нужно разделить её на $\sqrt{\Delta t}$.
\end{frame}



\begin{frame}{Историческая волатильность}
\centering
\begin{tikzpicture}
\begin{axis}[
  width=\textwidth,
  height=\textheight - 1cm,
  date coordinates in=x,
  date ZERO=2012-01-01,
  xtick={2012-01-01, 2014-01-01, 2016-01-01, 2018-01-01, 2020-01-01, 2022-01-01},
  minor xtick={2013-01-01, 2015-01-01, 2017-01-01, 2019-01-01, 2021-01-01},
  ytick={30, 40, ..., 90},
%  minor ytick={-0.75, -0.25, 0.25, 0.75, 1.25},
  xticklabel={\year},
  xmin=2012-01-01,
  xmax=2022-01-01,
  ymin=20,
  ymax=90,
  grid=both,
  %yticklabel={\pgfmathprintnumber{\tick}\%},
  ylabel={\small{Курс USDRUB}},
  xlabel near ticks,
  ylabel near ticks
]

\addplot[color = Set1-B, mark = none, very thick]
	table[
		x=date,
		y=fx_rate,
		col sep=comma
	]
	{cbr.USD.2012.2021.csv};

\end{axis}
\end{tikzpicture}

\scriptsize Данные: ЦБ РФ.
\end{frame}



\begin{frame}{Историческая волатильность}
\centering
\begin{tikzpicture}
\begin{axis}[
  width=\textwidth,
  height=\textheight - 1cm,
  date coordinates in=x,
  date ZERO=2012-01-01,
  xtick={2012-01-01, 2014-01-01, 2016-01-01, 2018-01-01, 2020-01-01, 2022-01-01},
  minor xtick={2013-01-01, 2015-01-01, 2017-01-01, 2019-01-01, 2021-01-01},
  ytick={0.1, 0.2, ..., 1},
%  minor ytick={-0.75, -0.25, 0.25, 0.75, 1.25},
  xticklabel={\year},
  xmin=2012-01-01,
  xmax=2022-01-01,
  ymin=0,
  ymax=0.9,
  grid=both,
  yticklabel={\pgfmathparse{\tick*100}\pgfmathprintnumber[precision=0]{\pgfmathresult}\%},
  ylabel={\small{Реализованная волатильность}},
  xlabel near ticks,
  ylabel near ticks
]

\addplot[color = Set1-B, mark = none, thick]
	table[
		x=mid_month,
		y=realized_vol,
		col sep=comma
	]
	{USDRUB_realized_vol.csv};

\end{axis}
\end{tikzpicture}

\scriptsize Данные: ЦБ РФ.
\end{frame}



\begin{frame}{Историческая волатильность}
\justify
Историческая волатильность --- не константа, что противоречит модели Блэка-Шоулза. В прошлом мы видели большой разброс в реализованной волатильности. Сложно выбрать репрезентативный период, который хорошо подходил бы тому опциону (скажем, на 3 месяца), который мы пытаемся оценить.

\justify
Прошлое не предсказывает будущее!
\end{frame}



\begin{frame}{Ожидаемая волатильность}
\justify
В модели Блэка-Шоулза цена опциона зависит от волатильности:
\begin{align*}
C_K = F(S_0, T, K, \sigma, r)
\end{align*}

\justify
Мы можем посмотреть на рыночные цены опционов и решить задачу в обратном направлении. 

\justify
Если участники рынка пользуются моделью Блэка-Шоулза, то какую волатильность они подставляют в формулу, чтобы получить те премии, которые мы наблюдаем?
\begin{align*}
\sigma = F^{-1}(S_0, T, K, r, C_K)
\end{align*}

\justify
Решение этой обратной задачи --- ожидаемая (\en{implied}) волатильность.
\end{frame}



\begin{frame}{Ожидаемая волатильность}
\justify
Опционы на USDRUB с экспирацией 16.12.2021.

\centering
\begin{tikzpicture}
\begin{axis}[
			width = \textwidth,
			height = \textheight - 2cm,
			domain=60:80,
			xtick={60,62,...,80},
			ytick={1,2,...,12},
			xmin=60, xmax=80,
			ymin=0, ymax=12,
			grid = major,
			xlabel={Страйк ($K$)},
			ylabel={Цена опциона, руб.}
]

	\addplot[color = Set1-A, mark = none, thick]
	table[
		x=strike,
		y=call,
		col sep=comma
	]
	{usdrub_implied_vol.csv};
	
	\addplot[color = Set1-B, mark = none, thick]
	table[
		x=strike,
		y=put,
		col sep=comma
	]
	{usdrub_implied_vol.csv};
\end{axis}
\end{tikzpicture}

\scriptsize Данные: Московская биржа.
\end{frame}



\begin{frame}{Ожидаемая волатильность}
\justify
Опционы на USDRUB с экспирацией 16.12.2021.

\centering
\begin{tikzpicture}
\begin{axis}[
			width = \textwidth,
			height = \textheight - 2cm,
			domain=60:80,
			xtick={60,62,...,80},
			ytick={0.02,0.04,...,0.20},
			xmin=60, xmax=80,
			ymin=0, ymax=0.2,
			yticklabel={\pgfmathparse{\tick*100}\pgfmathprintnumber[precision=0]{\pgfmathresult}\%},
			grid = major,
			xlabel={Страйк ($K$)},
			ylabel={Волатильность ($\sigma$)}
]

	\addplot[color = Set1-B, mark = none, thick]
	table[
		x=strike,
		y=iv,
		col sep=comma
	]
	{usdrub_implied_vol.csv};
\end{axis}
\end{tikzpicture}

\scriptsize Данные: Московская биржа.
\end{frame}



\begin{frame}{Ожидаемая волатильность}
\justify
Опционы на S\&P\,500 с экспирацией 17.12.2021.

\centering
\begin{tikzpicture}
\begin{axis}[
			width = \textwidth,
			height = \textheight - 2cm,
			domain=4000:5000,
			xtick={4000,4100,...,5000},
			ytick={50,100,...,600},
			xmin=4000, xmax=5000,
			ymin=0, ymax=600,
			grid = major,
			xlabel={Страйк ($K$)},
			ylabel={Цена опциона, \$},
			xticklabel={\pgfmathprintnumber[precision=0, 1000 sep={}]{\tick}}
]

	\addplot[color = Set1-A, mark = none, thick]
	table[
		x=strike,
		y=call,
		col sep=comma
	]
	{sp500_implied_vol.csv};
	
	\addplot[color = Set1-B, mark = none, thick]
	table[
		x=strike,
		y=put,
		col sep=comma
	]
	{sp500_implied_vol.csv};
\end{axis}
\end{tikzpicture}

\scriptsize Данные: barchart.com.
\end{frame}



\begin{frame}{Ожидаемая волатильность}
\justify
Опционы на S\&P\,500 с экспирацией 17.12.2021.

\centering
\begin{tikzpicture}
\begin{axis}[
			width = \textwidth,
			height = \textheight - 2cm,
			domain=4000:5000,
			xtick={4000,4100,...,5000},
			ytick={0.05,0.10,...,0.25},
			xmin=4000, xmax=5000,
			ymin=0, ymax=0.25,
			grid = major,
			xlabel={Страйк ($K$)},
			ylabel={Волатильность ($\sigma$)},
			xticklabel={\pgfmathprintnumber[precision=0, 1000 sep={}]{\tick}},
			yticklabel={\pgfmathparse{\tick*100}\pgfmathprintnumber[precision=0]{\pgfmathresult}\%},
]

	\addplot[color = Set1-B, mark = none, thick]
	table[
		x=strike,
		y=iv,
		col sep=comma
	]
	{sp500_implied_vol.csv};
\end{axis}
\end{tikzpicture}

\scriptsize Данные: barchart.com.
\end{frame}

\end{document}