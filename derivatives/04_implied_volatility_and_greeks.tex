\documentclass{beamer}

\usepackage{cmap}				% To be able to copy-paste russian text from pdf
\usepackage[T2A]{fontenc}
\usepackage[utf8]{inputenc}
\usepackage[russian]{babel}
\usepackage{textpos}
\usepackage{ragged2e}
\usepackage{amssymb}
\usepackage{ulem}
\usepackage{tikz}
\usepackage{pgfplots}
\usepackage{color}
\usepackage{cancel}
\usepackage{multirow}
\pgfplotsset{compat=1.17}
\usetikzlibrary{arrows,snakes,backgrounds,shapes}
\usepgfplotslibrary{groupplots,colorbrewer,dateplot,statistics}
\usepackage{animate}

\usepackage{amsfonts}
\usepackage{amsmath}
\usepackage{amssymb}
\usepackage{graphicx}
\usepackage{setspace}

\usepackage{enumitem}
\setitemize{label=\usebeamerfont*{itemize item}%
  \usebeamercolor[fg]{itemize item}
  \usebeamertemplate{itemize item}}

% remove navigation bar
\setbeamertemplate{navigation symbols}{} 

\usepackage{eurosym}
\renewcommand{\EUR}[1]{\textup{\euro}#1}

\title{Лекция 4. Улыбка волатильности. Греки}
\author{Артём Бакулин}
\date{28 октября 2021 г.}

\usetheme{Warsaw}
\usecolortheme{beaver}

\newcommand{\ru}[1]{\begin{otherlanguage}{russian}#1\end{otherlanguage}}
\newcommand{\en}[1]{\begin{otherlanguage}{english}#1\end{otherlanguage}}
\newcommand{\ruen}[2]{#1 (\en{#2})}

\begin{document}



\begin{frame}
\titlepage
\end{frame}



\begin{frame}{Напоминание: модель Блэка-Шоулза}
\justify
В модели Блэка-Шоулза процентное изменение цены $dS_t/S_t$ за каждый малый период $dt$ имеет нормальное распределение с трендом $\mu$ и волатильностью $\sigma$:
\begin{align*}
\frac{dS_t}{S_t} = \mu dt + \sigma\varepsilon\sqrt{dt}, \quad \varepsilon \sim \mathcal{N}(0, 1)
\end{align*}

\justify
Если сложить все малые колебания за интервал длиной $T$, то цена базового актива имеет логнормальное распределение:
\begin{align*}
S(T) &= S_0 \exp\left[\left(r - \dfrac{\sigma^2}{2}\right)T + \sigma\varepsilon\sqrt{T}\right], \quad \varepsilon \sim \mathcal{N}(0, 1) \\
\ln\left(S(T)\right) &= \ln\left(S_0\right) + \left(r - \dfrac{\sigma^2}{2}\right)T + \sigma\varepsilon\sqrt{T}
\end{align*}
\end{frame}



\begin{frame}{Напоминание: модель Блэка-Шоулза}
\justify
Цена колл-опциона со страйком $K$ и сроком погашения $T$ зависит от текущей цены базового актива $S_0$, его волатильности $\sigma$ и безрисковой процентной ставки $r$:
\begin{align*}
C &= S_0N(d_1) - Ke^{-rT}N(d_2)
\end{align*}
где
\begin{align*}
d_1 &= \dfrac{1}{\sigma\sqrt{T}}\left( \ln\left(\dfrac{S_0}{K}\right) + \left(r + \dfrac{\sigma^2}{2}\right)T\right) \\
d_2 &= \dfrac{1}{\sigma\sqrt{T}}\left( \ln\left(\dfrac{S_0}{K}\right) + \left(r - \dfrac{\sigma^2}{2}\right)T\right) \\
N(x) &= \dfrac{1}{\sqrt{2\pi}}\int\limits_{-\infty}^x e^{-\dfrac{t^2}{2}}dt
\end{align*}
\end{frame}



\begin{frame}{Историческая волатильность}
\justify
Если мы живём в мире Блэка-Шоулза, то нам достаточно знать волатильность базового актива, чтобы посчитать справедливую цену любого опциона. Почему бы не посмотреть на историю?

\justify
Рассмотрим временной ряд исторических цен базового актива $S_0,S_1,...,S_n$, взятых с фиксированным шагом $\Delta t$. Например, если у нас есть дневные данные, то $\Delta t = 1/250$ (по количеству рабочих дней в году). 

\justify
Посчитаем логарифмические доходности (\en{log returns}), которые близки к простым процентным доходностям:
\begin{align*}
r_i &= \ln \left( \dfrac{S_{i}}{S_{i-1}} \right) \\
\ln \left(\dfrac{S_{i}}{S_{i-1}} \right) &= \ln \left[1 + \left(\dfrac{S_{i}}{S_{i-1}} - 1 \right) \right] \approx \dfrac{S_{i}}{S_{i-1}} - 1
\end{align*}
\end{frame}



\begin{frame}{Историческая волатильность}
\justify
В модели Блэка-Шоулза все лог-доходности $r_i$ --- н.о.р.с.в. Их выборочное стандартное отклонение --- историческая (\en{historical}) или реализованная (\en{realized}) волатильность.

\begin{align*}
\hat{r} &= \frac{1}{n}\sum\limits_{i=1}^{n}r_i \\
\hat{\sigma} &= \sqrt{\frac{1}{n-1}\sum\limits_{i=1}^{n}(r_i - \hat{r})^2}
\end{align*}

\justify
Чтобы перевести <<дневную>> волатильность в проценты годовых, нужно разделить её на $\sqrt{\Delta t}$.
\end{frame}



\begin{frame}{Историческая волатильность}
\centering
\begin{tikzpicture}
\begin{axis}[
  width=\textwidth,
  height=\textheight - 1cm,
  date coordinates in=x,
  date ZERO=2012-01-01,
  xtick={2012-01-01, 2014-01-01, 2016-01-01, 2018-01-01, 2020-01-01, 2022-01-01},
  minor xtick={2013-01-01, 2015-01-01, 2017-01-01, 2019-01-01, 2021-01-01},
  ytick={30, 40, ..., 90},
%  minor ytick={-0.75, -0.25, 0.25, 0.75, 1.25},
  xticklabel={\year},
  xmin=2012-01-01,
  xmax=2022-01-01,
  ymin=20,
  ymax=90,
  grid=both,
  %yticklabel={\pgfmathprintnumber{\tick}\%},
  ylabel={\small{Курс USDRUB}},
  xlabel near ticks,
  ylabel near ticks
]

\addplot[color = Set1-B, mark = none, very thick]
	table[
		x=date,
		y=fx_rate,
		col sep=comma
	]
	{cbr.USD.2012.2021.csv};

\end{axis}
\end{tikzpicture}

\scriptsize Данные: ЦБ РФ.
\end{frame}



\begin{frame}{Историческая волатильность}
\centering
\begin{tikzpicture}
\begin{axis}[
  width=\textwidth,
  height=\textheight - 1cm,
  date coordinates in=x,
  date ZERO=2012-01-01,
  xtick={2012-01-01, 2014-01-01, 2016-01-01, 2018-01-01, 2020-01-01, 2022-01-01},
  minor xtick={2013-01-01, 2015-01-01, 2017-01-01, 2019-01-01, 2021-01-01},
  ytick={0.1, 0.2, ..., 1},
%  minor ytick={-0.75, -0.25, 0.25, 0.75, 1.25},
  xticklabel={\year},
  xmin=2012-01-01,
  xmax=2022-01-01,
  ymin=0,
  ymax=0.9,
  grid=both,
  yticklabel={\pgfmathparse{\tick*100}\pgfmathprintnumber[precision=0]{\pgfmathresult}\%},
  ylabel={\small{Реализованная волатильность}},
  xlabel near ticks,
  ylabel near ticks
]

\addplot[color = Set1-B, mark = none, thick]
	table[
		x=mid_month,
		y=realized_vol,
		col sep=comma
	]
	{USDRUB_realized_vol.csv};

\end{axis}
\end{tikzpicture}

\scriptsize Данные: ЦБ РФ.
\end{frame}



\begin{frame}{Историческая волатильность}
\justify
Историческая волатильность --- не константа, что противоречит модели Блэка-Шоулза. В прошлом мы видели большой разброс в реализованной волатильности. Сложно выбрать репрезентативный период, который хорошо подходил бы тому опциону (скажем, на 3 месяца), который мы пытаемся оценить.

\justify
Прошлое не предсказывает будущее!
\end{frame}



\begin{frame}{Ожидаемая волатильность}
\justify
В модели Блэка-Шоулза цена опциона зависит от волатильности:
\begin{align*}
C_K = F(S_0, T, K, \sigma, r)
\end{align*}

\justify
Мы можем посмотреть на рыночные цены опционов и решить задачу в обратном направлении. 

\justify
Если участники рынка пользуются моделью Блэка-Шоулза, то какую волатильность они подставляют в формулу, чтобы получить те премии, которые мы наблюдаем?
\begin{align*}
\sigma = F^{-1}(S_0, T, K, r, C_K)
\end{align*}

\justify
Решение этой обратной задачи --- ожидаемая (\en{implied}) волатильность.
\end{frame}



\begin{frame}{Ожидаемая волатильность}
\justify
Опционы на USDRUB с экспирацией 16.12.2021.

\centering
\begin{tikzpicture}
\begin{axis}[
			width = \textwidth,
			height = \textheight - 2cm,
			domain=60:84,
			xtick={60,62,...,84},
			ytick={1,2,...,13},
			xmin=60, xmax=84,
			ymin=0, ymax=13,
			grid = major,
			xlabel={Страйк ($K$)},
			ylabel={Цена опциона, руб.}
]

	\addplot[color = Set1-A, mark = none, thick]
	table[
		x=strike,
		y=call,
		col sep=comma
	]
	{usdrub_implied_vol.csv};
	
	\addplot[color = Set1-B, mark = none, thick]
	table[
		x=strike,
		y=put,
		col sep=comma
	]
	{usdrub_implied_vol.csv};
\end{axis}
\end{tikzpicture}

\scriptsize Данные: Московская биржа.
\end{frame}



\begin{frame}{Ожидаемая волатильность}
\justify
Опционы на USDRUB с экспирацией 16.12.2021.

\centering
\begin{tikzpicture}
\begin{axis}[
			width = \textwidth,
			height = \textheight - 2cm,
			domain=60:84,
			xtick={60,62,...,84},
			ytick={0.02,0.04,...,0.24},
			xmin=60, xmax=84,
			ymin=0, ymax=0.24,
			yticklabel={\pgfmathparse{\tick*100}\pgfmathprintnumber[precision=0]{\pgfmathresult}\%},
			grid = major,
			xlabel={Страйк ($K$)},
			ylabel={Волатильность ($\sigma$)}
]

	\addplot[color = Set1-B, mark = none, thick]
	table[
		x=strike,
		y=iv,
		col sep=comma
	]
	{usdrub_implied_vol.csv};
\end{axis}
\end{tikzpicture}

\scriptsize Данные: Московская биржа.
\end{frame}



\begin{frame}{Ожидаемая волатильность}
\justify
Опционы на S\&P\,500 с экспирацией 17.12.2021.

\centering
\begin{tikzpicture}
\begin{axis}[
			width = \textwidth,
			height = \textheight - 2cm,
			domain=4000:5000,
			xtick={4000,4100,...,5000},
			ytick={50,100,...,600},
			xmin=4000, xmax=5000,
			ymin=0, ymax=600,
			grid = major,
			xlabel={Страйк ($K$)},
			ylabel={Цена опциона, \$},
			xticklabel={\pgfmathprintnumber[precision=0, 1000 sep={}]{\tick}}
]

	\addplot[color = Set1-A, mark = none, thick]
	table[
		x=strike,
		y=call,
		col sep=comma
	]
	{sp500_implied_vol.csv};
	
	\addplot[color = Set1-B, mark = none, thick]
	table[
		x=strike,
		y=put,
		col sep=comma
	]
	{sp500_implied_vol.csv};
\end{axis}
\end{tikzpicture}

\scriptsize Данные: barchart.com.
\end{frame}



\begin{frame}{Ожидаемая волатильность}
\justify
Опционы на S\&P\,500 с экспирацией 17.12.2021.

\centering
\begin{tikzpicture}
\begin{axis}[
			width = \textwidth,
			height = \textheight - 2cm,
			domain=4000:5000,
			xtick={4000,4100,...,5000},
			ytick={0.05,0.10,...,0.25},
			xmin=4000, xmax=5000,
			ymin=0, ymax=0.25,
			grid = major,
			xlabel={Страйк ($K$)},
			ylabel={Волатильность ($\sigma$)},
			xticklabel={\pgfmathprintnumber[precision=0, 1000 sep={}]{\tick}},
			yticklabel={\pgfmathparse{\tick*100}\pgfmathprintnumber[precision=0]{\pgfmathresult}\%},
]

	\addplot[color = Set1-B, mark = none, thick]
	table[
		x=strike,
		y=iv,
		col sep=comma
	]
	{sp500_implied_vol.csv};
\end{axis}
\end{tikzpicture}

\scriptsize Данные: barchart.com.
\end{frame}



\begin{frame}{Улыбка волатильности}
\justify
Почти на всех рынках наблюдается зависимость ожидаемой (\en{implied}) волатильности от страйка опциона. Дальние \en{out of the money}\ опционы стоят дороже (в терминах волатильности), чем можно было бы ожидать.
\begin{itemize}
\item Улыбка (\en{smile}). Например, на рынке FX.
\item Ухмылка (\en{skew, smirk}). Например, на рынке акций.
\end{itemize}

\justify
Может ли быть так, что базовый актив ведёт себя по-разному в зависимости от того, какой опцион (с каким страйком) сейчас оценивают участники рынка? Нет!

\justify
Либо все участники опционного рынка сошли с ума, либо модель Блэка-Шоулза не до конца описывает реальность. Предположение о постоянной волатильности и логнормальном распределении будущей цены базового актива выглядит слишком строгим.
\end{frame}



\begin{frame}{Опционная бабочка}
\justify
Попробуем оценить, какое распределение, если не лог-нормальное, ожидают участники рынка.

\justify
Рассмотрим комбинацию опционов <<бабочка>> (\en{butterfly, fly}):
\begin{itemize}
\item Купленный колл со страйком $K - \delta$.
\item Два проданных колла со страйком $K$.
\item Купленный колл со страйком $K + \delta$.
\end{itemize}

\justify
Купленные опционы --- <<крылья>> (\en{wings}), а проданные --- <<тельце>> (\en{belly}).
\end{frame}



\begin{frame}{Опционная бабочка}
\justify
Пример: бабочка со страйками 70, 72 и 74.

\centering
	\begin{tikzpicture}
		\begin{axis}[
			domain=64:80,
			%axis lines=middle,
			xtick={64,66,...,80},
			ytick={-6,-5,...,6},
			xmin=64, xmax=80,
			ymin=-6, ymax=6,
			%x label style={at={(axis description cs: 0.5, -0.1)}, anchor=north},
			%y label style={at={(axis description cs:-0.1,1)},anchor=south},
			grid = major,
			xlabel={Курс в дату экспирации},
			ylabel={Выплата (payoff)},
			%scaled x ticks=false
		]
		
	\addplot[Set1-B, very thick, dashed] {(\x > 74)*(\x - 74) + 0.15};
  	\addplot[Set1-C, very thick, dashed] {(\x > 70)*(\x - 70) + 0.15};
  	\addplot[Set1-D, very thick, dashed] {2*(\x > 72)*(72 - \x) - 0.15};
 
 	\addplot[Set1-A, very thick] {(\x > 74)*(\x - 74) + (\x > 70)*(\x - 70) + 2*(\x > 72)*(72 - \x)};
 
   \draw[thick, color=black] (axis cs: 60, 0) -- (axis cs: 90, 0);
\end{axis}
\end{tikzpicture}
\end{frame}



\begin{frame}{Опционная бабочка}
\justify
Пример: бабочка со стайками 71, 72 и 73.

\centering
	\begin{tikzpicture}
		\begin{axis}[
			width = \textheight,
			height = \textheight*0.5,
			domain=64:80,
			%axis lines=middle,
			xtick={64,66,...,80},
			ytick={0,...,6},
			xmin=64, xmax=80,
			ymin=-1, ymax=3,
			%x label style={at={(axis description cs: 0.5, -0.1)}, anchor=north},
			%y label style={at={(axis description cs:-0.1,1)},anchor=south},
			grid = major,
			xlabel={Курс в дату экспирации},
			ylabel={Выплата (payoff)},
			%scaled x ticks=false
		]
		
 	\addplot[Set1-A, very thick, samples at={64,64.1,...,80}] {(\x > 71)*(\x - 71) + (\x > 73)*(\x - 73) + 2*(\x > 72)*(72 - \x) + 0.05};
 
   \draw[thick, color=black] (axis cs: 60, 0) -- (axis cs: 90, 0);
\end{axis}
\end{tikzpicture}

\justify
По мере того, как разность между <<крыльями>> уменьшается, бабочка превращается в ставку на то, что цена базового актива остановится вблизи центрального страйка.
\end{frame}



\begin{frame}{Опционная бабочка}
\justify
Бабочки со страйками $K-1$, $K$ и $K+1$. Экспирация 16.12.2021.

\centering
\begin{tikzpicture}
\begin{axis}[
			width = \textwidth,
			height = \textheight - 2cm,
			domain=60:84,
			xtick={60,62,...,84},
			ytick={0.02,0.04,...,0.18},
			xmin=60, xmax=84,
			ymin=0, ymax=0.18,
			yticklabel={\pgfmathprintnumber[fixed, fixed zerofill, precision=2]{\tick}},
			grid = major,
			xlabel={Страйк ($K$)},
			ylabel={Цена бабочки, руб.}
]

	\addplot[color = Set1-B, mark = none, thick]
	table[
		x=strike,
		y=fly,
		col sep=comma
	]
	{usdrub_fly.csv};
\end{axis}
\end{tikzpicture}

\scriptsize Данные: Московская биржа.
\end{frame}



\begin{frame}{Ожидаемое распределение}
\centering
\begin{tikzpicture}
\begin{axis}[
			width = \textwidth,
			height = \textheight - 2cm,
			domain=60:84,
			xtick={60,62,...,84},
			ytick={0.02,0.04,...,0.18},
			xmin=60, xmax=84,
			ymin=0, ymax=0.18,
			yticklabel={\pgfmathprintnumber[fixed, fixed zerofill, precision=2]{\tick}},
			grid = major,
			xlabel={Страйк ($K$)},
			ylabel={Плотность распределения},
			legend entries = {
				\small Ожидаемое,
				\small Логнормальное
			},
  			legend cell align={left},
 		 	legend style={at={(0.97,0.97)},anchor=north east}
]

	\addplot[color = Set1-B, mark = none, thick]
	table[
		x=strike,
		y=implied_density,
		col sep=comma
	]
	{usdrub_implied_density.csv};
	
	\addplot[color = Set1-A, mark = none, dashed, thick]
	table[
		x=strike,
		y=lognormal_density,
		col sep=comma
	]
	{usdrub_implied_density.csv};
\end{axis}
\end{tikzpicture}

\scriptsize Данные: Московская биржа.
\end{frame}



\begin{frame}{Ожидаемое распределение}
\justify
Обычно цены опционов подразумевают распределение, отличное от логнормального. Часто можно видеть толстые хвосты и скошенность влево или вправо.

\justify
Мы не знаем, как в точности объясняется этот эффект:
\begin{itemize}
\item Участники рынка верно оценивают истинную вероятность экстремальных исходов (больших изменений цены базового актив).
\item Участникам рынка настолько больно от экстремальных исходов, что они готовы переплатить за страховку. Это стремление застраховаться и избежать риска увеличивает рыночную цену опционов относительно <<фундаментально обоснованной>>.
\end{itemize}

\justify
Как обычно, если мы можем использовать ликвидные опционы для хэджирования риска и репликации более сложных экзотических продуктов, то нам не важно, какое объяснение верное.
\end{frame}



\begin{frame}{Модели волатильности}
\justify
Существование улыбки волатильности опровергает гипотезу о геометрическом броуновском движении с постоянной волатильностью. Как можно изменить модель движения базового актива, чтобы новая модель объясняла улыбку волатильности (давала те же премии, которые мы видим на рынке).

\justify
\begin{itemize}
\item Локальная волатильность.
\item Стохастическая волатильность.
\item Стохастическая локальная волатильность.
\item Стохастическая локальная волатильность с прыжками.
\item ...
\end{itemize}
\end{frame}



\begin{frame}{Локальная волатильность}
\justify
Геометрическое броуновское движение:
\begin{align*}
\frac{dS_t}{S_t} = \mu dt + \sigma\varepsilon\sqrt{dt}, \quad \varepsilon \sim \mathcal{N}(0, 1), \sigma = const
\end{align*}

\justify
Локальная волатильность:
\begin{align*}
\frac{dS_t}{S_t} = \mu dt + \sigma(S_t, t)\varepsilon\sqrt{dt}, \quad \varepsilon \sim \mathcal{N}(0, 1)
\end{align*}

\justify
$\sigma(S_t, t)$ --- зависимость волатильности от цены базового актива и времени. Например, предположим, что если пара доллар-рубль значительно отклонится от текущего уровня (прыгнет до 80), то на рынке начнётся паника и волатильность будет выше, чем в спокойные времена при курсе 70.
\end{frame}



\begin{frame}{Стохастическая волатильность}
\justify
Геометрическое броуновское движение:
\begin{align*}
\frac{dS_t}{S_t} = \mu dt + \sigma\varepsilon\sqrt{dt}, \quad \varepsilon \sim \mathcal{N}(0, 1), \sigma = const
\end{align*}

\justify
Модель SABR (\en{Stochastic Alpha-Beta-Rho}):
\begin{align*}
dF_t &= \sigma_t(F_t)^\beta \sqrt{dt}\varepsilon, \quad \varepsilon \sim \mathcal{N}(0, 1) \\
d\sigma_t &= \alpha\sigma_t\sqrt{dt}\psi, \quad \psi \sim \mathcal{N}(0, 1) \\
Cov(\varepsilon, \psi) &= \rho
\end{align*}

\justify
$\alpha$ --- волатильность волатильности.

$\beta$ --- коэффициент скошенности.

$\rho$ --- корреляция между волатильностью и базовым активом.
\end{frame}



\begin{frame}{Калибрация моделей}
\justify
Все модели волатильности подгоняются (\en{fitted}) или калибруются (\en{calibrated}) к рынку. Мы подбираем такие значения внутренних параметров модели (например, $\alpha$, $\beta$ и $\rho$ в SABR), чтобы модель <<лучше всего>> воспроизводила наблюдаемые цены ликвидных ванильных опционов. После этого модель можно использовать для оценки более сложных экзотических продуктов.

\begin{align*}
&\text{Рыночные цены} \Rightarrow \\
&\text{Внутренние параметры} \Rightarrow \\
&\text{Цены произвольных опционов}
\end{align*}

\justify
Объективная реальность, данная нам в ощущении --- рыночные цены ликвидных ванильных опционов. Модели и параметры --- наши предположения, которые позволяют <<интерполировать>> цены неликвидных деривативов.
\end{frame}

\end{document}