\documentclass{beamer}

\usepackage{cmap}				% To be able to copy-paste russian text from pdf
\usepackage[T2A]{fontenc}
\usepackage[utf8]{inputenc}
\usepackage[russian]{babel}
\usepackage{textpos}
\usepackage{ragged2e}
\usepackage{amssymb}
\usepackage{ulem}
\usepackage{tikz}
\usepackage{pgfplots}
\usepackage{color}
\usepackage{cancel}
\usepackage{multirow}
\pgfplotsset{compat=1.17}
\usetikzlibrary{arrows,snakes,backgrounds,shapes}
\usepgfplotslibrary{groupplots,colorbrewer,dateplot,statistics}
\usepackage{animate}

\usepackage{amsfonts}
\usepackage{amsmath}
\usepackage{amssymb}
\usepackage{graphicx}
\usepackage{setspace}

\usepackage{enumitem}
\setitemize{label=\usebeamerfont*{itemize item}%
  \usebeamercolor[fg]{itemize item}
  \usebeamertemplate{itemize item}}

\title{Лекция 2. Процентные свопы}
\author{Артём Бакулин}
\date{14 октября 2021 г.}

\usetheme{Warsaw}
\usecolortheme{beaver}

\newcommand{\ru}[1]{\begin{otherlanguage}{russian}#1\end{otherlanguage}}
\newcommand{\en}[1]{\begin{otherlanguage}{english}#1\end{otherlanguage}}
\newcommand{\ruen}[2]{#1 (\en{#2})}

\begin{document}



\begin{frame}
\titlepage
\end{frame}



\begin{frame}{Напоминание: текущая стоимость}
\justify
\alert{Текущая стоимость} (\en{present value}) выплаты в $N$ рублей через $T$ лет --- сумма, 
которую участники рынка готовы заплатить сегодня за право получить эту выплату.

\justify
Сколько рублей вы готовы заплатить сегодня (\en{PV}) за право получить 1\,060\,000 (\en{future value, FV}) рублей через год? Предположим, что на рынке есть идеальный безрисковый депозит под 6\% с простой капитализацией.

\begin{align*}
PV = \frac{FV}{1+rT} = \frac{1\,060\,000}{1 + 6\%} = 1\,000\,000
\end{align*}

\justify
Никто не захочет платить больше, чем 1\,000\,000 сегодняшних рублей за 1\,060\,000 будущих рублей, потому что выгоднее будет вложить 1\,000\,000 под 6\% на год. Никто не захочет продавать 1\,060\,000 будущих рублей дешевле, чем за 1\,000\,000 сегодняшних, потому что выгоднее будет взять в кредит 1\,000\,000 под 6\% и погасить его через год.
\end{frame}



\begin{frame}{Напоминание: коэффициент дисконтирования}
\justify
\alert{Коэффициент дисконтирования} (\en{discount factor}) для будущей даты $T$ --- текущая стоимость права получить 1 единицу валюты в день $T$. Другая формулировка: сколько рублей нужно иметь сегодня, чтобы сделать из них 1 рубль к моменту $T$?

\justify
Ответ зависит от того, какая безрисковая ставка $r$ нам доступна:
\begin{align*}
\delta_T &= \frac{1}{1 + rT} \quad \text{(простые проценты)} \\
\delta_T &= \frac{1}{e^{r^*T}} = e^{-r^*T} \quad \text{(непрерывные проценты)}
\end{align*}

\justify
Например, если безрисковая процентная ставка равна 6\%, то коэффициент дисконтирования на 1 год равен
\begin{align*}
\delta_1 = \frac{1}{1.06} \approx 0.9434
\end{align*}
\end{frame}



\begin{frame}{Напоминание: форвардная ставка}
\centering
\begin{tikzpicture}
		\draw [->,>=triangle 90] (0, 0) -- (8.5, 0);

		\draw [->,>=triangle 45] (1,0) node[anchor=north east, inner sep=1pt]{\setlength\tabcolsep{1pt}\begin{tabular}{r} 0 \\ $\delta_0=1.0$\end{tabular}} .. controls (1.5, 1) and (3.5, 1) .. (4,0) node[anchor=north, inner sep=1pt]{\begin{tabular}{c}$T_1=1$ \\ $\delta_1 = 0.9254$\end{tabular}} node[pos=0.5,anchor=south]{$r_1=5\%$};

		\draw [->,>=triangle 45] (4,0) .. controls (4.5, 1) and (6.5, 1) .. (7,0) node[anchor=north west, inner sep=1pt]{\setlength\tabcolsep{1pt}\begin{tabular}{l}$T_2=2$ \\ $\delta_2 = 0.8772$\end{tabular}} node[pos=0.5,anchor=south]{$x=8.57\%$};

		\draw [->,>=triangle 45] (1,0) .. controls (1.5, -1.75) and (6.5, -1.75) .. (7,0) node[pos=0.5,anchor=north]{$r_2=7\%$};
	\end{tikzpicture}
	
\justify
Следующие параметры описывают одну и ту же наблюдаемую действительность:

1) Ставка на период от сегодня до 1 года --- 5\%, ставка от сегодня до 2 лет --- 7\%.

2) Ставка на период от сегодня до 1 года --- 5\%, (<<форвардная>>) ставка от 1 года до 2 лет --- 8.57\%.

3) Коэффициент дисконтирования на сегодня --- 1.0, на 1 год --- 0.9254, на 2 года --- 0.8772. 
\end{frame}



\begin{frame}{European Interbak Offered Rate}
\justify
\alert{EURIBOR} --- индикативная процентная ставка (\en{interest rate benchmark}) по беззалоговым кредитам в евро среди банков Еврозоны. Публикуется \en{European Money Market Institute (EMMI)}.

\justify
По какой ставке сферический крупный банк может занять евро у другого банка или финансовой организации на открытом рынке? 

\justify
\centering
\begin{tabular}{l|r}
Срок (tenor)     & EURIBOR 08.10.2021 \\ \hline
1W (1 неделя)    & -0.568\% \\
1M (1 месяц)     & -0.561\% \\
3M (3 месяца)    & -0.548\% \\
6M (6 месяцев)   & -0.518\% \\
12M (12 месяцев) & -0.482\% 
\end{tabular}

\justify
*EURIBOR --- ставка без капитализации в конвенции ACT/360.
\end{frame}



\begin{frame}{Вычисление EURIBOR}
\justify
В вычислении EURIBOR участвуют 18 крупных банков Еврозоны и UK. Каждый банк ежедневно отправляет свою ставку по каждому из пяти сроков (от недели до года). Банки обязаны следовать методологии, которая определяет 3 <<уровня>>:

\justify 
1) Cредневзвешенная ставка реальных сделок банка в этот день и на этот срок. Каждая сделка должна быть не меньше 20 млн.

\justify
2) Интерполяция ставок реальных сделок на соседние сроки (например, вычисление 6M из 3M и 12M), либо экстраполяция сделок предыдущих дней на данный срок.

\justify
3) Ставки сделок по похожим инструментам, сделки менее 20 миллионов, котировки деривативов, сделки с нефинансовыми организациями, экспертное суждение. 

\justify
15\% самых больших и 15\% самых маленьких котировок отбрасываются. Арифметическое среднее оставшихся 70\%, округлённое до третьего знака --- EURIBOR.
\end{frame}



\begin{frame}{Вычисление EURIBOR}
\centering
\begin{tikzpicture}
		\draw [->,>=triangle 90] (0, 0) -- (9.5, 0);

		\draw [dashed] (0.5,0) node[anchor=north]{$T-1$} .. controls (1.0, 0.5) and (2.5, 0.5) .. (3,0) node[anchor=north]{$T+1$} node[pos=0.5,anchor=south]{spot lag};

		\draw [->,>=triangle 45] (3,0) .. controls (4, 1) and (7, 1) .. (8,0) node[anchor=north]{$T+1+3M$} node[pos=0.5,anchor=south]{$EURIBOR_{T}$};

		\node[anchor=north] at (1.75, 0) {$T$};
		
		\node[circle, fill, inner sep=1.5pt] at (0.5, 0) {};
		\node[circle, fill, inner sep=1.5pt] at (1.75, 0) {};
		\node[circle, fill, inner sep=1.5pt] at (3.0, 0) {};
		\node[circle, fill, inner sep=1.5pt] at (8.0, 0) {};
	\end{tikzpicture}
	
\justify
T-1 (вчера):  банки заключали сделки и собирали статистику.

T (сегодня): в 11:00 опубликовано среднее арифметическое вчерашних сделок --- EURIBOR на три месяца за дату T.

T+1 (завтра): кредиты, о которых договорились вчера, вступят в силу (банки-заёмщики получат евро).

T+1+3M (через три месяца): кредиты истекут (банки-заёмщики вернут деньги).

\justify
EURIBOR публикуется в дни, когда работает платёжная система Европейского Центробанка TARGET2. Три месяца прибавляются по конвенции \en{modified following business day}.
\end{frame}



\begin{frame}{Конвенция о переносе дней}
\justify
Кредит на три месяца начался в пятницу 29 октября 2021 года. Когда он закончится, если 29 января 2022 года --- суббота?

\justify
Конвенция \en{modified following / business month end}:

\justify
1) Если дата начала --- последний рабочий день текущего месяца, то дата окончания --- тоже последний рабочий день месяца. Пример: 30.09.2021 (чт) --- 31.12.2021 (пт).

\justify
2) Если дата окончания выпала на выходной, то она сдвигается вперёд на ближайший рабочий день, если только ближайший рабочий день --- не в следующем месяце. Пример: 01.10.2021 (пт) --- 03.01.2022 (пн).

\justify
3) В противном случае дата окончания сдвигается назад. Пример: 29.11.2021 (пн) --- 28.02.2022 (пн).
\end{frame}



\begin{frame}{Euro short-term rate}
\justify
\alert{ESTER} --- индикативная ставка по беззалоговым кредитам сроком на один день (\en{overnight}) среди банков Еврозоны. Публикуется Европейским центральным банком (ЕЦБ).

\justify
\centering
\begin{tabular}{l|r}
Срок   & ESTER 07.10.2021 \\ \hline
1 день & -0.568\%
\end{tabular}

\justify
В определении ESTER участвует панель из 48 банков. Каждый банк в течение дня сообщает ЕЦБ обо всех своих сделках (когда банк берёт кредит на 1 день). ЕЦБ отбрасывает 25\% самых маленьких и 25\% самых больших ставок и вычисляет среднее арифметическое.
\end{frame}



\begin{frame}{Вычисление ESTER}
\justify
\centering
\begin{tikzpicture}
		\draw [->,>=triangle 90] (0, 0) -- (4.0, 0);

		\draw [->, >= triangle 45] (0.5,0) node[anchor=north]{$T$} .. controls (1.0, 0.75) and (2.5, 0.75) .. (3,0) node[anchor=north]{$T+1$} node[pos=0.5,anchor=south]{$ESTER_T$};
\end{tikzpicture}

\justify
T (сегодня): банки берут кредиты (получают евро) и сообщают о сделках в ЕЦБ.

T+1 (завтра): ЕЦБ опубликует среднюю ставку вчерашних сделок --- ESTER за вчера. Кредиты банков истекают (банки отдают евро).

\justify
Как и EURIBOR, ESTER определяется в те дни, когда работает платёжная система TARGET2.
\end{frame}



\newcommand{\plotBenchmarkRate}[2] {
	
	\addplot[
		color = #2,
		mark = none,
		thick
	]
	table[
		x=date,
		y=#1,
		col sep=comma
	]
	{euro_benchmark.csv};
}

\begin{frame}{ESTER и EURIBOR}
\centering
\begin{tikzpicture}
\begin{axis}[
  width=\textwidth,
  height=\textheight - 1cm,
  date coordinates in=x,
  date ZERO=2012-01-01,
  xtick={2012-01-01, 2014-01-01, 2016-01-01, 2018-01-01, 2020-01-01, 2022-01-01},
  minor xtick={2013-01-01, 2015-01-01, 2017-01-01, 2019-01-01, 2021-01-01},
  ytick={-0.5, 0, 0.5, 1.0, 1.5},
  minor ytick={-0.75, -0.25, 0.25, 0.75, 1.25},
  xticklabel={\year},
  xmin=2012-01-01,
  xmax=2022-01-01,
  ymin=-0.75,
  ymax=1.5,
  grid=both,
  yticklabel={\pgfmathprintnumber{\tick}\%},
%  ylabel={\small{Курс USDRUB}},
  xlabel near ticks,
  ylabel near ticks,
  legend entries = {
      EURIBOR 3M,
      ESTER
  },
  legend cell align={left}
]

	\plotBenchmarkRate{3m}{Set1-A}
	\plotBenchmarkRate{ester}{Set1-B}
	
	\draw[thick, color=black] (axis cs: 2012-01-01, 0) -- (axis cs: 2022-01-01, 0);
\end{axis}
\end{tikzpicture}

\scriptsize Данные: ECB, EMMI.
\end{frame}

\end{document}