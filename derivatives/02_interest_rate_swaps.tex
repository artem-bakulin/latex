\documentclass{beamer}

\usepackage{cmap}				% To be able to copy-paste russian text from pdf
\usepackage[T2A]{fontenc}
\usepackage[utf8]{inputenc}
\usepackage[russian]{babel}
\usepackage{textpos}
\usepackage{ragged2e}
\usepackage{amssymb}
\usepackage{ulem}
\usepackage{tikz}
\usepackage{pgfplots}
\usepackage{color}
\usepackage{cancel}
\usepackage{multirow}
\pgfplotsset{compat=1.17}
\usetikzlibrary{arrows,snakes,backgrounds,shapes}
\usepgfplotslibrary{groupplots,colorbrewer,dateplot,statistics}
\usepackage{animate}

\usepackage{amsfonts}
\usepackage{amsmath}
\usepackage{amssymb}
\usepackage{graphicx}
\usepackage{setspace}

\usepackage{enumitem}
\setitemize{label=\usebeamerfont*{itemize item}%
  \usebeamercolor[fg]{itemize item}
  \usebeamertemplate{itemize item}}

% remove navigation bar
\setbeamertemplate{navigation symbols}{} 

\usepackage{eurosym}
\renewcommand{\EUR}[1]{\textup{\euro}#1}

\title{Процентные свопы}
\author{Артём Бакулин}
\date{13 марта 2023 г.}

\usetheme{Warsaw}
\usecolortheme{beaver}

% remove navigation bar
\setbeamertemplate{navigation symbols}{}

\setbeamertemplate{page number in head/foot}[totalframenumber] 

\newcommand{\ru}[1]{\begin{otherlanguage}{russian}#1\end{otherlanguage}}
\newcommand{\en}[1]{\begin{otherlanguage}{english}#1\end{otherlanguage}}
\newcommand{\ruen}[2]{#1 (\en{#2})}

\begin{document}



\begin{frame}
\titlepage
\end{frame}

\begin{frame}{Сложные проценты}
\justify
<<Сложные проценты --- самая могущественная формула сила во Вселенной>> 
(ошибочно приписывается Альберту Эйнштейну).

\justify
Пусть депозит под ставку (\en{rate}) $r$ действует (\en{term}) $T$ лет с 
частотой капитализации (\en{frequency})\ $f$ раз в год. Начальная сумма вклада 
$N_0$. В конце срока мы получим
\begin{align*}
N &= N_0\underbrace{\left(1 + \frac{r}{f}\right) \cdot \left(1 + \frac{r}{f}\right) \cdot ... \cdot \left(1 + \frac{r}{f}\right)}_{fT \text{\ множителей}} = 
N_0\left(1 + \frac{r}{f}\right)^{fT}
\end{align*}

Например, депозит на 1\,000\,000 под 12\% на 1 год с ежемесячной капитализацией:
\begin{align*}
1\,000\,000 \cdot \left(1 + \frac{0.12}{12}\right)^{12} \approx 1\,126\,825
\end{align*}
\end{frame}



\begin{frame}{Общеупотребительные методы капитализации}
\centering
\begin{tabular}{l|l|l}
Частота ($f$) & Русское название & Английское название \\ \hline 
0 & Без капитализации* & Zero \\
1 & Ежегодно & Annually \\
2 & Раз в полгода & Semi-annually \\
4 & Ежеквартально & Quarterly \\
12 & Ежемесячно & Monthly \\
365 & Ежедневно & Daily \\
$\infty$ & Непрерывно* & Continuously 
\end{tabular}

\justify
Простые проценты без капитализации (\en{zero compounding}), $f=0$:
\begin{align*}
N = N_0 (1 + rT)
\end{align*}

Непрерывная капитализация (\en{continuous compounding}), $f=\infty$:
\begin{align*}
N = N_0e^{rT}
\end{align*}
\end{frame}



\begin{frame}{Непрерывная капитализация процентов}
\justify
Предположим, что совсем-совсем небольшой процент прибавляется ко вкладу каждый 
час, каждую минуту, каждую миллисекунду. Частота капитализации $f$ стремится к 
бесконечности.
\begin{align*}
N &= N_0 \lim_{f \to +\infty} \left(1 + \frac{r}{f}\right)^{fT} = 
\begin{Bmatrix}x= f/r \\ f=xr \\ x \to +\infty \\ f \to +\infty\end{Bmatrix} = \\
&= N_0\lim_{x \to +\infty} \left(1 + \frac{r}{xr}\right) ^ {xrT} =
N_0{\underbrace{\left(\lim_{x \to +\infty} \left( 1 + \frac{1}{x} \right) ^ x\right)}_{e \approx 2.71828}}  ^ {rT} = \\
&= N_0e^{rT} \approx N_0(1 + rT)
\end{align*}
<<Непрерывные>> проценты не встречаются в реальной жизни, но заметно упрощают 
формулы в теоретических моделях.
\end{frame}



\begin{frame}{Капитализация процентов - 1}
\justify
Рассмотрим депозиты на 1\,000\,000 рублей под ставку 12\% на срок 1 год и 5 лет. 
Как конечная сумма вклада зависит от капитализации процентов?


\begin{table}
\centering
\begin{tabular}{l|r|r|r}
Начисление процентов & $f$ & 1 год & 5 лет \\ \hline
Без капитализации & 0         & 1\,120\,000 & 1\,600\,000 \\
Ежегодно          & 1         & 1\,120\,000 & 1\,762\,342 \\
Раз в полгода     & 2         & 1\,123\,600 & 1\,790\,848 \\
Ежеквартально     & 4         & 1\,125\,509 & 1\,806\,111 \\
Ежемесячно        & 12        & 1\,126\,825 & 1\,816\,697 \\
Ежедневно         & 365       & 1\,127\,475 & 1\,821\,939 \\
Непрерывно        & $+\infty$ & 1\,127\,497 & 1\,822\,119
\end{tabular}
\end{table}

\justifying
Чем выше ставка и дольше срок, тем важнее роль капитализации. Ежедневная капитализация процентов достаточно близка к непрерывной.
\end{frame}



\begin{frame}{Капитализация процентов - 2}

	\centering
	\begin{tikzpicture}
		\begin{axis}[
			xlabel=\text{Время, месяцы},
			ylabel=\text{Сумма вклада, рубли},
			xmin=0,
			xmax=24.5,
			width=\textwidth,
			height=\textheight - 1.5cm,
			ymin=100,
			ymax=130,
			legend style={at={(0.01, 0.99)}, anchor=north west}
		]
	

		\addplot[const plot, samples at={0,0.1,...,60}, very thick, color=Set1-A] {100 * (1 + 0.12/4) ^ (4*3*floor(x/3)/12.0)};
		\addlegendentry{$f=4$}
	
		\addplot[const plot, samples at={0,0.5,...,60}, very thick, color=Set1-B] {100 * (1 + 0.12/12) ^ (12*floor(x)/12.0)};
		\addlegendentry{$f=12$}
		
		\addplot[domain=0:60, color=Set1-C, very thick] {100 * exp(0.12*x/12)};
		\addlegendentry{$f=+\infty$}
		\end{axis}
	\end{tikzpicture}
	\scriptsize{Рост вклада с течением времени, $N_0=100$, $r=12\%$}
\end{frame}



\begin{frame}{Соглашение о подсчёте дней}
\justify
Сколько лет между 1 марта 2023 года и 1 апреля 2023 года?

\centering
\begin{tabular}{l|l}
Конвенция & Кол-во лет \\ \hline
ACT/365 & $31/365 \approx 0.08493$ \\
ACT/360 & $31/360 \approx 0.08611$ \\
30/360  & $30/360 \approx 0.08333$ \\
BUS/252 & $23/252 \approx 0.09127$ 
\end{tabular}

\justify
ACT --- считаем календарные дни. 30 --- считаем, что в месяце 30 дней. BUS --- считаем только рабочие дни.

\justify
Выбор способа начисления процентов --- рыночная конвенция. Правило большого пальца: в странах Британского Содружества ACT/365, в других странах --- ACT/360, в Бразилии --- BUS/252. Читайте мелкий шрифт! 
\end{frame}



\begin{frame}{Конвенция 30/360}
\justify
Сколько лет между датами $D_1.M_1.Y_1$ и $D_2.M_2.Y_2$?
\begin{align*}
T = \frac{360(Y_2-Y_1) + 30(M_2-M_1) + (D_2-D_1)}{360}
\end{align*}

\justify
Конвенция 30/360 возникла в докомпьютерную эпоху и позволяет быстро прикинуть результат в уме. Она по-прежнему живее всех живых (например, на рынке облигаций).
\begin{itemize}
\item Обратная совместимость
\item Размер платежей по кредитам не меняется в течение года
\end{itemize}

\vspace{\baselineskip}
\justify
*Несколько вариантов (30/360 Bond, 30/360 US, 30E/360, 30E/360 ISDA) отличаются тем, как они обрабатывают февраль и месяцы с 31 днями. Скажем спасибо библиотекам финансовых вычислений за то, что прячут от нас эти детали!
\end{frame}



\begin{frame}{Переход между способами капитализации}
\justify
Если вам нужна <<непрерывная>> ставка $r^*$, а есть только <<обычная>> $r$ (или наоборот), то всегда можно выразить одну через другую.
\begin{align*}
e^{r^*T} &= \left(1 + \frac{r}{f}\right)^{fT} \\
r^* &= f\ln \left(1 + \frac{r}{f}\right) \\
r &= f\left(e^{r^*/f} - 1\right)
\end{align*}

\justify
То же самое для <<простых>> процентов без капитализации:
\begin{align*}
e^{r^*T} &= 1+rT \\
r^* &= \frac{\ln(1+rT)}{T} \\
r &= \frac{e^{r^*T}-1}{T}
\end{align*}

\end{frame}



\begin{frame}{Текущая стоимость}
\justify
\alert{Текущая стоимость} (\en{present value}) выплаты в $N$ рублей через $T$ лет --- сумма, 
которую участники рынка готовы заплатить сегодня за право получить эту выплату.

\justify
Сколько рублей вы готовы заплатить сегодня (\en{PV}) за право получить 1\,120\,000 (\en{future value, FV}) рублей через год? Предположим, что на рынке есть идеальный безрисковый депозит под 12\% с простой капитализацией.

\begin{align*}
PV = \frac{FV}{1+rT} = \frac{1\,120\,000}{1 + 12\%} = 1\,000\,000
\end{align*}

\justify
Никто не захочет платить больше, чем 1\,000\,000 сегодняшних рублей за 1\,120\,000 будущих рублей, потому что выгоднее будет вложить 1\,000\,000 под 12\% на год. Никто не захочет продавать 1\,120\,000 будущих рублей дешевле, чем за 1\,000\,000 сегодняшних, потому что выгоднее будет взять в кредит 1\,000\,000 под 12\% и погасить его через год.
\end{frame}



\begin{frame}{Коэффициент дисконтирования}
\justify
\alert{Коэффициент дисконтирования} (\en{discount factor}) для будущей даты $T$ --- текущая стоимость права получить 1 единицу валюты в день $T$. Другая формулировка: сколько рублей нужно иметь сегодня, чтобы сделать из них 1 рубль к моменту $T$?

\justify
Ответ зависит от того, какая безрисковая ставка $r$ нам доступна:
\begin{align*}
\delta_T &= \frac{1}{1 + rT} \quad \text{(простые проценты)} \\
\delta_T &= \frac{1}{e^{r^*T}} = e^{-r^*T} \quad \text{(непрерывные проценты)}
\end{align*}

\justify
Например, если безрисковая процентная ставка равна 12\%, то коэффициент дисконтирования на 1 год равен
\begin{align*}
\delta_1 = \frac{1}{1.12} \approx 0.8929
\end{align*}
\end{frame}



\begin{frame}{Отложенный депозит}
\justify
Предположим, что на рынке доступны два депозита с <<нулевой>> капитализацией:
\begin{itemize}
\item На $T_1=1$ год под ставку $r_1=12\%$ 
\item На $T_2=2$ года под ставку $r_2=8\%$ 
\end{itemize}

Банк предлагает <<отложенный>> депозит: вы обязаны принести деньги через год на 1 год, а банк обязан зафиксировать ставку по депозиту $x$ уже сегодня. На какую ставку вы согласитесь?

\centering
\begin{tikzpicture}
		\draw [->,>=triangle 90] (0, 0) -- (8.5, 0);

		\draw [->,>=triangle 45] (1,0) node[anchor=north east]{$0$} .. controls (1.5, 1) and (3.5, 1) .. (4,0) node[anchor=north]{$T_1=1$} node[pos=0.5,anchor=south]{$r_1=12\%$};

		\draw [->,>=triangle 45] (4,0) .. controls (4.5, 1) and (6.5, 1) .. (7,0) node[anchor=north west]{$T_2=2$} node[pos=0.5,anchor=south]{$x$};

		\draw [->,>=triangle 45] (1,0) .. controls (1.5, -1) and (6.5, -1) .. (7,0) node[pos=0.5,anchor=north]{$r_2=8\%$};
	\end{tikzpicture}
\end{frame}



\begin{frame}{Отложенный депозит}
\centering
\begin{tikzpicture}
		\draw [->,>=triangle 90] (0, 0) -- (8.5, 0);

		\draw [->,>=triangle 45] (1,0) node[anchor=north east]{$0$} .. controls (1.5, 1) and (3.5, 1) .. (4,0) node[anchor=north]{$T_1=1$} node[pos=0.5,anchor=south]{$r_1=12\%$};

		\draw [->,>=triangle 45] (4,0) .. controls (4.5, 1) and (6.5, 1) .. (7,0) node[anchor=north west]{$T_2=2$} node[pos=0.5,anchor=south]{$x$};

		\draw [->,>=triangle 45] (1,0) .. controls (1.5, -1) and (6.5, -1) .. (7,0) node[pos=0.5,anchor=north]{$r_2=8\%$};
	\end{tikzpicture}
	
\justifying
Не должно быть разницы между депозитом на два года и цепочкой из двух депозитов:
\begin{align*}
(1 + r_1T_1)\Big(1+x(T_2-T_1)\Big) = 1 + r_2T_2
\end{align*}
\begin{align*}
x = \frac{\dfrac{1+r_2T_2}{1+r_1T_1} - 1}{T_2-T_1} = \frac{\dfrac{1 + 0.08 \cdot 2}{1 + 0.12} - 1}{2-1} \approx 3.57\%
\end{align*}

При любой другой ставке <<отложенный>> депозит будет не выгоден либо вкладчикам (и они не будут им пользоваться), либо банку (и он не будет его предлагать).
\end{frame}



\begin{frame}{Непрерывная капитализация процентов}
\justify
Пересчитаем простые проценты в непрерывные:
\begin{align*}
r_1^* &= \frac{\ln(1+r_1T_1)}{T_1} = \frac{\ln 1.12}{1} \approx 11.333\% \\
r_2^* &= \frac{\ln(1+r_2T_2)}{T_2} = \frac{\ln 1.14}{2} \approx 7.421\%
\end{align*}

\centering
\begin{tikzpicture}
		\draw [->,>=triangle 90] (0, 0) -- (8.5, 0);

		\draw [->,>=triangle 45] (1,0) node[anchor=north east]{$0$} .. controls (1.5, 1) and (3.5, 1) .. (4,0) node[anchor=north]{$T_1=1$} node[pos=0.5,anchor=south]{$r_1^*=11.333\%$};

		\draw [->,>=triangle 45] (4,0) .. controls (4.5, 1) and (6.5, 1) .. (7,0) node[anchor=north west]{$T_2=2$} node[pos=0.5,anchor=south]{$x^*$};

		\draw [->,>=triangle 45] (1,0) .. controls (1.5, -1) and (6.5, -1) .. (7,0) node[pos=0.5,anchor=north]{$r_2^*=7.421\%$};
	\end{tikzpicture}
\end{frame}



\begin{frame}{Непрерывная капитализация процентов}
\centering
\begin{tikzpicture}
		\draw [->,>=triangle 90] (0, 0) -- (8.5, 0);

		\draw [->,>=triangle 45] (1,0) node[anchor=north east]{$0$} .. controls (1.5, 1) and (3.5, 1) .. (4,0) node[anchor=north]{$T_1=1$} node[pos=0.5,anchor=south]{$r_1^*=11.333\%$};

		\draw [->,>=triangle 45] (4,0) .. controls (4.5, 1) and (6.5, 1) .. (7,0) node[anchor=north west]{$T_2=2$} node[pos=0.5,anchor=south]{$x^*$};

		\draw [->,>=triangle 45] (1,0) .. controls (1.5, -1) and (6.5, -1) .. (7,0) node[pos=0.5,anchor=north]{$r_2^*=7.421\%$};
	\end{tikzpicture}

\justifying
Не должно быть разницы между цепочкой из <<непрерывных>> годовых депозитов и двухгодовым депозитом:
\begin{align*}
e^{r_1^*T_1}e^{x^*(T_2-T_1)} = e^{r_2^*T_2}
\end{align*}
\begin{align*}
x^* = \frac{r_2^*T_2-r_1^*T_1}{T_2-T_1} = \frac{7.421\%\cdot2 - 11.333\%}{2 - 1} \approx 3.509\%
\end{align*}

Пересчитаем <<непрерывные>> обратно в простые:
\begin{align*}
x = \frac{e^{x^*(T_2-T_1)} - 1}{T_2 - T_1} = e^{0.03509} - 1 \approx 3.57\% 
\end{align*}
\end{frame}



\begin{frame}{Непрерывные проценты}
\justify
Наблюдаемая действительность --- то, что люди готовы вложить $100$ рублей, чтобы получить $112$ рублей через год. Это можно интерпретировать и как <<простую>> ставку 12\% , и как <<непрерывную>> ставку 11.33\%, и как коэффициент дисконтирования $1/1.12 = 0.8929$. Это разные точки зрения (или разные единицы измерения) на одно и то же явление.

\justify 
В зависимости от контекста может быть удобнее пользоваться разными абстракциями. Можно показывать пользователям <<простые>> проценты, математику будущих ставок программировать в терминах <<непрерывных>> ставок, а \en{present value}\ будущих выплат вычислять с помощью коэффициентов дисконтирования.
\end{frame}



\begin{frame}{European Interbank Offered Rate}
\justify
\alert{EURIBOR} --- индикативная процентная ставка (\en{interest rate benchmark}) по беззалоговым кредитам в евро среди банков Еврозоны. Публикуется \en{European Money Market Institute (EMMI)}.

\justify
По какой ставке сферический крупный банк может занять евро у другого банка или финансовой организации на открытом рынке? 

\justify
\centering
\begin{tabular}{l|r}
Срок (tenor)     & EURIBOR 08.03.2023 \\ \hline
1W (1 неделя)    & $2.382\%$ \\
1M (1 месяц)     & $2.565\%$ \\
3M (3 месяца)    & $2.944\%$ \\
6M (6 месяцев)   & $3.451\%$ \\
12M (12 месяцев) & $3.944\%$ 
\end{tabular}

\justify
*EURIBOR --- ставка без капитализации в конвенции ACT/360.
\end{frame}



\begin{frame}{Вычисление EURIBOR}
\justify
В вычислении EURIBOR участвуют 18 крупных банков Еврозоны и UK. Каждый банк ежедневно отправляет свою ставку по каждому из пяти сроков (от недели до года). Банки обязаны следовать методологии, которая определяет 3 <<уровня>>:

\justify 
1) Cредневзвешенная ставка реальных сделок банка в этот день и на этот срок. Каждая сделка должна быть не меньше 20 млн.

\justify
2) Интерполяция ставок реальных сделок на соседние сроки (например, вычисление 6M из 3M и 12M), либо экстраполяция сделок предыдущих дней на данный срок.

\justify
3) Ставки сделок по похожим инструментам, сделки менее 20 миллионов, котировки деривативов, сделки с нефинансовыми организациями, экспертное суждение. 

\justify
15\% самых больших и 15\% самых маленьких котировок отбрасываются. Арифметическое среднее оставшихся 70\%, округлённое до третьего знака --- EURIBOR.
\end{frame}



\begin{frame}{Вычисление EURIBOR}
\centering
\begin{tikzpicture}
		\draw [->,>=triangle 90] (0, 0) -- (9.5, 0);

		\draw [dashed] (0.5,0) node[anchor=north]{$T-1$} .. controls (1.0, 0.5) and (2.5, 0.5) .. (3,0) node[anchor=north]{$T+1$} node[pos=0.5,anchor=south]{spot lag};

		\draw [->,>=triangle 45] (3,0) .. controls (4, 1) and (7, 1) .. (8,0) node[anchor=north]{$T+1+3M$} node[pos=0.5,anchor=south]{$EURIBOR_{T}$};

		\node[anchor=north] at (1.75, 0) {$T$};
		
		\node[circle, fill, inner sep=1.5pt] at (0.5, 0) {};
		\node[circle, fill, inner sep=1.5pt] at (1.75, 0) {};
		\node[circle, fill, inner sep=1.5pt] at (3.0, 0) {};
		\node[circle, fill, inner sep=1.5pt] at (8.0, 0) {};
	\end{tikzpicture}
	
\justify
T--1 (вчера):  банки заключали сделки и собирали статистику.

T (сегодня): в 11:00 опубликовано среднее арифметическое вчерашних сделок --- EURIBOR на три месяца за дату T.

T+1 (завтра): кредиты, о которых договорились вчера, вступят в силу (банки-заёмщики получат евро).

T+1+3M (через три месяца): кредиты истекут (банки-заёмщики вернут деньги).

\justify
EURIBOR публикуется в дни, когда работает платёжная система Европейского Центробанка TARGET2. Три месяца прибавляются по конвенции \en{modified following business day}.
\end{frame}



\begin{frame}{Конвенция о переносе дней}
\justify
Кредит на три месяца начался 31 марта 2023 года (пт). Когда он закончится: 30 июня (пт) или 3 июля (пн)?

\justify
Конвенция \en{modified following / business month end}:

\justify
1) Если дата начала --- последний рабочий день текущего месяца, то дата окончания --- тоже последний рабочий день месяца. Пример: 31.03.2022 (пт) --- 30.06.2023 (пт).

\justify
2) Если дата окончания выпала на выходной, то она сдвигается вперёд на ближайший 
рабочий день, если только ближайший рабочий день --- не в следующем месяце. Пример: 
17.03.2023 (пт) --- 19.06.2023 (пн).

\justify
3) В противном случае дата окончания сдвигается назад. Пример: 29.11.2022 (вт) --- 28.02.2023 (вт).
\end{frame}



\begin{frame}{Euro short-term rate}
\justify
\alert{ESTER} --- индикативная ставка по беззалоговым кредитам сроком на один день (\en{overnight}) среди банков Еврозоны. Публикуется Европейским центральным банком (ЕЦБ).

\justify
\centering
\begin{tabular}{l|r}
Срок   & ESTER 08.03.2023 \\ \hline
1 день & $2.399\%$
\end{tabular}

\justify
В определении ESTER участвует панель из 48-ми банков. Каждый банк в течение дня сообщает ЕЦБ обо всех своих сделках (когда банк берёт кредит на 1 день). ЕЦБ отбрасывает 25\% самых маленьких и 25\% самых больших значений и вычисляет средневзвешенную ставку.
\end{frame}



\begin{frame}{Вычисление ESTER}
\justify
\centering
\begin{tikzpicture}
		\draw [->,>=triangle 90] (0, 0) -- (4.0, 0);

		\draw [->, >= triangle 45] (0.5,0) node[anchor=north]{$T$} .. controls (1.0, 0.75) and (2.5, 0.75) .. (3,0) node[anchor=north]{$T+1$} node[pos=0.5,anchor=south]{$ESTER_T$};
\end{tikzpicture}

\justify
T (сегодня): банки берут кредиты (получают евро) и сообщают о сделках в ЕЦБ.

T+1 (завтра): ЕЦБ опубликует среднюю ставку вчерашних сделок --- ESTER за вчера. Кредиты банков истекают (банки отдают евро).

\justify
Как и EURIBOR, ESTER определяется в те дни, когда работает платёжная система TARGET2.
\end{frame}



\newcommand{\plotBenchmarkRate}[2] {
	
	\addplot[
		color = #2,
		mark = none,
		thick
	]
	table[
		x=date,
		y=#1,
		col sep=comma
	]
	{euro_benchmark.csv};
}



\begin{frame}{ESTER и EURIBOR}
\centering
\begin{tikzpicture}
\begin{axis}[
  width=\textwidth,
  height=\textheight - 1cm,
  date coordinates in=x,
  date ZERO=2012-01-01,
  xtick={2012-01-01, 2014-01-01, 2016-01-01, 2018-01-01, 2020-01-01, 2022-01-01, 2024-01-01},
  minor xtick={2013-01-01, 2015-01-01, 2017-01-01, 2019-01-01, 2021-01-01, 2023-01-01},
  ytick={-1, 0, 1, 2, 3, 4},
  minor ytick={-0.5, 0.5, 1.5, 2.5, 3.5},
  xticklabel={\year},
  xmin=2012-01-01,
  xmax=2024-01-01,
  ymin=-1,
  ymax=4,
  grid=both,
  yticklabel={\pgfmathprintnumber{\tick}\%},
%  ylabel={\small{Курс USDRUB}},
  xlabel near ticks,
  ylabel near ticks,
  legend entries = {
  	   EURIBOR 6M,
      EURIBOR 3M,
      ESTER
  },
  legend cell align={left},
  legend style={at={(0.03,0.97)},anchor=north west}
]

	\plotBenchmarkRate{6m}{Set1-A}
	\plotBenchmarkRate{3m}{Set1-B}
	\plotBenchmarkRate{ester}{Set1-C}
	
	\draw[thick, color=black] (axis cs: 2012-01-01, 0) -- (axis cs: 2024-01-01, 0);
\end{axis}
\end{tikzpicture}

\scriptsize Данные: Banco de Espana.
\end{frame}



\begin{frame}{Фьючерсы на EURIBOR}
\justify
Биржа \en{ICE Futures Europe}\ (бывшая LIFFE) предлагает фьючерсы на трёхмесячный EURIBOR. Например, \alert{FEIM3} --- фьючерс на июнь 2023 года*.

\justify
\centering
\begin{tabular}{r|r}
Покупка & Продажа \\ \hline
96.29 & 96.30
\end{tabular}

\justify
Если мы <<купим>> 10 фьючерсов по 96.30, то:
\begin{itemize}
\justifying
\item Прямо сейчас мы ничего не платим.
\item 19 июня 2023 года** биржа установит финальную цену фьючерса $F = 100 - X$, где 
$X$ --- трёхмесячный EURIBOR, опубликованный в этот день.
\item Мы заработаем $10 \cdot (F - 96.30) \cdot \EUR{2\,500}$, где \EUR{2\,500} --- 
константа из спецификации фьючерса.
\end{itemize}

\justify
*FEI --- код контракта. M --- код месяца (H --- март, M --- июнь, U --- сентябрь, Z --- 
декабрь). 3 --- последняя цифра года.

\justify
**Третья среда месяца минус два рабочих дня.
\end{frame}



\begin{frame}{Фьючерсы на EURIBOR}
\justify
Я заранее знаю, что в июне я разбогатею на \EUR{1\,000\,000}. Я планирую положить деньги на депозит в надёжном банке примерно под EURIBOR, чтобы в сентябре поехать в отпуск. У меня есть \alert{процентный риск} (\en{interest rate risk}): я не знаю, сколько денег у меня будет к сентябрю.

\justify
Решение: купить один фьючерс на EURIBOR по 96.30.

\justify
Допустим, к 19 июня EURIBOR снизится до $2.50\%$.
\begin{itemize}
\justifying
\item Биржа установит финальную цену $100 - 2.50 = 97.50$.
\item Прибыль по фьючерсу $(97.50-96.30)\cdot\EUR{2\,500} = \EUR{3\,000}$.
\item Депозит под EURIBOR $2.50\%$ на три месяца принесёт

$\EUR{1\,000\,000} \cdot 0.025 / 4 = \EUR{6\,250}$.
\item Итого, я заработаю $\EUR{3\,000} + \EUR{6\,250} = \EUR{9\,250}$.
\end{itemize} 

\justify
Я заработал бы столько же, если бы EURIBOR был $3.70\%$: $\EUR{1\,000\,000} \cdot 0.037/4 =\EUR{9\,250}$.
\end{frame}



\begin{frame}{Фьючерсы на EURIBOR}
\justify
Допустим, мы купили июньский фьючерс по 96.30 в марте. В июне мы открыли депозит под
EURIBOR, каким бы он ни был, до сентября. Сколько мы заработаем?

\justify
\centering
\begin{tabular}{r|r|r|r|r}
EURIBOR   & Цена фьючерса & Фьючерс         & Депозит & Итого \\ \hline
$2.50\%$ & 97.50        & $\EUR{3\,000}$ & $\EUR{6\,250}$  & $\EUR{9\,250}$ \\
$3.00\%$ & 97.00        & $\EUR{1\,750}$ & $\EUR{7\,500}$  & $\EUR{9\,250}$ \\
$3.50\%$ & 96.50        & $\EUR{500}$    & $\EUR{8\,750}$  & $\EUR{9\,250}$ \\
$3.70\%$ & 96.30        & $\EUR{0}$       & $\EUR{9\,250}$  & $\EUR{9\,250}$ \\
$4.00\%$ & 96.00        & $-\EUR{750}$    & $\EUR{10\,000}$ & $\EUR{9\,250}$ \\
$4.50\%$ & 95.50        & $-\EUR{2\,000}$ & $\EUR{11\,250}$ & $\EUR{9\,250}$
\end{tabular}

\justify
Покупая фьючерс по цене $100-X$, мы фиксируем ставку будущего депозита на \EUR{1\,000\,000} под EURIBOR. Если мы продаём фьючерс по цене $100-X$, то мы фиксируем будущую выплату по кредиту под плавающую ставку EURIBOR.
\end{frame}



\begin{frame}{Процентный своп}
\justify
\alert{Процентный своп} (\en{interest rate swap}) --- дериватив, в котором две стороны 
обмениваются процентными платежами по фиксированной ставке и по плавающей (референсной) 
ставке. 

\justify
Параметры свопа:
\begin{itemize}
\justifying
\item Срок (\en{term, maturity}) --- дата последнего платежа. Например, 1 год.
\item Размер (\en{notional amount, principal amount}) --- сумма, с которой платятся проценты. Например, \EUR{1} миллион.
\item Купон (\en{coupon}) или ставка свопа (\en{swap rate}) --- фиксированная ставка, которую платит одна сторона. Например, $3.0\%$ годовых.
\item Референсная ставка (\en{reference rate}) --- тип плавающей ставки, которую платит вторая сторона. Например, 3M EURIBOR.
\end{itemize}

\justify
Купить своп --- согласиться получать плавающую ставку и платить за неё фиксированную ставку. Продать своп --- согласиться платить плавающую ставку и получать за неё фиксированную.
\end{frame}



\begin{frame}{Процентный своп}
\centering
\begin{tikzpicture}[thick, scale=0.75]
		\draw (0, 0) node[rectangle,draw,rounded corners,anchor=south,minimum height=1cm] {Компания A} -- (0, -7);
		\draw (9.5, 0) node[rectangle,draw,rounded corners,anchor=south,minimum height=1cm] {Компания B} -- (9.5, -7);

		\draw [dashed,->,>=triangle 90] (0, -1) node[label=left:{$T_0$}]{} -- (4.5, -1) node[pos=0.5,anchor=south]{$\EUR{1\,000\,000}$};

		\draw [dashed,->,>=triangle 90] (9.5, -1) node[label=right:{\text{фиксинг $E_1$}}]{} -- (5, -1) node[pos=0.5,anchor=south]{\euro 1\,000\,000};

		\draw [->,>=triangle 90] (0, -2.5) node[label=left:{$T_1 = \text{3M}$}]{} -- (4.5, -2.5) node[pos=0.5,anchor=south]{$\EUR{1\,000\,000} \cdot \dfrac{3.0\%}{4}$};

		\draw [snake=snake,->,>=triangle 90] (9.5, -2.5)  node[label=right:{\text{фиксинг $E_2$}}]{} -- (5, -2.5) node[pos=0.5,anchor=south] {$\EUR{1\,000\,000} \cdot \dfrac{E_1}{4}$};

		\draw [->,>=triangle 90] (0, -4) node[label=left:{$T_2 = \text{6M}$}]{} -- (4.5, -4) node[pos=0.5,anchor=south]{$\EUR{1\,000\,000} \cdot \dfrac{3.0\%}{4}$};

		\draw [snake=snake,->,>=triangle 90] (9.5, -4)  node[label=right:{\text{фиксинг $E_3$}}]{} -- (5, -4) node[pos=0.5,anchor=south] {$\EUR{1\,000\,000} \cdot \dfrac{E_2}{4}$};

		\draw [->,>=triangle 90] (0, -5.5) node[label=left:{$T_3 = \text{9M}$}]{} -- (4.5, -5.5) node[pos=0.5,anchor=south]{$\EUR{1\,000\,000} \cdot \dfrac{3.0\%}{4}$};

		\draw [snake=snake,->,>=triangle 90] (9.5, -5.5)  node[label=right:{\text{фиксинг $E_4$}}]{} -- (5, -5.5) node[pos=0.5,anchor=south] {$\EUR{1\,000\,000} \cdot \dfrac{E_3}{4}$};
		\draw [->,>=triangle 90] (0, -7) node[label=left:{$T_4 = \text{12M}$}]{} -- (4.5, -7) node[pos=0.5,anchor=south]{$\EUR{1\,000\,000} \cdot \dfrac{3.0\%}{4}$};

		\draw [snake=snake,->,>=triangle 90] (9.5, -7) -- (5, -7) node[pos=0.5,anchor=south] {$\EUR{1\,000\,000} \cdot \dfrac{E_4}{4}$};
\end{tikzpicture}
\end{frame}



\begin{frame}{Управление процентным риском}
\justify
Процентный своп позволяет превратить кредит под плавающую ставку (например, EURIBOR+1\%) в кредит под фиксированную ставку (например, 4.0\%) и наоборот.

\justify
\centering
	\begin{tikzpicture}[thick]

		\tikzstyle{company}=[rectangle,draw,rounded corners,minimum height=1.2cm,minimum width=3cm];
		\tikzstyle{fixed}=[->,>=triangle 90];
		\tikzstyle{floating}=[snake=snake,->,>=triangle 90];

		\node (A) at (0, 0) [company] {Компания A};
		\node (B) at (7, 0) [company] {Компания B};
		\node (Deutsche) at (0, -3) [company] {Deutsche};
		\node (Citi) at (7, -3) [company] {Citi};

		\draw [fixed] ([yshift=0.2cm]A.east) -- ([yshift=0.2cm]B.west) node[pos=0.5,anchor=south]{3.0\%};

		\draw [floating] ([yshift=-0.2cm]B.west) -- ([yshift=-0.2cm]A.east) node[pos=0.5,anchor=north]{EURIBOR};

		\draw [floating] (A.south) -- (Deutsche.north) node[pos=0.5,anchor=west]{EURIBOR + 1\%};

		\draw [fixed] (B.south) -- (Citi.north) node[pos=0.5,anchor=east]{$3.5\%$};
	\end{tikzpicture}

\justify
Итого: A платит 4.0\%, B платит EURIBOR + 0.5\%.
\end{frame}



\begin{frame}{Кредиты под плавающую ставку}
\justify
Предположим, что мы --- компания с невысоким кредитным рейтингом. Нам нужен кредит на 5 лет на постройку свечного заводика.

\begin{itemize}
\justifying
\item Занять деньги на 3 месяца стоит EURIBOR + 1\%. 
\item Занять деньги на 5 лет стоит 7.0\%.
\item Процентный своп на 5 лет стоит $3.0\%$.
\end{itemize}

\justify
Банки закладывают риск дефолта в стоимость кредита, причём чем дольше срок, тем дороже кредит. Вероятность того, что мы разоримся в ближайшие три месяца не такая высокая (в конце концов, мы же дожили до сегодня). Банк готов нам дать деньги на 1\% дороже, чем занимает деньги сам.

\justify
Чем дальше в будущее, чем выше неопределённость. Никто не может предсказать, что случится с нашей компанией за 5 лет. Поэтому банк даёт кредит на 5 лет под более высокую ставку --- фактически, под EURIBOR + 4\%.
\end{frame}



\begin{frame}{Кредиты под плавающую ставку}
\justify
Мы можем понадеяться, что сможем и в будущем занимать деньги под EURIBOR + 1\%. Если раз в три месяца заново брать кредит на 3 месяца под EURIBOR+1\%, и продать процентный своп на 5 лет по $3.0\%$, то можно зафиксировать стоимость кредита 4.0\%.

\justify
Минус такой стратегии: если в будущем наше кредитное качество снизится (банки будут опасаться, что наш проект не взлетает), то стоимость трёхмесячных кредитов вырастет. Например, в какой-то день придётся занять деньги под EURIBOR+10\%, и вся экономия сломается.
\end{frame}



\begin{frame}{Честная цена свопа}
\centering
\begin{tikzpicture}[thick, scale=0.7]
		\draw (0, 0) node[rectangle,draw,rounded corners,anchor=south,minimum height=1cm] {Компания A} -- (0, -3);
		\draw (7.5, 0) node[rectangle,draw,rounded corners,anchor=south,minimum height=1cm] {Компания B} -- (7.5, -3);

		\draw [->,>=triangle 90] (0, -1) node[label=left:{$T_1$}]{} -- (3.5, -1) node[pos=0.5,anchor=south]{$N \cdot x \cdot t_1$};

		\draw [snake=snake,->,>=triangle 90] (7.5, -1)  -- (4, -1) node[pos=0.5,anchor=south] {$N \cdot E_1 \cdot t_1$};
		
		\draw [->,>=triangle 90] (0, -2.5) node[label=left:{$T_2$}]{} -- (3.5, -2.5) node[pos=0.5,anchor=south]{$N \cdot x \cdot t_2$};

		\draw [snake=snake,->,>=triangle 90] (7.5, -2.5) -- (4, -2.5) node[pos=0.5,anchor=south] {$N \cdot E_2 \cdot t_2$};

		\draw (0, -3.5) -- (0, -4.5);				
		\draw (7.5, -3.5) -- (7.5, -4.5);

		\draw [->,>=triangle 90] (0, -4) node[label=left:{$T_i$}]{} -- (3.5, -4) node[pos=0.5,anchor=south]{$N \cdot x \cdot t_i$};

		\draw [snake=snake,->,>=triangle 90] (7.5, -4) -- (4, -4) node[pos=0.5,anchor=south] {$N \cdot E_i \cdot t_i$};
\end{tikzpicture}

\justify
$N$ --- notional, или размер свопа.

$T_i$ --- количество лет до $i$-го платежа.

$E_i$ --- ожидаемые значения EURIBOR.

$t_i = T_i - T_{i-1}$ --- количество лет, на которое начисляется EURIBOR.

$\delta_i$ --- коэффициент дисконтирования для момент времени $T_i$.

\justify
Какой должна быть честная цена свопа (купон) $x$?
\end{frame}



\begin{frame}{Честная цена свопа}
\justify
Текущие стоимости (PV) платежей компаний A и B должны совпадать. В противном случае одна из компаний подарила другой деньги просто так.

\justify
\begin{align*}
\sum\limits_{i=1}^{n} N \cdot x \cdot t_i \cdot \delta_i 
= \sum\limits_{i=1}^{n} N \cdot E_i \cdot t_i \cdot \delta_i
\end{align*}
\begin{align*}
x = \frac{\sum\limits_{i=1}^{n} E_i t_i \delta_i}{\sum\limits_{i=1}^{n} t_i \delta_i}
\end{align*}

\justify
Если кто-то сообщит нам будущие значения EURIBOR $E_i$ и коэффициенты дисконтирования $\delta_i$, то мы сможем посчитать цену любого процентного свопа.
\end{frame}



\begin{frame}{Честная цена свопа}
\justify
Предположим, что ставка дисконтирования равна нулю, поэтому все коэффициенты дисконтирования близки к 1, т.е. $\delta_i \approx1$.

\justify
Также будем игнорировать особенности календаря и будем считать, что все интервалы между платежами равны. Например, $t_i = t = 1/4$ для свопа с ежеквартальными платежами.

\begin{align*}
x = \frac{\sum\limits_{i=1}^{n} E_i t_i \delta_i}{\sum\limits_{i=1}^{n} t_i \delta_i}
\approx
\frac{\sum\limits_{i=1}^{n} E_i t}{\sum\limits_{i=1}^{n} t}
=
\frac{\sum\limits_{i=1}^{n} E_i}{n}
\end{align*}

\justify
Вывод от Капитана Очевидность: цена свопа --- среднее будущих значений EURIBOR за время жизни свопа.
\end{frame}



\begin{frame}{Вычисление ожидаемых значений EURIBOR}
\justify
Предположим, что рыночный терминал Reuters или Bloomberg показывает нам следующую информацию о рынке свопов на EURIBOR 3M.

\justify
\centering
\begin{tabular}{l|r}
Инструмент        & Цена (купон) \\ \hline
Последний фиксинг & 3.00\% \\
Своп 6M           & 3.40\% \\
Своп 9M           & 3.60\% \\
Своп 1Y           & 3.70\% \\
Своп 18M          & 3.75\% \\
Своп 2Y           & 3.60\%
\end{tabular}

\end{frame}



\begin{frame}{Вычисление ожидаемых значений EURIBOR}
\centering
\begin{tikzpicture}[thick, scale=0.7]
		\draw (0, 0) node[rectangle,draw,rounded corners,anchor=south,minimum height=1cm] {Компания A} -- (0, -2.5);
		\draw (7.5, 0) node[rectangle,draw,rounded corners,anchor=south,minimum height=1cm] {Компания B} -- (7.5, -2.5);

		\draw [->,>=triangle 90] (0, -1) node[label=left:{3M}]{} -- (3.5, -1) node[pos=0.5,anchor=south]{$3.40\%$};

		\draw [snake=snake,->,>=triangle 90] (7.5, -1)  -- (4, -1) node[pos=0.5,anchor=south] {$3.00\%$};
		
		\draw [->,>=triangle 90] (0, -2.5) node[label=left:{6M}]{} -- (3.5, -2.5) node[pos=0.5,anchor=south]{$3.40\%$};

		\draw [snake=snake,->,>=triangle 90] (7.5, -2.5) -- (4, -2.5) node[pos=0.5,anchor=south] {$E_2$};
\end{tikzpicture}

\justify
Мы знаем цену свопа на 6 месяцев ($3.40\%$). Также мы знаем и текущее значение EURIBOR ($3.00\%$), которое будет использовано для первого плавающего платежа. Отсюда мы можем вычислить <<ожидаемое>> значение $E_2$.
\begin{align*}
3.40\% + 3.40\% &= 3.00\% + E_2 \Rightarrow \\
E_2 &= 3.80\%
\end{align*}

\end{frame}



\begin{frame}{Отступление об <<ожиданиях>> рынка}
\justify
Можем ли мы быть уверены, что <<ожидаемое>> рынком значение $E_2=3.80\%$ --- правильное? Нет не можем. Скорее всего, в реальности значение EURIBOR будет другим. В среднем, процентные ставки оказываются чуть ниже, чем <<ожидаемые>> ставки, которые можно было вычислить из котировок свопов.

\justify
<<Ожидаемая>> рынком ставка --- это всего-навсего ставка, которая соответствует текущим ценам свопов. Это два эквивалентных взгляда на одно и то же явление:

1) Текущий фиксинг $3.00\%$, своп на 6 месяцев $3.40\%$.

2) Текущий фиксинг $3.00\%$, ставка через 3 месяца $3.80\%$.
\end{frame}



\begin{frame}{Отступление об <<ожиданиях>> рынка}
\justify
Не так важно, ошибается ли рынок или нет, когда оценивает будущее значение $E_2$. Важно, что в нашем распоряжении есть ликвидные инструменты, которые позволяют захеджировать процентный риск и избавиться от всей неопределённости, связанной с $E_2$.

\justify
Допустим, мы должны будем заплатить кому-то за кредит по ставке $E_2$. Купим процентный своп на 6 месяцев.

\centering
\begin{tabular}{l|r|r|r}
Дата    & \multicolumn{2}{c|}{Своп} & Кредит \\ \hline
Сегодня &          0  &  0            & 0 \\
3M      & $-3.40\%$  &  $+3.00\%$   & 0 \\
6M      & $-3.40\%$  &  $+E_2$       & $-E_2$
\end{tabular}

\justify
Всего по всем платежам мы заплатим $3.80\%$ вне зависимости от будущего значения EURIBOR $E_2$. Процентный своп позволяет нам жить в мире, в котором EURIBOR $E_2$ всегда оказывается равен $3.80\%$.
\end{frame}



\begin{frame}{Вычисление ожидаемых значений EURIBOR}
\centering
\begin{tikzpicture}[thick, scale=0.7]
		\draw (0, 0) node[rectangle,draw,rounded corners,anchor=south,minimum height=1cm] {Компания A} -- (0, -4);
		\draw (7.5, 0) node[rectangle,draw,rounded corners,anchor=south,minimum height=1cm] {Компания B} -- (7.5, -4);

		\draw [->,>=triangle 90] (0, -1) node[label=left:{3M}]{} -- (3.5, -1) node[pos=0.5,anchor=south]{$3.60\%$};

		\draw [snake=snake,->,>=triangle 90] (7.5, -1)  -- (4, -1) node[pos=0.5,anchor=south] {$3.00\%$};
		
		\draw [->,>=triangle 90] (0, -2.5) node[label=left:{6M}]{} -- (3.5, -2.5) node[pos=0.5,anchor=south]{$3.60\%$};

		\draw [snake=snake,->,>=triangle 90] (7.5, -2.5) -- (4, -2.5) node[pos=0.5,anchor=south] {$3.80\%$};
		
		\draw [->,>=triangle 90] (0, -4) node[label=left:{9M}]{} -- (3.5, -4) node[pos=0.5,anchor=south]{$3.60\%$};

		\draw [snake=snake,->,>=triangle 90] (7.5, -4) -- (4, -4) node[pos=0.5,anchor=south] {$E_3$};
\end{tikzpicture}

\justify
Знание цены свопа на 9 месяцев (3.60\%) и двух первых значений EURIBOR (3.00\% и 3.80\%) позволяет вычислить <<ожидаемое>> значение $E_3$.
\begin{align*}
3.60\% + 3.60\% + 3.60\% &= 3.0\% + 3.80\% + E_3 \Rightarrow \\
E_3 &= 4.00\%
\end{align*}
\end{frame}



\begin{frame}{Вычисление ожидаемых значений EURIBOR}
\centering
\begin{tikzpicture}[thick, scale=0.7]
		\draw (0, 0) node[rectangle,draw,rounded corners,anchor=south,minimum height=1cm] {Компания A} -- (0, -5.5);
		\draw (7.5, 0) node[rectangle,draw,rounded corners,anchor=south,minimum height=1cm] {Компания B} -- (7.5, -5.5);

		\draw [->,>=triangle 90] (0, -1) node[label=left:{3M}]{} -- (3.5, -1) node[pos=0.5,anchor=south]{$3.70\%$};

		\draw [snake=snake,->,>=triangle 90] (7.5, -1)  -- (4, -1) node[pos=0.5,anchor=south] {$3.00\%$};
		
		\draw [->,>=triangle 90] (0, -2.5) node[label=left:{6M}]{} -- (3.5, -2.5) node[pos=0.5,anchor=south]{$3.70\%$};

		\draw [snake=snake,->,>=triangle 90] (7.5, -2.5) -- (4, -2.5) node[pos=0.5,anchor=south] {$3.80\%$};
		
		\draw [->,>=triangle 90] (0, -4) node[label=left:{9M}]{} -- (3.5, -4) node[pos=0.5,anchor=south]{$3.70\%$};

		\draw [snake=snake,->,>=triangle 90] (7.5, -4) -- (4, -4) node[pos=0.5,anchor=south] {$4.00\%$};
		
		\draw [->,>=triangle 90] (0, -5.5) node[label=left:{12M}]{} -- (3.5, -5.5) node[pos=0.5,anchor=south]{$3.70\%$};

		\draw [snake=snake,->,>=triangle 90] (7.5, -5.5) -- (4, -5.5) node[pos=0.5,anchor=south] {$E_4$};
\end{tikzpicture}

\justify
Знание цены свопа на 12 месяцев (3.70\%) и трёх первых значений EURIBOR (3.00\%, 3.80\% и 4.00\%) позволяет вычислить <<ожидаемое>> значение $E_4$.
\begin{align*}
3.70\%+3.70\%+3.70\%+3.70\% &= 3.00\% + 3.80\% + 4.00\% + E_4 \Rightarrow \\
E_4 &= 4.00\%
\end{align*}
\end{frame}



\begin{frame}{Вычисление ожидаемых значений EURIBOR}
\centering
\begin{tikzpicture}[thick, scale=0.7]
		\draw (0, 0) node[rectangle,draw,rounded corners,anchor=south,minimum height=1cm] {Компания A} -- (0, -8.5);
		\draw (7.5, 0) node[rectangle,draw,rounded corners,anchor=south,minimum height=1cm] {Компания B} -- (7.5, -8.5);

		\draw [->,>=triangle 90] (0, -1) node[label=left:{3M}]{} -- (3.5, -1) node[pos=0.5,anchor=south]{$3.75\%$};

		\draw [snake=snake,->,>=triangle 90] (7.5, -1)  -- (4, -1) node[pos=0.5,anchor=south] {$3.00\%$};
		
		\draw [->,>=triangle 90] (0, -2.5) node[label=left:{6M}]{} -- (3.5, -2.5) node[pos=0.5,anchor=south]{$3.75\%$};

		\draw [snake=snake,->,>=triangle 90] (7.5, -2.5) -- (4, -2.5) node[pos=0.5,anchor=south] {$3.80\%$};
		
		\draw [->,>=triangle 90] (0, -4) node[label=left:{9M}]{} -- (3.5, -4) node[pos=0.5,anchor=south]{$3.75\%$};

		\draw [snake=snake,->,>=triangle 90] (7.5, -4) -- (4, -4) node[pos=0.5,anchor=south] {$4.00\%$};
		
		\draw [->,>=triangle 90] (0, -5.5) node[label=left:{12M}]{} -- (3.5, -5.5) node[pos=0.5,anchor=south]{$3.75\%$};

		\draw [snake=snake,->,>=triangle 90] (7.5, -5.5) -- (4, -5.5) node[pos=0.5,anchor=south] {$4.00\%$};
		
				\draw [->,>=triangle 90] (0, -7) node[label=left:{15M}]{} -- (3.5, -7) node[pos=0.5,anchor=south]{$3.75\%$};

		\draw [snake=snake,->,>=triangle 90] (7.5, -7) -- (4, -7) node[pos=0.5,anchor=south] {$E_5$};
		
				\draw [->,>=triangle 90] (0, -8.5) node[label=left:{18M}]{} -- (3.5, -8.5) node[pos=0.5,anchor=south]{$3.75\%$};

		\draw [snake=snake,->,>=triangle 90] (7.5, -8.5) -- (4, -8.5) node[pos=0.5,anchor=south] {$E_6$};
\end{tikzpicture}
\end{frame}



\begin{frame}{Вычисление ожидаемых значений EURIBOR}
\justify
У нас нет котировки свопа на 15 месяцев, есть только своп на 18 месяцев. Это даёт нам одно уравнение и два неизвестных значения EURIBOR $E_5$ и $E_6$.
\begin{align*}
3.00\%+3.80\%+4.00\%+4.00\%+E_5+E_6 = 6\cdot3.75\%
\end{align*}

\justify
Предположим, что будущие ставки изменяются линейно: на сколько $E_5$ больше $E_4$, на столько же $E_6$ больше, чем $E_5$.
\begin{align*}
\begin{cases}
3.00\%+3.80\%+4.00\%+4.00\%+E_5+E_6 = 6\cdot3.75\% \\
E_5 - 4.00\% = E_6 - E_5
\end{cases}
\end{align*}

С учётом предположения о линейной интерполяции:
\begin{align*}
\begin{cases}
E_5 = 3.90\% \\
E_6 = 3.80\%
\end{cases}
\end{align*}
\end{frame}



\begin{frame}{Вычисление ожидаемых значений EURIBOR}
\centering
\begin{tikzpicture}[thick, scale=0.7]
		\draw (0, 0) node[rectangle,draw,rounded corners,anchor=south,minimum height=1cm] {Компания A} -- (0, -2.5);
		\draw (7.5, 0) node[rectangle,draw,rounded corners,anchor=south,minimum height=1cm] {Компания B} -- (7.5, -2.5);

		\draw [->,>=triangle 90] (0, -1) node[label=left:{3M}]{} -- (3.5, -1) node[pos=0.5,anchor=south]{$S_{6M}t_1\alert{\delta_1}$};

		\draw [snake=snake,->,>=triangle 90] (7.5, -1)  -- (4, -1) node[pos=0.5,anchor=south] {$E_1t_1\alert{\delta_1}$};
		
		\draw [->,>=triangle 90] (0, -2.5) node[label=left:{6M}]{} -- (3.5, -2.5) node[pos=0.5,anchor=south]{$S_{6M}t_2\alert{\delta_2}$};

		\draw [snake=snake,->,>=triangle 90] (7.5, -2.5) -- (4, -2.5) node[pos=0.5,anchor=south] {$\alert{E_2}t_2\alert{\delta_2}$};
\end{tikzpicture}

\justify
Вернёмся в реальность и добавим в модель коэффициенты дисконтирования на 3 и 6 месяцев $\delta_1$ и $\delta_2$. Обычно они не известны заранее, поэтому у нас получается одно уравнение с тремя неизвестными.
\begin{align*}
S_{6M}t_1\alert{\delta_1} + S_{6M}t_2\alert{\delta_2} = E_1t_1\delta_1 + \alert{E_2}t_2\alert{\delta_2}
\end{align*}

\justify
Как быть?
\end{frame}



\begin{frame}{Вычисление ожидаемых значений EURIBOR}
\justify
Предположим, что ставки трёхмесячного EURIBOR являются безрисковыми. Это разумное предположение, потому что о проблемах у банка обычно становится известно заранее.

\justify
Мы можем построить цепочку из безрисковых депозитов под EURIBOR:

1) Выбрать надёжный банк и вложить деньги на три месяца под EURIBOR.

2) Подождать три месяца.

3) Снять деньги и вернуться к пункту 1.

\justify
\centering
\begin{tikzpicture}
		\draw [->,>=triangle 90] (0.5, 0) -- (11, 0);

		\draw [->,>=triangle 45] (0.5,0) node[anchor=north east]{0} .. controls (1, 0.5) and (2, 0.5) .. (2.5,0) node[anchor=north]{$T_1$} node[pos=0.5,anchor=south]{$E_1$};

		\draw [->,>=triangle 45] (2.5,0) .. controls (3, 0.5) and (4, 0.5) .. (4.5,0) node[anchor=north]{$T_2$} node[pos=0.5,anchor=south]{$E_2$};

		\draw [->,>=triangle 45] (4.5,0) .. controls (5, 0.5) and (6, 0.5) .. (6.5,0) node[anchor=north]{$T_3$} node[pos=0.5,anchor=south]{$E_3$};
		
		\draw [dashed] (6.5,0) .. controls (7, 0.5) and (8, 0.5) .. (8.5,0);
		
		\draw [->,>=triangle 45] (8.5,0) node[anchor=north]{$T_{n-1}$} .. controls (9, 0.5) and (10, 0.5) .. (10.5,0) node[anchor=north]{$T_n$} node[pos=0.5,anchor=south]{$E_n$};
	\end{tikzpicture}
%\justify
%Каждый раз на шаге 1 мы можем (но не обязаны) выбирать новый банк, если кредитный рейтинг предыдущего банка нам больше не нравится.
\end{frame}



\begin{frame}{Вычисление ожидаемых значений EURIBOR}
\centering
\begin{tikzpicture}
		\draw [->,>=triangle 90] (0.5, 0) -- (11, 0);

		\draw [->,>=triangle 45] (0.5,0) node[anchor=north]{$T_0$} .. controls (1, 0.5) and (2, 0.5) .. (2.5,0) node[anchor=north]{$T_1$} node[pos=0.5,anchor=south]{$E_1$};

		\draw [->,>=triangle 45] (2.5,0) .. controls (3, 0.5) and (4, 0.5) .. (4.5,0) node[anchor=north]{$T_2$} node[pos=0.5,anchor=south]{$E_2$};

		\draw [->,>=triangle 45] (4.5,0) .. controls (5, 0.5) and (6, 0.5) .. (6.5,0) node[anchor=north]{$T_3$} node[pos=0.5,anchor=south]{$E_3$};
		
		\draw [dashed] (6.5,0) .. controls (7, 0.5) and (8, 0.5) .. (8.5,0);
		
		\draw [->,>=triangle 45] (8.5,0) node[anchor=north]{$T_{n-1}$} .. controls (9, 0.5) and (10, 0.5) .. (10.5,0) node[anchor=north]{$T_n$} node[pos=0.5,anchor=south]{$E_n$};
		
		\node [anchor=north] at (1.5, 0) {$t_1$};
		\node [anchor=north] at (3.5, 0) {$t_2$};
		\node [anchor=north] at (5.5, 0) {$t_3$};
		\node [anchor=north] at (9.5, 0) {$t_n$};
	\end{tikzpicture}
	
\justify
Если мы вложили $N_0$ евро в цепочку депозитов, то сколько евро будет на счету к моменту времени $T_n$?
\begin{align*}
N_n = N_0(1 + E_1t_1)(1 + E_2t_2) ... (1 + E_nt_n)
\end{align*}

\justify
Чему равен коэффициент дисконтирования на дату окончания цепочки $T_n$?
\begin{align*}
\delta_n = \frac{1}{(1 + E_1t_1)(1 + E_2t_2) ... (1 + E_nt_n)}
\end{align*}
\end{frame}



\begin{frame}{Вычисление ожидаемых значений EURIBOR}
\centering
\begin{tikzpicture}[thick, scale=0.7]
		\draw (0, 0) node[rectangle,draw,rounded corners,anchor=south,minimum height=1cm] {Компания A} -- (0, -2.5);
		\draw (7.5, 0) node[rectangle,draw,rounded corners,anchor=south,minimum height=1cm] {Компания B} -- (7.5, -2.5);

		\draw [->,>=triangle 90] (0, -1) node[label=left:{3M}]{} -- (3.5, -1) node[pos=0.5,anchor=south]{$S_{6M}t_1\alert{\delta_1}$};

		\draw [snake=snake,->,>=triangle 90] (7.5, -1)  -- (4, -1) node[pos=0.5,anchor=south] {$E_1t_1\alert{\delta_1}$};
		
		\draw [->,>=triangle 90] (0, -2.5) node[label=left:{6M}]{} -- (3.5, -2.5) node[pos=0.5,anchor=south]{$S_{6M}t_2\alert{\delta_2}$};

		\draw [snake=snake,->,>=triangle 90] (7.5, -2.5) -- (4, -2.5) node[pos=0.5,anchor=south] {$\alert{E_2}t_2\alert{\delta_2}$};
\end{tikzpicture}

\justify
Мы только что связали будущие значения EURIBOR $E_i$ с коэффициентами дисконтирования $\delta_i$. Теперь у нас столько уже уравнений, сколько и неизвестных.
\begin{align*}
\begin{cases}
S_{6M}t_1\alert{\delta_1} + S_{6M}t_2\alert{\delta_2} = E_1t_1\delta_1 + \alert{E_2}t_2\alert{\delta_2} \\
\alert{\delta_1} = \dfrac{1}{1+E_1t_1} \\
\alert{\delta_2} = \dfrac{1}{(1+E_1t_1)(1+\alert{E_2}t_2)}
\end{cases}
\Rightarrow E_2 \approx 3.804\%
\end{align*}
\end{frame}



\begin{frame}{Метод bootstrap}
\justify
Если рассматривать EURIBOR как безрисковую ставку, то удобно представлять кривую будущих значений как совокупность коэффициентов дисконтирования $\delta_i$. Будущее значение EURIBOR для депозита, который начнётся через $T_1$ лет и закончится через $T_2 = T_1 + 0.25$ лет равно
\begin{align*}
E(T_1, T_2) = \frac{\dfrac{\delta_{T_1}}{\delta_{T_2}} - 1}{T_2 - T_1}
\end{align*}

\justify
Алгоритм пошагового расчёта коэффициентов дисконтирования и будущих ставок из котировок рыночных инструментов называется \alert{\en{bootstrap}}. Он реализован во многих финансовых библиотеках, таких как QuantLib и tf-quant-finance. 
\end{frame}



\begin{frame}{Пример: bootstrap}
\justify
Рассмотрим котировки свопов на трёхмесячный EURIBOR: 

\justify
\centering
\begin{tabular}{l|r|l|r}
Срок & Купон & Срок & Купон \\ \hline
3m   & 3.00\% & 4y   & 3.25\% \\
6m   & 3.40\% & 5y   & 3.15\% \\
9m   & 3.60\% & 6y   & 3.10\% \\
1y   & 3.70\% & 7y   & 3.05\% \\
18m  & 3.75\% & 8y   & 3.00\% \\
2y   & 3.60\% & 9y   & 3.00\% \\
3y   & 3.40\% & 10y  & 3.00\% 
\end{tabular}
\end{frame}



\begin{frame}{Пример: интерполяция}
\begin{tikzpicture}
\begin{axis}[
  width=\textwidth,
  height=\textheight - 1cm,
  date coordinates in=x,
  date ZERO=2012-01-01,
  xtick={2023-01-01, 2024-01-01, 2025-01-01, 2026-01-01, 2027-01-01, 2028-01-01, 2029-01-01, 2030-01-01},
  %minor xtick={2013-01-01, 2015-01-01, 2017-01-01, 2019-01-01, 2021-01-01},
  %ytick={-0.5, 0, 0.5, 1.0, 1.5},
  %minor ytick={-0.75, -0.25, 0.25, 0.75, 1.25},
  xticklabel={\year},
  xmin=2023-03-13,
  xmax=2030-01-01,
  %ymin=-0.75,
  %ymax=1.5,
  grid=major,
  ylabel={\small{EURIBOR, б.п.}},
  xlabel near ticks,
  ylabel near ticks,
  legend entries = {	
  	   Линейная,
  	   Сплайн
  },
  legend cell align={left},
  legend pos = north east
]

	\addplot[
		color = Set1-A,
		mark = none,
		very thick
	]
	table[
		x=date,
		y=linear,
		col sep=comma
	]
	{euribor_bootstrap.csv};
	
	\addplot[
		color = Set1-B,
		mark = none,
		very thick
	]
	table[
		x=date,
		y=spline,
		col sep=comma
	]
	{euribor_bootstrap.csv};
	
	\draw[thick, color=black] (axis cs: 2012-01-01, 0) -- (axis cs: 2030-01-01, 0);
\end{axis}
\end{tikzpicture}
\end{frame}



\begin{frame}{Пример: нестандартный своп}
\justify
Рассмотрим нестандартный своп, который начнётся через 2.5 года и закончится чрез 5 лет 
(30M--5Y).

\justify
Поскольку мы знаем будущие значения EURIBOR и коэффициентов дисконтирования, мы можем 
подобрать такой купон, что PV свопа было равно 0.

\justify
\centering
\begin{tabular}{l|l|l}
& Receive & Pay \\ \hline
Effective date & 15.09.2025 & 15.09.2025 \\ 
Maturity date  & 15.03.2028 & 15.03.2028 \\
Notional       & \EUR{1\,000\,000} & \EUR{1\,000\,000} \\
Frquency       & Semi-annually     & Quarterly \\
Rate           & \alert{2.7646\%}          & EURIBOR-3M   \\
Basis          & ACT/360           & ACT/360
\end{tabular}
\end{frame}



\begin{frame}{Пример: риск}
\justify
PV01 (present value of 1 bp) или DV01 (dollar value of 1 bp): на сколько изменится
PV, если процентная ставка изменится на 1 базисный пункт.

\justify
\centering
\begin{tabular}{l|r|r}
Срок  & PV01 & Размер хэджа \\ \hline
18m   & $-23$  евро/б.п. & $+\EUR{158\,000}$ \\
2y    & $+129$ евро/б.п. & $-\EUR{668\,000}$ \\
3y    & $+155$ евро/б.п. & $-\EUR{545\,000}$ \\
4y    & $-23$  евро/б.п. & $+\EUR{62\,000}$ \\
5y    & $-455$ евро/б.п. & $+\EUR{989\,000}$ \\ \hline
Итого & $-217$ евро/б.п. & $-\EUR{4\,000}$
\end{tabular}

\justify
Правило большого пальца для PV01: размер * срок / 10\,000.
\end{frame}



\begin{frame}{Кругом обман}
\centering
\makebox[\textwidth]{\includegraphics[width=\textwidth]{we_need_to_go_deeper.jpg}}
\end{frame}



\begin{frame}{Дисконтирование по EURIBOR}
\justify
Мы использовали трёхмесячный EURIBOR для дисконтирования будущих платежей. Но есть же, например, шестимесячный EURIBOR и свопы на шестимесячный EURIBOR. По аналогии, чтобы забутсрапить кривую шестимесячного EURIBOR мы должны были бы дисконтировать платежи по этой ставке (притвориться, что строим цепочку из шестимесячных депозитов).

\justify
Ещё хуже: есть базисные свопы (\en{tenor basis swap}), в которых одна сторона платит шестимесячный EURIBOR, а другая --- трёхмесячный с доплатой (базисом). По какой ставке  дисконтировать платежи в этих свопах? Хотелось бы, чтобы коэффициент дисконтирования на дату $T$ не зависел от того, в какой задаче мы его используем.

\justify
Кроме того, в определении EURIBOR участвуют банки с кредитным рейтингом BBB. Действительно ли это безрисковая ставка?
\end{frame}



\begin{frame}{Базисный процентный своп}
\justify
\begin{tikzpicture}[thick, scale=0.75]
		\draw (0, 0) node[rectangle,draw,rounded corners,anchor=south,minimum height=1cm] {Компания A} -- (0, -7);
		\draw (10, 0) node[rectangle,draw,rounded corners,anchor=south,minimum height=1cm] {Компания B} -- (10, -7);

		\draw [dashed,->,>=triangle 90] (0, -1) node[label=left:{$T_0$}]{} -- (4, -1) node[pos=0.5,anchor=south]{$\EUR{1\,000\,000}$};

		\draw [dashed,->,>=triangle 90] (10, -1) node[label=right:{\setlength\tabcolsep{1pt}\begin{tabular}{l}фиксинг\\$E_1^{3M}$ и $E_1^{6M}$\end{tabular}}]{} -- (4.5, -1) node[pos=0.5,anchor=south]{\euro 1\,000\,000};

		\node [label=left:{$3M$}] at (0, -2.5) {};

		\draw [snake=snake,->,>=triangle 90] (10, -2.5)  node[label=right:{\text{фиксинг $E_2^{3M}$}}]{} -- (4.5, -2.5) node[pos=0.5,anchor=south] {$\EUR{1\,000\,000} \cdot \dfrac{E_1^{3M} + 0.1\%}{4}$};

		\draw [snake=snake,->,>=triangle 90] (0, -4) node[label=left:{$6M$}]{} -- (4, -4) node[pos=0.5,anchor=south]{$\EUR{1\,000\,000} \cdot \dfrac{E_1^{6M}}{2}$};

		\draw [snake=snake,->,>=triangle 90] (10, -4)  node[label=right:{\setlength\tabcolsep{1pt}\begin{tabular}{l}фиксинг\\$E_3^{3M}$ и $E_2^{6M}$\end{tabular}}]{} -- (4.5, -4) node[pos=0.5,anchor=south] {$\EUR{1\,000\,000} \cdot \dfrac{E_2^{3M} + 0.1\%}{4}$};

		\node [label=left:{$9M$}] at (0, -5.5) {};
		
		\draw [snake=snake,->,>=triangle 90] (10, -5.5)  node[label=right:{\text{фиксинг $E_4^{3M}$}}]{} -- (4.5, -5.5) node[pos=0.5,anchor=south] {$\EUR{1\,000\,000} \cdot \dfrac{E_3^{3M} + 0.1\%}{4}$};
		\draw [snake=snake,->,>=triangle 90] (0, -7) node[label=left:{$12M$}]{} -- (4, -7) node[pos=0.5,anchor=south]{$\EUR{1\,000\,000} \cdot \dfrac{E_2^{6M}}{2}$};

		\draw [snake=snake,->,>=triangle 90] (10, -7) -- (4.5, -7) node[pos=0.5,anchor=south] {$\EUR{1\,000\,000} \cdot \dfrac{E_4^{3M} + 0.1\%}{4}$};
\end{tikzpicture}

\justify
$0.1\%$ --- базис или цена свопа.
\end{frame}



\begin{frame}{Ставки овернайт}
\justify
Если даже депозит в банке по EURIBOR на 3 или 6 месяцев --- слишком рискованная инвестиция, как можно уменьшить риск? Можно размещать деньги не на квартал, а на день (точнее, на ночь). Придётся каждое утро следить за новостями и выбирать наиболее надёжный банк на следующие сутки, зато получится очень надёжно.

\justify
Под какую ставку мы сможем вкладывать деньги на день? Скорее всего --- близко к ESTER, средней ставке по депозитам овернайт в крупных банках.

\justify
Осталось только разобраться с процентным риском. Начиная цепочку депозитов сейчас, при ESTER $2.40\%$, мы не знаем, по какой ставке будем открывать вклады через год. Вдруг ставки упадут ещё ниже?
\end{frame}



\begin{frame}{Overnight index swap}
\justify
Своп на индекс овернайт (\en{overnight index swap, OIS}) --- контракт, в котором одна сторона платит фиксированную ставку, а другая --- <<среднюю>> ставку овернайт (например, ESTER) за период.

\justify
\centering
\begin{tikzpicture}[thick, scale=0.75]
		\draw (0, 0) node[rectangle,draw,rounded corners,anchor=south,minimum height=1cm] {Компания A} -- (0, -2.5);
		\draw (10.5, 0) node[rectangle,draw,rounded corners,anchor=south,minimum height=1cm] {Компания B} -- (10.5, -2.5);

		\draw [->,>=triangle 90] (0, -1) node[label=left:{1Y}]{} -- (5, -1) node[pos=0.5,anchor=south]{$\EUR{1\,000\,000} \cdot (2.55\%)$};

		\draw [snake=snake,->,>=triangle 90] (10.5, -1) -- (5.5, -1) node[pos=0.5,anchor=south] {$\EUR{1\,000\,000} \cdot \overline{E_1}$};


		\draw [->,>=triangle 90] (0, -2.5) node[label=left:{2Y}]{} -- (5, -2.5) node[pos=0.5,anchor=south]{$\EUR{1\,000\,000} \cdot (2.55\%)$};

		\draw [snake=snake,->,>=triangle 90] (10.5, -2.5) -- (5.5, -2.5) node[pos=0.5,anchor=south] {$\EUR{1\,000\,000} \cdot \overline{E_2}$};
\end{tikzpicture}

\justify
$\overline{E_1}$ и $\overline{E_2}$ --- средние ставки ESTER ($e_i$) за первый и второй год:
\begin{align*}
\overline{E_1} &= \left(1 + \dfrac{e_1}{360}\right)
\left(1 + \dfrac{e_2}{360}\right)
...
\left(1 + \dfrac{e_{365}}{360}\right) - 1 \\
\overline{E_2} &= \left(1 + \dfrac{e_{366}}{360}\right)
\left(1 + \dfrac{e_{367}}{360}\right)...
\left(1 + \dfrac{e_{730}}{360}\right) - 1
\end{align*}

\justify
Фиксированная ставка $2.55\%$ --- цена свопа.
\end{frame}



\begin{frame}{Дисконтирование OIS}
\justify
Цепочка из однодневных депозитов под ESTER, в которой процентный риск захеджирован OIS свопом --- лучшее приближение теоретической безрисковой процентной ставки. Текущий рыночный консенсус --- использовать дисконтирование по OIS для оценки деривативов.

\justify
Дело в шляпе! У нас есть котировки OIS свопов, из которых методом bootstrap можно вычислить и будущие значения ESTER, и связанные с ними коэффициенты дисконтирования. Потом эти коэффициенты дисконтирования можно использовать для вычисления значений EURIBOR из свопов на EURIBOR.
\end{frame}



\begin{frame}{Дисконтирование OIS}
\justify
К сожалению, рынок OIS ликвиден только для свопов на сроки до одного-двух лет. Дальше участники предпочитают свопы EURIBOR-OIS (одна сторона платит трёхмесячный EURIBOR, другая --- среднюю ESTER плюс базис). Нужно одновременно вычислять и значения EURIBOR, и значения ESTER, и коэффициенты дисконтирования.

\justify
<<До кучи>> можно добавить в модель и шестимесячный EURIBOR. Его тоже обычно восстанавливают не из простых свопов, а из базисных свопов 3M-6M.

\justify
Получается одна огромная задача оптимизации, которая одновременно подбирает будущие значения ESTER, трёхмесячного EURIBOR и шестимесячного EURIBOR.
\end{frame}



%\begin{frame}{Демонстрация}
%\begin{itemize}
%\item Калибрация кривых ESTER и EURIBOR.
%\item Вычисление цены свопа на шестимесячный EURIBOR.
%\end{itemize}
%\end{frame}



\begin{frame}{Кругом обман}
\centering
\makebox[\textwidth]{\includegraphics[width=\textwidth]{we_need_to_go_deeper.jpg}}
\end{frame}



\begin{frame}{Если всего этого мало}
\justify
Коэффициенты дисконтирования, которые нужно применять к платежам по сделке, зависят от соглашения о гарантийном обеспечении (\en{credit support annex, CSA}). Начисление процентов на гарантийное обеспечение по ставке овернайт --- стандартная рыночная практика. Поэтому дисконтирование по OIS работает довольно часто, не не всегда.

\justify
Кроме того, кривые OIS в двух валютах не воспроизводят цены форвардов, потому что не являются идеальными безрисковыми ставками. Существует валютный базис (\en{cross-currency basis}), который тоже нужно учитывать при оценке деривативов.

\justify
Подробности в \en{Marc Henrard, Interest Rate Modelling in the Multi-Curve Framework (2014)}.
\end{frame}



\begin{frame}{Бонус: LIBOR}
\justify
В 2023 году прекратит существование ставка LIBOR (\en{London Interbank Offered Rate}). Она определяется по результатам опроса панели банков. Каждый рабочий день банки-участники отвечают на следующий вопрос:

\justify
\en{At what rate could you borrow funds, were you to do so by asking for and then accepting inter-bank offers in a reasonable market size just prior to 11am London time?}

\justify
25\% ответов снизу и сверху отбрасываются. По оставшимся считается средняя арифметическая ставка.

\justify
Когда-то LIBOR считали для 10 валют и 15 различных сроков от овернайта до 1 года. Сейчас --- на 5 валют и 7 сроков.
\end{frame}



\begin{frame}{Бонус: LIBOR}
\justify
Значения LIBOR на 11.10.2021 (базисные пункты):
\justify
\centering
\begin{tabular}{l|r|r|r|r|r}
Срок & CHF       & EUR       & GBP      & JPY      & USD \\ \hline
ON   & $-78.120$ & $-58.957$ & $4.025$  & $-9.883$ & $7.263$ \\
1W   & $-81.540$ & $-58.142$ & $4.425$  & $-8.400$ & $7.613$ \\
1M   & $-78.280$ & $-57.229$ & $6.413$  & $-7.050$ & $8.538$ \\
2M   & $-76.000$ & $-55.943$ & $9.175$  & $-5.050$ & $10.175$ \\
3M   & $-76.300$ & $-56.914$ & $11.363$ & $-7.933$ & $12.175$ \\
6M   & $-71.480$ & $-53.829$ & $24.325$ & $-4.483$ & $15.650$ \\
12M  & $-58.200$ & $-48.657$ & $51.938$ & $6.317$  & $25.663$
\end{tabular}
\end{frame}



\begin{frame}{Бонус: LIBOR}
\justify
LIBOR --- результат экспертного опроса. Методология, вообще говоря, не требует, чтобы банки называли свои процентные ставки исходя из реально заключённых сделок.

\justify
В начале 2010-х выяснилось, что сотрудники нескольких банков вошли в сговор и заранее договаривались отправлять чуть заниженные или завышенные показания. Так банки манипулировали значением LIBOR и зарабатывали дополнительные деньги на процентных свопах, привязанных к LIBOR.

\justify
Итоги скандала: миллиардные штрафы, тюремные сроки, коренная реформа всего института референсных ставок.
\end{frame}



\begin{frame}{Бонус: LIBOR}
\justify
Долларовый LIBOR уступит место новой ставке \alert{SOFR} (\en{Secured Overnight Funding Rate}). Это ставка однодневных кредитов под залог ценных бумаг. В отличие от LIBOR, этот индикатор усредняет реальные сделки на огромном рынке.

\justify
SOFR заменит LIBOR во всех контрактах. Там, где нужна была одна ставка трёхмесячного LIBOR, появится средняя ставка SOFR за три месяца (по аналогии с OIS свопами) плюс откалиброванная константа 0.26\%. Начиная с 01.01.2022 все новые контракты должны использовать SOFR, а не LIBOR.
\end{frame}
\end{document}