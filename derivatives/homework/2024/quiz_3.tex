\documentclass[a4paper,12pt]{extarticle}
\usepackage{cmap}				% To be able to copy-paste russian text from pdf		
\usepackage[utf8]{inputenc}
\usepackage[T2A]{fontenc}
\usepackage[margin=1in]{geometry}
\usepackage[english, russian]{babel}

\usepackage{libertine}
\usepackage{libertinust1math}
   
\begin{document}
\thispagestyle{empty}

\noindent 1. Рассмотрим купленный европейский пут-опцион на пару доллар-рубль со страйком 60 рублей за доллар. Текущий курс доллар-рубль 90. Чему, скорее всего, равна дельта такого опциона?

A. Близка к $0\%$.

B. Близка к $-25\%$

C. Близка к $-50\%$

B. Близка к $-100\%$

\vspace{0.5cm}

\noindent 2. Трейдер Артём купил несколько ванильных коллов и путов и регулярно их дельта-хеджирует. Артём заметил, что периодически рынок базового актива резко двигается вверх или вниз, но опционная позиция зарабатывает деньги при любом направлении движения. Какой риск помогает Артёму?

A. Дельта

B. Гамма

C. Вега

D. Тета

\vspace{0.5cm}

\noindent 3. Почему, если трейдер хочет захеджировать гамму, ему придётся торговать опционами? Почему не получится захеджировать гамму сделкой с базовым активом?

A. Гамма базового актива и гамма опциона имеют противоположный знак.

B. Рынок базового актива недостаточно ликвидный, чтобы набрать достаточно гаммы.

C. Гамма базового актива равна нулю, поэтому сделка с ним не изменит гамму портфеля.

D. На биржах торгуются либо опционы, либо базовый актив, но не оба одновременно.

\vspace{0.5cm}

\noindent 4. Доходность годовой бескупонной облигации большого банка составляет 7\%. Такая же безрисковая государственная облигация имеет доходность 5\%. При банкротстве банка recovery rate составит 50\%. Какова вероятность дефолта банка в течение года, по мнению участников рынка?

A. 1\%

B. 2\%

C. 3\%

D. 4\%

\vspace{0.5cm}

\noindent 5. Почему мы не используем кредитные рейтинги для оценки вероятностей дефолта эмитентов и для ценообразования кредитных свопов?

A. Рейтинги отражают относительную вероятность дефолта, а абсолютная зависит от других факторов, таких как бизнес-цикл.

B. Рейтинги публикуются с задержкой.

C. Агентства рейтингуют только высококачественные бумаги.

D. Это требование пост-кризисного законодательства.
\end{document}