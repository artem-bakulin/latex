\documentclass[a4paper,14pt]{extarticle}
\usepackage{cmap}				% To be able to copy-paste russian text from pdf			
\usepackage[utf8]{inputenc}
\usepackage[T2A]{fontenc}
\usepackage[margin=1in]{geometry}
\usepackage[english, russian]{babel}

\usepackage{amsmath}
\usepackage{amsfonts}
\usepackage{siunitx}

\usepackage[hyphens]{url}
\urlstyle{same}
\usepackage{hyperref}

\usepackage{tikz}
\usepackage{pgfplots}
\pgfplotsset{compat=1.17}
%\usetikzlibrary{arrows,snakes,backgrounds,shapes}
\usepgfplotslibrary{colorbrewer}

\usepackage{libertine}
\usepackage{libertinust1math}

\newcommand{\ru}[1]{\begin{otherlanguage}{russian}#1\end{otherlanguage}}
\newcommand{\en}[1]{\begin{otherlanguage}{english}#1\end{otherlanguage}}

\begin{document}


\noindent \textbf{Срок сдачи --- среда 1 мая 23:59 MSK.}

\vspace{\baselineskip}

\noindent Это бонусное домашнее задание на программирование можно сдать в виде ноутбука на Python. Пожалуйста, организуйте свой код так,
чтобы все промежуточные этапы, описанные в задании, очевидным образом соответствовали блокам вашего кода. У проверяющего должна 
быть возможность убедиться, что промежуточные результаты правильные. Если где-то ошибка, например на этапе 3, то было бы хорошо, чтобы 
проверяющий мог подменить результаты этого этапа и запустить ноутбук дальше. Так вы не потеряете баллы за все последующие этапы.

\section{Структурная нота (7 баллов)}

Посчитайте цену структурной ноты --- барьерного опциона на корзину из акций, 
купоны по которой зависят от траектории цены худшей из четырёх акций. В 
материалах на my.NES вы найдёте файл Excel, который повторяет все расчёты.

Структурная нота имеет номинал \$100\,000 и срок погашения 5 лет. В конце каждого года нота может выплатить (а может и не выплатить) фиксированный купон 15\%. Если не случится досрочного погашения (об этом ниже), то инвестор получит назад весь номинал через 5 лет.

Рассмотрим корзину из четырёх акций A, B, C, D. Каждая акция сейчас стоит
\$100, не платит дивидендов, имеет волатильность в модели Блэка-Шоулза 30\%.
Попарные корреляции между акциями одинаковые и равны 0.40. Безрисковая 
процентная ставка 5\% (непрерывная капитализация).

В самый первый день в момент выпуска ноты фиксируется текущая цена каждой из
акций $S_{A,0}$, $S_{B,0}$, $S_{C,0}$, $S_{D,0}$. Каждый купон по ноте
--- это европейский цифровой опцион на худшую из четырёх акций в корзине со страйком
$K=80\%$ от начальной цены. Более формально, выплата в конце года $t$ равна
\begin{align*}
V_t &= \begin{cases}\$100\,000 \cdot 15\%, M_t > 80\% \\
\$0, M_t \le 80\%
\end{cases}\\
M_t &= \min\left(\frac{S_{A,t}}{S_{A,0}}; \frac{S_{B,t}}{S_{B,0}}; \frac{S_{C,t}}{S_{C,0}}; \frac{S_{C,t}}{S_{C,0}}\right) 
\end{align*}

В ноте есть барьер с <<выбиванием>> (knock out). Если в дату купона худшая акция стоит $B=130\%$ своей начальной цены или выше, то нота досрочно погашается, а инвестор досрочно получает весь номинал $\$100\,000$. В качестве компенсации владелец также получает фиксированную выплату (\en{rebate}) $R=5\%$ от номинала, то есть $\$5\,000$.

Ни в дату пробития барьера, ни в последующие даты купонов, нота уже не выплачивает опционную часть. Например, если уже в конце первого года худшая акция стоит 130\% начальной цены, то владелец ноты получит 5\% компенсации за пробитие барьера, получит назад номинал ноты, но не получит дополнительные 15\% обычного купона.

\subsection{Случайные величины (1 балл)}

В риск-нейтральном мире модели Блэка-Шоулза цены активов следуют лог-нормальному
распределению. Предположим, что в начале периода цена акции $S_0$, а длина
периода до следующего измерения цены равна $\tau$ лет. Тогда цена акции на 
конец периода --- случайная величина
\begin{align}
S(\tau) = S_0\exp\left[\left(r - q - \dfrac{\sigma^2}{2}\right)\tau +
\sigma\xi\sqrt{\tau}\right], \quad \xi \sim \mathcal{N}(0, 1)
\label{brownian_final_price}
\end{align}
Здесь $S(\tau)$ --- цена акции на конец периода длиной $\tau$, $r$ ---
безрисоквая ставка, $q$ --- дивидендная доходность, $\sigma$ --- волатильность акции, $\xi$ --- стандартная 
нормальная величина, которая задаёт рост или падение акции относительно 
матожидания.

В нашем случае нам нужно проследить динамику корзины из 4-х акций на 
протяжении 5-ти периодов, чтобы узнать выплаты по структурной ноте в конце 
первого года, второго, и так далее. Итого один <<сценарий>> задётся 20-ю 
стандартными нормальными величинами $\xi_i^j$, где $i=1..4$ --- номер акции, 
$j=1..5$ --- номер периода (первый год, второй и так далее).

Из условия мы знаем, что в пределах одного периода акции связаны корреляцией
0.4. То есть для любого периода $j$ четыре случайные величины $\xi_1^j,...,
\xi_4^j$ связаны корреляцией 0.4. В модели Блэка-Шоулза нет <<памяти>>, поэтому
для разных периодов $j$ и $k$ соответствующие им случайные величины $\xi_i^j$ и
$\xi_i^k$ являются независимыми.

Получается, что один сценарий в методе Монте-Карло --- это 5 групп по 4 
скоррелированные случайные величины. Если вы хотите вычислить цену ноты на 
выборке из 10\,000 сценариев, вам нужно 50\,000 наборов по 4 скоррелированные 
величины. Если вы используете Python, то вы можете воспользоваться методом
\href{https://numpy.org/doc/stable/reference/random/generated/numpy.random.Generator.multivariate\_normal.html}{multivariate\_normal}.

Не забудьте зафиксировать в коде какой-то конкретный seed, иначе каждый раз
ваш код будет выдавать разные результаты!

Таблица \ref{random_sample_table} --- пример того, как могут выглядеть случайные симуляции. 
\begin{table}[h]
\centering
\begin{tabular}{l|r|r|r|r}
Набор & $\xi_1$ & $\xi_2$ & $\xi_3$ & $\xi_4$ \\ \hline
1 & $0.770$ &  $0.288$ & $-1.116$ &  $0.622$ \\
2 &$-2.060$ & $-0.295$ & $-1.288$ &  $0.205$ \\
3 & $0.398$ &  $0.194$ &  $0.621$ & $-0.222$ \\
4 & $1.568$ &  $0.702$ &  $1.592$ &  $0.917$ \\
5 &$-2.049$ & $-0.868$ & $-1.196$ & $-0.630$ \\
<...> & <...> & <...> & <...> & <...>
\end{tabular}
\caption{Пример нескольких случайных скоррелированных наборов по 4 переменные}
\label{random_sample_table}
\end{table}

\subsection{Траектории цен акций (2 балла)}

Теперь мы можем перейти к ценам акций из корзины на каждую дату выплаты купонов.
Для этого нужно последовательно применить формулу (\ref{brownian_final_price}).
Самое сложное --- не напутать с индексами. Например, первая четвёрка случайных
чисел задаёт движение цен в первый период первого сценария, вторая --- второй
период в том же сценарии и так далее. В конечном итоге должно получиться что-то
вроде таблицы \ref{table_stock_prices}.

\begin{table}[h]
\centering
\begin{tabular}{l|l|r|r|r|r|r}
Сценарий & Акция & 1 год & 2 года & 3 года & 4 года & 5 лет \\
\hline
1 & A & \$122.88 &  \$64.60 &  \$70.98 & \$110.81 &  \$58.45 \\
  & B & \$106.34 &  \$94.95 &  \$98.15 & \$118.17 &  \$88.83 \\
  & C &  \$69.78 &  \$46.24 &  \$54.34 &  \$85.45 &  \$58.22 \\
  & D & \$117.54 & \$121.90 & \$111.24 & \$142.87 & \$115.34 \\
\hline
2 & A &  \$75.13 & \$101.71 & \$111.90 &  \$94.56 & \$150.59 \\
  & B &  \$74.59 &  \$57.54 &  \$58.53 &  \$34.52 &  \$22.76 \\
  & C & \$117.70 & \$104.97 & \$172.62 & \$146.31 & \$114.91 \\
  & D &  \$77.03 &  \$70.02 &  \$97.54 &  \$70.80 &  \$49.77
\end{tabular}
\caption{Пример двух сценариев изменения цен акций}
\label{table_stock_prices}
\end{table}

Чтобы убедиться, что ваши симуляции цен акций имеют смысл, посчитайте
какой-нибудь ванильный опцион. Например, колл-опцион на акцию B с датой 
исполнения 
5 лет и страйком \$100. Средняя дисконтированная выплата на вашей выборке должна 
более-менее совпасть с теоретической премией из формулы Блэка-Шоулза.

\subsection{Цена структурной ноты (2 балла)}

Теперь вам нужно посчитать, какими будут платежи по структурной ноте в каждом
из сценариев. Нужно ещё раз перечитать описание контракта и аккуратно учесть все 
случаи. Советую в качестве промежуточного результата посчитать цену худшей акции
в процентах от начальной цены. У вас получится что-то вроде таблицы 
\ref{table_note_payments}.

\begin{table}[h]
\centering
\small
\begin{tabular}{l|l|r|r|r|r|r}
Сценарий & Параметр & 1 год  & 2 года & 3 года & 4 года & 5 лет \\
\hline
Барьер не пробит & Худшая акция & 90\% & 85\% &	80\%	& 65\% & 85\% \\
            & Выплаты      & \$15\,000 &       \$15\,000 & \$0 &    \$0 & \$115\,000 \\
\hline
Барьер пробит & Худшая акция & 100\% & 95\% & 120\%	& 130\% & 120\% \\
            & Выплаты      & \$15\,000 &  \$15\,000  &   \$15\,000 &    \$105\,000 & \$0 \\ 

\end{tabular}
\caption{Выплаты по структурной ноте в нескольких сценариях}
\label{table_note_payments}
\end{table}

Теперь осталось только дисконтировать все платежи по безрисковой ставке и 
посчитать среднюю дисконтированную выплату по ноте во всех сценариях. Это и 
будет её теоретическая средняя цена.

\subsection{Дельта и вега (1 балл)}

Посчитайте дельту и вегу структурной ноты. Конечно, аналитической формулы нет,
поэтому нужно запустить метод Монте-Карло несколько раз.

Запомните, сколько нота стоит сейчас. Затем увеличьте цену акции A на 1\%, а 
страйк оставьте прежним. Так мы моделируем ситуацию, что мы только что купили 
ноту с зафиксированным начальной ценой акции A на уровне \$100, и сразу же после этого акция А подрожала.
Сколько нота стоит сейчас? Если цена ноты изменилась на \$50, то можно оценить 
дельту как $\$50 / 0.01 = \$5\,000$. То есть в ноте столько же риска, как если
бы мы купили акций A на \$5\,000. 

Теперь посчитайте вегу. Увеличьте волатильность одной из акций на 1 процентный 
пункт. Изменение цены ноты даст нам вегу, которую обычно так и измеряют ---
в долларах на процентный пункт.

Теперь мы знаем не только цену ноты, но и как можно приблизительно реплицировать 
её при помощи торговли акциями и ванильными опционами на отдельные акции.

\subsection{Seed variance (1 балл)}

Посчитайте seed variance вашего метода Монте-Карло. Попробуйте 10 или 20
разных значений seed и посмотрите, как изменяется цена ноты. Посчитайте среднюю
цену по всем seed и выборочное стандартное отклонение.

\end{document}