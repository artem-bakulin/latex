\documentclass[a4paper,14pt]{extarticle}
\usepackage{cmap}				% To be able to copy-paste russian text from pdf			
\usepackage[utf8]{inputenc}
\usepackage[T2A]{fontenc}
\usepackage[margin=1in]{geometry}
\usepackage[english, russian]{babel}

\usepackage{amsmath}
\usepackage{amsfonts}
\usepackage{siunitx}

\usepackage[hyphens]{url}
\urlstyle{same}
\usepackage{hyperref}

\usepackage{tikz}
\usepackage{pgfplots}
\pgfplotsset{compat=1.17}
%\usetikzlibrary{arrows,snakes,backgrounds,shapes}
\usepgfplotslibrary{colorbrewer}

\usepackage{libertine}
\usepackage{libertinust1math}
\usepackage{eurosym}

\newcommand{\ru}[1]{\begin{otherlanguage}{russian}#1\end{otherlanguage}}
\newcommand{\en}[1]{\begin{otherlanguage}{english}#1\end{otherlanguage}}

\begin{document}

\noindent Экзамен состоит из тестовой части на 10 вопросов (суммарно 10 баллов)  
и 4 задач (суммарно 30 баллов). Задачи можно сдавать на бумаге, в 
виде Excel с расчётами, в виде ноутбука на Python --- главное, чтобы был 
понятен ход решения. Во время экзамена вы можете пользоваться чем угодно, кроме
помощи других людей и нейросетей.

\section{Тест (10 баллов)}

\noindent В каждом вопросе ровно один вариант ответа правильный.

\vspace{\baselineskip}

\noindent 1. Инвестбанк работает маркет-мейкером валютной пары \en{GBPUSD}\ (фунт-доллар). Он купил у одного клиента 1\,000\,000 GBP по курсу 1.25\underline{10} (за 1 фунт дают 1.2510 доллара) и продал другому клиенту 1\,000\,000 GBP по 1.25\underline{05}. Сколько банк заработал или потерял на этой паре сделок?

A. Потерял 500 долларов

B. Потерял 500 фунтов

C. Заработал 500 долларов

D. Заработал 500 фунтов
 
\vspace{\baselineskip}

\noindent 2. Спот-курс валютной пары EURJPY (евро-иена) равен 160. Процентная ставка в евро 2\%, в иенах 0\% (простые проценты без капитализации). Чему равен справедливый форвардный курс для форварда с поставкой через 3 месяца (1/4 года)?

A. 156.9

B. 159.2

C. 160.8

D. 163.2

\vspace{\baselineskip}

\noindent 3. Евровый депозит на год приносит 4\% годовых, а депозит на два года ---
3\% годовых (обе ставки с непрерывной капитализацией). Какова справедливая ставка по
отложенному депозиту, который начнётся через год и закончится через два?

A. 1\%

B. 2\%

C. 3\%

D. 4\%

\vspace{\baselineskip}

\noindent 4. Почему некоторые заёмщики предпочитают кредиты под плавающую ставку, привязанную к стоимости фондирования банков?

A. В этом случае клиенты могут использовать в качестве залога долговые ценные бумаги, а не только физические активы.

B. Такие кредиты можно не учитывать в балансе по МСФО..

C. Некоторые виды кредитов, такие как ипотека, в принципе невозможно структурировать с фиксированной процентной ставкой.

D. В среднем такой кредит может обойтись дешевле, потому что банк не будет закладывать в ставку \en{term risk premium}.

\vspace{\baselineskip}

\noindent 5. Июньский фьючерс на EURIBOR стоит 96.0. Трейдер продал один такой фьючерс.
В каком случае он заработает прибыль на своей ставке?

A. Если и только если в июне EURIBOR окажется больше 0.96\%

B. Если и только если в июне EURIBOR окажется больше 4.0\%

C. Если и только если в июне EURIBOR окажется меньше 0.96\%

D. Если и только если в июне EURIBOR окажется меньше 4.0\%

\vspace{\baselineskip}

\noindent 6. Один крупный американский производитель самолётов обеспокоен возможным ростом цен на алюминий. Ему хотелось бы, чтобы через год он мог купить алюминий хотя бы по \$2\,700 за тонну или дешевле. Какую опционную сделку нужно заключить? 

A. Купить пут со страйком \$2\,700

B. Продать пут со страйком \$2\,700

С. Купить колл со страйком \$2\,700

В. Продать колл со страйком \$2\,700

\vspace{\baselineskip}

\noindent 7. Трейдер купил европейский ванильный пут на нефть со страйком \$80 за премию
\$2 и купил европейский ванильный колл со страйком \$80 за премию \$1. В  момент исполнения 
нефть стоит \$78 за баррель. Сколько суммарно заработал или потерял трейдер, учитывая заплаченные премии.

A. Потерял \$3

B. Потерял \$1

C. Заработал \$1

D. Заработал \$3

\vspace{\baselineskip}

\noindent 8. Трейдер Артём купил по дешёвке колл-опцион на валютную пару USDRUB (доллар-рубль). Однако выяснилось, что Артём проспал пресс-конференцию Эльвиры Набиуллиной, на которой она объявила о снижении процентной ставки в рублях на 2 процентных пункта. Риск-система сразу же показала Артёму, что опцион в его портфеле подешевел, хотя ни курс доллар-рубль, ни волатильность опционов не поменялись. Какой риск нанёс Артёму удар в спину?

A. Вега

B. Гамма

C. Дельта

D. Ро

\vspace{\baselineskip}

\noindent 9. Годовая бескупонная облигация крупного английского банка имеет доходность 2\%. Такая же безрисковая годовая бескупонная государственная облигация Соединённого Королевства имеет доходность 1.5\%. В случае банкротства банка \en{recovery rate}\ составит 50\%. Чему равна вероятность дефолта банка по мнению участников рынка?

A. 4\%

B. 3\%

C. 2\%

D. 1\%

\vspace{\baselineskip}

\noindent 10. Почему азиатский опцион при прочих равных обычно дешевле, чем европейский ванильный опцион?

A. Не существует аналитической формулы для цены азиатского опциона

B. Азиатский опцион сложнее хеджировать, если к моменту экспирации цена базового актива оказывается вблизи страйка

C. Усреднение цены базового актива в течение жизни азиатского опциона уменьшает неопределённость и разброс (волатильность) будущей выплаты

D. В азиатские опционы всегда встраиваются барьеры, которые делают их дешевле

\section{Задачи (30 баллов)} 

\subsection{Артём и забытый форвард (6 баллов)}

Три месяца назад трейдер Артём купил форвард на пару EURUSD (евро-доллар) в количестве $N=10\,000\,000$ евро. В то время спот-курс был на уровне $S_0=1.06$ (за 1 евро давали 1.06 доллара), безрисковая ставка в долларах составляла $r_0=5\%$, в евро $q=3\%$ (обе ставки с непрерывной капитализацией), до даты поставки оставалось ровно $T_0=1/2$ года. Артём купил форвард ровно по справедливому курсу, без бид-аск спреда.

К сожалению, форвард не попал в риск-систему из-за технической ошибки, а Артём про него забыл. Форвард нашли только сегодня, когда до даты поставки осталось $T_1=1/4$ года. Спот-курс евро-доллар сегодня $S_1=1.05$. Безрисковая ставка в долларах тоже поменялась, теперь она составляет $r_1=5.5\%$ (непрерывная капитализация).

Чему был равен форвардный курс в самом начале (три месяца назад, за полгода до поставки по форварду)? Чему равен форвардный курс для той же даты поставки сегодня? Предположим, что Артём закроет позицию, купив или продав новый форвард с поставкой через $T_1=1/4$ года так, чтобы платежи в евро схлопнулись в ноль. Какую прибыль или убыток он зафиксирует? Дайте ответ в сегодняшних долларах (то есть с учётом дисконтирования по долларовой ставке).

\subsection{Кредитный своп (6 баллов)}

В таблице \ref{bond_yields} приведены доходности бескупонных государственных облигаций, которые мы считаем безрисковыми, и бескупонных корпоративных облигаций некоторого эмитента. Все доходности даны в виде процентных ставок с ежегодной капитализацией. Будем считать, что в случае дефолта по корпоративной облигации владелец получит остаточную стоимость в ту же дату, в которую должнен был получить номинал облигации. \en{Recovery rate}\ в случае дефолта составит $R=40\%$.

\begin{table}[h]
\centering
\begin{tabular}{l|r|r}
Срок & Безрисковая & Корпоративная \\ \hline
1 год & 4.4\% & 4.6\% \\
2 года & 4.5\% & 4.7\% \\
3 года & 4.6\% & 4.9\% \\
4 года & 4.8\% & 5.2\% \\
5 лет & 5.0\% & 5.5\% 
\end{tabular}
\caption {Доходности облигаций}
\label{bond_yields}
\end{table}

Рассмотрим пятилетний кредитный дефолтный своп, в котором выплата купона происходит раз в год. При наступлении страхового случая покупатель свопа получит вылату в следующую купонную дату. Выплатой накопленного купона можно пренебречь.

Чему равен честный купон в таком свопе? Пожалуйста, не пренебрегайте дисконтированием будущих платежей по безрисковой ставке.

Предположим, что на рынке станадартный купон в свопах на эмитентов инвестиционного рейтинга --- 1\% годовых. Номинал нашего свопа $N=\$1\,000\,000$. Сколько долларов покупатель должен заплатить продавцу (или наоборот) в начале жизни свопа, чтобы своп с фиксированным купоном 1\% остался честным?

\subsection{Улучшенная бабочка (9 баллов)}

Улучшенная бабочка --- опционная структура, которая задаётся страйком $K$, <<размахом>> $dK$ и размером <<улучшения>> $I$. Выплата по улучшенной бабочке $V(S_T)$ зависит только от цены базового актива на момент экспирации $S_T$. График выплаты приведён на рисунке \ref{fly_payoff}. Выплата определяется по следующей формуле:
\begin{align*}
V(S_T) = \begin{cases}0, S_T \le K - dK \\
I + dK - |S_T - K|, K - dK < S_T \le K + dK \\
0, S_T > K + dK
\end{cases}
\end{align*}

\begin{figure}[h]
\centering
\begin{tikzpicture}
\begin{axis}[
	width = 15cm,
	height = 6cm,
	xmin = 100,
	xmax = 200,
	ymin = 0,
	ymax = 27,
	xtick = {135, 150, 165},
	xticklabels = {$K-dK$, $K$, $K+dK$},
	ytick = {0, 10},
	yticklabels = {0, $I$}
]

 \addplot[color=red, very thick, domain=100:135] {0.1};
\addplot[color=red, very thick, domain=135.001:164.999] {25 - abs(\x-150)};
 \addplot[color=red, very thick, domain=165:200] {0.1};
\end{axis}
\end{tikzpicture}
\caption{Выплата по структуре <<улучшенная бабочка>>}
\label{fly_payoff}
\end{figure}

Как составить такую улучшенную бабочку из европейских ванильных и/или европейских цифровых опционов? Предположим, что мы живём в мире Блэка-Шоулза. Текущая цена акции $S=\$150$, волатильность $\sigma=30\%$, безрисковая ставка $r=5\%$, дивидендная доходность $q=0\%$, до экспирации бабочки осталось $T=0.5$ года. Сколько стоит бабочка со страйком $K=\$150$, размахом $dK=\$25$ и улучшением $I=\$10$?

Посчитайте вегу такой бабочки  в трёх сценариях: если текущая цена акции $S_1=\$100$, $S_2=\$150$, $S_3=200$. Дайте ответ в долларах на процентный пункт.

\subsection{Хеджирование структурной ноты (9 баллов)}

Структурная нота с частичной защитой капитала устроена следующим образом. Инвестор вкладывает начальный капитал $N=\$100\,000$ и условно <<покупает>> на них 20 <<акций>> индекса S\&P\,500 по рыночной цене $S_0=\$5\,000$ за штуку.

Если через $T=1$ год индекс снижается до уровня $S_T < S_0$, но остаётся выше уровня защиты $B=\$4\,800$, то инвестор участвует в убытках полностью. Он получит назад $N/S_0 \cdot S_T$ долларов. Например, если индекс упал до \$4\,900, то инвестор потеряет по \$100 на каждой из 20 <<акций>> или \$2\,000 всего.

Если индекс падает до уровня защиты $B=\$4\,800$ или ниже, то вступает в действие защита. Инвестор получит назад $N/S_0 \cdot B$ долларов, даже если индекс улетит в ноль. Другими словами, возможные убытки инвестора ограничены \$200 на акцию индекса.

Взамен за защиту инвестор соглашается на ограниченное участие в прибыли от роста индекса. Коэффициент участия $p$ --- параметр ноты, который вы должны вычислить. Если в конце срока индекс забирается выше текущего уровня, то инвестор получит назад $N(1+ p \cdot (S_T - S_0)/S_0)$ долларов. Например, если индекс вырастет за год до \$5\,300, а коэффициент участия равен 60\%, то инвестор заработает по \$180 с каждой акции индекса или \$3\,600 всего.

Предположим, что мы живём в мире Блэка-Шоулза. Волатильность индекса $\sigma=25\%$, безрисковая ставка $r=5\%$, дивидендная доходность $q=2\%$. Чему равен честный коэффициент участия $p$? По необходимости используйте цифровые опционы и/или ванильные опционы и/или безрисковые депозиты и/или сделки с базовым активом.

Представьте, что вы \underline{продали} клиенту такую ноту. Вы не смогли найти на рынке необходимые для репликации опционы, поэтому вы будете заниматься дельта-репликацией. Сколько единиц базового актива (<<акций>> индекса) вам нужно купить или продать прямо сейчас по рыночной цене $S_0$, чтобы дельта вашего портфеля (проданная нота плюс базовый актив) стала равна нулю?


\end{document}