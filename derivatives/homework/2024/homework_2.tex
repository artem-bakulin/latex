\documentclass[a4paper,14pt]{extarticle}
\usepackage{cmap}				% To be able to copy-paste russian text from pdf			
\usepackage[utf8]{inputenc}
\usepackage[T2A]{fontenc}
\usepackage[margin=1in]{geometry}
\usepackage[english, russian]{babel}

\usepackage{amsmath}
\usepackage{amsfonts}

\usepackage[hyphens]{url}
\urlstyle{same}
\usepackage{hyperref}

\usepackage{tikz}
\usepackage{pgfplots}
\pgfplotsset{compat=1.17}
\usepgfplotslibrary{colorbrewer}

\usepackage{libertine}
\usepackage{libertinust1math}

\newcommand{\ru}[1]{\begin{otherlanguage}{russian}#1\end{otherlanguage}}
\newcommand{\en}[1]{\begin{otherlanguage}{english}#1\end{otherlanguage}}

\begin{document}


\noindent \textbf{Срок сдачи --- воскресенье 7 апреля 23:59 MSK.}

\vspace{\baselineskip}

\noindent Домашнее задание состоит из 6 теоретических задач (суммарно 15 баллов). Задачи 
можно сдавать в виде файлов .doc или .pdf, в виде Excel с расчётами, в виде ноутбука на Python --- главное, чтобы был понятен ход решения. Вы получите 1 бонусный балл, если сдадите всё домашнее задание в виде ноутбука на Python.

\section{Ванильный колл (2 балла)}

Рассмотрим одношаговую биномиальную модель. Шаг в дереве составляет 1 месяц ($\tau=1/12$
года). Акция, которая на платит дивидендов, стоит сегодня $S=\$100$. Акция
может вырасти либо в $u=1.2$ раза, либо в $d=0.8$ раза. Безрисковая процентная ставка $r=0\%$. 
Сколько стоит европейский колл-опцион со страйком $K=\$105$ и датой исполнения через 1 месяц?
Предположим, что на рынке такой колл стоит \$5. Предъявите арбитражную стратегию, которая
позволит заработать на ошибке рынка.

\section{Цифровой пут (2 балла)}

Мы обсудили на лекции, что европейский цифровой колл можно приблизить комбинацией ванильных опционов. Мы сказали, что приближение неточное, потому что для точной репликации нам бы потребовалось бесконечное количество опционов. Так устроен мир математических теорий, в котором рыночные цены могу принимать любое значение из множества действительных чисел $\mathbb{R}$ .

Мы живём в реальном мире, в котором цены, как правило, дискретные, а не непрерывные. Например, референсный курс \en{EURUSD World Money / Reuters (WMR) London 4pm}, к которому привязаны многие деривативы, округляется до 5 знака после запятой. Этот курс в принципе не может принять значение 1.12\underline{34}56, он может быть либо 1.12\underline{34}5, либо 1.12\underline{34}6. Аналогично, межбанковская торговая система \en{EBS}\ в принципе не позволит выставить заявку с ценой длиннее, чем 5 знаков после запятой.

Рассмотрим европейский цифовой пут на валютную пару \en{EURUSD}\ (евро-доллар) со страйком $K=1.07\underline{00}$. Этот опцион выплатит $N=100\,000$ долларов, если в дату экспирации опубликованный курс \en{WMR London 4pm}\ будет в точности 1.07\underline{00} или ниже. Текущий спот-курс евро-доллар $S=1.08$, срок экспирации $T=0.25$ лет, волатильность в модели Блэка-Шоулза $\sigma=10\%$, безрисковые ставки в долларах и евро $r=5\%$ и $q=3\%$ соответственно (непрерывная капитализация процентов). Сколько стоит такой цифровой пут в модели Блэка-Шоулза?

В вашем распоряжении ликвидный рынок ванильных путов, привязанных к тому же референсному курсу \en{WMR London 4pm}. Вы можете покупать и продавать ванильные путы с любыми страйками. Как можно в точности реплицировать цифровой пут? Сколько и каких путов нужно продать или купить? Сколько стоит такой реплицирующий портфель в модели Блэка-Шоулза? 

Указание. В модели Блэка-Шоулза первая валюта пары (евро) --- <<акция>>, вторая (доллар) --- <<деньги>>.

\section{Стрэддл с барьерами (2 балла)}

Стрэддл с барьерами похож на обычный стрэддл из лекций, но превращается в тыкву, если цена базового актива уходит очень далеко от страйка. График выплаты приведён на рисунке \ref{staddle_payoff}. Такой стрэддл задаётся центральным страйком $K$, нижним барьером $B_1$ и верхним барьером $B_2$, причём $B_1 < K < B_2$. Выплата по барьерному стрэддлу в зависимости от цены базового актива на момент экспирации $S(T)$ равна
\begin{align*}
Payoff = \begin{cases}
0, S(T) \le B_1\\
|S(T) - K|, B_1 < S(T) < B_2 \\
0, S(T) \ge B_2
\end{cases}
\end{align*}

\begin{figure}[h]
\centering
\begin{tikzpicture}
\begin{axis}[
	width = 15cm,
	height = 6cm,
	xmin = 100,
	xmax = 200,
	ymin = 0,
	ymax = 27,
	xtick = {125, 150, 175},
	xticklabels = {$B_1$, $K$, $B_2$},
	ytick = {0}
]

 \addplot[color=red, very thick, domain=100:125] {0.1};
\addplot[color=red, very thick, domain=125.001:174.999] {abs(\x-150)};
 \addplot[color=red, very thick, domain=175:200] {0.1};
\end{axis}
\end{tikzpicture}
\caption{Выплата по структуре <<стрэддл с барьерами>>}
\label{staddle_payoff}
\end{figure}

Как реплицировать такой стрэддл  при помощи европейских ванильных и/или цифровых опционов? Предположим, что мы 
живём в мире Блэка-Шоулза. Акция, которая не платит дивидендов, стоит $S=\$100$. Волатильность акции $\sigma=20\%$, безрисковая ставка $r=5\%$ (непрерывная капитализация). Сколько стоит барьерный стрэддл со сроком погашения $T=1$ год, страйком $K=\$105$, барьерами $B_1=\$80$ и $B_2=\$130$?

Указание. Считайте, что цифровой опцион платит 1 единицу валюты при точном попадании в страйк.

\section{Стрэддл в биномиальном дереве (3 балла)}

Рассмотрим трёхшаговую биномиальную модель. Каждый шаг составляет 3 месяца ($\tau=0.25$ года). Акция, которая не платит дивидендов, стоит сейчас $S=\$200$. Безрисковая процентная ставка $r=4\%$ (простые проценты без капитализации). На каждом шаге акция может либо вырасти в $u=1.1$ раза, либо упасть в $d=0.9$ раза.

Сколько стоит стрэддл (комбинация из ванильных колла и пута) со страйком $K=\$207$ и датой экспирации через 9 месяцев (на третьем шаге дерева)? Рассчитайте также дельту в промежуточных узлах дерева. Решите сами, как вам удобнее: посчитать сразу стрэддл, или посчитать колл и пут по отдельности.

\section{Купоны и волатильность (3 балла)}

Рассмотрим структурный продукт, в котором инвестор зарабатывает повышенный купонный доход при сильном движении рынка в любом направлении.

Предположим, что клиент инвестирует номинал $N=\$100\,000$ в ноту, привязанную к индексу \en{S\&P\,500}. Срок жизни ноты $T=1$ год. В конце года инвестор гарантированно получит назад весь номинал. Текущий уровень индекса $S=\$5\,200$.

Четыре раза в течение жизни (через $t_1=0.25$, $t_2=0.5$, $t_3=0.75$ и $T=1$ год) нота может выплатить (а может и не выплатить)  купон в $x\%$ годовых от инвестированного номинала. Например, если $x=4\%$, и случилась выплата через $t_2=0.5$ года, то инвестор получает $N\cdot x\cdot(t_2-t_1) = \$100\,000\cdot4\%\cdot(0.5-0.25)=\$1\,000$.

Условие выплаты каждого из купонов: в купонную дату индекс \en{S\&P\,500}\ должен находиться либо на уровне $K_1=\$4\,700$ и ниже, либо на уровне $K_2=5\,700$ и выше. Если индекс останавливается между $K_1$ и $K_2$, то в эту дату купона не будет. Выплата или невыплата предыдущего купона не влияет на последующие.

Как реплицировать такую ноту при помощи безрискового депозита и/или европейских ванильных и/или европейских цифровых опционов?

Волатильность индекса в модели Блэка-Шоулза $\sigma=20\%$. Безрисковая процентная ставка $r=5\%$, дивидендная  доходность индекса $q=2\%$ (обе ставки с непрерывной капитализацией). Какова справедливая купонная ставка $x\%$ в такой ноте (проценты годовых, без капитализации)?

\section{Ограниченное участие в убытках (3 балла)}

Структурная нота предлагает инвестировать в индекс \en{S\&P\,500}\ с ограниченным участием в убытках от возможного снижения индекса.

Инвестор покупает ноту за $N=\$100\,000$. Текущий уровень индекса $S_0=\$5\,000$, то есть инвестор покупает 20 <<акций>> индекса. Если через $T=1$ год индекс оказывается между текущим уровнем $S_0$ и страйком $K=\$5\,200$, то инвестор получает назад свой капитал и ничего не зарабатывает сверху.

Если индекс вырос выше, чем $K=\$5\,200$, то инвестор не только получает назад номинал $N$, но и участвует в прибыли от роста индекса по формуле $(S_T - K) \cdot N / S_0$, где $S_T$ --- уровень индекса через год. Например, если индекс вырос до $S_T = \$5\,500$, то инвестор зарабатывает $(\$5\,500-\$5\,200)\cdot\$100\,000 / \$5\,000 = \$6\,000$, то есть по \$300 с каждой из 20 купленных <<акций>> индекса. 

Если за год индекс упал относительно сегодняшнего уровня, то инвестор начинает участвовать в убытках и получает назад не весь вложенный номинал, а только часть. Коэффициент участия в убытках $x\%$ --- параметр ноты, который вам предстоит вычислить. Формально, если индекс упал до уроня $S_T$, то убыток инвестора равен $x \cdot (S_T - S_0) \cdot N/S_0$. 

Например, предположим, что $x=60\%$, а индекс упал до $S_T=\$4\,500$. Инвестор теряет $60\% \cdot (\$4\,500 - \$5\,000) \cdot \$100\,000/\$5\,000 = -\$6\,000$ и получает в конце года \$94\,000. Если бы инвестор участвовал в убытках полностью, он потерял бы по \$500 долларов с каждой <<акции>> индекса или \$10\,000 в сумме.

Как реплицировать такую ноту при помощи безрискового депозита и/или европейских ванильных и/или европейских цифровых опционов? Волатильность индекса в модели Блэка-Шоулза $\sigma=20\%$. Безрисковая процентная ставка $r=5\%$, дивидендная  доходность индекса $q=2\%$ (обе ставки с непрерывной капитализацией). Каков справедливый коэффициент участия в убытках $x\%$?
\end{document}