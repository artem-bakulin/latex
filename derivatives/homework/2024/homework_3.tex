\documentclass[a4paper,14pt]{extarticle}
\usepackage{cmap}				% To be able to copy-paste russian text from pdf			
\usepackage[utf8]{inputenc}
\usepackage[T2A]{fontenc}
\usepackage[margin=1in]{geometry}
\usepackage[english, russian]{babel}

\usepackage{amsmath}
\usepackage{amsfonts}

\usepackage[hyphens]{url}
\urlstyle{same}
\usepackage{hyperref}

\usepackage{tikz}
\usepackage{pgfplots}
\pgfplotsset{compat=1.17}
%\usetikzlibrary{arrows,snakes,backgrounds,shapes}
\usepgfplotslibrary{colorbrewer}

\usepackage{libertine}
\usepackage{libertinust1math}

\newcommand{\ru}[1]{\begin{otherlanguage}{russian}#1\end{otherlanguage}}
\newcommand{\en}[1]{\begin{otherlanguage}{english}#1\end{otherlanguage}}

\begin{document}


\noindent \textbf{Срок сдачи --- воскресенье 21 апреля 23:59 MSK.}

\vspace{\baselineskip}

\noindent Домашнее задание состоит из 6 теоретических задач (суммарно 15 баллов). Задачи 
можно сдавать в виде файлов .doc или .pdf, в виде Excel с расчётами, в виде ноутбука на Python --- главное, чтобы был понятен ход решения. Вы получите 1 бонусный балл, если сдадите всё домашнее задание в виде ноутбука на Python.

\section{Артём и 25-й колл (2 балла)}

Начинающий трейдер Артём позвонил в другой банк в соседнем небоскрёбе и спросил, сколько сейчас стоят \en{out-of-the-money}\ коллы на валютную пару GBPUSD (фунт-доллар). Ему ответили, что трёхмесячный 25C стоит 8\% и повесили трубку, прежде чем Артём успел хоть что-то уточнить.

Артём вспомнил, что 25C --- это такой страйк, на котором колл-опцион имеет дельту 25\%, а котировка, которую ему сказали --- это волатильность в модели Блэка-Шоулза. Артём посмотрел в \en{Bloomberg}, что текущий спот-курс равен $S=1.26$, безрисковая процентная ставка в долларах $r=5\%$, в фунтах $q=2\%$, до экспирации трёхмесячного опциона осталось $T=0.25$ года.

Помогите Артёму вычислить, какой же страйк (человекочитаемый курс фунт-доллар) ему предложили, и сколько долларов нужно заплатить за колл-опцион номиналом $N=1\,000\,000$ фунтов. Вы можете либо вывести формулу, которая вычисляет страйк из дельты, либо подобрать такой страйк численно с точностью до 0.01.

\section{Стрэнгл (2 балла)}

Структура стрэнгл (\en{strangle}) --- это немного потолстевший стрэдл. Стрэнгл состоит из ванильного пута со страйком $K_1$ и ванильного колла со страйком $K_2$, причём $K_1 < K_2$. Такой стрэнгл дешевле стрэдла, но для выхода в плюс придётся ждать большего движения рынка ниже $K_1$ или выше $K_2$.

Предположим, что мы живём в мире Блэка-Шоулза. Акция, которая не платит дивидендов, стоит $S=\$100$, её волатильность $\sigma=30\%$, безрисковая процентная ставка $r=5\%$. Сейчас наш портфель состоит из единственной позиции --- купленного стрэнгла со страйками $K_1=\$95$ и $K_2=\$105$ со сроком погашения $T=0.5$ года.

Сколько стоит такой стрэдл сейчас? Какова его дельта? Сколько единиц базового актива нужно купить или продать, чтобы дельта комбинированного портфеля (стрэнгл плюс базовый актив) была равна нулю.

Предположим, что мы выполнили дельта-хеджирование. Прошло время $dt=1/360$ (наступил новый рабочий день), и до экспирации стрэнгла осталось $T-dt$ лет.

Рассчитайте нашу прибыль или убыток относительно вчерашнего дня в трёх сценариях:  если акция стоит $S_1=\$98$ (подешевела), если она стоит $S_2=\$100$ (осталась на месте), если она стоит $S_3=\$102$ (подорожала).


\section{Метод Монте-Карло (2 балла)}

Методом Монте-Карло можно проверять других численные алгоритмы. <<Боевая>> модель может, скажем, численно решать дифференциальные уравнения на разностной решётке. Но можно же посчитать тот же дериватив <<в лоб>> методом Монте-Карло и проверить, что числа если и отличаются, то не сильно.

Предположим, что <<боевая>> модель оценила некоторый экзотический дериватив в $V=\$100$. Метод Монте-Карло на $n=10\,000$ симуляций считает, что цена $\hat{V}=\$99$. Выборочное стандартное отклонение значений в методе Монте-Карло $\hat{\sigma}=\$15$. Коллега, который написал <<боевую>> модель уверяет вас, что всё в порядке, ведь результаты достаточно близки, а метод Монте-Карло в принципе сам по себе неточный и нужно больше симуляций.

Докажите, что всё не так просто. Посчитайте двухсторонний 99\% доверительный интервал, в котором находится истинный средний результат метода Монте-Карло (тот, к котором метод сошёлся бы,если бы мы сделали бесконечно много симуляций). Если $N(x)$ --- функция распределения нормального распределения, то $N(0.995) = 2.576$

\section{Дельта-гамма хеджирование (3 балла)}

Рассмотрим модель Блэка-Шоулза. Индекс \en{S\&P\,500}\ находится на уровне $S=\$5\,200$. Волатильность индекса $\sigma=20\%$, дивидендная доходность $q=2\%$, безрисковая процентная ставка $r=5\%$ (непрерывная капитализация).

Наш портфель состоит из единственной позиции --- проданного европейского колл-опциона со страйком $K_1=\$5\,500$ и сроком исполнения $T=0.5$ лет. Считайте, что премию за опцион мы получили и потратили давным-давно, поэтому сейчас портфель имеет отрицательное \en{present value}.

Выполним гамма-хеджирование. Мы нашли на рынке ещё один опцион --- европейский пут со страйком $K_2=\$4\,700$ и сроком погашения $T=0.5$ лет. Сколько единиц пута нужно купить или продать, чтобы общая гамма портфеля (изначальный проданный колл плюс хеджирующий пут) была равна нулю?

Скорее всего, дельта портфеля из колла и пута не будет равна нулю. Сколько единиц базового актива нужно купить или продать, чтобы теперь и дельта портфеля (пут, колл, базовый актив) оказалась равной нулю? Будем считать, что мы можем <<купить>> или <<продать>> индекс либо в виде ETF-а либо в виде короткого фьючерса.

Как изменится PV портфеля (чему будет равна наша прибыль убыток), если сразу после хеджирования индекс скачком двинется к уровню $S_1=\$5\,300$ или уровню $S_2=\$5\,100$? Время до экспирации остаётся прежним.

\section{Вега цифрового опциона (3 балла)}

Трейдер Артём прочитал на Википедии, что в модели Блэка-Шоулза вега ванильного опциона всегда положительная. Это <<очевидно>>, потому что вся формула веги --- произведение положительных величин. Артём задумался, справедливо ли это утверждение и для цифровых опционов.

Акция, которая не платит дивидендов ($q=0\%$), имеет волатильность в модели Блэка-Шоулза $\sigma=20\%$. Безрисковая процентная ставка $r=5\%$ (непрерывная капитализация). Рассмотрим цифровой колл-опцион со страйком $K=\$100$ и сроком $T=0.5$ года. Номинал опциона (выигрыш, который получит владелец, если акция заберётся выше страйка) равен $N=\$1\,000$.

Посчитайте вегу в двух сценариях: если текущая цена акции $S_1=\$90$ и если акция стоит $S_2=\$110$. Дайте ответ в долларах на процентный пункт. Вы можете либо вывести формулу веги цифрового колла, взяв производную, либо посчитать её численно (прибавить к текущей волатильности 0.01\% и посмотреть, как изменится цена). Помните, что если вы дифференцируете формулу Блэка-Шоулза, то вега получится не в долларах на процентный пункт, а в долларах на единичку процентной ставки (в 100 раз больше).

Вычислите или подберите с точностью $\$0.01$ такую цену акции $S_0$, при которой вега нашего цифрового колла будет равна 0.

\section{Репликация кредитного свопа (3 балла)}

Безрисковая государственная бескупонная облигация со сроком погашения $T=5$ лет имеет
доходность $g=3\%$ (ежегодная капитализация). Доходность рискованной корпоративной 
бескупонной облигации с тем же сроком погашения $c=5\%$. При дефолте по корпоративной
облигации \en{recovery rate}\ составит $R=40\%$, а выплата случится в предполагаемую 
дату погашения. Рассмотрим кредитный дефолтный своп сроком на 5 лет, в котором купоны выплачиваются 
ежегодно. При дефолте выплата страховки произойдёт в дату следующего купона. Выплаты
накопленного купона нет.

Посчитайте справедливый купон в таком свопе. Пожалуйста, не пренебрегайте 
дисконтированием.

Мы продали клиенту такой своп номиналом $N=\$1\,000\,000$ по справедливой 
цене. В момент заключения сделки её \en{present value}\ равно нулю. Посмотрим, каков наш 
рыночный риск, и как им управлять.

Предположим, что доходность корпоративной облигации выросла на $dc=0.01\%$ (1 базисный 
пункт). Чему теперь равен честный купон в кредитном свопе (он изменится, потому что 
поменялась вероятность дефолта)? Если мы получаем от клиента старый купон, а вероятность 
дефолта выросла, то сколько мы заработали или потеряли (т.е. чему равно наше новое 
\en{PV}, учитывая, что исходное \en{PV}\ было \$0)? 

Теперь посчитайте, как изменилась цена самой корпоративной облигации при росте 
доходности на $dc=0.01\%$. Сколько облигаций нам нужно было бы купить или продать в самом начале, когда доходность ещё не успела измениться, чтобы общая прибыль по позиции из свопа и облигации при росте доходности на $dc$ была равна нулю? Считайте, что номинал корпоративной облигации \$1.

Проделайте то же самое упражнение для государственной облигации. Предположим, что её доходность изменилась на $dg=0.01\%$. На сколько изменится  \en{PV} свопа? Сколько государственных облигаций нужно было купить или продать с самого начала, чтобы общая цена портфеля не изменилась?

Напоминание: цена бескупонной облигации связана с доходностью соотношением $P = 1/(1+r)^T$.
\end{document}