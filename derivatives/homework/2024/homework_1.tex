\documentclass[a4paper,14pt]{extarticle}
\usepackage{cmap}				% To be able to copy-paste russian text from pdf			
\usepackage[utf8]{inputenc}
\usepackage[T2A]{fontenc}
\usepackage[margin=1in]{geometry}
\usepackage[english, russian]{babel}

\usepackage{amsmath}
\usepackage{amsfonts}

\usepackage[hyphens]{url}
\urlstyle{same}
\usepackage{hyperref}

\usepackage{libertine}
\usepackage{libertinust1math}

\newcommand{\ru}[1]{\begin{otherlanguage}{russian}#1\end{otherlanguage}}
\newcommand{\en}[1]{\begin{otherlanguage}{english}#1\end{otherlanguage}}

\begin{document}


\noindent \textbf{Срок сдачи --- воскресенье 24 марта 23:59 MSK.}

\vspace{\baselineskip}

\noindent Домашнее задание состоит из 6 теоретических задач (суммарно 15 баллов). Задачи 
можно сдавать в виде файлов .doc или .pdf, в виде Excel с расчётами, в виде ноутбука на Python --- главное, чтобы был понятен ход решения. Вы получите 1 бонусный балл, если сдадите всё домашнее задание в виде ноутбука на Python.


\section{Отложенный депозит (2 балла)}

Безрисковая бескупонная государственная облигация со сроком погашения $T_1=2$ года стоит
сейчас $P_1=94\%$ номинала. Другая безрисковая бескупонная облигация со сроком погашения $T_2=4$ года стоит $P_2 = 88\%$ номинала. Какова справедливая ставка по отложенному депозиту (непрерывная капитализация), который начнётся через 2 года и закончится через 4?

\section{Валютный своп (2 балла)}

Валютный своп (\en{foreign exchange swap}) --- комбинация из покупки спота и продажи форварда (или из продажи спота и покупки форварда). Участники финансового рынка используют валютные свопы, чтобы конвертировать кредиты или депозиты из одной валюты в другую.

Представьте, что вы --- российский банк. Вы не можете напрямую занимать доллары. Однако вы можете привлекать рубли и заключать сделки на рынке валютных спотов и форвардов. Предположим, что спот-курс пары USDRUB (доллар-рубль) $S=90$ рублей за доллар, годовой форвардный курс ($T=1$ год) $F=97$. Вы можете занимать рубли по ставке $r=15\%$ (простые проценты без капитализации).

Как при помощи доступных вам инструментов составить синтетический кредит на \$1\,000\,000? Какая получится эффективная процентная ставка (снова простые проценты без капитализации)?

\section{Артём и золото (2 балла)}

Трейдер Артём изучает рынок драгоценных металлов. Артём выяснил, что в мире два основных центра торговли золотом: спотовый внебиржевой рынок в Лондоне и рынок фьючерсов на золото на Чикагской бирже \en{(Chicago Mercantile Exchange, CME)}. 

Спотовая цена в Лондоне составляет $S=\$2\,100$ долларов за унцию. Декабрьский фьючерс (поставка через $T=0.5$ года) в Чикаго стоит $F=\$2\,155$ за унцию. Безрисковая процентная ставка в долларах на полгода составляет $r=5\%$ (непрерывная капитализация). Артём считает, что сможет при необходимости хранить золото в течение полугода в сейфе в офисе совершенно бесплатно ($q=0\%$).

Какова теоретический честный курс фьючерса при таком соотношении спотовой цены, фьючерса, процентной ставки и стоимости хранения? Предложите арбитражную стратегию, которая позволит заработать на ошибке рынка. Считайте, что Артём может как вкладывать, так и занимать доллары под безрисковую ставку.

Когда Артём пришёл с этой блестящей торговой стратегией к начальнику, тот напомнил, что, вообще-то, Лондон и Чикаго находятся по разные стороны Атлантического океана. Кроме того, в Лондоне обычно торгуются слитки по 400 унций, а в Чикаго - по 1000. Прежде чем золото пересечёт океан, его ещё придётся переплавить из слитков одного размера в другой размер. При какой суммарной стоимости переплавки и транспортировки (в долларах за унцию) арбиртражная стратегия выйдет хотя бы в ноль?

Указание: представьте, что имеете дело с валютным форвардом на валютную пару XAUUSD.

Интересный факт: обычно разность между Чикаго и Лондоном не превышает \$1--\$2 за унцию. При большей разности арбитраж становится экономически оправданным. Хедж-фонды буквально покупают золото на одном берегу, везут его самолётами на другой как обычный коммерческий груз, а потом платят металлургическим заводам за переплавку. Как вы можете догадаться, в начале 2020 года в разгар локдаунов самолёты не летали, а заводы не работали. Арбитраж был невозможен, и разность цен доходила до \$50 за унцию. Отличный пример несовпадения теории (цены должны быть равны) и практики (если рынки изолированны друг от друга, то цены могут разойтись).

\section{Беспоставочный форвард (3 балла)}

Тайваньский производитель чипов договорился о поставке партии чипов в США. По условиям 
контракта, через 3 месяца он получит $N=\$10\,000\,000$ от своего клиента из США. Экспортёр планирует через 3 месяца продать все полученные американские доллары (USD) и купить свои местные тайваньские (TWD). 

Беспоставоный форвард на пару USDTWD (доллар США - тайваньский доллар) стоит $F=35$ (35 тайваньских долларов за 1 американский). Должен ли импортёр купить или продать беспоставочный форвард, чтобы захеджировать 
валютный риск?

Убедитесь, что хедж действительно работает в трёх сценариях: спот-курс 
через 3 месяца $S_1=30$, $S_2=35$, $S_3=40$. В каждом из этих сценариев посчитайте, 
сколько американских долларов импортёр заработает или потеряет на беспоставочном форварде, сколько американских долларов он продаст в рынок, и сколько тайваньских долларов он в итоге получит.

\section{Форвард с дивидендами (3 балла)}

Некоторая привилегированная акция стоит $S=100$ рублей. Через 3 месяца акция может выплатить (а может и не выплатить) фиксированный дивиденд $D=5$ рублей. Полугодовой форвард на эту акцию ($T=0.5$ года) стоит $F=101$ рубль. Безрисковая процентная ставка в рублях, по которой можно и занимать, и размещать рубли, $r=10\%$ (простые проценты без капитализации). Продажа акций в короткую (шорт) разрешена и бесплатна.

Какую вероятность $p$ выплаты дивиденда участники рынка закладывают в цену форварда? Реинвестированием полученного дивиденда пренебречь.

Указание. Представьте, что дивиденда совершенно точно не будет. Тогда на рынке явно есть арбитражная стратегия. Как она выглядит и сколько принесёт арбитражёру? Какой убыток она принесёт, если дивиденд каким-то чудом всё-таки выплатят? Помните, что если вы продали акцию в короткую (зашортили), а она возьми да и выплати дивиденд, то вам придётся заплатить дивиденд владельцу акции из своего кармана. При какой вероятности выплаты дивиденда псевдо-арбитражная стратегия в среднем выходит в ноль?

\section{Базисный своп (3 балла)}

В таблице ниже представлены котировки рынка свопов на трёхмесячный EURIBOR (одна
сторона раз в квартал платит фиксированный купон, а вторая --- трёхмесячный 
EURIBOR) и свопов на шестимесячный EURIBOR (одна сторона раз в полгода платит 
фиксированный купон, вторая --- шестимесячный EURIBOR).

\begin{table}[h]
\centering
\begin{tabular}{l|r|r}
Инструмент         & 3M EURIBOR & 6M EURIBOR \\ \hline
Последний фиксинг  & 3.50\%     & 3.80\% \\
Своп на 1 год      & 3.40\%     & 3.60\% \\
Своп на 2 года     & 3.20\%     & 3.40\%
\end{tabular}
\end{table}

Пренебрегая дисконтированием и используя предположение о линейной интерполяции,
рассчитайте <<ожидаемые>> значения трёхмесячного и шестимесячного EURIBOR.

Базисный своп 3M-6M (tenor-basis swap, TBS) --- процентный своп, в котором 
платежи обеих сторон привязаны к плавающим ставкам. Первая сторона раз в 
полгода платит шестимесячный EURIBOR. Вторая сторона раз в квартал платит 
трёхмесячный EURIBOR плюс $x\%$, где $x$ --- цена свопа, о которой стороны и 
должны договориться. Этот $x$ ещё называют <<базис>>. Пример такого свопа есть в 
слайдах с лекции. Cвоп позволяет, например, превратить кредит под трёхмесячный 
EURIBOR в кредит под шестимесячный EURIBOR.

Используя полученные ранее <<ожидаемые>> значения EURIBOR, посчитайте честный
базис для двухлетнего базисного свопа 3M-6M. Дисконтированием по-прежнему можно
пренебречь. Видите ли вы более лёгкий способ вычислить базис?

\end{document}