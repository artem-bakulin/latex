\documentclass[a4paper,12pt]{extarticle}
\usepackage{cmap}				% To be able to copy-paste russian text from pdf		
\usepackage[utf8]{inputenc}
\usepackage[T2A]{fontenc}
\usepackage[margin=1in]{geometry}
\usepackage[english, russian]{babel}

\usepackage{libertine}
\usepackage{libertinust1math}
   
\begin{document}
\thispagestyle{empty}

\noindent 1. Трейдер Артём купил 1\,000\,000 валютной пары евро-доллар (EURUSD) по курсу 1.08\underline{51} (1.0851 доллара за 1 евро). Сразу после этого честный рыночный курс сдвинулся до уровня 1.08\underline{52}. Сколько денег заработал или потерял Артём?

А. Заработал 100 долларов

B. Потерял 100 долларв

C. Заработал 100 евро

D. Потерял 100 евро

\vspace{0.5cm}

\noindent 2. Спот-курс валютной пары доллар-рубль (USDRUB) составляет 90 рублей за доллар. Процентная ставка в рублях 10\%, в долларах 4\% (простые проценты без капитализации). Сколько должен стоить форвард с поставкой через 1 год?

A. 86.50

B. 93.60

C. 95.20

D. 99.00

\vspace{0.5cm}

\noindent 3. Июньский фьючерс на EURIBOR сейчас стоит 96.0. Какое значение EURIBOR 
участники рынка ожидают увидеть в июне?

A. $0.4\%$

B. $0.96\%$.

C. $4.0\%$

D. $9.6\%$.

\vspace{0.5cm}

\noindent 4. В полугодовом свопе на EURIBOR-3M два плавающих платежа: через 3 месяца и 
через 6 месяцев от сегодня. Рыночная цена этого свопа 4\%. Сегодня был опубликован 
EURIBOR 3\%. Какое будущее значение EURIBOR через 3 месяца <<ожидает>> рынок? 
Дисконтированием пренебречь.

A. 3\%.

B. 4\%

C. 5\%

D. 6\%

\vspace{0.5cm}

\noindent 5. Почему график прошлых значений процентной ставки ESTER выглядит как ступенчатая функция?

А. Процентная ставка ESTER публикуется раз в 3 месяца, а между публикациями остаётся постоянной.

B. Рынок овернайт кредитов в евро слишком неликвидный.

С. График следует за изменениями кредитного рейтинга государств Еврозоны.

D. Скачки на графике отражают решения Европейского центрального банка об изменении процентных ставок.
\end{document}