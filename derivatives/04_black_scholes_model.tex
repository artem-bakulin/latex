
\documentclass{beamer}

\usepackage{cmap}				% To be able to copy-paste russian text from pdf
\usepackage[T2A]{fontenc}
\usepackage[utf8]{inputenc}
\usepackage[russian]{babel}
\usepackage{textpos}
\usepackage{ragged2e}
\usepackage{amssymb}
\usepackage{ulem}
\usepackage{tikz}
\usepackage{pgfplots}
\usepackage{color}
\usepackage{cancel}
\usepackage{multirow}
\pgfplotsset{compat=1.17}
\usetikzlibrary{arrows,snakes,backgrounds,shapes}
\usepgfplotslibrary{groupplots,colorbrewer,dateplot,statistics}
\usepackage{animate}

\usepackage{amsfonts}
\usepackage{amsmath}
\usepackage{amssymb}
\usepackage{graphicx}
\usepackage{setspace}
\usepackage{cancel}

\usepackage{enumitem}
\setitemize{label=\usebeamerfont*{itemize item}%
  \usebeamercolor[fg]{itemize item}
  \usebeamertemplate{itemize item}}

% remove navigation bar
\setbeamertemplate{navigation symbols}{} 

\usepackage{eurosym}
\renewcommand{\EUR}[1]{\textup{\euro}#1}

\title{Модель Блэка-Шоулза}
\author{Артём Бакулин}
\date{25 марта 2024 г.}

\usetheme{Warsaw}
\usecolortheme{beaver}

\setbeamertemplate{page number in head/foot}[totalframenumber] 

\newcommand{\ru}[1]{\begin{otherlanguage}{russian}#1\end{otherlanguage}}
\newcommand{\en}[1]{\begin{otherlanguage}{english}#1\end{otherlanguage}}
\newcommand{\ruen}[2]{#1 (\en{#2})}

\begin{document}



\begin{frame}
\titlepage
\end{frame}


\newcommand{\drawStockNode}[5]{

	\node (#5)
	[
		draw,
		rectangle,
		rounded corners,
		inner sep = 0pt,
		outer sep = 0pt,
		minimum width = 2.4cm,
		minimum height = 0.55cm,
		align = center
	]
	at (#3, #4)
	{
		\begin{tabular}{c|c}
		#1 & #2
		\end{tabular}
	};
}

\newcommand{\drawStockLink}[4]{

	\draw[
		->,
		>=triangle 90
	]
	(#1.east) -- (#2.west)
	node[
		pos = 0.5,
		anchor = #4
	]
	{#3};
}

\newcommand{\drawOneStepBinomialTree}{
	\drawStockNode{\$100}{?}{0}{0}{S0_node}
	\drawStockNode{\$120}{\$20}{4}{ 1}{Su_node}
	\drawStockNode{\$80}{\$0}{4}{-1}{Sd_node}
	
	\drawStockLink{S0_node}{Su_node}{$90\%$}{south east}	
	\drawStockLink{S0_node}{Sd_node}{$10\%$}{north east}
}



\renewcommand{\drawOneStepBinomialTree}{
	\drawStockNode{$S_0$}{?}{0}{0}{S0_node}
	\drawStockNode{$S_0u$}{$V_u$}{4}{ 1}{Su_node}
	\drawStockNode{$S_0d$}{$V_d$}{4}{-1}{Sd_node}
	
	\drawStockLink{S0_node}{Su_node}{$p$}{south east}	
	\drawStockLink{S0_node}{Sd_node}{$1 - p$}{north east}
}

\begin{frame}{Напоминание: биномиальная модель}
\centering
\begin{tikzpicture}
	\drawOneStepBinomialTree
\end{tikzpicture}

\justify
1. Текущая цена акции $S_0$.

\justify
2. Цена акции может либо вырасти до $S_0\cdot u$ ($u>1$), либо снизиться до $S_0 \cdot d$ ($d<1$).

\justify
3. Длина одного периода $\tau$ лет, безрисковая  процентная ставка $r$, причём $d < 1+r\tau < u$.

\justify
4. В случае роста или падения акции опцион принесет (будет иметь value) либо $V_u$, либо $V_d$.
\end{frame}



\begin{frame}{Напоминание: биномиальная модель -- 2}
\centering
\begin{tikzpicture}
	\drawOneStepBinomialTree
\end{tikzpicture}

\justify
Рассмотрим портфель, состоящий из $\Delta$ акций и долга $L$. 
\begin{equation*}
\begin{cases}
L(1+r\tau) + \Delta S_0 u = V_u \\
L(1+r\tau) + \Delta S_0 d = V_d
\end{cases}
\end{equation*}

\begin{equation*}
\begin{cases}
\Delta = \dfrac{V_u - V_d}{S_0(u-d)} \\
L = \dfrac{V_du - V_ud}{(1+r\tau)(u-d)}
\end{cases}
\end{equation*}
\end{frame}



\begin{frame}{Напоминание: биномиальная модель -- 3}
\centering
\begin{tikzpicture}
\drawOneStepBinomialTree
\end{tikzpicture}

\justify
Цена опциона равна цене реплицирующего портфеля:
\begin{align*}
C &= \Delta S_0 +L = \\
 &= \dfrac{V_u-V_d}{(u-d)\cancel{S_0}}\cancel{S_0} + \dfrac{V_du -V_ud}{(1+r\tau)(u-d)} = \\
 &= \dfrac{qV_u +(1-q)V_d}{1+r\tau},
\end{align*}
где
\begin{equation*}
q = \dfrac{1+r\tau - d}{u-d} \text{ --- <<риск-нейтральная вероятность>>}
\end{equation*}
\end{frame}



\renewcommand{\drawStockLink}[2]{

	\draw[
		->,
		>=triangle 45
	]
	(#1.east) -- (#2.west)
	{};
}

\renewcommand{\drawStockNode}[5]{

	\node (#5)
	[
		draw,
		rectangle,
		rounded corners,
		inner sep = 1pt,
		outer sep = 0pt,
		minimum width = 1.5cm
	]
	at (#3, #4)
	{
		\centering
		\begin{tabular}{c}
		#1 \\ \hline #2
		\end{tabular}
	};
}

\newcommand{\nodeVerticalStep}{0.7}
\newcommand{\nodeHorizontalStep}{2.75}

\begin{frame}{Напоминание: биномиальная модель -- 4}
\centering
\begin{tikzpicture}
\drawStockNode{$\$100$}{\only<1-7>{?}\only<8->{\$14.8}}{0}{0}{S0_node}

\drawStockNode{$\$120$}{\only<1-5>{?}\only<6->{\$25.8}}{\nodeHorizontalStep}{\nodeVerticalStep}{Su_node}
\drawStockNode{$\$80$}{\only<1-6>{?}\only<7->{\$3.8}}{\nodeHorizontalStep}{-\nodeVerticalStep}{Sd_node}

\drawStockNode{$\$144$}{\only<1-2>{?}\only<3->{\$44}}{2*\nodeHorizontalStep}{2*\nodeVerticalStep}{Suu_node}
\drawStockNode{$\$96$}{\only<1-3>{?}\only<4->{\$7.6}}{2*\nodeHorizontalStep}{0}{Sud_node}
\drawStockNode{$\$64$}{\only<1-4>{?}\only<5->{\$0}}{2*\nodeHorizontalStep}{-2*\nodeVerticalStep}{Sdd_node}

\drawStockNode{$\$172.8$}{\only<1>{?}\only<2->{\$72.8}}{3*\nodeHorizontalStep}{3*\nodeVerticalStep}{Suuu_node}
\drawStockNode{$\$115.2$}{\only<1>{?}\only<2->{\$15.2}}{3*\nodeHorizontalStep}{\nodeVerticalStep}{Suud_node}
\drawStockNode{$\$76.8$}{\only<1>{?}\only<2->{\$0}}{3*\nodeHorizontalStep}{-\nodeVerticalStep}{Sudd_node}
\drawStockNode{$\$51.2$}{\only<1>{?}\only<2->{\$0}}{3*\nodeHorizontalStep}{-3*\nodeVerticalStep}{Sddd_node}

\drawStockLink{S0_node}{Su_node}
\drawStockLink{S0_node}{Sd_node}

\drawStockLink{Su_node}{Suu_node}
\drawStockLink{Su_node}{Sud_node}

\drawStockLink{Sd_node}{Sud_node}
\drawStockLink{Sd_node}{Sdd_node}

\drawStockLink{Suu_node}{Suuu_node}
\drawStockLink{Suu_node}{Suud_node}

\drawStockLink{Sud_node}{Suud_node}
\drawStockLink{Sud_node}{Sudd_node}

\drawStockLink{Sdd_node}{Sudd_node}
\drawStockLink{Sdd_node}{Sddd_node}
\end{tikzpicture}

\justify
Предположим, что $u=1.2$, $d=0.8$, $S_0=\$100$, $r=0\%$. Сколько стоит колл со страйком $K=100$?

\justify
<<Риск-нейтральная вероятность>>:
\begin{align*}
q = \dfrac{1+r\tau - d}{u - d} = \dfrac{1 - 0.8}{1.2 - 0.8} = 0.5
\end{align*}
\end{frame}



\begin{frame}{Напоминание: биномиальная модель -- 5}
\centering
\begin{tikzpicture}
\drawStockNode{$\$100$}{$\Delta=0.58$}{0}{0}{S0_node}

\drawStockNode{$\$120$}{$\Delta=0.76$}{\nodeHorizontalStep}{\nodeVerticalStep}{Su_node}
\drawStockNode{$\$80$}{$\Delta=0.24$}{\nodeHorizontalStep}{-\nodeVerticalStep}{Sd_node}

\drawStockNode{$\$144$}{$\Delta=1.0$}{2*\nodeHorizontalStep}{2*\nodeVerticalStep}{Suu_node}
\drawStockNode{$\$96$}{$\Delta=0.4$}{2*\nodeHorizontalStep}{0}{Sud_node}
\drawStockNode{$\$64$}{$\Delta=0.0$}{2*\nodeHorizontalStep}{-2*\nodeVerticalStep}{Sdd_node}

\drawStockNode{$\$172.8$}{$\Delta=1$}{3*\nodeHorizontalStep}{3*\nodeVerticalStep}{Suuu_node}
\drawStockNode{$\$115.2$}{$\Delta=1$}{3*\nodeHorizontalStep}{\nodeVerticalStep}{Suud_node}
\drawStockNode{$\$76.8$}{$\Delta=0$}{3*\nodeHorizontalStep}{-\nodeVerticalStep}{Sudd_node}
\drawStockNode{$\$51.2$}{$\Delta=0$}{3*\nodeHorizontalStep}{-3*\nodeVerticalStep}{Sddd_node}

\drawStockLink{S0_node}{Su_node}
\drawStockLink{S0_node}{Sd_node}

\drawStockLink{Su_node}{Suu_node}
\drawStockLink{Su_node}{Sud_node}

\drawStockLink{Sd_node}{Sud_node}
\drawStockLink{Sd_node}{Sdd_node}

\drawStockLink{Suu_node}{Suuu_node}
\drawStockLink{Suu_node}{Suud_node}

\drawStockLink{Sud_node}{Suud_node}
\drawStockLink{Sud_node}{Sudd_node}

\drawStockLink{Sdd_node}{Sudd_node}
\drawStockLink{Sdd_node}{Sddd_node}
\end{tikzpicture}

\justify
Справедливая безарбитражная цена опциона не зависит от вероятностей изменения цены вверх и вниз. Если мы можем \alert{динамически} ребалансировать портфель, состоящий из акций и долга, то мы можем реплицировать любой опцион.

\justify
Эта стратегия называется \alert{дельта-хеджирование} (\en{delta hedging}).
\end{frame}



\newcommand{\highlightStockLink}[6]{
	\draw[
		color=#4,
		very thick,
		->,
		>=triangle 45
	]
	(#1.east) -- (#2.west)
	node[
		pos=#5,
		anchor=#6
	]
	{#3};
}

\newcommand{\highlightStockLinkUp}[3]{
	\highlightStockLink{#1}{#2}{$q$}{#3}{0.5}{south}
}

\newcommand{\highlightStockLinkDown}[3]{
	\highlightStockLink{#1}{#2}{$1-q$}{#3}{0.15}{west}
}

\begin{frame}{Напоминание: риск-нейтральная вероятность}
\centering
\begin{tikzpicture}
\drawStockNode{$S_0$}{?}{0}{0}{S0_node}

\drawStockNode{$S_0u$}{?}{\nodeHorizontalStep}{\nodeVerticalStep}{Su_node}
\drawStockNode{$S_0d$}{?}{\nodeHorizontalStep}{-\nodeVerticalStep}{Sd_node}

\drawStockNode{$S_0u^2$}{?}{2*\nodeHorizontalStep}{2*\nodeVerticalStep}{Suu_node}
\drawStockNode{$S_0ud$}{?}{2*\nodeHorizontalStep}{0}{Sud_node}
\drawStockNode{$S_0d^2$}{?}{2*\nodeHorizontalStep}{-2*\nodeVerticalStep}{Sdd_node}

\drawStockNode{$S_0u^3$}{$V_3$}{3*\nodeHorizontalStep}{3*\nodeVerticalStep}{Suuu_node}
\drawStockNode{$S_0u^2d$}{$V_2$}{3*\nodeHorizontalStep}{\nodeVerticalStep}{Suud_node}
\drawStockNode{$S_0ud^2$}{$V_1$}{3*\nodeHorizontalStep}{-\nodeVerticalStep}{Sudd_node}
\drawStockNode{$S_0d^3$}{$V_0$}{3*\nodeHorizontalStep}{-3*\nodeVerticalStep}{Sddd_node}

\only<1-2>{
	\drawStockLink{S0_node}{Su_node}
	\drawStockLink{S0_node}{Sd_node}

	\drawStockLink{Su_node}{Suu_node}
	\drawStockLink{Su_node}{Sud_node}

	\drawStockLink{Sd_node}{Sud_node}
	\drawStockLink{Sd_node}{Sdd_node}

	\drawStockLink{Suu_node}{Suuu_node}
	\drawStockLink{Suu_node}{Suud_node}

	\drawStockLink{Sud_node}{Suud_node}
	\drawStockLink{Sud_node}{Sudd_node}

	\drawStockLink{Sdd_node}{Sudd_node}
	\drawStockLink{Sdd_node}{Sddd_node}
}

\only<3>{
	\highlightStockLinkUp{S0_node}{Su_node}{Set1-A}
	\highlightStockLinkUp{Su_node}{Suu_node}{Set1-A}
	\highlightStockLinkDown{Suu_node}{Suud_node}{Set1-A}
}

\only<4>{
	\highlightStockLinkUp{S0_node}{Su_node}{Set1-A}
	\highlightStockLinkDown{Su_node}{Sud_node}{Set1-A}
	\highlightStockLinkUp{Sud_node}{Suud_node}{Set1-A}
}

\only<5>{
	\highlightStockLinkDown{S0_node}{Sd_node}{Set1-A}
	\highlightStockLinkUp{Sd_node}{Sud_node}{Set1-A}
	\highlightStockLinkUp{Sud_node}{Suud_node}{Set1-A}
}

\end{tikzpicture}

\justify
Риск-нейтральная вероятность: $q = \dfrac{1 + rT - d}{u - d}$.

\justify
Цена дериватива сегодня:
\begin{align*}
V = \frac{q^3V_3 + \only<1>{3q^2(1-q)}\only<2->{\alert{3q^2(1-q)}}V_2 + 3q(1-q)^2V_1 + (1-q)^3V_0}{(1+rT)^3}
\end{align*}
\end{frame}



\begin{frame}{Напоминание: риск-нейтральная вероятность - 2}
\justify
Если представить, что $q$ --- вероятность движения акции вверх, то $3q^2(1-q)$ --- вероятность того, что акция дважды вырастет и один раз упадёт (не важно, в каком порядке). Тогда акция будет стоить $S_0u^2d$, а дериватив принесёт прибыль $V_2$.

\justify
\centering
\begin{tabular}{l|l|l}
Цена акции & Прибыль & <<Вероятность>> \\ \hline
$S_0u^3$   & $V_3$   & $q^3$ \\
$S_0u^2d$  & $V_2$   & $3q^2(1-q)$ \\
$S_0ud^2$  & $V_1$   & $3q(1-q)^2$ \\ 
$S_0d^3$   & $V_0$   & $(1-q)^3$ 
\end{tabular}

\justify
Цена дериватива похожа на дисконтированное <<математическое ожидание>> прибыли.
\begin{align*}
V = \frac{q^3V_3 + 3q^2(1-q)V_2 + 3q(1-q)^2V_1 + (1-q)^3V_0}{(1+rT)^3}
\end{align*}
\end{frame}



\begin{frame}{Биномиальная модель и модель Блэка-Шоулза}
Для дерева, состоящего из $n$ шагов:
\begin{align*}
V &= \dfrac{\sum\limits_{k=0}^{n} C^k_nq^k(1-q)^{n-k}V(S_0u^kd^{n-k})}{(1+r\tau)^n} \\
C^k_n &= \dfrac{n!}{k!(n-k)!}
\end{align*}

\justify
Что будет, если вместо произвольной функции $V(S)$ взять функцию $max(S-K,0)$, как в колл-опционе, и устремить $n$ к бесконечности? Получится формула Блэка-Шоулза.

\vspace{\baselineskip}
Строгое доказательство с помощью закона больших чисел и центральной
предельной теоремы здесь (28 страниц):

\url{http://www.math.cmu.edu/~handron/21_370/BS.pdf}
\end{frame}



\begin{frame}{Демонстрация: доска Гальтона}

\url{https://www.mathsisfun.com/data/quincunx.html}
\end{frame}



\begin{frame}{Центральная предельная теорема}
\justify
Пусть $\xi_i$ --- независимые одинаково распределённые случайные величины, которые имеют конечное математическое ожидание $\mathbb{E}\xi_i=\mu$ и конечную дисперсию $\operatorname{Var}(\xi_i) = \sigma^2$. Тогда имеет место сходимость по распределению:
\begin{align*}
\lim_{n \to \infty} \sqrt{n}\frac{\dfrac{1}{n}\sum\limits_{i=1}^{n}\xi_i - \mu}{\sigma} = \mathcal{N}(0, 1)
\end{align*}

\justify
$\mathcal{N}(0, 1)$ --- стандартное нормальное распределение.

\justify
Это --- центральная предельная теорема (ЦПТ). Среднее арифметическое большого количества реализаций случайной величины следует нормальному распределению.
\end{frame}



\begin{frame}{Геометрическое броуновское движение}
\justify
Предположим, что до экспирации опциона осталось $T$ лет, и мы раздробили этот интервал на $N$ шагов длиной по $dt = T/N$ лет каждый.

\justify
Пусть на каждом самом маленьком шаге цена акции растёт в среднем на $
\mu$ процентов годовых, а также колеблется вокруг этого среднего со 
стандартным отклонением (волатильностью) $\sigma$.
\begin{align*}
\frac{S_{i+1} - S_i}{S_i} = \mu dt + \sigma\xi_i\sqrt{dt}, \quad \xi_i \sim \mathcal{N}(0, 1)
\end{align*}

\justify
Это геометрическое броуновское движение. $\xi_i$ ---случайные шоки, которые заставляют цену акций колебаться вокруг среднего тренда. 

\justify
Геометрическое броуновское движение --- предел биномиального дерева, если предположить, что
\begin{align*}
u = e^{\mu dt + \sigma\sqrt{dt}}, \quad d = e^{\mu dt - \sigma\sqrt{dt}}
\end{align*}
\end{frame}



\newcommand{\plotBrownianMotion}[2] {
	
	\addplot[
		color = #2,
		mark = none,
		thick
	]
	table[
		x=t,
		y=s,
		col sep=comma
	]
	{#1};
	
	\addplot[
		color = #2,
		mark = none,
		thick,
		dashed,
		forget plot
	] 
	table[
		x=t,
		y=trend,
		col sep=comma
	]
	{#1};
}



\begin{frame}{Геометрическое броуновское движение}
\centering
\begin{tikzpicture}
\begin{axis}[
  width=\textwidth,
  height=\textheight - 1cm,
  xlabel near ticks,
  ylabel near ticks,
  xmin=0, xmax=1,
  legend entries = {
  	   {$\mu=5\%, \sigma=10\%, N=100$},
      {$\mu=10\%, \sigma=40\%, N=500$}
  },
  legend cell align={left},
  legend style={at={(0.97,0.03)},anchor=south east}
]

	\plotBrownianMotion{gbm_sample_100.csv}{Set1-A}
	\plotBrownianMotion{gbm_sample_500.csv}{Set1-B}
\end{axis}
\end{tikzpicture}
\end{frame}



\begin{frame}{Геометрическое движение}
\justify
Предположим, что $\sigma=0$. Никакой случайности нет, перед нами детерминистический закон:
\begin{align*}
\frac{S_{i+1} - S_i}{S_i} = \mu dt \quad \Rightarrow \quad S_{i+1} = S_i(1 + \mu dt)
\end{align*}

\justify
На каждом шаге $dt$ акция растёт на $\mu$ процентов годовых (простые проценты без капитализации). Чему равна цена акции через время $T$ после $N$ шагов по $dt=T/N$ каждый?
\begin{align*}
S_T &= S_N = S_{N-1}\left(1+\mu\dfrac{T}{N}\right) 
= S_{N-2}\left(1+\mu\dfrac{T}{N}\right)\left(1+\mu\dfrac{T}{N}\right) = \\
&= S_0\left(1+\mu\dfrac{T}{N}\right)^N \to S_0e^{\mu T}
\end{align*}

\justify
Это уже знакомые нам <<непрерывные>> проценты.
\end{frame}



\begin{frame}{Добавляем случайность}
\justify
Пусть акция не просто растёт на каждом шаге, а колеблется вокруг тренда $\mu$ вверх и 
вниз случайным образом.

\justify
Каким именно случайным? Если колебания цены акции отражают
приходящую информацию (<<шоки>>), и эти шоки не зависят друг от друга, то можно 
надеяться, что они складываются в нормальное распределение.
\begin{align*}
\frac{S_{i+1} - S_i}{S_i} = \mu dt + \sigma\xi_i\sqrt{dt}, \quad \xi_i \sim \mathcal{N}(0, 1)
\end{align*}

Здесь $\xi_i$ --- источник случайных шоков, $\sigma$ --- волатильность или коэффициент масштабирования. Обычные единицы измерения волатильности --- проценты годовых, как у тренда $\mu$ и процентных ставок.
\end{frame}



\begin{frame}{Геометрическое броуновское движение}
\justify
Если геометрическое броуновское движение начинается с цены базового актива $S_0$ с трендом $\mu$ и волатильностью $\sigma$, то цена базового актива через время $T$ лет $S_T$ есть случайная величина:
\begin{align*}
S_T = S_0\cdot \exp\left[\left(\mu - \frac{\sigma^2}{2}\right)T + \sigma\xi\sqrt{T}\right], \quad \xi \sim \mathcal{N}(0, 1)
\end{align*} 

\justify
Логарифм изменения цены (грубо: процентный прирост) есть случайная величина, которая следует нормальному распределению:
\begin{align*}
\ln\left(\frac{S_T}{S_0}\right) = \ln S_T - \ln S_0 = \left(\mu - \frac{\sigma^2}{2}\right)T + \sigma\xi\sqrt{T} \quad \xi \sim \mathcal{N}(0, 1)
\end{align*}

\justify
Поэтому говорят, что цена базового актива следует лог-нормальному распределению.
\end{frame}



\begin{frame}{Лог-нормальное распределение}
\centering
\begin{tikzpicture}
	\begin{axis}[
		width = \textwidth,
		height = \textheight - 1cm,
		xmin = -5, xmax = 5,
		ymin = 0, ymax = 0.7,
		grid = major,
		legend entries = {Нормальное, Лог-нормальное},
		legend pos = north west
	]
		
		\addplot[color=Set1-A, thick] table[x=x, y=norm_density, col sep=comma] {norm_and_lognorm_density.csv};
		
				\addplot[color=Set1-B, thick] table[x=x, y=lognorm_density, col sep=comma] {norm_and_lognorm_density.csv};
	\end{axis}
\end{tikzpicture}
\end{frame}



\begin{frame}{Модель Блэка-Шоулза}
\justify
1. Базовый актив не платит дивидендов, его цена $S_0$.

\justify
2. Цена актива следует геометрическому броуновскому движению с трендом $\mu$ и волатильностью $\sigma$:
\begin{align*}
\dfrac{dS}{S} = \mu dt + \sigma\xi\sqrt{dt}, \quad \xi \sim \mathcal{N}(0,1)
\end{align*}

\justify
3. Безрисковая процентная ставка $r$ постоянна.

\justify
4. Идеальный ликвидный рынок без транзакционных издержек и без возможностей для арбитража, на котором мы можем заниматься дельта-хеджированием сколь угодно часто.
\end{frame}



\begin{frame}{Формула Блэка-Шоулза}
Цена европейского колл-опциона со страйком $K$ задается формулой:
\begin{align*}
C_{BS} &= S_0N(d_1) - Ke^{-rT}N(d_2)
\end{align*}
где
\begin{align*}
d_1 &= \dfrac{1}{\sigma\sqrt{T}}\left( \ln\left(\dfrac{S_0}{K}\right) + \left(r + \dfrac{\sigma^2}{2}\right)T\right) \\
d_2 &= \dfrac{1}{\sigma\sqrt{T}}\left( \ln\left(\dfrac{S_0}{K}\right) + \left(r - \dfrac{\sigma^2}{2}\right)T\right) \\
N(x) &= \dfrac{1}{\sqrt{2\pi}}\int\limits_{-\infty}^x e^{-\frac{t^2}{2}}dt
\end{align*}

\justify
Цена не зависит от тренда $\mu$!
\end{frame}



\begin{frame}{Формула Блэка-Шоулза с дивидендами}
\justify
Если базовый актив имеет дивидендную доходность $q$  и дивиденды выплачиваются непрерывно (валюта, индекс акций, нефть в хранилище):
\begin{align*}
C_{BS} &= S_0e^{-qT}N(d_1) - Ke^{-rT}N(d_2)
\end{align*}
где
\begin{align*}
d_1 &= \dfrac{1}{\sigma\sqrt{T}}\left( \ln\left(\dfrac{S_0}{K}\right) + \left(r -q + \dfrac{\sigma^2}{2}\right)T\right) \\
d_2 &= \dfrac{1}{\sigma\sqrt{T}}\left( \ln\left(\dfrac{S_0}{K}\right) + \left(r -q- \dfrac{\sigma^2}{2}\right)T\right)
\end{align*}
\end{frame}



\begin{frame}{Модель Блэка-Шоулза: пример}
\justify
Пример: колл-опцион со страйком $K=100$, сроком $T=0.25$ лет, $r=5\%$, $q=0\%$.

\centering
\begin{tikzpicture}
\begin{axis}[
			domain=92:108,
			xtick={92,94,...,108},
			ytick={0,1,2,...,10},
			xmin=92, xmax=108,
			ymin=0, ymax=10,
			grid = major,
			xlabel={Курс сегодня ($S_0$)},
			ylabel={Цена опциона},
  legend entries = {
  	   $\sigma=25\%$,
  	   $\sigma=10\%$,
  	   $\sigma=5\%$
  },
  legend cell align={left},
  legend style={at={(0.03,0.97)},anchor=north west}
]

	\addplot[color = Set1-A, mark = none, thick]
	table[
		x=S,
		y=C_25,
		col sep=comma
	]
	{call_price.csv};
	
	\addplot[color = Set1-B, mark = none, thick]
	table[
		x=S,
		y=C_10,
		col sep=comma
	]
	{call_price.csv};
	
	\addplot[color = Set1-C, mark = none, thick]
	table[
		x=S,
		y=C_5,
		col sep=comma
	]
	{call_price.csv};
	
	\addplot[Set1-D, very thick, dashed] {(\x >= 100)*(\x - 100) + 0.05};
\end{axis}
\end{tikzpicture}
\end{frame}



\begin{frame}{Временная и внутренняя стоимость}
\justify
Величину $\max(S - K, 0)$ называют \alert{внутренней стоимостью} колл-опциона (\en{intrinsic value}). Столько денег мы бы заработали на опционе, если бы могли исполнить его прямо сейчас.

\justify
Если $C$ --- цена колл-опциона, то величина $C - \max(S - K, 0)$ называется \alert{временной стоимостью} (\en{time value}). Она показывает, насколько опцион дороже, чем простая внутренняя стоимость.

\justify
Чем выше волатильность, и чем больше времени осталось до экспирации, тем выше временная стоимость.
\end{frame}


\begin{frame}{Формула Блэка-Шоулза: интерпретация}
\begin{align*}
C_{BS} &= S_0e^{-qT}N(d_1) - Ke^{-rT}N(d_2)
\end{align*}

\justify
$S_0e^{-qT}N(d_1)$ --- цена опциона <<актив или ничего>> (\en{asset-or-nothing call}). Этот опцион выплачивает 1 единицу базового актива, если цена выше страйка.

\justify
$e^{-rT}N(d_2)$ --- цена опциона <<деньги или ничего>> (\en{cash-or-nothing call}). Этот опцион выплачивает 1 единицу денег, если цена актива выше страйка.
\end{frame}



\begin{frame}{Цифровые опционы}
\justify
Цифровые колл и пут со страйком $K$:
\begin{align*}
C_K = e^{-rT}N(d_2), \quad P_K = e^{-rT}N(-d_2) \\
d_2 = \dfrac{1}{\sigma\sqrt{T}}\left( \ln\left(\dfrac{S_0}{K}\right) + \left(r -q- \dfrac{\sigma^2}{2}\right)T\right)
\end{align*}

\justify
Предположим, что $r=q=0$ и $T=1$. Что такое $N(d_2)$?
\begin{align*}
N(d_2) &= \mathbb{P}(\xi \le d_2) = 
\mathbb{P}(-\xi \le d_2) = \mathbb{P}\left(-\xi \le \dfrac{\ln S_0 - \ln_K}{\sigma} - \frac{\sigma}{2}\right) = \\
&= \mathbb{P}\left(S_0\exp\left(-\dfrac{\sigma^2}{2} + \xi\sigma\right) \ge K\right),
\quad \xi \sim \mathcal{N}(0,1)
\end{align*}

Если представить, что цена базового актива следуют геометрическому броуновскому 
движению с трендом $r-q$ и волатильностью $\sigma$, то $N(d_2)$ --- <<вероятнтость>>
того, что конечная цена окажется выше страйка $K$.

\end{frame}




\begin{frame}{Приближённое вычисление: ATMF-колл}
\justify
Рассмотрим \en{at-the-money-forward (ATMF)}\ колл-опцион со страйком $K = F = S_0e^{(r-q)T}$.

\begin{align*}
d_1 &= \dfrac{1}{\sigma\sqrt{T}}\left( \ln\left(\dfrac{S_0}{S_0e^{(r-q)T}}\right) + \left(r -q + \dfrac{\sigma^2}{2}\right)T\right) = \frac{\sigma \sqrt{T}}{2} \\
d_2 &= \dfrac{1}{\sigma\sqrt{T}}\left( \ln\left(\dfrac{S_0}{S_0e^{(r-q)T}}\right) + \left(r -q- \dfrac{\sigma^2}{2}\right)T\right) = -\frac{\sigma \sqrt{T}}{2}
\end{align*}

Тогда:
\begin{align*}
C_{atmf} &= S_0e^{-qT}N(d_1) - S_0e^{(r-q)T}e^{-rT}N(d_2) = \\
&= S_0e^{-qT}\left[
N\left(\frac{\sigma \sqrt{T}}{2}\right) - N\left(-\frac{\sigma \sqrt{T}}{2}\right)
\right]
\end{align*}
\end{frame}



\begin{frame}{Приближённое вычисление: ATMF-колл - 2}
\justify
Разложим $N(x)$ в ряд Тейлора:
\begin{align*}
N(x) \approx \frac{x}{\sqrt{2\pi}} \quad \Rightarrow
N(x) - N(-x) \approx \frac{2x}{\sqrt{2\pi}}
\end{align*}

Тогда:
\begin{align*}
N\left(\frac{\sigma \sqrt{T}}{2}\right) - N\left(-\frac{\sigma \sqrt{T}}{2}\right)
\approx
\frac{\sigma\sqrt{T}}{\sqrt{2\pi}}
\approx
0.4\sigma\sqrt{T}
\end{align*}

Если также предположить, что $q\approx 0\%$ и $e^{-qT} \approx 1$, то
\begin{align*}
C_{atmf} \approx 0.4 S_0 \sigma \sqrt{T}
\end{align*}
\end{frame}



\begin{frame}{Приближённое вычисление: ATMF-колл - 3}
\justify
Пример: спот-курс EURRUB равен $S=100$. Форвард на три месяца стоит $F=102$. Волатильность равна $\sigma=10\%$. Сколько стоит ATMF-колл со страйком $K=102$?

\begin{align*}
C_{atmf} \approx 0.4 \cdot S_0 \cdot \sigma \cdot \sqrt{T} = \\
0.4 \cdot 100 \cdot 0.1 \cdot \sqrt{\dfrac{1}{4}} \approx 2
\end{align*}

Единицы измерения --- рубли за право купить 1 евро.
\end{frame}



\begin{frame}{Динамическое хеджирование: пример}
\justify
Рассмотрим акцию, которая не платит дивидендов, стоит $S_0=\$100$ и следует 
геометрическому броуновскому движению с трендом $\mu=5\%$ и волатильностью
$\sigma=20\%$. Безрисковая ставка $r=0\%$.

\justify
Рассмотрим европейский колл-опцион со страйком $K=100$, сроком исполнения $T=1$ год. 
Представим, что нам не удалось его купить на рынке и мы занимаемся динамической 
репликацией.

\justify
Предположим, что сейчас у нас на счету есть $\$7.97$, и мы можем перебалансировать портфель $N=6$ раз в год (всего раз в два месяца).
\end{frame}



\begin{frame}{Динамическое хеджирование: цена акции}
\centering
\pgfplotsset{cycle list/Dark2}
\begin{tikzpicture}
\begin{axis}[
	width = \textwidth,
	height = \textheight - 0.5cm,
	xlabel = {Время (годы)},
	ylabel = {Цена базового актива ($S_t$)},
	xmin = 0, xmax = 1,
	ymin = 70, ymax = 140,
	grid = major
]	
    \foreach \n in {1,2,...,20} {
        \addplot+[thick, mark = *]
        table [col sep=comma, x=t, y=S_\n]
        {black_scholes_hedging_6_stock.csv};
    }
\end{axis}
\end{tikzpicture}
\end{frame}



\begin{frame}{Динамическое хеджирование: дельта}
\centering
\pgfplotsset{cycle list/Dark2}
\begin{tikzpicture}
\begin{axis}[
	width = \textwidth,
	height = \textheight - 0.5cm,
	xlabel = {Время (годы)},
	ylabel = {Кол-во акций в портфеле ($\Delta$)},
	xmin = 0, xmax = 1,
	ymin = 0, ymax = 1,
	ytick = {0, 0.1,...,1.1},
	grid = major
]	
    \foreach \n in {1,2,...,20} {
        \addplot+[thick, mark = *]
        table [col sep=comma, x=t, y=bs_delta_\n]
        {black_scholes_hedging_6_delta.csv};
    }
\end{axis}
\end{tikzpicture}
\end{frame}



\begin{frame}{Динамическое хеджирование: цена портфеля}
\centering
\pgfplotsset{cycle list/Dark2}
\begin{tikzpicture}
\begin{axis}[
	width = \textwidth,
	height = \textheight - 0.5cm,
	xlabel = {Время (годы)},
	ylabel = {Цена портфеля},
	xmin = 0, xmax = 1,
	ymin = -5, ymax = 40,
	grid = major,
	ytick = {-10, -5, ..., 50}
]	
    \foreach \n in {1,2,...,20} {
        \addplot+[thick, mark = *]
        table [col sep=comma, x=t, y=balance_\n]
        {black_scholes_hedging_6_balance.csv};
    }
    
    \addplot[black, very thick, solid, domain = 0:1] {0};
\end{axis}
\end{tikzpicture}
\end{frame}



\begin{frame}{Динамическое хеджирование: цена портфеля - 2}
\centering
\pgfplotsset{cycle list/Dark2}
\begin{tikzpicture}
\begin{axis}[
	width = \textwidth,
	height = \textheight - 0.5cm,
	xlabel = {Цена базового актива через $T=1$ год ($S_T$)},
	ylabel = {Цена портфеля},
	xmin = 70, xmax = 130,
	ymin = -5, ymax = 30,
	grid = major,
	ytick = {-10, -5, ..., 50}
]	

    \foreach \n in {1,2,...,20} {
        \addplot+[thick, mark = *, only marks]
        table [col sep=comma, x=S_\n, y=balance_\n]
        {black_scholes_hedging_6_summary.csv};
    }
    
    \addplot[black, very thick, solid, domain = 0:150] {0};
\end{axis}
\end{tikzpicture}
\end{frame}



\begin{frame}{Что, если хеджировать чаще?}
\justify
Зависимость цены реплицирующего портфеля через $T=1$ год от цены акции очень похожа на функцию выплаты европейского
колл-опциона. Зависимость неточная, но мы и перебалансировали портфель всего-то раз 2 месяца.

\justify
Что, если перебалансировать портфель 4 раза в день ($4 \cdot 365$ раз в год)? 
\end{frame}



\begin{frame}{Что, если хеджировать чаще - 2?}
\centering
\begin{tikzpicture}
\begin{axis}[
	width = \textwidth,
	height = \textheight - 0.5cm,
	xlabel = {Цена базового актива через $T=1$ год ($S_T$)},
	ylabel = {Цена портфеля},
	xmin = 70, xmax = 130,
	ymin = -5, ymax = 30,
	grid = major,
	ytick = {-10, -5, ..., 50}
]	

   \addplot[thick, mark = o, only marks, color=Set1-A]
     table [col sep=comma, x=S, y=balance]
     {black_scholes_hedging_1420.csv};    
    
    \addplot[black, very thick, solid, domain = 0:150] {0};
\end{axis}
\end{tikzpicture}
\end{frame}



\begin{frame}{Динамическое хеджирование: выводы}
\justify
Механистическая торговая стратегия, которая регулярно докупает или продаёт базовый актив, может привести к такому же
результату (итоговой выплате), что и <<настоящий>> европейский колл-опцион.

\justify
Инвестбанки могли бы продавать клиентам не <<настоящие>> опционы, а правильно настроенных торговых роботов. Захотел
купить европейский опцион --- настраиваешь торговый робот и отправляешь его на рынок заниматься дельта-хеджированием.
В мире нулевых транзакционных издержек и идеально ликвидных рынков <<настоящие>> опционы были бы не нужны.

\justify
Кстати, куда пропали \$7.97, которые были у нас на счёте с самого начала? Они перераспределились: когда они \$0,
когда-то --- выигрыш по опциону.
\end{frame}



\begin{frame}{Кругом обман}
\centering
\makebox[\textwidth]{\includegraphics[width=\textwidth]{we_need_to_go_deeper.jpg}}
\end{frame}



\begin{frame}{Историческая волатильность}
\justify
Если мы живём в мире Блэка-Шоулза, то нам достаточно знать волатильность базового актива, чтобы посчитать справедливую цену любого опциона. Почему бы не посмотреть на историю?

\justify
Рассмотрим временной ряд исторических цен базового актива $S_0,S_1,...,S_n$, взятых с фиксированным шагом $\Delta t$. Например, если у нас есть дневные данные, то $\Delta t = 1/250$ (по количеству рабочих дней в году). 

\justify
Посчитаем логарифмические доходности (\en{log returns}), которые близки к простым процентным доходностям:
\begin{align*}
r_i &= \ln \left( \dfrac{S_{i}}{S_{i-1}} \right) \\
\ln \left(\dfrac{S_{i}}{S_{i-1}} \right) &= \ln \left[1 + \left(\dfrac{S_{i}}{S_{i-1}} - 1 \right) \right] \approx \dfrac{S_{i}}{S_{i-1}} - 1
\end{align*}
\end{frame}



\begin{frame}{Историческая волатильность -- 2}
\justify
В модели Блэка-Шоулза все лог-доходности $r_i$ --- н.о.р.с.в. Их выборочное стандартное отклонение --- \alert{историческая} (\en{historical}) или \alert{реализованная} (\en{realized}) волатильность.

\begin{align*}
\hat{r} &= \frac{1}{n}\sum\limits_{i=1}^{n}r_i \\
\hat{\sigma} &= \sqrt{\frac{1}{n-1}\sum\limits_{i=1}^{n}(r_i - \hat{r})^2}
\end{align*}

\justify
Чтобы перевести <<дневную>> волатильность в проценты годовых, нужно разделить её на $\sqrt{\Delta t}$.
\end{frame}



\begin{frame}{Историческая волатильность -- 3}
\centering
\begin{tikzpicture}
\begin{axis}[
  width=\textwidth,
  height=\textheight - 1cm,
  date coordinates in=x,
  date ZERO=2012-01-01,
  xtick={2012-01-01, 2014-01-01, 2016-01-01, 2018-01-01, 2020-01-01, 2022-01-01, 2024-01-01},
  minor xtick={2013-01-01, 2015-01-01, 2017-01-01, 2019-01-01, 2021-01-01, 2023-01-01},
  ytick={30, 40, ..., 120},
%  minor ytick={-0.75, -0.25, 0.25, 0.75, 1.25},
  xticklabel={\year},
  xmin=2012-01-01,
  xmax=2025-01-01,
  ymin=20,
  ymax=130,
  grid=both,
  %yticklabel={\pgfmathprintnumber{\tick}\%},
  ylabel={\small{Курс USDRUB}},
  xlabel near ticks,
  ylabel near ticks
]

\addplot[color = Set1-B, mark = none, very thick]
	table[
		x=date,
		y=fx_rate,
		col sep=comma
	]
	{cbr.USD.2012.2024.csv};

\end{axis}
\end{tikzpicture}

\scriptsize Данные: ЦБ РФ.
\end{frame}



\begin{frame}{Историческая волатильность -- 4}
\centering
\begin{tikzpicture}
\begin{axis}[
  width=\textwidth,
  height=\textheight - 1cm,
  date coordinates in=x,
  date ZERO=2012-01-01,
  xtick={2012-01-01, 2014-01-01, 2016-01-01, 2018-01-01, 2020-01-01, 2022-01-01, 2024-01-01},
  minor xtick={2013-01-01, 2015-01-01, 2017-01-01, 2019-01-01, 2021-01-01, 2023-01-01},
  ytick={0.1, 0.2, ..., 1},
%  minor ytick={-0.75, -0.25, 0.25, 0.75, 1.25},
  xticklabel={\year},
  xmin=2012-01-01,
  xmax=2025-01-01,
  ymin=0,
  ymax=0.9,
  grid=both,
  yticklabel={\pgfmathparse{\tick*100}\pgfmathprintnumber[precision=0]{\pgfmathresult}\%},
  ylabel={\small{Реализованная волатильность}},
  xlabel near ticks,
  ylabel near ticks
]

\addplot[color = Set1-B, mark = none, thick]
	table[
		x=mid_month,
		y=realized_vol,
		col sep=comma
	]
	{USDRUB_realized_vol.csv};

\end{axis}
\end{tikzpicture}

\scriptsize Данные: ЦБ РФ.
\end{frame}



\begin{frame}{Историческая волатильность -- 5}
\justify
Историческая волатильность --- не константа, что противоречит модели Блэка-Шоулза. В прошлом мы видели большой разброс в реализованной волатильности. Сложно выбрать репрезентативный период, который хорошо подходил бы тому опциону (скажем, на 3 месяца), который мы пытаемся оценить.

\justify
Прошлое не предсказывает будущее!
\end{frame}



\begin{frame}{<<Ожидаемая>> волатильность}
\justify
В модели Блэка-Шоулза цена опциона зависит от волатильности:
\begin{align*}
C_K = F(S_0, T, K, \sigma, r, q)
\end{align*}

\justify
Мы можем посмотреть на рыночные цены опционов и решить задачу в обратном направлении. 

\justify
Если участники рынка пользуются моделью Блэка-Шоулза, то какую волатильность они подставляют в формулу, чтобы получить те премии, которые мы наблюдаем?
\begin{align*}
\sigma = F^{-1}(S_0, T, K, r, q, C_K)
\end{align*}

\justify
Решение этой обратной задачи --- \alert{<<ожидаемая>> (\en{implied})} волатильность.
\end{frame}



\begin{frame}{<<Ожидаемая>> волатильность -2}
\justify
Опционы на USDRUB с датой исполнения 20.06.2024.

\centering
\begin{tikzpicture}
\begin{axis}[
			width = \textwidth,
			height = \textheight - 2cm,
			xmin=70, xmax=120,
			ymin=0, ymax=25,
			xtick distance=5,
			ytick distance=5,
			grid = major,
			xlabel={Страйк ($K$)},
			ylabel={Цена опциона, руб.}
]

	\addplot[color = Set1-A, mark = none, thick]
	table[
		x=strike,
		y=call,
		col sep=comma
	]
	{usdrub_implied_vol.csv};
	
	\addplot[color = Set1-B, mark = none, thick]
	table[
		x=strike,
		y=put,
		col sep=comma
	]
	{usdrub_implied_vol.csv};
\end{axis}
\end{tikzpicture}

\scriptsize Данные: Московская биржа.
\end{frame}



\begin{frame}{<<Ожидаемая>> волатильность - 3}
\justify
Опционы на USDRUB с датой исполнения 20.06.2024.

\centering
\begin{tikzpicture}
\begin{axis}[
			width = \textwidth,
			height = \textheight - 2cm,
			xmin=70, xmax=120,
			ymin=0.1, ymax=0.30,
			xtick distance=5,
			ytick distance=0.02,
			yticklabel={\pgfmathparse{\tick*100}\pgfmathprintnumber[precision=0]{\pgfmathresult}\%},
			grid = major,
			xlabel={Страйк ($K$)},
			ylabel={Волатильность ($\sigma$)}
]

	\addplot[color = Set1-B, mark = none, thick]
	table[
		x=strike,
		y=iv,
		col sep=comma
	]
	{usdrub_implied_vol.csv};
\end{axis}
\end{tikzpicture}

\scriptsize Данные: Московская биржа.
\end{frame}



\begin{frame}{<<Ожидаемая>> волатильность - 4}
\justify
Опционы на S\&P\,500 с датой исполнения 21.06.2024.

\centering
\begin{tikzpicture}
\begin{axis}[
			width = \textwidth,
			height = \textheight - 2cm,
			xmin=4800, xmax=5800,
			ymin=0, ymax=500,
			xtick distance=100,
			ytick distance=50,
			grid = major,
			xlabel={Страйк ($K$)},
			ylabel={Цена опциона, \$},
			xticklabel={\pgfmathprintnumber[precision=0, 1000 sep={}]{\tick}}
]

	\addplot[color = Set1-A, mark = none, thick]
	table[
		x=strike,
		y=call,
		col sep=comma
	]
	{sp500_implied_vol.csv};
	
	\addplot[color = Set1-B, mark = none, thick]
	table[
		x=strike,
		y=put,
		col sep=comma
	]
	{sp500_implied_vol.csv};
\end{axis}
\end{tikzpicture}

\scriptsize Данные: barchart.com.
\end{frame}



\begin{frame}{<<Ожидаемая>> волатильность - 5}
\justify
Опционы на S\&P\,500 с датой исполнения 21.06.2024.

\centering
\begin{tikzpicture}
\begin{axis}[
			width = \textwidth,
			height = \textheight - 2cm,
			xmin=4800, xmax=5800,
			ymin=0.1, ymax=0.20,
			xtick distance=100,
			ytick distance=0.02,
			grid = major,
			xlabel={Страйк ($K$)},
			ylabel={Волатильность ($\sigma$)},
			xticklabel={\pgfmathprintnumber[precision=0, 1000 sep={}]{\tick}},
			yticklabel={\pgfmathparse{\tick*100}\pgfmathprintnumber[precision=0]{\pgfmathresult}\%},
]

	\addplot[color = Set1-B, mark = none, thick]
	table[
		x=strike,
		y=iv,
		col sep=comma
	]
	{sp500_implied_vol.csv};
\end{axis}
\end{tikzpicture}

\scriptsize Данные: barchart.com.
\end{frame}



\begin{frame}{Улыбка волатильности}
\justify
Почти на всех рынках наблюдается зависимость <<ожидаемой>> (\en{implied}) волатильности от страйка опциона. Дальние \en{out of the money}\ опционы стоят дороже (в терминах волатильности), чем можно было бы ожидать.
\begin{itemize}
\item Улыбка (\en{smile}). Например, на рынке FX.
\item Ухмылка (\en{skew, smirk}). Например, на рынке акций.
\end{itemize}

\justify
Может ли быть так, что базовый актив ведёт себя по-разному в зависимости от того, какой опцион (с каким страйком) сейчас оценивают участники рынка? Нет!

\justify
Либо все участники опционного рынка сошли с ума, либо модель Блэка-Шоулза не до конца описывает реальность. Предположение о постоянной волатильности и логнормальном распределении будущей цены базового актива выглядит слишком строгим.
\end{frame}



\begin{frame}{Опционная бабочка}
\justify
Попробуем оценить, какое распределение, если не логнормальное, ожидают участники рынка.

\justify
Рассмотрим комбинацию опционов <<бабочка>> (\en{butterfly, fly}):
\begin{itemize}
\item Купленный колл со страйком $K - \delta$.
\item Два проданных колла со страйком $K$.
\item Купленный колл со страйком $K + \delta$.
\end{itemize}

\justify
Купленные опционы --- <<крылья>> (\en{wings}), а проданные --- <<тельце>> (\en{belly}).
\end{frame}



\begin{frame}{Опционная бабочка - 2}
\justify
Пример: бабочка со страйками 98, 100 и 102.

\centering
	\begin{tikzpicture}
		\begin{axis}[
			%axis lines=middle,
			domain=92:108,
			xtick={92,94,...,108},
			ytick={-6,-5,...,6},
			xmin=92, xmax=108,
			ymin=-6, ymax=6,
			%x label style={at={(axis description cs: 0.5, -0.1)}, anchor=north},
			%y label style={at={(axis description cs:-0.1,1)},anchor=south},
			grid = major,
			xlabel={Курс в дату экспирации},
			ylabel={Выплата (payoff)},
			%scaled x ticks=false
		]
		
	\addplot[Set1-B, very thick, dashed] {(\x > 102)*(\x - 102) + 0.15};
  	\addplot[Set1-C, very thick, dashed] {(\x > 98)*(\x - 98) + 0.15};
  	\addplot[Set1-D, very thick, dashed] {2*(\x > 100)*(100 - \x) - 0.15};
 
 	\addplot[Set1-A, very thick] {(\x > 102)*(\x - 102) + (\x > 98)*(\x - 98) + 2*(\x > 100)*(100 - \x)};
 
   \draw[thick, color=black] (axis cs: 60, 0) -- (axis cs: 120, 0);
\end{axis}
\end{tikzpicture}
\end{frame}



\begin{frame}{Опционная бабочка - 3}
\justify
Пример: бабочка со стайками 99, 100 и 101.

\centering
	\begin{tikzpicture}
		\begin{axis}[
			width = \textheight,
			height = \textheight*0.5,
			domain=92:108,
			%axis lines=middle,
			xtick={92,94,...,108},
			ytick={0,...,6},
			xmin=92, xmax=108,
			ymin=-1, ymax=3,
			%x label style={at={(axis description cs: 0.5, -0.1)}, anchor=north},
			%y label style={at={(axis description cs:-0.1,1)},anchor=south},
			grid = major,
			xlabel={Курс в дату экспирации},
			ylabel={Выплата (payoff)},
			%scaled x ticks=false
		]
		
 	\addplot[Set1-A, very thick, samples at={92,92.1,...,108}] {(\x > 99)*(\x - 99) + (\x > 101)*(\x - 101) + 2*(\x > 100)*(100 - \x) + 0.05};
 
   \draw[thick, color=black] (axis cs: 60, 0) -- (axis cs: 120, 0);
\end{axis}
\end{tikzpicture}

\justify
По мере того, как разность между <<крыльями>> уменьшается, бабочка превращается в ставку на то, что цена базового актива остановится вблизи центрального страйка.
\end{frame}



\begin{frame}{Опционная бабочка - 4}
\justify
Бабочки со страйками $K-1$, $K$ и $K+1$. Экспирация  20.06.2024.

\centering
\begin{tikzpicture}
\begin{axis}[
			width = \textwidth,
			height = \textheight - 2cm,
			%domain=60:84,
			xmin=70, xmax=120,
			ymin=0, ymax=0.08,
			xtick distance=5,
			ytick distance=0.01,
			scaled y ticks = false,
			yticklabel={\pgfmathprintnumber[fixed, fixed zerofill, precision=2]{\tick}},
			grid = major,
			xlabel={Страйк ($K$)},
			ylabel={Цена бабочки, руб.}
]

	\addplot[color = Set1-B, mark = none, thick]
	table[
		x=strike,
		y=fly,
		col sep=comma
	]
	{usdrub_fly.csv};
\end{axis}
\end{tikzpicture}

\scriptsize Данные: Московская биржа.
\end{frame}



\begin{frame}{<<Ожидаемое>> распределение}
\centering
\begin{tikzpicture}
\begin{axis}[
			width = \textwidth,
			height = \textheight - 2cm,
			xmin=70, xmax=120,
			ymin=0, ymax=0.08,
			xtick distance=5,
			ytick distance=0.01,
			scaled y ticks = false,
			yticklabel={\pgfmathprintnumber[fixed, fixed zerofill, precision=2]{\tick}},
			grid = major,
			xlabel={Страйк ($K$)},
			ylabel={Плотность распределения},
			legend entries = {
				\small Ожидаемое,
				\small Логнормальное
			},
  			legend cell align={left},
 		 	legend style={at={(0.97,0.97)},anchor=north east}
]

	\addplot[color = Set1-B, mark = none, thick]
	table[
		x=strike,
		y=implied_density,
		col sep=comma
	]
	{usdrub_implied_density.csv};
	
	\addplot[color = Set1-A, mark = none, dashed, thick]
	table[
		x=strike,
		y=lognormal_density,
		col sep=comma
	]
	{usdrub_implied_density.csv};
\end{axis}
\end{tikzpicture}

\scriptsize Данные: Московская биржа.
\end{frame}



\begin{frame}{<<Ожидаемое>> распределение - 2}
\justify
Обычно цены опционов подразумевают распределение, отличное от логнормального. Часто можно видеть толстые хвосты и скошенность влево или вправо.

\justify
Мы не знаем, как в точности объясняется этот эффект:

\justify
1. Участники рынка верно оценивают истинную вероятность экстремальных исходов (больших изменений цены базового актив).

\justify
2. Участникам рынка настолько больно от экстремальных исходов, что они готовы переплатить за страховку. Это стремление застраховаться и избежать риска увеличивает рыночную цену опционов относительно <<фундаментально обоснованной>>.

\justify
Как обычно, если мы можем использовать ликвидные опционы для хеджирования риска и репликации более сложных экзотических продуктов, то нам не важно, какое объяснение верное.
\end{frame}



\begin{frame}{Модели волатильности}
\justify
Существование улыбки волатильности опровергает гипотезу о геометрическом броуновском движении с постоянной волатильностью. Как можно изменить модель движения базового актива, чтобы новая модель объясняла улыбку волатильности (давала те же премии, которые мы видим на рынке)?

\justify
\begin{itemize}
\item Локальная волатильность.
\item Стохастическая волатильность.
\item Стохастическая локальная волатильность.
\item Стохастическая локальная волатильность с прыжками.
\item ...
\end{itemize}
\end{frame}



\begin{frame}{Локальная волатильность}
\justify
Геометрическое броуновское движение:
\begin{align*}
\frac{dS_t}{S_t} = \mu dt + \sigma\xi\sqrt{dt}, \quad \xi \sim \mathcal{N}(0, 1), \sigma = const
\end{align*}

\justify
Локальная волатильность:
\begin{align*}
\frac{dS_t}{S_t} = \mu dt + \sigma(S_t, t)\xi\sqrt{dt}, \quad \xi \sim \mathcal{N}(0, 1)
\end{align*}

\justify
$\sigma(S_t, t)$ --- зависимость волатильности от цены базового актива и времени. Например, предположим, что если пара доллар-рубль значительно отклонится от текущего уровня (прыгнет до 120), то на рынке начнётся паника и волатильность будет выше, чем в <<спокойные>> времена при курсе 90.
\end{frame}



\begin{frame}{Стохастическая волатильность}
\justify
Геометрическое броуновское движение:
\begin{align*}
\frac{dS_t}{S_t} = \mu dt + \sigma\xi\sqrt{dt}, \quad \xi \sim \mathcal{N}(0, 1), \sigma = const
\end{align*}

\justify
Модель SABR (\en{Stochastic Alpha-Beta-Rho}):
\begin{align*}
dF_t &= \sigma_t(F_t)^\beta \sqrt{dt}\xi, \quad \xi \sim \mathcal{N}(0, 1) \\
d\sigma_t &= \alpha\sigma_t\sqrt{dt}\psi, \quad \psi \sim \mathcal{N}(0, 1) \\
Cov(\xi, \psi) &= \rho
\end{align*}

\justify
$\alpha$ --- волатильность волатильности.

$\beta$ --- коэффициент скошенности.

$\rho$ --- корреляция между волатильностью и базовым активом.
\end{frame}



\begin{frame}{Калибрация моделей}
\justify
Все модели волатильности подгоняются (\en{fitted}) или калибруются (\en{calibrated}) к рынку. Мы подбираем такие значения внутренних параметров модели (например, $\alpha$, $\beta$ и $\rho$ в SABR), чтобы модель <<лучше всего>> воспроизводила наблюдаемые цены ликвидных ванильных опционов. После этого модель можно использовать для оценки более сложных экзотических продуктов.

\begin{align*}
&\text{Рыночные цены} \Rightarrow \\
&\text{Внутренние параметры} \Rightarrow \\
&\text{Цены произвольных опционов}
\end{align*}

\justify
Объективная реальность, данная нам в ощущении --- рыночные цены ликвидных ванильных опционов. Модели и параметры --- наши предположения, которые позволяют <<интерполировать>> цены неликвидных деривативов.
\end{frame}



\begin{frame}{Бонус: вариационный своп}
\justify
\alert{Вариационный своп} (\en{variance swap}) --- финасовый дериватив, выплата которому зависит от реализованной дисперсии базового актива за заданный период.

\justify
Рассмотрим вариационный своп со страйком $\sigma_K^2$ (волатильность в квадрате). Номинал свопа $N$ долларов,в референсном интервале $n$ торговых дней. В конце жизни свопа покупатель получит выплату
\begin{align*}
P = N(\sigma_R^2 - \sigma_K^2)
\end{align*}

Здесь $\sigma_R^2$ --- реализованная дневная дисперсия базового актива:
\begin{align*}
\sigma_R^2 = \frac{252}{n-1}\sum\limits_{i=1}^{n}\left(\ln\frac{S_i}{S_{i-1}}\right)^2
\end{align*}
\end{frame}



\begin{frame}{Бонус: ценообразование вариационного свопа}
\justify
Хозяйке на заметку: вариационный своп длиной в $T$ лет можно реплицировать портфелем ванильных колл-опционов с тем же сроком экспирации. Каждый опцион входит в реплицирующий портфель с собственным весом $1/K^2$.

\begin{align*}
\sigma^2_K &= \frac{2e^{rT}}{T}\left(
\int\limits_{0}^{F}\frac{P(K)}{K^2}dK + 
\int\limits_{F}^{\infty}\frac{C(K)}{K^2}dK
\right) \\
F &= Se^{rT}
\end{align*}

\justify
Эта формула не зависит от предположений модели Блэка-Шоулза, она верна в любой модели, примерно как паритет опционов колл и пут.

\justify
Мы можем вычислить справедливый страйк вариационного свопа (мнение участников рынка относительно будущей дисперсии) из рыночных премий ванильных опционов.
\end{frame}



\begin{frame}{Бонус: индекс VIX}
\justify
Чикагская опционная биржа (\en{The Chicago Board Options Exchange, CBOE}) публикует индекс \en{VIX (Volatility IndeX)}, который измеряет волатильность 30-дневных опционов на индекс \en{S\&P\,500}.

\justify
Это не <<ожидаемая>> волатильность из модели Блэка-Шоулза (которая зависела бы от страйка), а цена вариационного свопа, вычисленная из котировок всех ликвидных ванильных опционов (то есть принимаются во внимание все стайки).

\justify
Индекс \en{VIX} часто называют <<индексом страха>>. Когда инвесторы паникуют, они бросаются покупать страховки от кризиса. Рыночные цены опционов растут, и вместе с ними растёт индекс \en{VIX}.

\justify
У индекса \en{VIX}\ есть брат-близнец \en{RVOL}, который измеряет реализованную волатильность индекса \en{S\&P\,500}.
\end{frame}



\begin{frame}{Бонус: индекс VIX и реализованная волатильность}
\centering
\begin{tikzpicture}
\begin{axis}[
  width=\textwidth,
  height=\textheight - 1cm,
  date coordinates in=x,
  date ZERO=2012-01-01,
  xtick={2002-01-01, 2005-01-01, 2010-01-01, 2015-01-01, 2020-01-01, 2025-01-01},
  minor xtick={2003-01-01, 2004-01-01, 2006-01-01, 2007-01-01, 2008-01-01, 2009-01-01, 2011-01-01, 2012-01-01, 2013-01-01, 2014-01-01, 2016-01-01, 2017-01-01, 2018-01-01, 2019-01-01, 2021-01-01, 2022-01-01, 2023-01-01, 2024-01-01},
  ytick={10, 20, ..., 100},
%  minor ytick={-0.75, -0.25, 0.25, 0.75, 1.25},
  xticklabel={\year},
  xmin=2002-01-01,
  xmax=2025-01-01,
  ymin=0,  ymax=100,
  grid=both,
  yticklabel={\pgfmathprintnumber{\tick}\%},
  xlabel near ticks,
  ylabel near ticks,
legend entries = {
				\small VIX,
				\small RVOL
			},
  			legend cell align={left},
			legend pos=north west
]

\addplot[color = Set1-A, mark = none,  thick]
	table[
		x=date,
		y=vix,
		col sep=comma
	]
	{vix_and_rvol.csv};

\addplot[color = Set1-B, mark = none, dotted, very thick]
	table[
		x=date,
		y=rvol,
		col sep=comma
	]
	{vix_and_rvol.csv};
\end{axis}
\end{tikzpicture}
\small Данные: \en{CBOE}
\end{frame}



\begin{frame}{Бонус: насколько глубока кроличья нора?}
\justify
1. Есть рынок акций США.

\justify
2. Есть индекс S\&P\,500 рынка акций США.

\justify
3. Есть фьючерсы и опционы на индекс S\&P\,500 рынка акций США.

\justify
4. Есть индекс \en{VIX} --- индекс волатильности опционов на индекс S\&P\,500 рынка акций США.

\justify
5. Есть фьючесры и опционы на индекс \en{VIX}\ волатильности опционов на индекс S\&P\,500 рынка акций США.
\end{frame}


\end{document}

