\documentclass[pdf,12pt]{beamer}

\usepackage{cmap}				% To be able to copy-paste russian text from pdf

\usepackage[utf8]{inputenc}
\usepackage[russian]{babel}
\usepackage{graphicx}
\usepackage{ragged2e} 
\usepackage{amssymb}
%\usepackage{amsmath, amsfonts}
\usepackage{ulem}
%\usepackage{indentfirst}
%\usepackage{fancyhdr}
\usepackage{tikz}
\usepackage{pgfplots}
\usepackage{color}
\usepackage{cancel}
\usepackage{multirow}
\usetikzlibrary{arrows,snakes,backgrounds,shapes}
\usepackage{animate}

\title{Вывод формулы Блэка-Шоулза}
\author{Артём Бакулин}
\date{21 октября 2021 г.}
\usetheme{Warsaw}
\usecolortheme{beaver}

\begin{document}

\begin{frame}
\titlepage
\end{frame}

\begin{frame}{Бонус: вывод формулы Блэка-Шоулза}
\justify
Рассмотрим биномиальное дерево из $N$ шагов, которое начинается с цены акции $S_0$. Европейский ПУТ имеет длину $T$ лет и страйк $K$. Безрисковая процентная ставка $r$.

\vspace{\baselineskip}
На каждом шаге цена может измениться в $u$ или в $d$ раз:
\begin{align*}
S_{i+1} &= \begin{cases}
S_iu\text{, где } u = 1 + \mu \dfrac{T}{N} + \sigma\sqrt{\dfrac{T}{N}} \\
S_id\text{, где } d = 1 + \mu \dfrac{T}{N} - \sigma\sqrt{\dfrac{T}{N}}\end{cases}
\end{align*}
Здесь $\mu$ --- тренд, $\sigma \ge 0$ -- волатильность.
\vspace{\baselineskip}

Потребуем дополнительно чтобы
\begin{align*}
d < 1+rT/N < u
\end{align*}
\end{frame}

\begin{frame}{Бонус: вывод формулы Блэка-Шоулза}
<<Риск-нейтральная вероятность>>:
\begin{align*}
p &= \dfrac{1+r\dfrac{T}{N} - d}{u - d} = \dfrac{\left(r-\mu\right)\dfrac{T}{N} + \sigma\sqrt{\dfrac{T}{N}}}{2\sigma\sqrt{\dfrac{T}{N}}} 
= \dfrac{1}{2} + \dfrac{r -\mu}{2\sigma}\sqrt{\dfrac{T}{N}} \\
1-p &= \dfrac{1}{2} - \frac{r -\mu}{2\sigma}\sqrt{\dfrac{T}{N}}
\end{align*}
Цена опциона в дереве из $N$ шагов:
\begin{equation*}
P = \dfrac{\sum\limits_{k=0}^{N} C^k_Np^k(1-p)^{N-k}\max(K - S_0u^kd^{N-k},0)}{\left(1+rT/N \right)^N}
\end{equation*}

\justify
Найти предел при $N \to \infty$ поможет теория вероятности, даже несмотря на то, что $p$ и $1-p$ не являются вероятностями.
\end{frame}

\begin{frame}{Бонус: вывод формулы Блэка-Шоулза}
Перейдем к дереву из логарифмов:

\centering
\begin{tikzpicture}
	\node(s) at (0 ,0) {$\ln S_i$};
	\node(su) at (3, 1) [anchor=west] {$\ln S_iu = \ln S_i + \ln u$};
	\node(sd) at (3, -1) [anchor=west] {$\ln S_id = \ln S_i + \ln d$};

	\draw (s) [->] -- (su) node[above,midway] {$p$} ;
	\draw (s) [->] -- (sd) node[below,midway] {$1-p$};
\end{tikzpicture}

\justify
Если интерпретировать $p$ как вероятность, то логарифм конечной цены акции $\ln S_N$ --- случайная величина:
\begin{equation*}
\ln S_N = \ln S_0 + \sum\limits_{i=0}^{N}X_i \text{,}
\end{equation*}
где все $X_i$ --- н.о.р.с.в. и имеют распределение
\begin{equation*}
X_i = \begin{cases}
\ln u \text{ с вероятностью } $p$ \\
\ln d \text{ с вероятностью } $1-p$
\end{cases}
\end{equation*}
\end{frame}

\begin{frame}{Бонус: вывод формулы Блэка-Шоулза}
Payoff опциона зависит от логарифма конечной цены:
\begin{equation*}
V(S_N) = \max(K - S_N ,0) \Rightarrow V(\ln S_N) = \max(K - e^{\ln S_N}, 0)
\end{equation*}

\justify
Тогда формулу цены опциона можно записать как математическое ожидание функции от случайной величины $\ln S_N$:
\begin{equation*}
P = \dfrac{\mathbb{E}\left(\max(K - e^{\ln S_N},0)\right)}{(1+rT/N)^N}
\end{equation*}
Хорошие новости: закон больших чисел и центральная предельная теорема позволяют вычислить предел числителя. Предел знаменателя --- просто $e^{rT}$.
\end{frame}

\begin{frame}{Бонус: вывод формулы Блэка-Шоулза}
Найдем математическое ожидание и дисперсию $X_i$:
\begin{equation*}
X_i = \begin{cases}
\ln u \text{ с вероятностью } $p$ \\
\ln d \text{ с вероятностью } $1-p$
\end{cases}
\end{equation*}

\justify
Разложим $\ln u$ и $\ln d$ по формуле Тейлора:
\begin{equation*}
\ln(1+x) = x - \dfrac{x^2}{2} + O(x^3)
\end{equation*}
Нас будут интересовать только слагаемые порядка $\dfrac{1}{N}$ и $\dfrac{1}{\sqrt{N}}$.
\end{frame}

\begin{frame}{Бонус: вывод формулы Блэка-Шоулза}
\begin{align*}
\ln u &= \ln \left(1 + \mu \dfrac{T}{N} + \sigma\sqrt{\dfrac{T}{N}} \right) = \\ 
&=
\mu \dfrac{T}{N} + \sigma\sqrt{\dfrac{T}{N}} - \dfrac{\mu^2T^2}{2N^2} - \dfrac{\sigma^2T}{2N} - \mu\sigma\left(\dfrac{T}{N}\right)^{1.5} + O\left(\frac{1}{N^{1.5}}\right) = \\
&=  \left(\mu - \dfrac{\sigma^2}{2}\right)\dfrac{T}{N} + \sigma\sqrt{\dfrac{T}{N}} + O\left(\frac{1}{N^{1.5}}\right)
\end{align*}
Аналогично,
\begin{align*}
\ln d &= \ln \left(1 + \mu \dfrac{T}{N} - \sigma\sqrt{\dfrac{T}{N}} \right) = \left(\mu -\dfrac{\sigma^2}{2}\right)\dfrac{T}{N} -\sigma\sqrt{\dfrac{T}{N}} + O\left(\frac{1}{N^{1.5}}\right)
\end{align*}
\end{frame}

\begin{frame}{Бонус: вывод формулы Блэка-Шоулза}
Математическое ожидание:
\begin{align*}
\mathbb{E}X_i &= p\ln u + (1-p)\ln d = \\
&= \left( \dfrac{1}{2} + \dfrac{r -\mu}{2\sigma}\sqrt{\dfrac{T}{N}} \right)
\left(\left(\mu - \dfrac{\sigma^2}{2}\right)\dfrac{T}{N} + \sigma\sqrt{\dfrac{T}{N}} + O\left(\frac{1}{N^{1.5}}\right) \right) + \\
&+ \left( \dfrac{1}{2} - \dfrac{r -\mu}{2\sigma}\sqrt{\dfrac{T}{N}} \right)
\left(\left(\mu - \dfrac{\sigma^2}{2}\right)\dfrac{T}{N} -\sigma\sqrt{\dfrac{T}{N}} + O\left(\frac{1}{N^{1.5}}\right) \right) =\\
&= \left(r - \dfrac{\sigma^2}{2}\right)\dfrac{T}{N} + O\left(\frac{1}{N^{1.5}}\right)
\end{align*}
Ожидание ожидаемо сходится к $0$ при $N \to \infty$.
\end{frame}

\begin{frame}{Бонус: вывод формулы Блэка-Шоулза}
Дисперсия:
\begin{align*}
Var(X_i) &= \mathbb{E}(X_i^2) - (\mathbb{E}X_i)^2 =  p\ln^2u + (1-p)\ln^2d - (\mathbb{E}X_i)^2
\end{align*}
Заметим, что
\begin{equation*}
(\mathbb{E}X_i)^2 = \left(\left(r - \dfrac{\sigma^2}{2}\right)\dfrac{T}{N} + O\left(\frac{1}{N^{1.5}}\right)\right)^2 = O\left(\frac{1}{N^2}\right)
\end{equation*}
Кроме того, 
\begin{align*}
\ln^2 u &=  \left((\mu - \dfrac{\sigma^2}{2})\dfrac{T}{N} + \sigma\sqrt{\dfrac{T}{N}} + O\left(\frac{1}{N^{1.5}}\right)\right)^2 = \sigma^2\dfrac{T}{N} + O\left(\frac{1}{N^{1.5}}\right) \\
\ln^2 d &=  \left((\mu - \dfrac{\sigma^2}{2})\dfrac{T}{N} - \sigma\sqrt{\dfrac{T}{N}} + O\left(\frac{1}{N^{1.5}}\right)\right)^2 = \sigma^2\dfrac{T}{N} + O\left(\frac{1}{N^{1.5}}\right)
\end{align*}
\end{frame}

\begin{frame}{Бонус: вывод формулы Блэка-Шоулза}
Тогда:
\begin{align*}
Var(X_i) &= p\ln^2u + (1-p)\ln^2d + O\left(\dfrac{1}{N^2}\right) = \\
&= p\left(\sigma^2\dfrac{T}{N} + O\left(\frac{1}{N^{1.5}}\right)\right) + \\ 
&+ (1-p)\left(\sigma^2\dfrac{T}{N} + O\left(\frac{1}{N^{1.5}}\right)\right)
+  O\left(\dfrac{1}{N^2}\right) = \\
& = \left(\sigma^2\dfrac{T}{N} + O\left(\frac{1}{N^{1.5}}\right)\right)
\left( p + 1 - p\right) + O\left(\dfrac{1}{N^2}\right) = \\
&= \sigma^2\dfrac{T}{N} + O\left(\frac{1}{N^{1.5}}\right)
\end{align*}
\end{frame}

\begin{frame}{Бонус: вывод формулы Блэка-Шоулза}
Вернемся к случайной величине $\ln S_N$:
\begin{align*}
\ln S_N &= \ln S_0 + \sum X_i 
= \ln S_0 + N \cdot \left( \dfrac{1}{N}\sum X_i \right) = \\
&= \ln S_0 + N \cdot \left( \dfrac{1}{N}\sum X_i - \left(r - \dfrac{\sigma^2}{2}\right)\dfrac{T}{N} + \left(r - \dfrac{\sigma^2}{2}\right)\dfrac{T}{N}\right) = \\
&= \ln S_0 +  \left(r - \dfrac{\sigma^2}{2}\right)T
+\sigma\sqrt{T} \cdot \sqrt{N}\dfrac{\dfrac{1}{N}\sum X_i - \left(r - \dfrac{\sigma^2}{2}\right)\dfrac{T}{N}}{\sigma\sqrt{T/N}} = \\
&= \ln S_0 +  \left(r - \dfrac{\sigma^2}{2}\right)T + \sigma\sqrt{T}Z
\end{align*}
\justify
где $Z\sim N(0,1)$ при $N \to \infty$ согласно центральной предельной теореме.
\end{frame}

\begin{frame}{Бонус: вывод формулы Блэка-Шоулза}
\justify
Зная распределение $\ln S_N$, мы можем записать математическое ожидание любой непрерывной и ограниченной функции $f(x)$ от $\ln S_N$:
\begin{equation*}
\mathbb{E}f(\ln S_N) = \int\limits_{-\infty}^{+\infty}f\left(\ln S_0 + \left(r-\dfrac{\sigma^2}{2}\right)T + \sigma\sqrt{T}x\right)\phi(x)dx \text{,}
\end{equation*}
где
\begin{equation*}
\phi(x) = \dfrac{1}{\sqrt{2\pi}}e^{-\frac{x^2}{2}}
\end{equation*}
--- плотность стандартного нормального распределения.
\end{frame}

\begin{frame}{Бонус: вывод формулы Блэка-Шоулза}
\justify
Функция выплаты пут-опциона
\begin{equation*}
f(\ln S_N) = \max(K - e^{\ln S_N},0)
\end{equation*}
является ограниченной, поэтому мы можем записать цену опциона пут:
\begin{align*}
P &= e^{-rT}\mathbb{E}(\max(K - e^{\ln S_N},0))+ =\\
&= e^{-rT}\int\limits_{-\infty}^{+\infty}\max(K - e^{\ln S_0 + \left(r-\frac{\sigma^2}{2}\right)T + \sigma\sqrt{T}x},0) \phi(x)dx
\end{align*}
\end{frame}

\begin{frame}{Бонус: вывод формулы Блэка-Шоулза}
\justify
Теперь нужно найти те значения $x$, при которых функция под интегралом отлична от нуля.
\begin{align*}
&K - e^{\ln S_0 + \left(r-\frac{\sigma^2}{2}\right)T + \sigma\sqrt{T}x} > 0 \\
&\ln S_0 + \left(r-\frac{\sigma^2}{2}\right)T + \sigma\sqrt{T}x < \ln K \\
&x < \dfrac{\ln\frac{K}{S_0} -  \left(r-\frac{\sigma^2}{2}\right)T}{\sigma\sqrt{T}}
\end{align*}
Обозначим
\begin{equation*}
d_2 = \dfrac{1}{\sigma\sqrt{T}}\left(\ln\frac{S_0}{K} + \left(r - \frac{\sigma^2}{2}\right)T \right)
\end{equation*}
Тогда
\begin{equation*}
x < -d_2
\end{equation*}
\end{frame}

\begin{frame}{Бонус: вывод формулы Блэка-Шоулза}
\justify
При $x \ge -d_2$ функция под интегралом равна нулю, поэтому можно интегрировать только до $-d_2$:
\begin{align*}
P &= e^{-rT}\int\limits_{-\infty}^{-d_2}(K - e^{\ln S_0 + \left(r-\frac{\sigma^2}{2}\right)T + \sigma\sqrt{T}x}) \phi(x)dx = \\
&= Ke^{-rT}\int\limits_{-\infty}^{-d_2}\phi(x)dx + \\
&+ \dfrac{S_0}{\sqrt{2\pi}} \int\limits_{-\infty}^{-d_2}e^{-\frac{\sigma^2}{2}T + \sigma\sqrt{T}x} \cdot e^{\frac{-x^2}{2}}dx
\end{align*}
Первое слагаемое -- это просто $Ke^{-rT}N(-d_2)$, где $N(x)$ --- функция распределения стандартного нормального распределения.
\end{frame}

\begin{frame}{Бонус: вывод формулы Блэка-Шоулза}
Последнее усилие --- рассмотреть второе слагаемое:
\begin{align*}
\dfrac{S_0}{\sqrt{2\pi}} \int\limits_{-\infty}^{-d_2}e^{-\frac{\sigma^2}{2}T + \sigma\sqrt{T}x -\frac{x^2}{2}}dx
=
\dfrac{S_0}{\sqrt{2\pi}} \int\limits_{-\infty}^{-d_2}e^{-\frac{1}{2}(x - \sigma\sqrt{T})^2}dx
\end{align*}
Сделаем замену $y = x - \sigma\sqrt{T}$, тогда при $x=-d_2$
\begin{align*}
y &= x - \sigma\sqrt{T} = -d_2 - \sigma\sqrt{T} = \\
&= -\dfrac{1}{\sigma\sqrt{T}}\left(\ln\frac{S_0}{K} + \left(r - \frac{\sigma^2}{2}\right)T \right) - \sigma\sqrt{T} = \\
&= -\dfrac{1}{\sigma\sqrt{T}}\left(\ln\frac{S_0}{K} + \left(r + \frac{\sigma^2}{2}\right)T \right) = -d_1
\end{align*}
\end{frame}

\begin{frame}{Бонус: вывод формулы Блэка-Шоулза}
С учетом подстановки получаем:
\begin{align*}
\dfrac{S_0}{\sqrt{2\pi}} \int\limits_{-\infty}^{-d_2}e^{-\frac{1}{2}(x - \sigma\sqrt{T})^2}dx
=
\dfrac{S_0}{\sqrt{2\pi}} \int\limits_{-\infty}^{-d_1}e^{-\frac{y^2}{2}}dy
=
S_0N(-d_1)
\end{align*}
Окончательно цена опциона пут равна
\begin{align*}
P = Ke^{-rT}N(-d_2) + S_0N(-d_1)
\end{align*}
где
\begin{align*}
d_1 &= \dfrac{1}{\sigma\sqrt{T}}\left(\ln\frac{S_0}{K} + \left(r + \frac{\sigma^2}{2}\right)T \right) \\
d_2 &= \dfrac{1}{\sigma\sqrt{T}}\left(\ln\frac{S_0}{K} + \left(r - \frac{\sigma^2}{2}\right)T \right)
\end{align*}
\end{frame}
\end{document}