\documentclass{beamer}

\usepackage{cmap}				% To be able to copy-paste russian text from pdf
\usepackage[T2A]{fontenc}
\usepackage[utf8]{inputenc}
\usepackage[english]{babel}
\usepackage{textpos}
\usepackage{ragged2e}
\usepackage{amssymb}
\usepackage{ulem}
\usepackage{tikz}
\usepackage{pgfplots}
\usepackage{color}
\usepackage{cancel}
\usepackage{multirow}
\pgfplotsset{compat=1.17}
\usetikzlibrary{arrows,snakes,backgrounds,shapes}
\usepgfplotslibrary{groupplots,colorbrewer,dateplot,statistics}
\usepackage{animate}

\usepackage{amsfonts}
\usepackage{amsmath}
\usepackage{amssymb}
\usepackage{graphicx}
\usepackage{setspace}

\usepackage{enumitem}
\setitemize{label=\usebeamerfont*{itemize item}%
  \usebeamercolor[fg]{itemize item}
  \usebeamertemplate{itemize item}}

% remove navigation bar
\setbeamertemplate{navigation symbols}{}

\setbeamertemplate{page number in head/foot}[totalframenumber] 

\usepackage{eurosym}
\renewcommand{\EUR}[1]{\textup{\euro}#1}

\title{Foreign Exchange Market. Forwards}
\author{Artem Bakulin}
\date{October 15, 2024}

\usetheme{Warsaw}
\usecolortheme{beaver}

\begin{document}

\begin{frame}
\titlepage
\end{frame}



\begin{frame}{Foreign exchange spot}
\justify
\alert{Foreign exchange spot, or FX spot} is a trade in which two parties exchange two currencies "almost immediately" ("on the spot"), usually on the second business day.

\justify
Buying a currency pair, such as EURUSD, involves receiving the first currency (EUR) and paying the second one (USD). For example, buying 1\,000\,000 of EURUSD spot at exchange rate of 1.0921 results in the following payments:
\justify

\centering
\begin{tabular}{l|r|r}
Date                          & EUR & USD \\ \hline
Tue 15.10.2024 (today)  & 0   & 0   \\
Thu 17.10.2024 (spot)     & +1\,000\,000 & $-1\,092\,100$
\end{tabular}
\end{frame}



\begin{frame}{Figures and pips}
\justify
Some market conventions originate from pre-computer era, when people had to negotiate
over phone.

\begin{align*}
\underbrace{\text{\Large 1.09}}_{\text{Figures}}\underbrace{\text{\Large 21}}_{\text{Pips}}
\end{align*}
\justify

\alert{Big figures} are  the most significant digits that are typically not pronounced. \alert{Pips} are the third and the fourth digits after the decimal point (1st and 2nd in USDJPY), and it is essential to negotiate them.

\justify
--- Good morning. Euro [EURUSD], 10 [million], please?

--- 21--23. [We buy at 1.09\underline{21}, we sell at 1.09\underline{23}].

--- Sold. [We pay you 10\,000\,000 EUR and receive 10\,9\underline{21}\,000 USD].
\end{frame}



\begin{frame}{Over-the-counter market and market-makers}
\justify
FX market operates primarily \alert{over-the-counter (OTC)}. There is no single central trading platform or an exchange to which every participant would have to connect to.

\justify
Most companies and individuals purchase currency from market-makers, usually from large investment banks. \alert{Market-makers}, literally, make buying and selling rates for their customers:

\centering
\begin{tabular}{c|c|c|c}
Bid & Mid & Ask/offer & Spread \\ \hline
1.09\underline{21} & 1.09\underline{22} & 1.09\underline{23} & 2 pips
\end{tabular}

\justify
\alert{Bid} is the rate at which customers sell the first currency. \alert{Ask} is the rate at which customers buy the first currency. \alert{Mid} is the average of the bid and the ask. \alert{Spread} is the difference between the ask and the bid.

\justify
When a client requests a price (a quote), they specify the size of the trade, but do not indicate whether they intend to buy or to sell!
\end{frame}



\begin{frame}{Market-maker's profit}
\justify
Suppose that a bank has sold 1\,000\,000 EUR at 1.09\underline{23} to one client, and has managed to buy 1\,000\,000 EUR at 1.09\underline{21} from another client. How much money has the bank made?

\centering
\begin{tabular}{l|r|r}
Trade & EUR & USD \\
\hline
Sell at 1.09\underline{23} & $-1\,000\,000$ & $+1\,092\,300$ \\
Buy at 1.09\underline{21} & $+1\,000\,000$ & $-1\,092\,100$ \\
\hline
Net & 0 & $+200$
\end{tabular}

\justify
Do you have to buy Euros before selling Euros? Not necessarily, because usually FX spot trades  settle on the second business day.

\justify
This market-maker seems to be making money out of thin air. Where is the catch? Could everyone follow the same strategy?
\end{frame}



\begin{frame}{Market risk}
\justify
\alert{Market risk} is the possibility of incurring a loss due to changes in market prices.

\justify
1. Client A bought 1\,000\,000 EUR at 1.09\underline{23} from a market-maker.

\justify
2. For some reason, clients who could be selling are inactive.

\justify
3. Breaking news: the European Central Bank (ECB) has hiked interest rates, and Euro is strengthening. Other market-makers are now quoting 1.09\underline{51}/1.09\underline{53}.

\justify
4. Client B who is willing to sell requests a quote. The market-maker must offer at least 1.09\underline{51}. Otherwise the client will prefer competitors' offers.

\justify
5. Summary: the market-maker sold EUR at 1.09\underline{23} to client A, and bought EUR at 1.09\underline{51} from client B. Net loss is \$2\,800.
\end{frame}



\begin{frame}{Managing market risk}
\justify
The only way to eliminate the market risk entirely is to shut down all trading systems, terminate traders and close the bank. What could we do to maintain market risk at an acceptable and reasonable level?

\justify
1. Widen the spread.
\justify
2. Offer skewed quotes.
\justify
3. Close our position on the market.
\end{frame}



\begin{frame}{1. Widen the spread}
\justify
Wider bid/ask spread provides a margin of safety and acts as a buffer in case the market moves
rapidly.

\justify
Scenario 1 (all trades are 1\,000\,000 EUR) :

1. Quote 1.09\underline{21}/1.09\underline{23} to client A. Client A buys at \underline{23}.

2. The market moves up by 30 pips to 1.09\underline{51}/1.09\underline{53}.

3. Client B sells at 1.09\underline{51}. Loss: $-\$2\,800$.

\justify
Scenario 2 (all trades are 1\,000\,000 EUR) :

1. Quote 1.08\underline{97}/1.09\underline{47} to client A. Client A buys at \underline{47}.

2. The market moves up by 30 pips to 1.09\underline{27}/1.09\underline{77}.

3. Client B sells at 1.09\underline{27}. Profit: $+\$20\,000$.

\justify
Bad news: clients do not like excessively wide spreads and will gladly trade with our competitors.
\end{frame}



\begin{frame}{2. Skewed quotes}
\justify
What if we offer more favorable conditions on trades that reduce our market risk?
\justify
1. Quote 1.09\underline{21}/1.09\underline{23} to client A. Client A buys at \underline{23}.

\justify
2. Client B requests a quote, and we quote 1.09\underline{22}/1.09\underline{24}.

\justify
2a. The bid (left hand side) \underline{22} is more favorable now. In case client B is willing to sell, there is high probability that they will prefer us over competitors. We sold at \underline{23} and bought at \underline{22}. Profit is $+\$100$. Less risk, less profit.

\justify
2b. The ask (right hand side) \underline{24} is less favorable. In case client B is willing to buy, they will probably trade with our competitors. If they still trade with us, we charge an extra pip for the increased risk.
\end{frame}



\begin{frame}{3. Trade on the market }
\justify
We could potentially ask another market-maker to make a quote for us.

\justify
1. We quoted 1.09\underline{21}/1.09\underline{23} to client A. Client A bought at \underline{23}.

\justify
2. Lets call our competitors across the street and request a quote. We act as a client, and they are a market-maker.

\justify
3. They quote 1.09\underline{20.5}/1.09\underline{22.5}. We buy at \underline{22.5}.

\justify
4. We sold at \underline{23}, we bought at \underline{22.5}, we made \$50. This profit is even less than with strategy 2, but it comes with lower risk.

\justify
Buy the way, why did our competitor skew the price to the left? Presumably, they have recently bought from client B at \underline{21}. Now they are employing strategy 2 --- they are offering a better price \underline{22.5}.

\justify
This way two market-makers can collaborate to lock in a profit and eliminate market risk.
\end{frame}



\begin{frame}{Market-maker's handbook}
\justify
So, do you need to predict the future in order to make money by market-making? No!

\justify
Market-maker's handbook:

\justify
1. Adjust the spread. Balance the risk of clients trading away with the expected volatility of the market (balance caution and greed).

\justify
2. Skew your quotes in case your risk exceeds your comfort level.

\justify
3. Keep in mind that you can trade with other market-makers as necessary.

\justify
Market-making is not risk-free. Each individual trade may result in a loss. However, if you execute a sufficiently large number of trades, and your systems function properly, you can make money on average in the long run.
\end{frame}



\begin{frame}{Market-makers and the fair price}
\justify
How does a market-maker know what is the \alert{fair} price?

\justify
Junior trader Artem joined a large bank and was put in charge of market-making EURUSD. Isolated in a bunker, he is cut off from external communication channels. The sole source of information available to Artem are clients' requests and completed trades.

\justify
09:00 --- Client A requested a quote. Artem quoted 1.09\underline{21}/\underline{23}.

09:00 --- Client A bought at \underline{23}.

\justify
09:01 --- Client B requested a quote. Artem quoted 1.09\underline{21}/\underline{23}.

09:01 --- Client B bought at \underline{23}.

\justify
09:02 --- Client C requested a quote. Artem quoted 1.09\underline{21}/\underline{23}.

09:02 --- Client C bought at \underline{23}.

\justify
Should Artem be worried at this point?
\end{frame}



\begin{frame}{Market-makers and the fair price - 2}
\justify
Artem faces increased market risk due to a series of unidirectional right-hand side trades. What is his next step? Artem must begin skewing his quotes to the right (strategy 2).

\justify
09:03 --- Client D requested a quote. Artem quoted 1.09\underline{22}/\underline{24}.

09:03 --- Nevertheless, Client D bought at \underline{24}.

\justify
09:04 --- Client E requested a quote. Artem quoted 1.09\underline{23}/\underline{25}.

09:04 --- Finally, client E sold at \underline{23}!

\justify
09:05 --- Client F requested a quote. Artem quoted 1.09\underline{23}/\underline{25}.

09:05 --- Client F bought at  \underline{25}!

\justify
Despite Artem's isolation, he successfully found an equilibrium price $\underline{23}/\underline{25}$, at which the demand from buyers matches the supply from sellers.
\end{frame}



\begin{frame}{Market-makers and the fair price - 3}
\justify
A ''fair'' market price balances the supply and demand. If clients are only buying or only selling, the quoted price
is probably incorrect (too low or too high).

\justify
A market-maker does not need to know the "fundamentally fair" exchange rate, and the existence of such a rate is questionable.

\justify
The market is essential, because it collects information from billions of economic agents (importers, exporters, speculators, investors) and consolidates this information into market prices. If it was possible to collect all relevant information, we would not need the market in the first place.

\justify
Market-makers merely help the invisible hand of the market find equilibrium. Market-makers do not attempt to forecast the exchange rate, although some of their clients may engage in speculative betting.
\end{frame}



\begin{frame}{Do we need market-makers?}
\justify
Market-makers are not able to forecast the future, and they just earn profits from bid/ask spreads. In this case, why do we need them?

\justify
What if we allow all market participants to trade directly with each other, and eliminate the need to pay spreads to \sout{banksters} market-makers?

\justify
1. Difficulties of finding a counterparty. This could be addressed by establishing a centralized exchange, a fintech marketplace, etc.

\justify
2. Counterparty risk. This could be mitigated by central clearing counterparties.

\justify
3. Adverse selection and information asymmetry.
\end{frame}



\begin{frame}{A natural experiment}
\justify
A natural experiment is taking place on the U.S. market for index credit default swaps.

\justify
1. The Dodd-Frank Act (2010) mandates trading venues to facilitate both kinds of trading: requesting a quote from a market-maker, and submitting an order into anonymous all-vs-all order book.

\justify
2. Clients usually prefer requesting quotes from market-makers.

\justify
3. In 95\% of cases, market-makers offer a better spread than the spread available on the anonymous market.

\justify
4. The wider was the spread quoted BEFORE a trade, the more the market moves against the market-maker AFTER the same trade.

\justify
\small{Pierre Collin-Dufresne, Benjamin Junge, and Anders B Trolle. \textit{Market structure and transaction costs of index CDSs}. Swiss Finance Institute Research Paper 18-40. 2018}
\end{frame}


\begin{frame}{Adverse selection}
\justify
You plan to buy 1\,000 GBP for an upcoming holiday. You place an order on an anonymous Fintech platform: "I will buy 1\,000 GBPEUR at 1.19 or better". Your order is filled almost instantly. Another user of this platform, an algorithmic hedge fund, has traded with you. Is this a good or a bad outcome?

\justify
On one hand, you have achieved your goal promptly. On the other hand, it is likely that you've fallen victim to a more informed participant. The hedge fund had sold you GBPEUR at 1.19 just seconds before the rate dropped to 1.18.

\justify
This is called \alert{adverse selection}. When you place an order you bear the risk that a better-informed participant will make money on you.
\end{frame}



\begin{frame}{Adverse selection: an example}
\justify
Suppose you are buying an asset. You place an order "I will buy the asset at $x$ or cheaper". Once you place an order, an uninformed participant will enter the market and will sell the asset to you at this price $x$. Then both of you will learn the true value of the asset: either \$90 or \$110 with 50\%/50\% probability.

\justify
Suppose you place an order to buy at $x=\$100$.

\centering
\begin{tabular}{l|l}
True value & Uninformed seller (100\%)  \\ \hline
\$90 (50\%) & Trade (loss $-\$10$) \\ \hline
\$110 (50\%) & Trade (gain $+\$10$) 
\end{tabular}

\justify
Which price $x$ allows you neither win nor lose on average?
\begin{align*}
0.5\cdot(\$110 - x) + 0.5\cdot(\$90 - x) = 0 \quad \Leftrightarrow \quad x = \$100
\end{align*}

\justify
In a competitive market you'll have to submit your order at \$100.00 or perhaps \$99.99. Otherwise you will never trade, because your competitors will always bid higher prices. In case sellers face a similar problem, they will be submitting orders to sell at \$100.00 or perhaps \$100.01.
\end{frame}



\begin{frame}{Adverse selection: an example (cont.)}
\justify
Let's introduce a bit of information asymmetry. There is 10\% probability that you will have to trade with an \alert{informed} seller, who knows the true value of the asset in advance.

\justify
Suppose you place an order to buy at $x=\$100$.

\centering
\begin{tabular}{l|l|l}
True value & Uninformed (90\%) & Informed (10\%) \\ \hline
\$90 (50\%) & Trade (loss $-\$10$) & Trade (loss $-\$10$) \\ \hline
\$110 (50\%) & Trade (gain $+\$10$) & \alert{No trade (\$0)} 
\end{tabular}

\justify
What is new equilibrium price $x$ for your order?
\begin{align*}
0.5\cdot\alert{0.9}\cdot(\$110 - x) + 0.5 \cdot (\$90 - x) = 0 \Leftrightarrow x \approx \$99.47
\end{align*}

\justify
You must act first by placing an order without knowing in advance who will trade with you. To be safe, you have to demand a discount of \$0.53. This represents the cost of adverse selection and information asymmetry.The bid/ask spread needs to be wider, and uninformed sellers bear the cost of the existence of informed sellers.
\end{frame}



\begin{frame}{Market-makers and adverse selection}
\justify
How could you avoid paying the cost of adverse selection? You could introduce yourself to the buyer and explain that you are a 
harmless retail investor rather than an algorithmic hedge fund.

\justify
A market-maker knows in advance \alert{who} is requesting a quote and can tailor the bid/ask spread accordingly. Informed clients encounter a wider spread, because their trades are hazardous. Uninformed clients enjoy a more narrow spread.

\justify
Many clients can expect a more favorable spread if they call a market-maker and introduce themselves. They would face wider bid/ask spread if they chose anonymous market where everyone could trade with everyone.

\justify
Market-makers accumulate historical data, asses individual trading patterns and estimate the appropriate bid/ask spread for each client. As a result, corporate clients can get a better spread than hedge funds, so that they do not bear the cost of adverse selection.
\end{frame}



\begin{frame}{Financial derivatives}
\justify
A \alert{financial derivative} is a contract whose profit or loss depends on, or is derived from, the behavior of another financial asset known as the \alert{underlying} asset.

\justify
Examples of underlying assets: foreign exchange, interest rates, securities, wheat, weather.

\justify
Examples of derivatives: forwards and futures, options, interest rate swaps, credit default swaps.

\justify
Market participants can use derivatives for risk-management (hedging) or to make bets on future changes in market prices (speculation). 
\end{frame}



\begin{frame}{Foreign exchange forwards}
\justify
\alert{Foreign exchange forward, or outright forward} is a contract, in which two parties commit to exchanging pre-determined amounts in two currencies on a pre-determined future date at a pre-determined rate. 

\justify
Example: buying 1\,000\,000 of EURUSD 1 year forward at forward rate of 1.112:

\centering
\begin{tabular}{l|r|r}
Date                          & EUR & USD \\ \hline
Thu 17.10.2024 (today)  & 0   & 0   \\
Mon 21.10.2024 (spot) & 0   & 0   \\
Tue 21.10.2025 (1Y)   & 1\,000\,000 & $-1\,112\,000$
\end{tabular}

\justify
The forward contract does not cost anything by itself (you do not pay anything today). To "buy" a forward means to accept an obligation to receive the first currency and pay the second currency. However you need to negotiate the exchange rate with the seller.
\end{frame}



\begin{frame}{Forward rate and forward points}
\justify
\alert{Forward rate} is the future exchange rate which parties of the forward contract negotiate in advance. Occasionally it is
referred to as the "price" of a forward.

\justify
Sometimes market participants prefer to use the difference between the forward rate and the spot rate, which is called \alert{forward points}. This is convenient, because the spot rate may be fluctuating and pulling the forward rate every millisecond. The difference between the two rates is more stable.

\vspace{\baselineskip}
\centering
\begin{tabular}{l|l}
Spot rate & 1.091 \\
Forward rate & 1.112 \\
\hline
Forward points & $1.112 - 1.091 = 0.021 = 210$ pips
\end{tabular}

\justify
* By convention, 1 pip in EURUSD is equal to 0.0001.
\end{frame}



\begin{frame}{Currency risk}
\justify
Suppose that we are a natural gas exporter from Russia. We signed a contract to deliver gas to Europe. In 1 year, we must deliver 1\,000\,000 cubic meters of gas, for which we will be paid 1\,000\,000 EUR. The production cost for this volume of gas is 100\,000\,000 RUB. Current EURRUB spot rate is 105.0.

\justify
\centering
\begin{tabular}{r|r|r|r|r|r}
EURRUB      & Revenue & \multicolumn{2}{c|}{Spot} & Expense & Profit \\
\cline{3-4}
in 1 year   & EUR   & EUR   & RUB & RUB & RUB \\ \hline
95.0        & $+1$  & $-1$ & $+95$         & $-100$  & $-5$ \\
105.0        & $+1$ & $-1$ & $+105$         & $-100$  & $+5$ \\
115.0        & $+1$ & $-1$ & $+115$          & $-100$ & $+15$ 
\end{tabular}

(all amounts are in millions)

\justify
This is \alert{currency risk}: our profit is subject to change depending on fluctuations of the FX spot rate.
\end{frame}



\begin{frame}{Hedging currency risk}
\justify
We can mitigate currency risk by entering into an FX forward contract. For instance, one large German investment bank from Frankfurt might buy EURRUB forward from us at a forward rate of 110.0.

\justify
\centering
\begin{tabular}{r|r|r|r|r|r}
EURRUB      & Revenue & \multicolumn{2}{c|}{Forward} & Expense & Profit \\
\cline{3-4}
in 1 year   & EUR     & EUR    & RUB              & RUB     & RUB   \\ \hline
95.0        & $+1$  & $-1$ & $+110$  & $-100$ & $+10$  \\
105.0        & $+1$  & $-1$ & $+110$  & $-100$ & $+10$  \\
115.0        & $+1$  & $-1$ & $+110$  & $-100$ & $+10$ 
\end{tabular}

\justify
A forward is our obligation. Even if EURRUB rises to 130.0 in 1 year, we remain committed to selling EUR at 110.0. The forward contract protects us from the risk of a loss (in case EURRUB declines) and caps the potential profit (in case EURRUB rises).

\justify
Do we need to possess 1\,000\,000 EUR today in order to sell a forward? No!
\end{frame}



\begin{frame}{Speculating on currency risk}
\justify
Lets dream that we have a crystal ball and we know for sure that EUR will depreciate in 1 year. How could we monetize our knowledge? Straightforward answer: sell 1 million of EURRUB on the spot. What if we do not have 1 million euro at hand?

\justify
Solution: sell a EURRUB 1 year forward at forward rate of 110.0. In 1 year  we will receive 110\,000\,000 RUB. We will buy EUR on the spot at the prevailing spot rate.

\justify
\centering
\begin{tabular}{r|r|r|r|r|r}
EURRUB      & \multicolumn{2}{c|}{Forward} & \multicolumn{2}{c|}{Spot} & Profit \\
\cline{2-5}
in 1 year & EUR     & RUB     & EUR     & RUB      & RUB \\ \hline
90.0      & $-1$ & $+110$  & $+1$ & $-90$ & $+20$ \\
110.0      & $-1$ & $+110$  & $+1$ & $-110$ & 0 \\
130.0     & $-1$& $+110$  & $+1$ & $-130$ & $-20$ \\
\end{tabular}
 
\justify
* A bank which will buy our forward will probably ask for a margin, but this margin will be less than 1\,000\,000 EUR.
\end{frame}



\begin{frame}{Fair forward rate}
\justify
What factors determine the "fair" forward rate, at which both the buyer and the seller will be happy to sign the contract? 

\justify
Suppose that you are a European investor. You have 10\,000 EUR today. In 1 year you will require RUB for a holiday trip to Sochi. How could you secure the amount of RUBs which you will possess in 1 year?
\end{frame}



\begin{frame}{Fair forward rate - 2}
\justify
Current EURRUB spot rate is $S_{eurrub}=100$, risk-free bank deposit in EUR earns $r_{eur}=3\%$, RUB deposit is $r_{rub}=20\%$. 
Forward market offers forwards at $F_{eurrub}=115.0$.

\justify
Strategy 1: buy RUB today

\centering
\begin{tabular}{l|r|r|r|r}
& \multicolumn{2}{c|}{Today} & \multicolumn{2}{c}{In 1 year} \\ \cline{2-5}
& EUR & RUB & EUR & RUB \\ \hline
Sell spot & $-10\,000$ & $+1\,000\,000$ & & \\
Deposit RUB & & $-1\,000\,000$ & & $+1\,200\,000$ \\ \hline
Total & $-10\,000$ & $0$ & & $+1\,200\,000$
\end{tabular}

\justify
Strategy 2: sell a forward

\centering
\begin{tabular}{l|r|r|r|r}
& \multicolumn{2}{c|}{Today} & \multicolumn{2}{c}{In 1 year} \\ \cline{2-5}
& EUR & RUB & EUR & RUB \\ \hline
Deposit EUR & $-10\,000$ & & $+10\,300$ & \\
Sell a forward &   &   & $-10\,300$ & $+1\,184\,500$ \\ \hline
Total & $-10\,000$ &   & $0$ & $+1\,184\,500$
\end{tabular}

\end{frame}



\begin{frame}{Fair forward rate - 3}
\justify
Is it plausible that two strategies yield different results?

\justify
\centering
\begin{tabular}{l|l}
Strategy 1 & Strategy 2 \\ \hline
+1\,200\,000 RUB  & +1\,184\,500 RUB \\
$10\,000 \cdot S_{eurrub} \cdot (1+r_{rub})$ & $10\,000 \cdot (1+r_{eur}) \cdot F_{eurrub}$
\end{tabular}

\justify
Strategy 1 is clearly better, and everyone will follow it. 

1) More investors sell EURRUB spot --- spot rate $S_{eurrub}$ gets lower.

2) Demand for RUB deposits is higher --- interest rate $r_{rub}$ gets lower.

3) Nobody needs deposits in EUR --- interest rate $r_{eur}$ gets higher. 

4) Nobody is selling forwards --- forward rate $F_{eurrub}$ gets higher.
\end{frame}



\begin{frame}{Fair forward rate - 4}
\justify
Suppose that forward rate on the market is higher, e.g. $F_{eurrub}=118$.


\justify
\centering
\begin{tabular}{l|l}
Strategy 1 & Strategy 2 \\ \hline
+1\,200\,000 RUB  & +1\,215\,400 RUB \\
$10\,000 \cdot S_{eurrub} \cdot (1+r_{rub})$ & $10\,000 \cdot (1+r_{eur}) \cdot F_{eurrub}$
\end{tabular}

\justify
This time strategy 2 is better. 

1) High demand for EUR deposits --- $r_{eur}$ gets lower.

2) More sellers of forwards --- $F_{eurrub}$ gets lower.

3) Nobody sells spot --- $S_{eurrub}$ gets higher. 

4) RUB deposits are unpopular --- $r_{rub}$ gets higher.
\end{frame}



\begin{frame}{Fair forward rate - 5}
\justify
At which forward rate $F_{eurrub}$ the market will be in equilibrium, and strategies 1 and 2 will be equivalent?

\begin{align*}
&10\,000 \cdot S_{eurrub} \cdot (1 + r_{rub}) = 10\,000 \cdot (1+r_{eur}) \cdot F_{eurrub} \Rightarrow \\
&F_{eurrub} = S_{eurrub} \frac{1 + r_{rub}}{1 + r_{eur}} = 100 \cdot \frac{1 + 0.20}{1 + 0.03} \approx 116.50
\end{align*}

\justify
Conclusion: fair forward rate only depends on current spot rate and current interest rates in the two currencies.

\justify
You do not need to forecast the future to trade forwards!
\end{frame}



\begin{frame}{Arbitrage}
\justify
Is there a guarantee that market prices will converge to equilibrium in finite time? Could the market potentially remain imbalanced for a long time?

\justify
When a client deposits cash with a bank at the risk-free rate, the bank borrows the money at the risk-free rate. Let's assume that some market participants are able to borrow money at the risk-free rate. How does this change our model?
\end{frame}



\begin{frame}{Arbitrage - 2}
\justify
Suppose that the forward rate is lower than the equilibrium rate.

$S_{eurrub}=100$, $r_{eur}=3\%$, $r_{rub}=20\%$, $F_{eurrub}=115$. 

\justify
\centering
\small{
\begin{tabular}{l|r|r|r|r}
& \multicolumn{2}{c|}{Today} & \multicolumn{2}{c}{In 1 year} \\ \cline{2-5}
& EUR & RUB & EUR & RUB \\ \hline
Borrow EUR     & $+10\,000.00$ &                                   & $-10\,300.00$ & \\
Sell spot            & $-9\,870.83$ & $+987\,083$ &                               & \\
Deposit RUB    &                                & $-987\,083$  &                                & $+1\,118\,450$ \\
Buy a forward &                               &                                 & $+10\,300.00$ & $-1\,118\,450$ \\ \hline
Total & $+129.17$ & 0 & 0 & 0
\end{tabular}
}

\justify
129 EUR out of nothing. Too good to be true?
\end{frame}



\begin{frame}{Arbitrage -  3}
\justify
Suppose that the forward rate is higher than the equilibrium rate.

 $S_{eurrub}=100$, $r_{eur}=3\%$, $r_{rub}=20\%$, $F_{eurrub}=118$. 

\justify
\centering
\small{
\begin{tabular}{l|r|r|r|r}
& \multicolumn{2}{c|}{Today} & \multicolumn{2}{c}{In 1 year} \\ \cline{2-5}
& EUR & RUB & EUR & RUB \\ \hline
Borrow RUB     &                               & $+1\,000\,000$  &                            & $-1\,200\,000$  \\
Buy spot            & $+10\,000.00$ & $-1\,000\,000$ &                               &   \\
Deposit EUR    &  $-9\,873.29$  &                                 & $+10\,169.49$ &  \\
Sell a forward &                               &                                & $-10\,169.49$ & $+1\,200\,000$ \\ \hline
Total & $+126.71$ & 0 & 0 & 0
\end{tabular}
}

\justify
We are making money out of thin air!
\end{frame}



\begin{frame}{Arbitrage - 4}
\justify
An \alert{arbitrage strategy} is a combination of trades which allows a market participant to earn risk-free profit with zero initial investment. This market participant is known as an \alert{arbitrageur}.

\justify
Flocks of greedy arbitrageurs watch the market day and night. If the market forward rate deviates from the correct value for just a second, they will swoop down on this mispricing and help the invisible hand of the market restore equilibrium. Everyone wants to make money out of thin air!

\justify
We will be pricing forwards and other derivatives with the assumption that a fair price should leave no opportunities for arbitrage.

\justify
"There is no such thing as a free lunch!"
\end{frame}



\begin{frame}{Replicating a forward}
\justify
Consider the following two strategies.

\justify
\small{
\begin{tabular}{l|r|r|r|r}
 & \multicolumn{2}{c|}{Today} & \multicolumn{2}{c}{In 1 year} \\ \cline{2-5}
Strategy  1& EUR & RUB & EUR & RUB \\ \hline
Buy a forward &                              &                                & $+10\,300$ & $-1\,200\,000$
\end{tabular}
}

\justify
\small{
\begin{tabular}{l|r|r|r|r}
& \multicolumn{2}{c|}{Today} & \multicolumn{2}{c}{In 1 year} \\ \cline{2-5}
Strategy 2 & EUR & RUB & EUR & RUB \\ \hline
Borrow RUB &                             & $+1\,000\,000$ &                    & $-1\,200\,000$ \\
Buy spot        & $+10\,000$ & $-1\,000\,000$  &                     &                                  \\
Deposit EUR & $-10\,000$  &                                  & $+10\,300$ & \\  \hline
Total & 0 & 0 &  $+10\,300$ & $-1\,200\,000$
\end{tabular}
}

\justify
Strategy 2 perfectly \alert{replicates} strategy 1. Imagine that these two strategies are hidden inside two black boxes. You can only observe EUR and RUB cash flows produced by the black boxes. You will never be able to find out which black box contains the "true" forward and which contains the synthetic strategy.
\end{frame}



\begin{frame}{Pricing derivatives via replication}
\justify
A bun costs \$0.30, a piece of meat costs \$2.00, salad costs \$1.00, sauce costs \$0.50. What would be the fair price of a burger on a perfectly efficient and liquid market? \$3.80 plus assembly cost, which is close to zero on financial markets.

\justify
The price of a derivative is literally derived from the prices of underlying market instruments. It is more like a relative value (if ingredients cost X then what is the price of their combination?) rather than absolute (what is the fundamentally fair price of a burger?).

\justify
Suppose that we have sold a burger to a client at \$3.81 and managed to assemble it from ingredients at \$3.80. We have locked in a profit of \$0.01, even if the market went crazy and the fundamentally fair price of a burger is \$1.00 or \$10.00.

\justify
Even better, a market maker can buy a burger from client A at \$3.79 and sell the same burger to client B at \$3.81.
\end{frame}



\begin{frame}{Arbitrage-free forward rate}
\justify
There will be no arbitrage opportunities on the market if and only if forward rate depends on spot rate and interest rates.

\begin{align*}
F_{xxxyyy} &= S_{xxxyyy} \frac{1 + r_{yyy}T_{yyy}}{1 + r_{xxx}T_{xxx}} \\
%F_{xxxyyy} &= S_{xxxyyy} e^{r_{yyy}^*T_{yyy} - r_{xxx}^*T_{xxx}} \\
%F_{xxxyyy} &= S_{xxxyyy} \frac{\delta_{xxx}(T)}{\delta_{yyy}(T)}%
\end{align*}

Here $r_{xxx}$ is interest rate (annualized), $T_{xxx}$ is number of years between today and delivery date.

\justify
Does forward rate predict future spot rate? No!
\end{frame}



\begin{frame}{Foreign exchange futures}
\justify
\alert{Foreign exchange futures} are exchange-traded contracts, which allow to exchange one currency for another at a pre-determined rate on a pre-determined future date.

\justify
What is difference between futures and forwards?

\justify
1. Futures are traded on an exchange, forwards are traded OTC.

\justify
2. Futures are standardized contracts with a specific lot size and specific delivery dates. Forwards are customizable.

\justify
3. Trading futures requires posting cash on a margin account. This is optional in forward contracts.

\justify
Fair futures rate is the same as fair forward rate.
\end{frame}



\begin{frame}{Futures and predicting future spot rate}
\center
\begin{tikzpicture}
\begin{axis}[
  width=\textwidth,
  height=\textheight - 1cm,
  date coordinates in=x,
  date ZERO=2014-01-01,
  xtick={2014-02-01,2014-04-01, 2014-06-01, 2014-08-01, 2014-10-01, 2014-12-01},
  xticklabel={\day.\month.14},
  xmin=2014-01-01,
  xmax=2014-12-31,
  ymin=30,
  ymax=62,
  grid=major,
  ylabel={\small{USDRUB rate}},
  xlabel near ticks,
  ylabel near ticks,
  legend entries = {
      Spot rate (Bank of Russia),
      December 2014 futures (Moscow Exchange)
  },
  legend pos=north west,
  %legend style={font=\tiny},
  legend cell align={left}
]
\addplot[color=Set1-A, mark=none, thick] table[x=date, y=cbr_spot_rate, col sep=comma]{Si-12.14.csv};
\addplot[color=Set1-B, mark=none, thick] table[x=date, y=futures_price, col sep=comma]{Si-12.14.csv};
\end{axis}
\end{tikzpicture}

\scriptsize Data: Central Bank of Russia, Moscow Exchange.
\end{frame}




\begin{frame}{Carry trade}
\justify
\alert{Currency carry trade} is an investment strategy which borrows cash in currencies that have lower interest rates (e.g. JPY) and invests into currencies that have higher interest rates (e.g. USD). A simple example: buying a USDJPY future (USD interest rates are higher).

\justify
On average, in the long run, high-yield currencies depreciate not as dramatically as one could expect from the interest rate differential.

\justify
Doskov, Nikolay and Swinkels, Laurens. \textit{Empirical Evidence on the Currency Carry Trade, 1900--2012}. Journal of International Money and Finance 51, pp. 370-389. 2015.

\end{frame}



\begin{frame}{Partially convertible currencies}
\justify
A few currencies are not fully convertible. Moreover, foreign exchange derivatives may be banned.

\justify
Examples:

1. Brazilian Real (BRL)

2. Chinese Renminbi (CNY)

3. Indian Rupee (INR)

4. Korean Won (KRW)

5. Taiwan Dollar (TWD)

...

\end{frame}



\begin{frame}{The impossible trinity}
\justify
International finance trilemma (the impossible trinity): you can choose any two options out of three. 

1. Fixed exchange rate of a national currency

2. Free capital flow across the border.

3. Independent monetary policy.

\justify
For example, suppose that Russia decided to peg USDRUB at 24.0 as if today was 2007.

1. U.S. Fed is raising interest rates --- investors buy more dollars

2. To maintain the peg, Central Bank of Russia has to sell dollars from its reserves.

3. The reserves are large but finite. Sooner or later the Central Bank has to either raise RUB interest rates following U.S. Fed move, or let USDRUB fluctuate.

\end{frame}



\begin{frame}{Case study: China}

\justify
1. Capital flow (buying stocks and bonds) is restricted and requires case by case approval by the Government.

\justify
2. Current operations (paying for goods and services) are not restricted.

\justify
3. After submitting documents (e.g. a contract for goods or services) one is allowed to buy or sell CNY on  China Foreign Exchange Trading System (CFETS).

\justify
4. People's Bank of China is a counterparty in 70\% of trades on CFETS and has full control over exchange rate.

\justify
5. Derivatives are banned!

\end{frame}



\begin{frame}{Non-deliverable forward}
\justify
How could companies manage currency risk in case forwards are banned? They need a special derivative: a \alert{non-deliverable forward (NDF)}.

\justify
A hypothetical non-deliverable forward between Citi and Deutsche:

\justify
\centering
\begin{tabular}{l|l}
	Contract & Non-deliverable forward		\\
	Deutsche "sells"  & 1\,000\,000 USD	\\
	Deutsche "buys" & 7\,000\,000 CNY		\\
	Rate		 		      & 7.00 						\\
	Delivery date	   & 21.11.2024 \\
	Reference rate	   & People's Bank of China rate	\\
	Fixing date		   & 19.11.2024 \\
	Today (reference)	& 17.10.2024 \\
	Spot date (reference) & 21.10.2024
\end{tabular}
\end{frame}



\begin{frame}{Mechanics of a non-deliverable forward}
\justify
There are no payments at inception (on October 17).

\justify
On the fixing date (November 19) the PBoC announces its official exchange rate, e.g. 6.90.

\justify
Deutsche guessed that the rate will be lower, so

A. Deutsche "sells" 1\,000\,000 USD.

B. Deutsche "buys" $7\,000\,000 \text{CNY} = \dfrac{7\,000\,000}{6.90} = 1\,014\,493 \text{ USD}$.

\justify
On delivery date (November 21) Citi will pay Deutsche the net gain 14\,493 USD.

\justify
Market participants do not have to report their non-deliverable forwards to the Government of China. A non-deliverable forward is an ideal tool for speculators.
\end{frame}



\begin{frame}{Hedging currency risk with an NDF}
\justify
A fruit company from the U.S. has to pay 7\,000\,000 CNY to its supplier in a month.

1. Sell 1\,000\,000 of USDCNY non-deliverable forward at 7.00.

2. On the fixing date buy 7\,000\,000 CNY at prevailing (and unknown) spot rate  $S$.

3. On the delivery date receive both the USD gain on the NDF and the CNY from the spot trade.

\justify
\centering
\small
\begin{tabular}{l|r|r|r|r}
& \multicolumn{2}{c|}{In case $S=6.90$} & \multicolumn{2}{c}{In case $S=7.10$} \\ \cline{2-5}
Transaction & USD & CNY & USD & CNY \\ \hline
NDF & $+14\,493$ & & $-14\,084$ & \\
Spot & $-1\,014\,493$ & $+7\,000\,000$ & $-985\,916$ & $+7\,000\,000$ \\
Pay the supplier & & $-7\,000\,000$ & & $-7\,000\,000$ \\ \hline
Total & $-1\,000\,000$ & 0 & $-1\,000\,000$ & 0
\end{tabular}
\end{frame}



\begin{frame}{Pricing a non-deliverable forward}
\justify
Capital controls make void arguments of market equilibrium and no arbitrage opportunities. Price of a non-deliverable forward may deviate from the theoretical forward rate, and nobody will be able to arbitrage it away.

\justify
A non-deliverable forward for 1 month is closer to an underlying asset (i.e. spot) although technically it is a derivative. The invisible hand of the market is looking for such a NDF rate at which the demand (those willing to pay to suppliers in China or to make bet on USD strengthening) is equal to the supply (those willing to repatriate profit from China or to speculate on CNY strengthening).
\end{frame}

\end{document}