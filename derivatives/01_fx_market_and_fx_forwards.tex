\documentclass{beamer}

\usepackage{cmap}				% To be able to copy-paste russian text from pdf
\usepackage[T2A]{fontenc}
\usepackage[utf8]{inputenc}
\usepackage[russian]{babel}
\usepackage{textpos}
\usepackage{ragged2e}
\usepackage{amssymb}
\usepackage{ulem}
\usepackage{tikz}
\usepackage{pgfplots}
\usepackage{color}
\usepackage{cancel}
\usepackage{multirow}
\pgfplotsset{compat=1.17}
\usetikzlibrary{arrows,snakes,backgrounds,shapes}
\usepgfplotslibrary{groupplots,colorbrewer,dateplot,statistics}
\usepackage{animate}

\usepackage{amsfonts}
\usepackage{amsmath}
\usepackage{amssymb}
\usepackage{graphicx}
\usepackage{setspace}

\usepackage{enumitem}
\setitemize{label=\usebeamerfont*{itemize item}%
  \usebeamercolor[fg]{itemize item}
  \usebeamertemplate{itemize item}}

% remove navigation bar
\setbeamertemplate{navigation symbols}{}

\setbeamertemplate{page number in head/foot}[totalframenumber] 

\usepackage{eurosym}
\renewcommand{\EUR}[1]{\textup{\euro}#1}

\title{Валютный рынок. Валютные форварды}
\author{Артём Бакулин}
\date{4 марта 2024 г.}

\usetheme{Warsaw}
\usecolortheme{beaver}

\newcommand{\ru}[1]{\begin{otherlanguage}{russian}#1\end{otherlanguage}}
\newcommand{\en}[1]{\begin{otherlanguage}{english}#1\end{otherlanguage}}
\newcommand{\ruen}[2]{#1 (\en{#2})}

\newcommand{\eurusdBigFigures}{1.08}
\newcommand{\eurusdBigFiguresMinusOne}{1.07}
\newcommand{\eurusdBigFiguresPlusTwo}{1.10}
\newcommand{\eurusdBigFiguresPerMillion}{1\,08}
\newcommand{\eurusdBigFiguresPlusTwoPerMillion}{1\,10}
\newcommand{\eurusdBusinessDate}{Пн 04.03.2024}
\newcommand{\eurusdSpotDate}{Ср 06.03.2024}
\newcommand{\eurusdOneYearDate}{Чт 06.03.2025}

\begin{document}



\begin{frame}
\titlepage
\end{frame}



\begin{frame}{Валютный спот}
\justify
\alert{Валютная спот-сделка} (\en{foreign exchange spot, FX spot}) --- сделка, в которой стороны меняют одну валюту на другую <<почти сразу>> (\en{on the spot}), обычно на второй рабочий день.

\justify
Купить валютную пару, например EURUSD, означает согласиться получить первую валюту (EUR) и отдать вторую (USD). Например, купить спот 1\,000\,000 EURUSD по курсу \eurusdBigFigures21 означает следующие платежи:

\justify
\centering
\begin{tabular}{l|r|r}
Дата                          & EUR & USD \\ \hline
\eurusdBusinessDate\ (сегодня)  & 0   & 0   \\
\eurusdSpotDate\ (спот)     & +1\,000\,000 & $-\eurusdBigFiguresPerMillion2\,100$
\end{tabular}
\end{frame}



\begin{frame}{Фигуры и пипы}
\justify
Некоторые традиции валютного рынка сложились в докомпьютерную эру, когда нужно
было договариваться о сделках по телефону.
\begin{align*}
\underbrace{\text{\Large \eurusdBigFigures}}_{\text{Figures}}\underbrace{\text{\Large 21}}_{\text{Pips}}
\end{align*}
\justify
<<Фигуры>> (\en{big figures}) ---  старшие значащие цифры, которые можно пропустить в 
разговоре. <<Пипы>> (\en{pips}) --- третья и четвёртая цифры после запятой (первая и вторая 
для \en{USDJPY}), о которых и нужно договориться.

\justify
--- Добрый день, евро (EURUSD), 1 (миллион), пожалуйста?

--- 21--23. (Покупаем по \eurusdBigFigures\underline{21}, продаём по \eurusdBigFigures\underline{23}).

--- Продали. (Отдадим вам 1\,000\,000 EUR, получим \eurusdBigFiguresPerMillion2\,100 USD).
\end{frame}



\begin{frame}{Внебиржевой рынок и маркет-мейкеры}
\justify
Валютный рынок, в основном, является \alert{внебиржевым} (over-the-counter, OTC). Не 
существует одной биржи (exchange), к которой подключались бы все участники торгов.

\justify
Большинство компаний и граждан покупают валюту у маркет-мейкеров (market makers), в 
основном, у крупных инвестбанков. Маркет-мейкеры, буквально, производят цены покупки 
и продажи для своих клиентов:

\centering
\begin{tabular}{c|c|c|c}
Bid & Mid & Ask/offer & Spread \\ \hline
\eurusdBigFigures\underline{21} & \eurusdBigFigures\underline{22} & \eurusdBigFigures\underline{23} & \underline{2}
\end{tabular}

\justify
Bid --- курс, по которому клиент продаёт первую валюту. Ask --- курс, по которому клиент 
покупает первую валюту. Mid --- среднее между bid и ask. Спред (spread) --- разность между 
ask и bid.

\justify
Когда клиент просит цену, он говорит объём сделки, но не уточняет, будет ли он покупать или продавать!
\end{frame}



\begin{frame}{Прибыль маркет-мейкера}
\justify
Предположим, что банк продал одному клиенту 1\,000\,000 EUR по \eurusdBigFigures\underline{23} и
смог купить у другого клиента 1\,000\,000 EUR по \eurusdBigFigures\underline{21}. Сколько 
заработал банк?

\centering
\begin{tabular}{l|r|r}
Сделка & EUR & USD \\
\hline
Продажа по \eurusdBigFigures\underline{23} & $-1\,000\,000$ & $+\eurusdBigFiguresPerMillion2\,300$ \\
Покупка по \eurusdBigFigures\underline{21} & $+1\,000\,000$ & $-\eurusdBigFiguresPerMillion2\,100$ \\
\hline
Итого & 0 & $+200$
\end{tabular}

\justify
Нужно ли сначала купить евро, прежде чем продать? Не обязательно, ведь расчёты 
происходят на второй рабочий день.

\justify
Если маркет-мейкер делает деньги из воздуха, то в чём подвох? Почему так не могут
делать все?
\end{frame}



\begin{frame}{Рыночный риск}
\justify
\alert{Рыночный риск} (\en{market risk}) --- возможность оказаться в убытке в результате изменения рыночных цен.

\justify
1. Клиент A купил у маркет-мейкера 1\,000\,000 EUR по \eurusdBigFigures\underline{23}.

\justify
2. По какой-то причине клиенты-продавцы пока бездействуют.

\justify
3. Вышла новость: ЕЦБ повышает ставки, евро укрепляется. Другие маркет-мейкеры теперь предлагают цены \eurusdBigFigures\underline{51}/\eurusdBigFigures\underline{53}.

\justify
5. Клиент-продавец Б всё-таки дошёл до нашего маркет-мейкера и просит котировку.
Придётся предложить ему как минимум \eurusdBigFigures\underline{51}, чтобы он не ушёл к конкурентам.

\justify
6. Итого: продали клиенту А по \eurusdBigFigures\underline{23}, купили у клиента Б по
\eurusdBigFigures\underline{51}. Убыток: \$2\,800.
\end{frame}



\begin{frame}{Управление рыночным риском}
\justify
Единственный способ полностью избавиться от рыночного риска --- выключить 
все торговые системы, уволить трейдеров и закрыть бизнес. Что можно сделать
для того, чтобы поддерживать приемлемый уровень рыночного риска?

\justify
1. Расширить спред.

\justify
2. Предлагать клиентам скошенные (\en{skewed}) цены.

\justify
3. Закрыть позицию на рынке.
\end{frame}



\begin{frame}{1. Расширение спреда}
\justify
Более широкий спред даёт дополнительный запас прочности на случай резкого
движения рынка.

\justify
Сценарий 1 (все сделки по 1\,000\,000 EUR) :

1. Клиент А получил спред \eurusdBigFigures\underline{21}/\eurusdBigFigures\underline{23} и купил по \underline{23}.

2. Рынок сдвинулся на 30 пипов вверх до \eurusdBigFigures\underline{51}/\eurusdBigFigures\underline{53}.

3. Клиент Б продал по \eurusdBigFigures\underline{51}. Убыток: $-\$2\,800$.

\justify
Сценарий 2 (все сделки по 1\,000\,000 EUR) :

1. Клиент А получил спред \eurusdBigFiguresMinusOne\underline{97}/\eurusdBigFigures\underline{47} и купил по \underline{47}.

2. Рынок сдвинулся на 30 пипов вверх до \eurusdBigFigures\underline{27}/\eurusdBigFigures\underline{77}.

3. Клиент Б продал по \eurusdBigFigures\underline{27}. Прибыль: $+\$20\,000$.

\justify
Проблема: клиенты не любят слишком широкие спреды и с радостью убегут к 
конкурентам.
\end{frame}



\begin{frame}{2. Скошенные цены}
\justify
Что если предлагать клиентам более выгодные цены на те сделки, которые сокращают
наш рыночный риск?

\justify
1. Клиент А получил спред \eurusdBigFigures\underline{21}/\eurusdBigFigures\underline{23} и купил по \underline{23}.

\justify
2. Клиент Б обратился за котировкой, и мы предложили ему \eurusdBigFigures\underline{22}/\eurusdBigFigures\underline{24}.

\justify
2a. Bid (левая цена) \underline{22} стал выгоднее. Если клиент Б пришёл 
продавать, то с большей вероятностью он продаст нам, а не конкурентам.
Мы продали по \underline{23} и купили по \underline{22}. Прибыль: $+\$100$ (меньше, чем полный спред \$200, зато и риск меньше).

\justify
2b. Ask (правая цена) \underline{24} стал хуже. Если клиент Б пришёл покупать,
то скорее всего он уйдёт к конкурентам. Если же он купит у нас, то пусть заплатит за то, что увеличит наш риск.
\end{frame}



\begin{frame}{3. Закрытие позиции на рынке}
\justify
Что если обратиться к другому маркет-мейкеру?

\justify
1. Клиент А получил спред \eurusdBigFigures\underline{21}/\eurusdBigFigures\underline{23} и купил по \underline{23}.

\justify
2. Позвоним коллегам из другого банка в соседнем небоскрёбе и попросим 
котировку (как будто мы -- клиент, а они --- маркет-мейкер)

\justify
3. Нам предложили \eurusdBigFigures\underline{20.5}/\eurusdBigFigures\underline{22.5}. Мы купили по \underline{22.5}.

\justify
4. Продали по \underline{23}, купили по \underline{22.5} --- заработали \$50. 
Ещё меньше,чем в стратегии 2, зато совсем наверняка.

\justify
Кстати, а почему коллеги выдали нам скошенную влево цену? Возможно, они сами недавно купили у клиента Б по \underline{21} и теперь стимулируют нас, чтобы мы 
купили у них по супер-цене \underline{22.5}.

\justify
Так два маркет-мейкера могут оба зафиксировать прибыль и избавиться от риска.
\end{frame}



\begin{frame}{Прибыль и риск маркет-мейкера}
\justify
Нужно ли уметь предсказывать будущие курсы валют, чтобы зарабатывать маркет-мейкерством на валютном рынке? Нет!

\justify
Памятка начинающего маркет-мейкера:

\justify
1. Управляй спредом, балансируя риск оттока клиентов и амплитуду колебаний рынка (осторожность против жадности).

\justify
2. Начинай сдвигать цены влево или вправо, если твой риск начинает расти выше комфортного уровня.

\justify
3. Помни, что можешь избавиться от рыночного риска на рынке, если нужно.

\justify
Стратегия маркет-мейкера не является безрисковой. Каждая отдельно взятая сделка
может принести убыток. Но если сделок много, и все системы настроены правильно,
то в среднем на длинной дистанции всё будет хорошо.
\end{frame}



\begin{frame}{Маркет-мейкер и справедливая цена}
\justify
Откуда маркет-мейкер знает <<справедливую>> рыночную цену?

\justify
Артём устроился младшим трейдером ДжиПиСитиБанк и отвечает за пару EURUSD. Артём сидит в бункере без связи с внешним миром. Единственное, что он наблюдает --- сделки клиентов.

\justify
09:00 --- Клиент А просит котировку. Артём предложил \eurusdBigFigures\underline{21}/\underline{23}. 
09:00 --- Клиент А купил по \underline{23}.

\justify
09:01 --- Клиент Б просит котировку. Артём предложил \eurusdBigFigures\underline{21}/\underline{23}.
09:01 --- Клиент Б купил по \underline{23}.

\justify
09:02 --- Клиент В просит котировку. Артём предложил \eurusdBigFigures\underline{21}/\underline{23}. 
09:02 --- Клиент В купил по \underline{23}.

\justify
Не пора ли Артёму насторожиться?
\end{frame}



\begin{frame}{Маркет-мейкер и справедливая цена - 2}
\justify
Что сделает Артём, получив подряд несколько сделок на покупку, которые увеличили 
его риск? Начнёт сдвигать цену вправо (применит стратегию номер 2).

\justify
09:03 --- Клиент Г просит котировку. Артём предложил \eurusdBigFigures\underline{22}/\underline{24}.
09:03 --- Клиент Г всё равно купил по \underline{24}.

\justify
09:04 --- Клиент Д просит котировку. Артём предложил \eurusdBigFigures\underline{23}/\underline{25}.
09:04 --- Клиент Д наконец-то продал по \underline{23}!

\justify
09:05 --- Клиент Е просит котировку. Артём предложил \eurusdBigFigures\underline{23}/\underline{25}.
09:05 --- Клиент Е купил по \underline{25}!

\justify
Изоляция от мира не помешала Артёму нащупать равновесную цену --- ту, при которой поток продавцов примерно равен потоку покупателей.
\end{frame}



\begin{frame}{Маркет-мейкер и справедливая цена - 3}
\justify
<<Справедливая>> рыночная цена балансирует спрос и предложение.
Если клиенты только покупают или только продают, то с большой вероятностью цена 
неправильная (слишком высокая или низкая).

\justify
Маркет-мейкеру не нужно знать <<фундаментально обоснованный>> курс валюты. Не факт, что этот <<фундаментально обоснованный>> курс вообще существует.

\justify
Для того и нужен рынок, чтобы собирать информацию от миллионов экономических 
агентов (импортёров, экспортёров, спекулянтов, инвесторов) и обобщать её в виде
рыночной цены. Если бы был способ собрать всю релевантную информацию, то нам бы
был не нужен сам рынок.

\justify
Маркет-мейкеры только помогают невидимой руке рынка найти равновесие. Пытаться 
угадать направление движение курса могут некоторые клиенты, но не сам маркет-мейкер.
\end{frame}



\begin{frame}{Нужны ли маркет-мейкеры?}
\justify
Если маркет-мейкеры не умеют предсказывать будущее и только зарабатывают на 
разнице курсов покупки и продажи, подобно обменнику у метро, то зачем они вообще 
нужны?

\justify
Почему бы всем участникам рынка не торговать друг с другом напрямую и не платить
спред банкирам-кровопийцам?

\justify
1. Сложности с поиском контрагента. Это можно решить одной общей биржей,
маркет-плейсом, финтехом и т.д.

\justify
2. Риск контрагента. Решается сделками через централизованный клиринг.

\justify
3. Неблагоприятный отбор.
\end{frame}



\begin{frame}{Естественный эксперимент}
\justify
Естественный эксперимент на рынке индексных кредитных свопов в США:

\justify
1. Закон обязывает торговые площадки предоставлять оба метода торговли: и обращение к маркет-мейкеру, и анонимный рынок всех со всеми.

\justify
2. Клиенты чаще предпочитают обращение к маркет-мейкеру.

\justify
3. В 95\% случаев клиенты получают от маркет-мейкера спред лучше, чем на анонимном рынке.

\justify
4. Чем шире спред, который маркет-мейкер выставил ДО сделки, тем сильнее рынок двигается против маркет-мейкера ПОСЛЕ сделки.

\justify
\small{Pierre Collin-Dufresne, Benjamin Junge, and Anders B Trolle. \textit{Market structure and transaction costs of index CDSs}. Swiss Finance Institute Research Paper 18-40. 2018}
\end{frame}


\begin{frame}{Неблагоприятный отбор}
\justify
Вы ищете, у кого бы купить 1\,000 евро на отпуск и разметили на финтех-платформе
объявление <<куплю 1\,000 евро по 100>>. Через секунду ваша заявка 
исполнена: с вами поторговал алгоритмический хэдж-фонд. Хорошо это или плохо?

\justify
С одной стороны, вы сделали ровно то, что хотели. С другой стороны, высока 
вероятность, что вас облапошили: хэдж-фонд осведомлён лучше вас и продал вам
евро по 100 за секунду до того, как евро упал до 99.

\justify
Происходит \alert{неблагоприятный отбор} (\en{adverse selection}). Выставляя 
заявку, вы рискуете, что с вами поторгует более информированный участник рынка.
\end{frame}



\begin{frame}{Неблагоприятный отбор: пример}
\justify
Вы покупаете некоторый актив. Вы можете разместить заявку <<куплю актив А по цене $x$ или дешевле>>. Как только вы выставите заявку, на рынок выйдет неинформированный продавец, который продаст вам актив по вашей цене. Сразу после сделки вы оба узнаете истинную стоимость актива: либо \$90, либо \$110 с вероятностью 50\%/50\% .

\justify
Предположим, вы выставили заявку с ценой $x = \$100$:

\centering
\begin{tabular}{l|l}
Истинная стоимость & Неинформированный продавец (100\%)  \\ \hline
\$90 (50\%) & Сделка (убыток $-\$10$) \\ \hline
\$110 (50\%) & Сделка (прибыль $+\$10$) 
\end{tabular}

\justify
При какой цене $x$  вы в среднем ни теряете, ни зарабатываете?
\begin{align*}
0.5\cdot(\$110 - x) + 0.5\cdot(\$90 - x) = 0 \quad \Leftrightarrow \quad x = \$100
\end{align*}

\justify
На конкурентном рынке вам придётся предложить как минимум \$100 или хотя бы  
\$99.99, а иначе конкуренты предложат цену лучше. 
\end{frame}



\begin{frame}{Неблагоприятный отбор: пример (продолжение)}
\justify
Добавим асимметрию информации. С вероятностью 10\% вы будете иметь дело с информированным продавцом, который точно знает конечную цену базового актива.

\justify
Предположим, вы выставили заявку с ценой $x = \$100$:

\centering
\begin{tabular}{l|l|l}
Истинная стоимость & Неинформ. (90\%) & Информ. (10\%) \\ \hline
\$90 (50\%) & Сделка ($-\$10$) & Сделка ($-\$10$)\\ \hline
\$110 (50\%) & Сделка ($+\$10$) &\alert{ Нет сделки (\$0)} 
\end{tabular}

\justify
Какая теперь безубыточная цена в заявке $x$?
\begin{align*}
0.5\cdot\alert{0.9}\cdot(\$110 - x) + 0.5(\$90 - x) = 0 \Leftrightarrow x \approx \$99.47
\end{align*}

\justify
Вы должны действовать первым и разместить заявку, не зная, с кем придётся заключить сделку. Вам придётся на всякий случай требовать скидку в 53 цента с каждого продавца. Это цена отрицательнго отбора и асиммерии информации. Bid/ask спред становится шире. Неинформированные продавцы расплачиваются за существование информированных.
\end{frame}



\begin{frame}{Маркет-мейкеры и неблагоприятный отбор}
\justify
Как не платить цену неблагоприятного отбора за других? Можно 
представиться покупателю и объяснить, что вы простой частный инвестор, а не хэдж-фонд.

\justify
Маркет-мейкер знает, кто именно к нему обратился, и может предложить разный спред разным клиентам. Информированные получают спред пошире (потому что их сделки опасны для маркет-мейкера), неинформированные --- поуже.

\justify
Многим клиентам выгоднее обратиться к маркет-мейкеру, назвать себя и 
получить более выгодный спред, чем искать счастья на анонимном рынке, на котором
каждый может торговать с каждым.

\justify
Маркет-мейкеры накапливают данные о всех предыдущих сделках клиентов и
могут точно вычислить требуемый спред для каждого их них. В результате корпоративные клиенты могут получить спред получше, а не платить цену неблагоприятного отбора за хэдж-фонды.
\end{frame}



\begin{frame}{Финансовые деривативы}
\justify
\alert{Финансовый дериватив} (\en{financial derivative}) --- контракт, доход по которому зависит 
от (выводится из) поведения другого финансового актива, который называется базовым 
(\en{underlying}).

\justify
Примеры базовых активов: валютные курсы, процентные ставки, ценные бумаги, пшеница, нефть, погода.

\justify
Примеры деривативов: форварды и фьючерсы, опционы, процентные свопы, кредитные дефолтные свопы.

\justify
Деривативы можно использовать для управления риском (хэджирование) и для того, чтобы заработать деньги на движении рынка (спекуляция). 
\end{frame}



\begin{frame}{Валютный форвард}
\justify
\alert{Валютный форвард} (\en{FX forward, outright forward}) --- сделка, в которой стороны берут 
на себя обязательства произвести обмен валюты в заданную дату в будущем по 
фиксированному курсу и в фиксированном объёме. 

\justify
Пример: покупка форварда EURUSD сроком 1 год по курсу \eurusdBigFiguresPlusTwo\underline{20} в размере 1\,000\,000 EUR:

\centering
\begin{tabular}{l|r|r}
Дата                          & EUR & USD \\ \hline
\eurusdBusinessDate\ (сегодня)  & 0   & 0   \\
\eurusdSpotDate\ (спот) & 0   & 0   \\
\eurusdOneYearDate\ (1Y)   & 1\,000\,000 & $-\eurusdBigFiguresPlusTwoPerMillion 2\,000$
\end{tabular}

\justify
Сам по себе форвард обычно нисколько не стоит (сегодня ничего платить не надо). <<Купить>> форвард --- взять на себя обязательство получить первую валюту и заплатить вторую по фиксированному курсу. Главное договориться об этому курсе обмена.
\end{frame}



\begin{frame}{Форвардный курс и форвардные пункты}
\justify
Важнейший параметр, о котором должны сторговаться участники сделки --- будущий курс обмена, который называется \alert{форвардный курс} (\en{forward rate}). Его ещё можно назвать <<ценой>> форварда.

\justify
Иногда используют не сам форвардный курс, а разность между форвардом и спотом --- \alert{форвардные пункты} (\en{forward points}). Это удобно, потому что спот-курс меняется каждую миллисекунду и тянет за собой форвард, а разность между ними меняется реже и не так сильно.

\vspace{\baselineskip}
\centering
\begin{tabular}{l|l}
Спот-курс & \eurusdBigFigures\underline{10} \\
Форвардный курс & \eurusdBigFiguresPlusTwo\underline{20} \\
\hline
Форвардные пункты & $\eurusdBigFiguresPlusTwo\underline{20} - \eurusdBigFigures\underline{10} = 0.02\underline{10 }= 210$ <<пипов>>
\end{tabular}

\justify
* 1 <<пип>> в паре EURUSD --- 0.0001.
\end{frame}



\begin{frame}{Валютный риск}
\justify
Предположим, что мы --- российский экспортёр рогов и копыт. Мы заключили контракт на поставку рогов в Европу по рогопроводу <<Северный олень 2>>. По контракту мы должны через год поставить 100 кубометров рогов, и тогда же через год нам заплатят 1\,000\,000 евро. Допустим, себестоимость партии рогов 95\,000\,000 рублей. Текущий курс EURRUB 100.0.

\justify
\centering
\begin{tabular}{r|r|r|r|r|r}
EURRUB      & Выручка & \multicolumn{2}{c|}{Спот} & Расходы & Прибыль \\
\cline{3-4}
через год   & EUR     & EUR    & RUB              & RUB     & RUB   \\ \hline
90.0        & $+1$  & $-1$ & $+90$          & $-95$ & $-5$ \\
100.0        & $+1$  & $-1$ & $+100$          & $-95$ & $+5$ \\
110.0        & $+1$  & $-1$ & $+110$          & $-95$ & $+15$
\end{tabular}
(Все суммы указаны в миллионах)

\justify
Это \alert{валютный риск} (\en{FX risk}): наша прибыль может измениться из-за колебаний валютного курса.
\end{frame}



\begin{frame}{Хэджирование валютного риска}
\justify
Чтобы избежать риска, можно заключить форвард. Например, крупный немецкий инвестбанк из 
Франкфурта может купить у нас годовой форвард EURRUB по курсу 105.

\justify
\centering
\begin{tabular}{r|r|r|r|r|r}
EURRUB      & Выручка & \multicolumn{2}{c|}{Форвард} & Расходы & Прибыль \\
\cline{3-4}
через год   & EUR     & EUR    & RUB              & RUB     & RUB   \\ \hline
90.0          & $+1$  & $-1$ & $+105$          & $-95$ & $+10$ \\
100.0        & $+1$  & $-1$ & $+105$          & $-95$ & $+10$ \\
110.0        & $+1$  & $-1$ & $+105$          & $-95$ & $+10$
\end{tabular}

\justify
Форвард --- обязательный к исполнению контракт. Даже если через год евро укрепится до 
110 рублей, мы будем обязаны продать евро по 105. Форвард защищает не только от риска 
потерь (если евро подешевеет), но и от риска неожиданно высокой прибыли (если евро 
укрепится).

\justify
Нужно ли иметь 1\,000\,000 евро сегодня, чтобы продать форвард EURRUB? Нет!
\end{frame}



\begin{frame}{Спекуляция на валютном риске}
\justify
Помечтаем, что мы -- экстрасенсы и точно знаем, что за год евро сильно подешевеет. Как мы можем на этом заработать? Простой путь: продать 1 миллион евро сейчас (<<на споте>>). Но что, если под рукой нет свободного миллиона евро?

\justify
Решение: продадим кому-нибудь годовой форвард EURRUB по 105. Через год нам принесут 105\,000\,000 рублей, а мы купим на них евро по текущему спот-курсу.

\justify
\centering
\begin{tabular}{r|r|r|r|r|r}
EURRUB      & \multicolumn{2}{c|}{Форвард} & \multicolumn{2}{c|}{Спот} & Прибыль \\
\cline{2-5}
через год & EUR     & RUB     & EUR     & RUB      & RUB \\ \hline
90.0      & $-1$ & $+105$  & $+1$  & $-90$  & $+15$ \\
100.0      & $-1$ & $+105$  & $+1$  & $-100$  & $+5$ \\
110.0     & $-1$ & $+105$  & $+1$   & $-110$ & $-5$ \\
\end{tabular}
 
\justify
* Банк, которому мы продадим форвард, скорее всего потребует от нас залог на случай, если наша ставка не сыграет. Но залог будет меньше, чем миллион евро.
\end{frame}



\begin{frame}{Честный форвардный курс}
\justify
От чего зависит <<справедливый>> форвардный курс, при котором продавец и покупатель будут рады заключить контракт? 

\justify
Представьте, что вы --- инвестор из Европы. Сейчас у вас есть 1 миллион евро, но через год вам понадобятся рубли для поездки в Сочи. Как вы можете зафиксировать сумму рублей, которой вы будете владеть через год? 
\end{frame}



\begin{frame}{Честный форвардный курс - 2}
\justify
Текущий спот-курс EURRUB $S_{eurrub}=100$, безрисковый депозит в евро 
приносит $r_{eur}=3\%$, а в рублях --- $r_{rub}=12\%$. 
На рынке форвардов можно заключить контракт по курсу $F_{eurrub}=107$.

\justify
Стратегия 1: купить рубли сегодня

\centering
\begin{tabular}{l|r|r|r|r}
& \multicolumn{2}{c|}{Сегодня} & \multicolumn{2}{c}{Через 1 год} \\ \cline{2-5}
& EUR & RUB & EUR & RUB \\ \hline
Продать спот & $-1$ & $+100$ & & \\
Вложить RUB & & $-100$ & & $+112$ \\ \hline
Итого & $-1$ & $0$ & & $+112$
\end{tabular}

\justify
Стратегия 2: продать форвард

\centering
\begin{tabular}{l|r|r|r|r}
& \multicolumn{2}{c|}{Сегодня} & \multicolumn{2}{c}{Через 1 год} \\ \cline{2-5}
& EUR & RUB & EUR & RUB \\ \hline
Вложить EUR & $-1$ & & $+1.030$ & \\
Продать форвард &   &   & $-1.030$ & $+110.210$ \\ \hline
Итого & $-1$ &   & $0$ & $+110.210$
\end{tabular}
\end{frame}



\begin{frame}{Честный форвардный курс - 3}
\justify
Могут ли две стратегии приводить к разным результатам?

\justify
\centering
\begin{tabular}{l|l}
Стратегия 1 & Стратегия 2 \\ \hline
+112\,000\,000 RUB  & +110\,210\,000 RUB \\
$1\,000\,000 \cdot S_{eurrub} \cdot (1+r_{rub})$ & $1\,000\,000 \cdot (1+r_{eur}) \cdot F_{eurrub}$
\end{tabular}

\justify
Стратегия 1 явно лучше, и все будут пользоваться ей. 

1) Больше желающих продать евро --- курс $S_{eurrub}$ снижается.

2) Выше спрос на депозиты в рублях --- ставка $r_{rub}$ снижается.

3) Никому не нужны депозиты в евро --- ставка $r_{eur}$ растёт. 

4) Никто не продаёт форварды --- цена форварда $F_{eurrub}$ растёт.
\end{frame}



\begin{frame}{Честный форвардный курс - 4}
\justify
Что, если форвардный курс выше, например $F_{eurrub}=110$?


\justify
\centering
\begin{tabular}{l|l}
Стратегия 1 & Стратегия 2 \\ \hline
+112\,000\,000 RUB  & +113\,300\,000 RUB \\
$1\,000\,000 \cdot S_{eurrub} \cdot (1+r_{rub})$ & $1\,000\,000 \cdot (1+r_{eur}) \cdot F_{eurrub}$
\end{tabular}

\justify
Теперь стратегия 2 лучше. 

1) Выше спрос на депозиты в евро --- ставка $r_{eur}$ снижается.

2) Больше продавцов форвардов --- курс $F_{eurrub}$ снижается.

3) Никто не продаёт евро на споте --- курс $S_{eurrub}$ растёт. 

4) Депозиты в рублях не нужны --- ставка $r_{rub}$ растёт.
\end{frame}



\begin{frame}{Честный форвардный курс - 5}
\justify
При каком форвардном курсе $F_{eurrub}$ рынок будет в равновесии, а стратегии 1 и 2 будут приводить к одинаковым результатам?

\begin{align*}
&1\,000\,000 \cdot S_{eurrub} \cdot (1 + r_{rub}) = 1\,000\,000 \cdot (1+r_{eur}) \cdot F_{eurrub} \Rightarrow \\
&F_{eurrub} = S_{eurrub} \frac{1 + r_{rub}}{1 + r_{eur}} = 100 \cdot \frac{1 + 0.12}{1 + 0.03} \approx 108.7379
\end{align*}

\justify
Вывод: честная цена форварда зависит только от текущего спот-курса и соотношения процентных ставок. Не нужно предсказывать будущее, чтобы торговать форвардами!
\end{frame}



\begin{frame}{Арбитраж}
\justify
Где гарантия, что рынок сойдётся к равновесию за конечное время? Вдруг он может пребывать в неравновесном состоянии сколь угодно долго?

\justify
Когда клиент открывает вклад в банке под безрисковую ставку, то банк де-факто берёт у клиента кредит под эту ставку. Что изменится в нашей модели, если допустить, что некоторые участники умеют не только открывать вклады, но и брать кредиты?
\end{frame}



\begin{frame}{Арбитраж - 2}
\justify
Допустим, что цена форварда ниже равновесной.

$S_{eurrub}=100$, $r_{eur}=3\%$, $r_{rub}=12\%$, $F_{eurrub}=107$. 

\justify
\begin{tabular}{l|r|r|r|r}
& \multicolumn{2}{c|}{Сегодня} & \multicolumn{2}{c}{Через 1 год} \\ \cline{2-5}
& EUR & RUB & EUR & RUB \\ \hline
Занять EUR     & $+1$ &                                   & $-1.030$ & \\
Продать спот           & $-1$ & $+100$ &                               & \\
Вложить RUB    &                                & $-100$  &                                & $+112$ \\
Купить форвард &                               &                                 & $+1.047$ & $-112$ \\ \hline
Итого & 0 & 0 & $+0.017$ & 0
\end{tabular}

\justify
17\,000 евро через год при нулевых начальных инвестициях. Слишком хорошо, чтобы быть правдой.
\end{frame}



\begin{frame}{Арбитраж - 3}
\justify
Пусть цена форварда выше равновесной.

$S_{eurrub}=100$, $r_{eur}=3\%$, $r_{rub}=12\%$, $F_{eurrub}=110$. 

\justify
\begin{tabular}{l|r|r|r|r}
& \multicolumn{2}{c|}{Сегодня} & \multicolumn{2}{c}{Через 1 год} \\ \cline{2-5}
& EUR & RUB & EUR & RUB \\ \hline
Занять RUB     &                               & $+100$  &                            & $-112$  \\
Купить спот            & $+1$ & $-100$ &                               &   \\
Вложить EUR    &  $-1$  &                                 & $+1.030$ &  \\
Продать форвард &                               &                                & $-1.030$ & $+113.3$ \\ \hline
Итого & 0 & 0 & 0 & $+1.3$
\end{tabular}

\justify
Снова деньги из воздуха!
\end{frame}



\begin{frame}{Арбитраж - 4}
\justify
Совокупность сделок, в результате которой участник рынка может ничем не рискуя заработать ненулевую прибыль при нулевых начальных вложениях, называется \alert{арбитражем} (\en{arbitrage}). Самого этого участника называют арбитражёром (\en{arbitrageur}).

\justify
Толпы алчных арбитражёров рыскают по рынку и выискивают малейшие возможности для арбитража. Если рыночная цена форварда отклонится от равновесной хотя бы на мгновение, арбитражёры налетят коршунами и помогут невидимой руке рынка исправить ошибку (\en{mispricing}). Каждый хочет сделать деньги из воздуха! 

\justify
Мы будем оценивать форварды и другие деривативы так, чтобы цена дериватива не оставляла возможностей для арбитража.
\end{frame}



\begin{frame}{Репликация форварда}
\justify
Рассмотрим следующие две стратегии

\justify
\small{
\begin{tabular}{l|r|r|r|r}
 & \multicolumn{2}{c|}{Сегодня} & \multicolumn{2}{c}{Через 1 год} \\ \cline{2-5}
Стратегия 1& EUR & RUB & EUR & RUB \\ \hline
Купить форвард по 108.74&                              &                                & $+1.030$ & $-112$
\end{tabular}
}

\justify
\small{
\begin{tabular}{l|r|r|r|r}
& \multicolumn{2}{c|}{Сегодня} & \multicolumn{2}{c}{Через 1 год} \\ \cline{2-5}
Стратегия 2 & EUR & RUB & EUR & RUB \\ \hline
Занять RUB &                             & $+100$ &                    & $-112$ \\
Купить спот        & $+1$ & $-100$  &                     &                                  \\
Вложить EUR & $-1$  &                                  & $+1.030$ & \\  \hline
Итого & 0 & 0 &  $+1.030$ & $-112$
\end{tabular}
}

\justify
Стратегия 2 полностью \alert{реплицирует} стратегию 1. Представьте, что стратегии спрятали в два чёрных ящика. Всё, что вы видите --- что иногда из чёрных ящиков вылетают евро, а в ящики залетают рубли. Не заглядывая внутрь, вы никогда не угадаете, в каком ящике настоящий форвард, а в каком --- синтетическая стратегия из кредита, депозита и спот-сделки.
\end{frame}



\begin{frame}{Оценка деривативов через репликацию}
\justify
Булочка стоит 30 р., котлетка 200 р., салатик 100 р., майонезик 50 р. Сколько должен стоить бургер на идеальном эффективном рынке? 380 р. плюс стоимость сборки, которая на финансовых рынках близка к нулю.

\justify
Цена дериватива выводится (derived) из цены базовых инструментов. Оценка дериватива
не абсолютная (<<какова фундаментально обоснованная цена бургера?>>), а относительная (<<если стоимость ингредиентов X, то сколько стоит их комбинация?>>).

\justify
Если мы продали клиенту бургер за 381 р., а сами собрали его из ингредиентов за 380 р., то мы зафиксируем прибыль 1 р., даже если рынок сошёл с ума и фундаментально обоснованная цена бургера 1000 р. или 100 р.

\justify
Ещё лучше --- купить бургер у одного клиента за 379 р., и через минуту продать
другом клиенту за 381 р.
\end{frame}



\begin{frame}{Безарбитражная цена форварда}
\justify
На рынке не будет возможностей для арбитража, если форвардный курс будет следовать за спот-курсом и процентными ставками:

\begin{align*}
F_{xxxyyy} &= S_{xxxyyy} \frac{1 + r_{yyy}T_{yyy}}{1 + r_{xxx}T_{xxx}} \\
%F_{xxxyyy} &= S_{xxxyyy} e^{r_{yyy}^*T_{yyy} - r_{xxx}^*T_{xxx}} \\
%F_{xxxyyy} &= S_{xxxyyy} \frac{\delta_{xxx}(T)}{\delta_{yyy}(T)}%
\end{align*}

Здесь $r_{xxx}$ --- процентная ставка (проценты годовых), $T_{xxx}$ --- количество лет между сегодня и датой поставки по форварду.

\justify
Предсказывает ли форвардный курс будущий спот-курс? Нет!
\end{frame}



\begin{frame}{Биржевые фьючерсы}
\justify
\alert{Валютный фьючерс} (\en{FX future}) --- биржевой контракт, по которому можно обменять одну валюту на другую по фиксированному курсу в фиксированную дату в будущем.

\justify
Отличия фьючерса от форварда:
\begin{itemize}
\justifying
\item Фьючерс торгуется на бирже, форвард --- на внебиржевом рынке.
\item Фьючерс --- стандартизованный контракт (фиксированный размер лота, стандартные даты поставки --- третья среда каждого третьего месяца). В форвардах стороны могут договориться о чём угодно.
\item Чтобы торговать фьючерсами, нужно обязательно внести гарантийное обеспечение. В форвардном контракте --- как решат стороны.
\end{itemize}

\justify
Справедливый фьючерсный курс вычисляется так же, как и форвардный.
\end{frame}



\begin{frame}{Фьючерсы и предсказание будущего курса}
\center
\begin{tikzpicture}
\begin{axis}[
  width=\textwidth,
  height=\textheight - 1cm,
  date coordinates in=x,
  date ZERO=2014-01-01,
  xtick={2014-02-01,2014-04-01, 2014-06-01, 2014-08-01, 2014-10-01, 2014-12-01},
  xticklabel={\day.\month.14},
  xmin=2014-01-01,
  xmax=2014-12-31,
  ymin=30,
  ymax=62,
  grid=major,
  ylabel={\small{Курс USDRUB}},
  xlabel near ticks,
  ylabel near ticks,
  legend entries = {
      Спот-курс ЦБ РФ,
      Цена фьючерса Si-12.14
  },
  legend pos=north west,
  %legend style={font=\tiny},
  legend cell align={left}
]
\addplot[color=Set1-A, mark=none, thick] table[x=date, y=cbr_spot_rate, col sep=comma]{Si-12.14.csv};
\addplot[color=Set1-B, mark=none, thick] table[x=date, y=futures_price, col sep=comma]{Si-12.14.csv};
\end{axis}
\end{tikzpicture}

\scriptsize Данные: Московская Биржа.
\end{frame}



\begin{frame}{Беспоставочные (расчётные) форварды}
\justify
Многие валюты развивающихся стран не являются свободно-конвертируемыми. Кроме того, запрещены деривативы, такие как форварды.

\justify
Примеры:

1. Бразильский реал (BRL)

2. Китайский юань (CNY)

3. Индийская рупия (INR)

4. Корейский вон (KRW)

5. Тайваньский доллар (TWD)
\end{frame}



\begin{frame}{Беспоставочные (расчётные) форварды - 2}
\justify
Невозможная троица международных финансов (\en{international finance trilemma}): можно выбрать любые два пункта из трёх.

1. Фиксированный курс национальной валюты.

2. Свободное движение капитала через границу.

3. Независимая кредитно-денежная политика.

\justify
Например, мы хотим фиксированный курс 24 рублей за доллар, как будто на дворе 2007 год.

1. ФРС повышает ставки --- инвесторы продают рубли и покупают доллары, чтобы вложить их под более высокую ставку.

2. Чтобы удовлетворить спрос и удержать курс, ЦБ РФ продаёт доллары из резервов.

3. Резервы большие, но конечные. Всё равно придётся либо отпустить курс, либо повысить ставки вслед за ФРС.

\end{frame}



\begin{frame}{Пример: Китай}
\justify
1. Движение капитала (покупка долга или акций) ограничено и требует предварительного одобрения правительства Китая.

\justify
2. Текущие операции (оплата товаров и услуг) не ограничиваются. 

\justify
3. Предоставив подтверждающие документы, можно купить или продать юани на бирже CFETS (China Foreign Exchange Trading System)

\justify
4. Народный банк Китая выступает контрагентом в 70\% сделок на CFETS и имеет неограниченные возможности для манипулирования курсом.

\justify
5. Деривативы запрещены!
\end{frame}



\begin{frame}{Беспоставочный форвард}
\justify
Как управлять валютным риском, если форварды запрещены? Нужен \alert{беспоставочный форвард} (\en{non-deliverable forward, NDF}), он же расчётный форвард.

\justify
Беспоставочный форвард между Citi и Deutsche:

\justify
\centering
\begin{tabular}{l|l}
	Тип контракта 		   & Non-deliverable forward		\\
	Deutsche <<продаёт>>  & 1\,000\,000 долларов (USD)	\\
	Deutsche <<покупает>> & 7\,400\,000 юаней (CNY)		\\
	Курс		 		      & 7.40 						\\
	Дата поставки		   & 8 апреля 2024 г. (пн) \\
	Референсный курс	   & Курс Народного банка Китая	\\
	Дата фиксинга		   & 4 апреля 2024 г. (чт) \\
	Сегодня (справочно)	& 4 марта 2024 г. (пн) \\
	Спот-дата (справочно) & 6 марта 2024 г. (ср)
\end{tabular}
\end{frame}



\begin{frame}{Механика беспоставочного форварда}
\justify
При заключении контракта (4 марта) никто никому ничего не платит.

\justify
В дату фиксинга (4 апреля) Народный банк Китая публикует официальный курс, например 7.20.

\justify
Deutsche угадал, что курс снизится, поэтому:

1. Deutsche <<отдаёт>> 1\,000\,000 USD.

2. Deutsche <<получает>> $7\,400\,000 \text{CNY} = \dfrac{7\,400\,000}{7.20} = 1\,027\,778 \text{USD}$.

\justify
В дату поставки (8 апреля) Citi переведёт Deutsche выигрыш: 27\,778 USD.

\justify Банки не обязаны сообщать о сделке правительству Китая. Беспоставочный форвард --- мечта спекулянта.
\end{frame}



\begin{frame}{Хэджирование валютного риска}
\justify
Одна американская фруктовая компания должна заплатить поставщику 7\,400\,000 юаней через месяц.

1. Продаём форвард 1\,000\,000 USDCNY по 7.40.

2. В дату фиксинга покупаем 7\,400\,000 CNY по спот-курсу $S$.

3. В дату поставки получаем и выплату по форварду, и юани по спот-сделке.

\justify
\centering
\small
\begin{tabular}{l|r|r|r|r}
& \multicolumn{2}{c|}{В случае $S=7.20$} & \multicolumn{2}{c}{В случае $S=7.60$} \\ \cline{2-5}
Транзакция & USD & CNY & USD & CNY \\ \hline
Форвард & $+27\,778$ &  &  $-26\,316$& \\
Спот & $-1\,027\,778$ & $+7\,400\,000$ & $-973\,694$ & $+7\,400\,000$ \\
Оплата & & $-7\,400\,000$ & & $-7\,400\,000$ \\ \hline
Итого & $-1\,000\,000$ & 0 & $-1\,000\,000$ & 0
\end{tabular}\end{frame}



\begin{frame}{Ценообразование беспоставочного форварда}
\justify
Аргументы про рыночное равновесие и отсутствие арбитража не работают, если есть ограничения на движение капитала. Цена беспоставочного форварда может далеко отклониться от <<теоретической>>, и никто не сможет её заарбитражить.

\justify
Беспоставочный форвард на 1 месяц --- это скорее базовый актив, а не дериватив. Невидимая рука рынка ищет такую цену, при которой спрос (желающие заплатить поставщикам в Китае или поспекулировать на укреплении доллара) и предложение (желающие вывести прибыль из Китая или поспекулировать на укреплении юаня) уравновешиваются.
\end{frame}

\end{document}