\documentclass{beamer}

\usepackage{cmap}				% To be able to copy-paste russian text from pdf
\usepackage[T2A]{fontenc}
\usepackage[utf8]{inputenc}
\usepackage[english]{babel}
\usepackage{textpos}
\usepackage{ragged2e}
\usepackage{amssymb}
\usepackage{ulem}
\usepackage{tikz}
\usepackage{pgfplots}
\usepackage{color}
\usepackage{cancel}
\usepackage{multirow}
\pgfplotsset{compat=1.17}
\usetikzlibrary{arrows,snakes,backgrounds,shapes}
\usepgfplotslibrary{groupplots,colorbrewer,dateplot,statistics}
\usepackage{animate}

\usepackage{amsfonts}
\usepackage{amsmath}
\usepackage{amssymb}
\usepackage{graphicx}
\usepackage{setspace}

\usepackage{enumitem}
\setitemize{label=\usebeamerfont*{itemize item}%
  \usebeamercolor[fg]{itemize item}
  \usebeamertemplate{itemize item}}

% remove navigation bar
\setbeamertemplate{navigation symbols}{} 

\usepackage{eurosym}
\renewcommand{\EUR}[1]{\textup{\euro}#1}

\title{Present Value, Risk and Greeks}
\author{Artem Bakulin}
\date{November 9, 2023}

\usetheme{Warsaw}
\usecolortheme{beaver}

\newcommand{\ru}[1]{\begin{otherlanguage}{russian}#1\end{otherlanguage}}
\newcommand{\en}[1]{\begin{otherlanguage}{english}#1\end{otherlanguage}}
\newcommand{\ruen}[2]{#1 (\en{#2})}

\setbeamertemplate{page number in head/foot}[totalframenumber] 

\begin{document}



\begin{frame}
\titlepage
\end{frame}



\begin{frame}{Present value}
\justify
\alert{Present value (PV)} of a payment of $N$ dollars in $T$ years is the amount which market participants would be willing to pay to be entitled for this payment.

\justify
\alert{Present value} of a derivative or a portfolio of derivatives is the amount which market participants would hypothetically be willing to pay for this trade or the portfolio. How much would they pay us if we were to erase our name from all contracts and replace it with the buyer's name?

\end{frame}



\begin{frame}{Derivatives and IFRS}
\justify
Recent quarterly report by a major Frankfurt-based investment bank:

\justify
Assets (billions):

\en{Positive market values from derivative financial instruments}: \EUR{287.6}

\justify
Liabilities (billions):

\en{Negative market values from derivative financial instruments}: \EUR{271.3}

\justify
According to IFRS, the change in the present value (PV) of the derivative portfolio should be accounted for in the profits (or losses) of the current year, even if the payments on derivatives are scheduled to occur 100 years from now.
\end{frame}



\begin{frame}{Present value of a spot trade}
\justify
We have bought $N=1\,000\,000$ of EURUSD spot at the rate of $S_0 = 1.05$. Current market rate is $S=1.06$. What is our PV?

\justify
\centering
\begin{tabular}{c|c|c}
Date & EUR  & USD \\ \hline
T+2 & $+N$ & $-NS_0$
\end{tabular}

\justify
$N$ euro are worth $N$ euro (we ignore discounting from T+2 to T+0).

\justify
$NS_0$ dollats are worth $NS_0/S$ euro.

\begin{align*}
\text{PV} = N - N\frac{S_0}{S} = N\frac{S - S_0}{S} = \EUR{1\,000\,000} \cdot \frac{1.06 - 1.05}{1.06} = \EUR{9\,434}
\end{align*}

\justify
Calculating the present value (PV) from market prices (in this case, the spot rate) is called \alert{marking-to-market (MtM)}.
\end{frame}



\begin{frame}{Present value of a spot trade - 2}
\justify
We have bought $N=1\,000\,000$ of EURUSD spot at the rate of $S_0 = 1.05$. Current market rate is $S=1.06$. What is our PV?

\justify
\centering
\begin{tabular}{r|r|r}
         & EUR               & USD \\ \hline
Trade   & $+1\,000\,000$ & $-1\,050\,000$ \\
Close-out &    $-990\,566$ & $+1\,050\,000$ \\ \hline
Total    &      $+9\,434$ & $0$
\end{tabular}

\justify
If we decided to close the position in the market, we could nullify all payments in dollars and be left with a guaranteed fixed payment of \EUR{9\,434}. This is the present value (PV) of the transaction.
\end{frame}



\begin{frame}{Present value and market spread}
\justify
On the real market, there is a spread between buying and selling. Let's say the euro can be sold for 1.04 dollars and bought for 1.06. Two companies have agreed with each other on a transaction for 1 million euros at the exchange rate of 1.05.

\justify
The buyer:

\centering
\begin{tabular}{r|r|r}
                & EUR                   & USD \\ \hline
Buying at 1.05 & $+1\,000\,000$ & $-1\,050\,000$ \\
Closing at 1.04 & $-1\,009\,615$ & $+1\,050\,000$ \\ \hline
Total PV      &      $-9\,615$ & $0$
\end{tabular}

\justify
The seller:

\centering
\begin{tabular}{r|r|r}
                & EUR                   & USD \\ \hline
Selling at 1.05 & $-1\,000\,000$ & $+1\,050\,000$ \\
Closing at 1.06 & $+990\,566$ & $-1\,050\,000$ \\ \hline
Total PV      &      $-9\,434$ & $0$
\end{tabular} 
\end{frame}



\begin{frame}{Present value and market spread - 2}
\justify
Can all market participants be required to take the bid/ask spread into account when calculating PV and profit or loss? No, because in that case, a newly executed deal would be unprofitable for both parties.

\justify
Usually, PV is calculated either from the "fair" mid price between the bid and the ask or from the price of the last deal on the market.

\justify
Implication: if the transaction occurred at the mid price between the bid and the ask, then its PV is zero.
\end{frame}



\begin{frame}{Present value}
\justify
We sold a call option to the client for a premium of 100 euros. Similar options are priced at 90 euros on the market. We know this because it is either a liquid option, or the calibrated model interpolated this price from market prices of other options

\justify
PV of our position is $+10$ euro.
\begin{itemize}
\item Sold option: $-90$ euro.
\item Received premium: $+100$ euro.
\end{itemize}

\justify
In an ideal liquid market, other participants would be willing to pay 10 euros for our portfolio. If the price is lower, say 9.90, someone might offer 9.95, hoping to earn 0.05 euro.

\justify
Another interpretation: we can enter the market and buy the option for 90 euros. In that case, we would have 10 euros remaining, and the option position would disappear.
\end{frame}



\begin{frame}{Risk}
\justify
\alert{Risk} is the possibility that something will not go as planned, and the profit will turn out to be different from what we anticipated.

\justify
You cannot earn anything more than the risk-free interest rate if you don't take any risks. Financial companies deliberately take numerous risks:

\begin{itemize}
\item Market
\item Credit
\item Liquidity
\item Regulatory
\item Technical
\item Legal
\item Political
\end{itemize}
\end{frame}



\begin{frame}{Market risk}
\justify
\alert{Market risk} is the possibility that our present value (PV) will change (upward or downward) due to changes in market prices.

\justify
How could we  measure market risk? The standard approach is to calculate sensitivities of present value (PV) to changes in market variables. By how much will the PV of the portfolio change if the price of Brent futures increases by \$1?

\justify
There are standard sensitivities or the Greeks:
\begin{itemize}
\item Delta ($\Delta$) is sensitivity to the underlying asset price.
\item Vega ($\mathcal{V}$) is sensitivity to implied volatility.
\item Rho ($\rho$) is sensitivity to interest rates.
\item Theta ($\Theta$) is sensitivity to the passage of time.
\item Gamma ($\Gamma$) is sensitivity of Delta to the underlying asset price.
\end{itemize}
\end{frame}



\begin{frame}{Delta}
\justify
Delta is sensitivity of the portfolio present value $V$ to changes in the price of the underlying asset (for example, the currency spot rate) S$.$
\begin{align*}
\Delta = \frac{\partial V}{\partial S}
\end{align*}

\justify
With a small change in the underlying asset price $dS$, present value of a portfolio changes by
\begin{align*}
dV \approx \Delta\cdot dS
\end{align*}

\justify
This follows from the definition of a derivative:
\begin{align*}
F'(x_0) = \lim_{dx \to 0} \frac{F(x_0+dx) - F(x_0)}{dx} = \lim_{dx \to 0} \frac{dF}{dx}
\end{align*}
Consequently, if $dx$ is small then
\begin{align*}
F'(x_0) \approx \frac{dF}{dx} \quad \Leftrightarrow \quad dF \approx F'(x_0)dx
\end{align*}
\end{frame}



\begin{frame}{Delta of a spot trade}
\justify
Consider a spot trade: buying $N=1\,000\,000$ euro at the fair mid rate of $S_0=1.05$.
\begin{align*}
V = N - N\frac{S_0}{S} \quad \Rightarrow \quad \Delta(S) = \frac{\partial V}{\partial S} = N\frac{S_0}{S^2}
\end{align*}

\justify
Suppose that spot rate has not moved after the trade yet, $S_0=S$:
\begin{align*}
\Delta(S_0) = N\frac{S_0}{S_0^2} = \frac{N}{S_0}
\end{align*}

\justify
With every 1 pip ($dS=0.0001$) change in the spot rate, PV of the trade changes by 
\begin{align*}
dV \approx \Delta \cdot dS = \frac{N}{S_0}dS = \frac{\EUR{1\,000\,000}}{1.05}\cdot 0.0001 = \EUR{95.24}
\end{align*}
\end{frame}



\begin{frame}{Delta of a vanilla call}
\justify
In the Black-Scholes model, Delta of a vanilla call option at strike $K$ expiring in $T$ years is equal to:
\begin{align*}
\Delta = \frac{\partial C(S, K, T, \sigma, r, q)}{\partial S} = e^{-qT}N(d_1)
\end{align*}

\justify
If we have a complex option pricing model for which an analytical formula cannot be derived, Delta can be estimated numerically as finite difference:
\begin{align*}
\Delta \approx \frac{C(S+\delta,...) - C(S,...)}{\delta} \approx \frac{C(S+\delta,...) - C(S-\delta,...)}{2\delta}
\end{align*}

\justify
Delta of an option is the number of units of the underlying asset that needs to be bought to replicate the option for the next small time interval.
\end{frame}



\begin{frame}{Delta of a vanilla call - 2}
\justify
Example: a call option at strike $K=100$, expiration in $T=0.25$ years. $r=5\%$, $\sigma=10\%$.

\centering
\begin{tikzpicture}
\begin{axis}[
			domain=92:108,
			xtick distance=2,
			ytick distance=1,
			xmin=94, xmax=106,
			ymin=0, ymax=6,
			grid = major,
			xlabel={Underling asset price today ($S_0$)},
			ylabel={Option price}
]
	
	\addplot[color = Set1-A, mark = none, thick]
	table[
		x=S,
		y=C_10,
		col sep=comma
	]
	{call_price.csv};
	
	%\addplot[Set1-D, very thick, dashed] {(\x >= 72)*(\x - 72) + 0.05};
	
	\node[inner sep=2pt, circle, fill, color=Set1-A] at (axis cs: 100, 2.6648) {};
	
	\addplot[color=Set1-A, dashed, thick] {0.6083 * \x - 58.1936};
	
	\node[anchor=south west] at (axis cs: 96.5, 0) {$\Delta = 0.61$};
\end{axis}
\end{tikzpicture}
\end{frame}



\begin{frame}{Delta of a vanilla put}
\justify
Call-put parity in case the underlying asset does not pay dividends:
\begin{align*}
C - P = S - Ke^{-rT}
\end{align*}

\justify
Differentiate both parts of the equation:
\begin{align*}
\frac{\partial C}{\partial S} - \frac{\partial P}{\partial S} = \frac{\partial S}{\partial S} \quad \Leftrightarrow \quad \frac{\partial P}{\partial S} = \frac{\partial C}{\partial S} - 1
\end{align*}

\justify
This property is independent of assumptions about the stochastic process for the underlying asset. I.e. holds true not only in the Black-Scholes world.

\justify
In case the underlying asset offers dividend yield  $q$:
\begin{align*}
\frac{\partial P}{\partial S} = \frac{\partial C}{\partial S} - e^{-qT}
\end{align*}
\end{frame}


\begin{frame}{Delta of vanilla options}
\centering
\begin{tikzpicture}
\begin{axis}[
			xtick={0.5},
			xticklabel={$K$},
			ytick={-1,-0.75,...,1},
			xmin=0, xmax=1,
			ymin=-1, ymax=1,
			grid = major,
			xlabel={Underlying asset price ($S$)},
			ylabel={Option delta},
			legend entries = {
  	   Call,
  	   Put
  },
  legend cell align={left},
  legend style={at={(0.03,0.97)},anchor=north west}
]
	
	\addplot[color = Set1-A, mark = none, thick]
	table[
		x=S,
		y=call_delta,
		col sep=comma
	]
	{black_scholes_delta.csv};
	
	\addplot[color = Set1-B, mark = none, thick]
	table[
		x=S,
		y=put_delta,
		col sep=comma
	]
	{black_scholes_delta.csv};

   \draw[thick, color=black] (axis cs: 0, 0) -- (axis cs: 1, 0);
\end{axis}
\end{tikzpicture}
\end{frame}



\begin{frame}{Delta and the replicating portfolio}
\justify
How much of the asset and debt are needed to replicate the option?

\justify
\centering
\begin{tikzpicture}
\begin{axis}[
			domain=92:108,
			xtick=\empty,
			ytick={-2.23, 0},
			yticklabels={$L$, 0},
			xmin=92, xmax=108,
			ymin=-3, ymax=6,
			xlabel={Underling asset price ($S_0$)},
			ylabel={Option price}
]
	
	\addplot[color = Set1-A, mark = none, thick]
	table[
		x=S,
		y=C_10,
		col sep=comma
	]
	{call_price.csv};
	
	%\addplot[Set1-D, very thick, dashed] {(\x >= 72)*(\x - 72) + 0.05};
	
	\node[inner sep=2pt, circle, fill, color=Set1-A] at (axis cs: 100, 2.6648) {};
	
	\addplot[color=Set1-A, dashed, thick] {0.6083 * \x - 58.1936};
	
   \draw[thick, color=black] (axis cs: 0, 0) -- (axis cs: 200, 0);
	
	\node[anchor=south west] at (axis cs: 96.5, 0) {$\Delta$};
\end{axis}
\end{tikzpicture}
\end{frame}



\begin{frame}{Option quoting conventions}
\justify
Despite the fact that the Black-Scholes model doesn't work (doesn't describe reality), everyone continues to use it to negotiate option prices. It is more convenient to negotiate volatility than premiums to understand what is expensive and what is cheap.

\justify
Instead of such a dialogue:

--- What is price of a 3 months call at strike \$100?

--- \$5.

\justify
It turns out like this:

--- What is price of a 3 months call at strike \$100?

--- 20\%.

\justify
It is assumed that everyone can substitute volatility into the Black-Scholes formula and calculate the premium.
\end{frame}



\begin{frame}{Delta and strikes}
\justify
What happens to the volatility smile when the price of the underlying asset increases? Usually, it shifts in the same direction.

\centering
\begin{tikzpicture}
\begin{axis}[
			width = \textwidth,
			height = \textheight - 2cm,
			domain=90:114,
			xtick={90,92,...,114},
			ytick={0.05,0.1,...,0.5},
			xmin=90, xmax=114,
			ymin=0, ymax=0.50,
			yticklabel={\pgfmathparse{\tick*100}\pgfmathprintnumber[precision=0]{\pgfmathresult}\%},
			grid = major,
			xlabel={Strike ($K$)},
			ylabel={Implied volatility ($\sigma$)},
			legend entries = {
  	   $S=98$,
  	   $S=104$
  },
  legend cell align={left},
  legend style={at={(0.97,0.03)},anchor=south east}
]

	\addplot[color = Set1-B, mark = none, thick, dashed] {0.002*(\x - 98)^2 + 0.2};
	
	\addplot[color = Set1-B, mark = none, thick] {0.002*(\x - 104)^2 + 0.2};
	
	\draw[->, >= triangle 45, thick] (98, 0.175) -- (104, 0.175);
\end{axis}
\end{tikzpicture}
\end{frame}



\begin{frame}{Delta and strikes - 2}
\justify
Since the volatility smile may follow spot movements, it is necessary to negotiate deals quickly. Currently, a strike of \$100 corresponds to volatility of 20\%. In a minute it may have volatility of 22\% due to change in the spot price.

\justify
In the over-the-counter market, it is customary to operate with the following strikes, which are tied to deltas:
\begin{itemize}
\item DN (delta neutral) is a strike $K_{DN}$ at which the straddle of a call and a put has a Delta of 0.
\item 25C is a strike $K_{25C}$ at which the call has a delta of 0.25.
\item 25P is a strike $K_{25P}$ at which the put has a delta of $-0.25$.
\item 10C, 10P are similar to 25C and 25P.
\end{itemize}

\justify
Empirically, the implied volatility of such strikes may be more stable (sticky delta effect).
\end{frame}



\newcommand{\drawVolNode}[4] {

	\node[
		circle,
		color=Set1-B,
		fill,
		inner sep=2pt
	] at (#1, #2) {};
	
	\node[anchor=#4] at (#1, #2) {$\sigma_{#3}$};
}

\begin{frame}{Delta and strikes - 3}
\centering
\begin{tikzpicture}
\begin{axis}[
			width = \textwidth,
			height = \textheight - 2cm,
			xtick={0.10, 0.25, 0.5, 0.75, 0.90},
			xticklabels={10P, 25P, DN, 25C, 10C},
			ytick={0,0.05,0.1,...,0.4},
			xmin=0, xmax=1,
			ymin=0, ymax=0.4,
			yticklabel={\pgfmathparse{\tick*100}\pgfmathprintnumber[precision=0]{\pgfmathresult}\%},
			grid = major,
			xlabel={Strike ($K$)},
			ylabel={Implied volatility ($\sigma$)}
]

	\addplot[color = Set1-B, mark = none, thick, samples at={0,0.01,...,1}] {0.75*(\x - 0.5)^2 + 0.2};
	
	\drawVolNode{0.10}{0.320}{10P}{south west}
	\drawVolNode{0.25}{0.247}{25P}{south west}
	\drawVolNode{0.50}{0.200}{DN }{south}
	\drawVolNode{0.75}{0.247}{25C}{south east}
	\drawVolNode{0.90}{0.320}{10C}{south east}
\end{axis}
\end{tikzpicture}
\end{frame}



\begin{frame}{Parametrization of the volatility smile}
\justify
Volatilities of options at strikes 10P, 25P, DN, 25C, and 10C describe the shape of the volatility smile well. However, there is an inconvenience. When the market volatility increases, all five volatilities increase more or less uniformly ("a rising tide lifts all boats").

\justify
\alert{Risk-reversal ("risky")} RR25 consists of a bought 25C call and a sold 25P put.
\begin{align*}
\sigma_{RR25} = \sigma_{25C} - \sigma_{25P}
\end{align*}

\justify
\alert{Butterfly ("fly")} FLY25 is a bought 25C call and 25P put, sold $DN$ call and put.
\begin{align*}
\sigma_{FLY25} = 0.5\cdot(\sigma_{25P} - 2\sigma_{DN} + \sigma_{25C})
\end{align*}

\justify
RR and FLY formulas resemble the first and the second derivative.
\end{frame}



\begin{frame}{Parametrization of the volatility smile - 2}
\justify
Volatility of the DN straddle sets the overall level of the smile. Higher $\sigma_{DN}$, the higher the dispersion of the distribution of the underlying asset price.

\centering
\begin{tikzpicture}
\begin{axis}[
			width = \textwidth,
			height = \textheight - 2cm,
			xtick={0.10, 0.25, 0.5, 0.75, 0.90},
			xticklabels={10P, 25P, DN, 25C, 10C},
			ytick={0,0.02,0.04,...,0.24},
			xmin=0, xmax=1,
			ymin=0, ymax=0.24,
			yticklabel={\pgfmathparse{\tick*100}\pgfmathprintnumber[precision=0]{\pgfmathresult}\%},
			grid = major,
			xlabel={Strike ($K$)},
			ylabel={Implied volatility ($\sigma$)},
			legend entries = {
  	   $\sigma_{DN}=6\%$,
  	   $\sigma_{DN}=10\%$
  },
  legend cell align={left},
  legend style={at={(0.97,0.03)},anchor=south east}
]

	\addplot[color = Set1-B, mark = none, thick, dashed, samples at={0,0.01,...,1}] {0.5*(\x - 0.5)^2 + 0.06};
	

	\addplot[color = Set1-B, mark = none, thick, samples at={0,0.01,...,1}] {0.5*(\x - 0.5)^2 + 0.1};
	
	\draw[<->, >= triangle 45, thick] (0.5, 0.0) -- (0.5, 0.1);
	
	\drawVolNode{0.5}{0.1}{DN}{south}
\end{axis}
\end{tikzpicture}
\end{frame}



\begin{frame}{Parametrization of the volatility smile - 3}
\justify
Risk reversal sets the slope of the smile or the skewness of the underlying asset's distribution to the left or right.

\centering
\begin{tikzpicture}
\begin{axis}[
			width = \textwidth,
			height = \textheight - 2cm,
			xtick={0.10, 0.25, 0.5, 0.75, 0.90},
			xticklabels={10P, 25P, DN, 25C, 10C},
			ytick={0,0.02,0.04,...,0.24},
			xmin=0, xmax=1,
			ymin=0, ymax=0.24,
			yticklabel={\pgfmathparse{\tick*100}\pgfmathprintnumber[precision=0]{\pgfmathresult}\%},
			grid = major,
			xlabel={Strike ($K$)},
			ylabel={Implied volatility ($\sigma$)},
			legend entries = {
  	   $\sigma_{RR25}=0\%$,
  	   $\sigma_{RR25}=4\%$
  },
  legend cell align={left},
  legend style={at={(0.97,0.03)},anchor=south east}
]

	\addplot[color = Set1-B, mark = none, thick, dashed, samples at={0,0.01,...,1}] {0.5*(\x - 0.5)^2 + 0.1};
	

	\addplot[color = Set1-B, mark = none, thick, samples at={0,0.01,...,0.5}] {0.32*(\x - 0.5)^2 + 0.1};
	
	\addplot[color = Set1-B, mark = none, thick, samples at={0.5,0.51,...,1}] {0.96*(\x - 0.5)^2 + 0.1};
	
	
	\drawVolNode{0.25}{0.12}{25P}{north east}
	
	\drawVolNode{0.75}{0.16}{25C}{south east}
	
	\draw (0.25, 0.12) -- (0.75, 0.16);
	
	\draw (0.25, 0.12) -- (0.9, 0.12);
	\draw (0.75, 0.16) -- (0.9, 0.16);
	
	\draw[<->, >= triangle 45, thick] (0.87, 0.16) -- (0.87, 0.12) node[pos=0.5, anchor=west] {$\sigma_{RR25}$};
\end{axis}
\end{tikzpicture}
\end{frame}



\begin{frame}{Parametrization of the volatility smile - 4}
\justify
The butterfly sets the concavity of the smile or the fatness of the tails of the underlying asset's distribution.

\centering
\begin{tikzpicture}
\begin{axis}[
			width = \textwidth,
			height = \textheight - 2cm,
			xtick={0.10, 0.25, 0.5, 0.75, 0.90},
			xticklabels={10P, 25P, DN, 25C, 10C},
			ytick={0,0.02,0.04,...,0.24},
			xmin=0, xmax=1,
			ymin=0, ymax=0.24,
			yticklabel={\pgfmathparse{\tick*100}\pgfmathprintnumber[precision=0]{\pgfmathresult}\%},
			grid = major,
			xlabel={Strike ($K$)},
			ylabel={Implied volatility ($\sigma$)},
			legend entries = {
  	   $\sigma_{FLY25}=1\%$,
  	   $\sigma_{FLY25}=4\%$
  },
  legend cell align={left},
  legend style={at={(0.97,0.03)},anchor=south east}
]

	\addplot[color = Set1-B, mark = none, thick, dashed, samples at={0,0.01,...,1}] {0.16*(\x - 0.5)^2 + 0.1};

	\addplot[color = Set1-B, mark = none, thick, samples at={0.0,0.01,...,1}] {0.64*(\x - 0.5)^2 + 0.1};
	
	
	\drawVolNode{0.25}{0.14}{25P}{south west}
	
	\drawVolNode{0.75}{0.14}{25C}{south east}
	
	\drawVolNode{0.50}{0.10}{DN}{north}
	
	\draw (0.25, 0.14) -- (0.75, 0.14);

	\draw[<->, >= triangle 45, thick] (0.5, 0.10) -- (0.5, 0.14) node[pos=0.5, anchor=west] {$\sigma_{FLY25}$};
\end{axis}
\end{tikzpicture}
\end{frame}



\begin{frame}{Volatility surface}
\justify
Market volatility surface consists of DN, RR and FLY quotes for different terms.

\justify
\centering
\begin{tabular}{l|r|r|r|r|r}
Term & DN     & RR25   & RR10   & FLY25  & FLY10  \\ \hline
1M   & 6.00\% & 0.50\% & 0.90\% & 0.25\% & 0.70\% \\
2M   & 6.40\% & 0.60\% & 1.10\% & 0.30\% & 0.70\% \\
3M   & 6.50\% & 0.65\% & 1.15\% & 0.30\% & 0.75\% 
\end{tabular}

\justify
This is the observed reality. Next, we need to calibrate a model, such as SABR. We tune the internal parameters so that the model produces the same prices for DN, RR, and FLY.

\justify
If we succeed in this, then we can hope that the model will provide reasonable prices not only for liquid vanilla options but also for more complex and less liquid products, whose prices are not observable.
\end{frame}



\begin{frame}{Vega}
\justify
Suppose that a magic function $f$ picks the SABR parameters to reprice the market volatility surface:
\begin{align*}
(\alpha, \beta, \rho) = f(\sigma_{1M,DN}, \sigma_{1M,RR25}, ..., \sigma_{3M,FLY10})
\end{align*}

\justify
We also have a numerical procedure to compute PV ($V$) of our portfolio using the SABR model. Lets call it a function $g$.
\begin{align*}
V = g(\alpha, \beta, \rho) = g\Big(f(\sigma_{1M,DN}, \sigma_{1M,RR25}, ..., \sigma_{3M,FLY10})\Big)
\end{align*}

\justify
What will happen to the portfolio if market quotes change? To answer this question we compute the derivative of the portfolio's present value (PV) with respect to volatility --- the \alert{Vega}.
\begin{align*}
\mathcal{V}_{1M,RR25} &= \frac{\partial V}{\partial \sigma_{1M,RR25}} \approx \\
&\approx \frac{g\Big(f(...,\sigma_{1M,RR25}+\delta,...)\Big) - g\Big(f(...,\sigma_{1M,RR25},...)\Big)}{\delta}
\end{align*}
\end{frame}



\begin{frame}{Vega of a vanilla option}
\justify
Volatility $\sigma$ is constant in the Black-Scholes model, so it is odd to calculate the derivative. Nevertheless, both for calls and for puts Vega is equal to
\begin{align*}
\mathcal{V} &= \frac{\partial C}{\partial \sigma} = \frac{\partial P}{\partial \sigma} = Se^{-qT}\sqrt{T}N'(d_1) \\
N'(x) &= \frac{1}{\sqrt{2\pi}}e^{-\frac{x^2}{2}}
\end{align*}

\justify
Usually Vega is measured in dollars or euros per percentage point change in volatility.

\justify
Vega is positive both for puts and for calls. "Options like volatility".
\end{frame}



\begin{frame}{Vega of a vanilla option - 2}
\centering
\begin{tikzpicture}
\begin{axis}[
			xtick={0.5},
			xticklabel={$K$},
			ytick={\empty},
			xmin=0, xmax=1,
			ymin=0, ymax=0.25,
			grid = none,
			xlabel={Underlying asset price ($S$)},
			ylabel={Option Vega}
]
	
	\addplot[color = Set1-B, mark = none, thick]
	table[
		x=S,
		y=vega,
		col sep=comma
	]
	{black_scholes_vega.csv};
\end{axis}
\end{tikzpicture}
\end{frame}



\begin{frame}{Theta}
\justify
\alert{Theta} is the sensitivity of the present value (PV) to the passage of time. How much will the purchased option decrease in value when it becomes one day closer to expiration?

\begin{align*}
\Theta = -\frac{\partial V}{\partial T}
\end{align*}
The standard units of measurement are dollars or euros per calendar or trading day.

\justify
Theta of a call option in the Black-Scholes model:
\begin{align*}
\Theta = -\frac{\partial C}{\partial T} = - \frac{S\sigma e^{-qT}N'(d_1)}{2\sqrt{T}} -rKe^{-rT}N(d_2) + qSe^{-qT}N(d_1)
\end{align*}
\end{frame}



\begin{frame}{Theta of a call option}
\centering
\begin{tikzpicture}
\begin{axis}[
			xtick={0.5},
			xticklabels={$K$},
			ytick={0},
			scaled y ticks = false,
			xmin=0, xmax=1,
			ymax=0,
			grid = none,
			xlabel={Underlying asset price ($S$)},
			ylabel={Option Theta}
]
	
	\addplot[color = Set1-B, mark = none, thick]
	table[
		x=S,
		y=theta,
		col sep=comma
	]
	{black_scholes_theta.csv};
\end{axis}
\end{tikzpicture}
\end{frame}



\begin{frame}{Theta of a call option - 2}
\centering
\begin{tikzpicture}
\begin{axis}[
			xtick={\empty},
			ytick={0},
			xmin=0, xmax=1,
			ymin=-0.15, ymax=0,
			grid = major,
			xlabel={Time till expiration ($T$)},
			ylabel={Option Theta},
			legend entries = {
  	   Out of the money,
  	   In the money,
  	   At the money
  },
  legend cell align={left},
  legend style={at={(0.97,0.03)},anchor=south east}
]
	
	\addplot[color = Set1-A, mark = none, thick]
	table[
		x=T,
		y=theta_otm,
		col sep=comma
	]
	{black_scholes_theta_by_money.csv};
	
	\addplot[color = Set1-B, mark = none, thick]
	table[
		x=T,
		y=theta_itm,
		col sep=comma
	]
	{black_scholes_theta_by_money.csv};


	\addplot[color = Set1-C, mark = none, thick]
	table[
		x=T,
		y=theta_atm,
		col sep=comma
	]
	{black_scholes_theta_by_money.csv};
\end{axis}
\end{tikzpicture}
\end{frame}



\begin{frame}{Gamma}
\justify
\alert{Gamma} is sensitivity of the Delta to the underlying asset price. This is the second derivative of PV w.r.t. underlying asset price.
\begin{align*}
\Gamma = \frac{\partial V^2}{\partial^2 S} = \frac{\partial \Delta}{\partial S}
\end{align*}

\justify
The larger the absolute value of Gamma, the worse a linear combination of the underlying asset and debt replicates a derivative, and the more frequently we have to rebalance the replicating portfolio.

\justify
In the Black-Scholes model Gamma of a put is equal to Gamma of a call:
\begin{align*}
\Gamma = \frac{\partial C^2}{\partial^2 S} = \frac{\partial P^2}{\partial^2 S} =
\frac{e^{-qT}N'(d_1)}{S\sigma\sqrt{T}} 
\end{align*} 
\end{frame}



\begin{frame}{Gamma and delta-hedging}
\justify
Gamma reflects concavity of the price function. The more concave the function, the greater the losses from imperfect hedging.

\centering
\begin{tikzpicture}
\begin{axis}[
			domain=92:110,
			xtick distance=2,
			ytick distance=1,
			xmin=94, xmax=110,
			ymin=0, ymax=10,
			grid = major,
			xlabel={Underlying asset price today ($S$)},
			ylabel={Option price}
]
	
	\addplot[color = Set1-A, mark = none, thick]
	table[
		x=S,
		y=C_10,
		col sep=comma
	]
	{call_price.csv};
	
	%\addplot[Set1-D, very thick, dashed] {(\x >= 72)*(\x - 72) + 0.05};
	
	\node[inner sep=2pt, circle, fill, color=Set1-A] at (axis cs: 100, 2.6648) {};
	
	\node[inner sep=2pt, circle, fill, color=Set1-A] at (axis cs: 108, 9.3179) {};
	
	\addplot[color=Set1-A, dashed, thick] {0.6083 * \x - 58.1936};
	
	\node[anchor=south west] at (axis cs: 96.5, 0) {$\Delta = 0.61$};
	
	\draw[<->, >= triangle 45, thick] (108, 9.3179) -- (108, 7.5028);
\end{axis}
\end{tikzpicture}
\end{frame}



\begin{frame}{Gamma of a vanilla option}
\centering
\begin{tikzpicture}
\begin{axis}[
			xtick={0.5},
			xticklabels={$K$},
			ytick={0},
			scaled y ticks = false,
			xmin=0, xmax=1,
			ymin=0,
			grid = none,
			xlabel={Underlying asset price ($S$)},
			ylabel={Option Gamma}
]
	
	\addplot[color = Set1-B, mark = none, thick]
	table[
		x=S,
		y=gamma,
		col sep=comma
	]
	{black_scholes_gamma.csv};
\end{axis}
\end{tikzpicture}
\end{frame}



\begin{frame}{Gamma of a vanilla option - 2}
\centering
\begin{tikzpicture}
\begin{axis}[
			xtick={\empty},
			ytick={0},
			xmin=0, xmax=1,
			ymin=0, ymax=15,
			grid = major,
			xlabel={Time till expiration ($T$)},
			ylabel={Option Gamma},
			legend entries = {
		At the money,
  	   Out of the money,
  	   In the money
  },
  legend cell align={left},
  legend style={at={(0.97,0.97)},anchor=north east}
]
	
	\addplot[color = Set1-C, mark = none, thick]
	table[
		x=T,
		y=gamma_atm,
		col sep=comma
	]
	{black_scholes_gamma_by_money.csv};
	
	\addplot[color = Set1-A, mark = none, thick]
	table[
		x=T,
		y=gamma_otm,
		col sep=comma
	]
	{black_scholes_gamma_by_money.csv};
	
	\addplot[color = Set1-B, mark = none, thick]
	table[
		x=T,
		y=gamma_itm,
		col sep=comma
	]
	{black_scholes_gamma_by_money.csv};

\end{axis}
\end{tikzpicture}
\end{frame}



\begin{frame}{Gamma and Theta}
\justify
Consider a portfolio of an option and its underlying asset. In the Black-Scholes model portfolio value $V$ satisfies the following condition:
\begin{align*}
\frac{\partial V}{\partial t} + rS\frac{\partial V}{\partial S} + \frac{1}{2}\sigma^2S^2\frac{\partial^2 V}{\partial S^2} = rV
\end{align*}
\begin{align*}
-\Theta + rS\Delta + \frac{1}{2}\sigma^2S^2\Gamma = rV
\end{align*}

If we delta-hedge the portfolio regularly, then $\Delta=0$.
\begin{align*}
-\Theta + \frac{1}{2}\sigma^2S^2\Gamma = rV
\end{align*}

\justify
Large values of $\Gamma$ usually correspond to large values of $\Theta$. If we buy an option ($\Gamma > 0$), we profit from large movements in the underlying asset but lose Theta over time (time value of the option decreases). If we sell an option ($\Gamma < 0$), we lose money due to imperfect hedging but gain Theta.
\end{frame}



\begin{frame}{Gamma and Theta: an example}
\justify
Suppose we live in the Black-Scholes world, where we have $S=\$100$, $r=0\%$, $q=0\%$, $\sigma=20\%$. We have \alert{sold} a call option at strike $K=\$100$ which expires in$T=0.25$  years.

\justify
Fair premium for this option is $\$3.99$, its Delta is 0.52. We hedged the risk: we bought 0.52 of the stock and borrowed \$48.01. Our PV is zero, our Delta is zero.

\justify
One business day has passed ($dt = 1/252$). Stock price has changed. How much did we gain or lose on our delta-neutral position?
\end{frame}



\begin{frame}{Gamma and Theta: an example - 2}
\centering
\begin{tikzpicture}
\begin{axis}[
	height = \textheight-1cm,
	xmin = 96, xmax = 104,
	ymin = -0.3, ymax = 0.05,
	grid = major,
	xlabel = {Stock price tomorrow ($S$)},
	ylabel = {Portfolio PV, \$},
	y tick label style={
        /pgf/number format/.cd,
            fixed,
            fixed zerofill,
            precision=2,
        /tikz/.cd}
]

	\addplot[color=Set1-A, thick] table[x=S, y=full_pnl, col sep=comma] {gamma_pnl_example.csv};
	
	\draw[<->, >= triangle 45, thick] (100, 0.0) -- (100, 0.032) node[pos=0.5, anchor=west] {$\theta$};
	
	\draw[<->, >= triangle 45, thick] (103, 0.0) -- (103, -0.145) node[pos=0.5, anchor=west] {$\Gamma$};
	
   \draw[thick, color=black] (axis cs: 95, 0) -- (axis cs: 105, 0);
\end{axis}
\end{tikzpicture}
\end{frame}



\begin{frame}{Gamma and Vega}
\justify
In the Black-Scholes model Gamma and Vega differ by a constant multiplier.
\begin{align*}
\Gamma &= \frac{e^{-qT}N'(d_1)}{S\sigma\sqrt{T}} \\
\mathcal{V} &= Se^{-qT}\sqrt{T}N'(d_1)
\end{align*}

\justify
In some sense, both derivatives measure our sensitivity to the variance of the future value of $S$.

\justify
Gamma is major risk for short-dated options ($\sqrt{T}$ in the denominator), Vega is major risk for long-dated options ($\sqrt{T}$ in the numerator). 
\end{frame}



\begin{frame}{Delta-hedging and volatility}

\justify
Consider a non-dividend paying stock. Its current price is $S_0=\$100$. The stock price follows a geometric Brownian motion with trend $\mu=5\%$ and volatility $\sigma=20\%$. Risk-free rate is $r=0\%$.

\justify
Consider a European call option at strike $K=100$ that expires in $T=1$ year. Imagine that we were unable to buy this option in the market, so we have to replicate it dynamically ourselves.

\justify
Suppose that we have $\$7.97$ in  a bank account today. We are allowed to rebalance our replicating portfolio $N=4 \cdot 365$ times a year (4 times a day).

\end{frame}




\begin{frame}{Delta-hedging and volatility - 2}
\centering
\begin{tikzpicture}
\begin{axis}[
	width = \textwidth,
	height = \textheight - 0.5cm,
	xlabel = {Underlying asset price in$T=1$ year ($S_T$)},
	ylabel = {Portfolio value},
	xmin = 70, xmax = 130,
	ymin = -5, ymax = 30,
	grid = major,
	ytick = {-10, -5, ..., 50}
]	

   \addplot[thick, mark = o, only marks, color=Set1-A]
     table [col sep=comma, x=S, y=balance]
     {black_scholes_hedging_1420.csv};    
    
    \addplot[black, very thick, solid, domain = 0:150] {0};
\end{axis}
\end{tikzpicture}
\end{frame}



\begin{frame}{Delta-hedging and volatility - 3}
\justify
How do we compute $\Delta$ for our hedge? From the Black-Scholes formula. But the formula depends on 
\alert{implied} volatility! What if \alert{realized} volatility will turn out to be higher?

\justify
Suppose that a every time step realized volatility of the Brownian motion turns out to be $\sigma_{realized}=25\%$. We do not know this in advance, so we pick the hedge size assuming $\sigma_{implied}=20\%$. 
\end{frame}



\begin{frame}{Delta-hedging and volatility - 4}
\centering
\begin{tikzpicture}
\begin{axis}[
	width = \textwidth,
	height = \textheight - 0.5cm,
	xlabel = {Underlying asset price in $T=1$ year ($S_T$)},
	ylabel = {Portfolio value},
	xmin = 70, xmax = 130,
	ymin = -5, ymax = 30,
	grid = major,
	ytick = {-10, -5, ..., 50}
]	

	\addplot[very thick, dashed, color=Set1-A, domain=70:130] {(\x > 100) * (\x - 100) + 0.1};

   \addplot[thick, mark = o, only marks, color=Set1-A]
     table [col sep=comma, x=S, y=balance]
     {black_scholes_hedging_realized_vol.csv};    
    
    
    \addplot[black, very thick, solid, domain = 0:150] {0};
\end{axis}
\end{tikzpicture}
\end{frame}



\begin{frame}{Conclusion}
\justify
One can consider an option premium as the market's expectation of the profit through Gamma (how much the option holder will earn in total if they delta-hedge). On the other hand, the option premium compensates for the seller's losses (they will lose money on delta-hedging).

\justify
If the realized volatility turns out to be higher than implied, the option seller will lose money on hedging, while the buyer will profit.

\justify
On average, in the long run, it is more profitable to be selling Gamma (sell options, essentially selling crisis insurance to others). Sometimes this strategy fails: the market movement is so large that the losses on Gamma exceed the received premium.
\end{frame}

\end{document}


