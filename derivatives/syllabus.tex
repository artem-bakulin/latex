\documentclass[a4paper,14pt]{extarticle}
\usepackage{cmap}				% To be able to copy-paste russian text from pdf			
\usepackage[utf8]{inputenc}
\usepackage[T2A]{fontenc}
\usepackage[margin=1in]{geometry}
\usepackage[english]{babel}

\usepackage{amsmath}
\usepackage{amsfonts}

\usepackage{libertine}
\usepackage{libertinust1math}

\usepackage{parskip} % remove paragraph indentation

\title{Derivatives}
\author{Artem Bakulin}
\date{October 9, 2023}

\begin{document}

\maketitle

\section*{Course description}
This course is introduction to financial derivatives. We will have a look at some most common
classes, such as foreign exchange forwards, interest rate swaps, vanilla and exotic options, credit
default swaps. Every time we will be interested in two questions: what is economic purpose of a
derivative (who are natural buyers and sellers), and how do we price it?

Main goal of the course is to give you an idea of risk-neutral pricing and replication. Every
reasonable derivative can be replicated as a combination of some underlying assets (probably
with dynamic, but mechanical, trading). If you are trying to price a derivative by forecasting the
future, then probably you are doing something wrong.

After the course you will understand major classes of derivatives. You will learn which class of
derivatives can hedge which risk. You will also understand how market-makers price derivatives
and risk-manage their positions.

\section*{Course requirements}

Prerequisites:

1. Basic calculus (you know what is derivative of a function).

2. Basic probability theory (you know what is expected value and standard deviation).

\section*{Course contents}

1. Mechanics of foreign exchange market. Foreign exchange forwards. Hedging and speculation.
Arbitrage-free price of a forward. Non-deliverable forwards.
Reading: Hull, chapters 2-5.

2. Floating interest rates. Interest rate swaps and futures. Bootstrapping a yield curve from
market instruments. Managing interest rate risk. Overnight index swaps curve and discounting.
LIBOR-gate.
Reading: Hull, chapters 6, 7.

3. Vanilla options. Vanilla strategies. Single-step and multi-step binomial trees. Dynamic
replication (also known as delta-hedging).
Reading: Hull, chapters 10-13.

4. Geometric Brownian motion and Black-Scholes formula as the limit of binomial tree. Implied
volatility and volatility smile.
Reading: Hull, chapters 14,15,20

5. Market risk and option greeks: delta, gamma, vega, rho, theta. Link between gamma and theta,
gamma and vega.
Reading: Hull, chapter 19.

6. Credit default swaps. Connection between bonds, probabilities of default, and credit default
swaps. Wrong-way risk. Risk-neutral and real-world probabilities revisited.
Reading: Hull, chapters 24, 25

7. Monte-Carlo method. Risk-neutral probability and fundamental theorem of asset pricing.
Reading: Hull, chapter 21.

\section*{Course materials}

\subsection*{Required textbooks and materials}
Rule of thumb in derivatives: if you are not sure where to start, check Hull’s book.

John C Hull. \textit{Options, Futures, and Other Derivatives.} 11th ed. Pearson, 2022. ISBN:
9780136939979.

\subsection*{Additional reading}

These books go into depths of some particular topics. You do not have to read them to pass the
course. However, they demonstrate how deep is the rabbit hole, and how many aspects one
needs to take into account.

1. Strictly speaking, Kevin Rodger’s book is not about derivatives per se. However it offers great
insight into everyday life of a trader at large investment bank, and role of computers on modern
financial markets.

Kevin Rodgers. \textit{Why Aren't They Shouting?: A Banker’s Tale of Change, Computers and Perpetual
Crisis.} Random House Business, 2017. ISBN: 9781847941534.

2. I recommend Emmanuel Derman’s book as your second book about derivatives after Hull. On
one hand, it covers more advanced topics than Hull’s textbook. On the other hand, I like how the
book combines intuitive explanations with mathematical technicalities.

Emanuel Derman, Michael B. Miller, David Park. \textit{The Volatility Smile: An Introduction for Students
and Practitioners.} Wiley, 2016. ISBN: 9781118959169.

3. In case you are interested in theoretical foundations and mathematical proofs, have a look at
two Steven Shreve’s books.

Steven Shreve. \textit{Stochastic Calculus for Finance I: The Binomial Asset Pricing Model.} Springer, 2004.
ISBN: 9780387249681.

Steven Shreve. \textit{Stochastic Calculus for Finance II: Continuous-Time Models.} Springer, 2004. ISBN:
9780387401010.

4. Some people prefer Tomas Bjork’s book about mathematical foundations:

Tomas Bjork. \textit{Arbitrage Theory in Continuous Time.} 3rd ed. Oxford University Press, 2009. ISBN:
9780199574742.

5. Interest rate swaps may appear simple to someone, compared to options. Marc Henrard’s
book demonstrates how complex this area is, and how different it is from what is described in
textbooks.

Marc Henrard. \textit{Interest Rate Modelling in the Multi-curve Framework.} Palgrave Macmillan, 2014.
ISBN: 9781137374660.

6. Iain Clark describes complex stochastic volatility models that are frequently used on foreign
exchange market.

Iain Clark. \textit{Foreign Exchange Option Pricing: A Practitioner's Guide.} Wiley, 2010. ISBN: 9780470683682.

7. This course does not cover credit valuation adjustment (CVA) and other specific adjustments.
However they are extremely important. If you are not aware of them, you are risking losing
money in the long run.

Jon Gregory. \textit{The xVA Challenge: Counterparty Risk, Funding, Collateral, Capital and Initial Margin.}
4th ed. Wiley, 2020. ISBN: 9781119508977.

\section*{Sample problem for course evaluation}
A structured note works as follow. A client invests \$100\,000 notional at inception. Whatever
happens, the client will receive full notional \$100\,000 in 3 years (so the note offers 100\% capital
guarantee).

At the end of years 1, 2, and 3, the note may (or may not) pay out additional fixed coupons of X\%
of notional. This X\% coupon rate is parameter of the contract, which you will need to compute.
Coupon is paid out only in case on the payment date the S\&P\,500 index is higher than its level
today \$4\,000.

How can you replicate this note using cash deposits, vanilla options and/or digital options?
Assume that risk-free rate is 3\% (continuously compounded), Black-Scholes volatility of the
index is 20\%, dividend yield is 1\%. What is fair value of coupon X\%?
\end{document}