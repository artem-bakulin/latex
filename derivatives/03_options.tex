\documentclass{beamer}

\usepackage{cmap}				% To be able to copy-paste russian text from pdf
\usepackage[T2A]{fontenc}
\usepackage[utf8]{inputenc}
\usepackage[russian]{babel}
\usepackage{textpos}
\usepackage{ragged2e}
\usepackage{amssymb}
\usepackage{ulem}
\usepackage{tikz}
\usepackage{pgfplots}
\usepackage{color}
\usepackage{cancel}
\usepackage{multirow}
\pgfplotsset{compat=1.17}
\usetikzlibrary{arrows,snakes,backgrounds,shapes}
\usepgfplotslibrary{groupplots,colorbrewer,dateplot,statistics}
\usepackage{animate}

\usepackage{amsfonts}
\usepackage{amsmath}
\usepackage{amssymb}
\usepackage{graphicx}
\usepackage{setspace}

\usepackage{enumitem}
\setitemize{label=\usebeamerfont*{itemize item}%
  \usebeamercolor[fg]{itemize item}
  \usebeamertemplate{itemize item}}

% remove navigation bar
\setbeamertemplate{navigation symbols}{} 

\usepackage{eurosym}
\renewcommand{\EUR}[1]{\textup{\euro}#1}

\title{Лекция 3. Опционы}
\author{Артём Бакулин}
\date{21 октября 2021 г.}

\usetheme{Warsaw}
\usecolortheme{beaver}

\newcommand{\ru}[1]{\begin{otherlanguage}{russian}#1\end{otherlanguage}}
\newcommand{\en}[1]{\begin{otherlanguage}{english}#1\end{otherlanguage}}
\newcommand{\ruen}[2]{#1 (\en{#2})}

\begin{document}



\begin{frame}
\titlepage
\end{frame}



\begin{frame}{Ванильные опционы}
\justifying
\alert{Опцион колл (пут)} --- это контракт, который даёт владельцу \textit{право}, но не обязанность, купить (продать) базовый актив по заранее оговоренной цене в заранее определённую дату. 

\justify
Пример: колл-опцион, который даёт право (но не обязанность) купить 1\,000 долларов за 72\,000 рублей рублей в дату 15.12.2021. Сегодня за такой опцион нужно заплатить 500 рублей.

\justify
Такие опционы колл и пут называют <<ванильными>> (\en{plain vanilla}), потому что самое простое мороженое без ничего --- простое ванильное.
\end{frame}



\begin{frame}{Ванильные опционы}
\justify
<<Первый канал>> об убытках банка \en{Societe General}, 25.08.2008:

\justify
<<Источник агентства Reuters в самом банке на условиях анонимности сообщил, что в последнее время трейдер, о котором идет речь, покупал и продавал контракты на поставку обычной ванили. \alert{Той самой, которую используют в кулинарии}. Правда, ванили на 7 миллиардов долларов, в год не производят и во всем мире. Фактически речь идет о торговле воздухом. Хотя это вполне законно и используется повсеместно.>>

\justify
\url{https://www.newsru.com/russia/25Jan2008/vanilla.html}

\end{frame}



\begin{frame}{Ванильные опционы}
\justifying
\begin{itemize}
\item Колл (call) --- право купить актив, пут (put) --- право продать.
\item Страйк (strike) --- фиксированная цена, по которой можно будет купить или продать базовый актив.
\item Премия (premium) --- количество денег, которые нужно заплатить сегодня, чтобы получить опцион.
\item Выписать опцион (write an option) --- то же самое, что продать опцион.
\item Дата экспирации (expiration date) --- дата, в которую опцион прекращает действовать, если владелец его не использовал.
\item Европейский опцион --- воспользоваться правом можно только в дату экспирации.
\item Американский опцион --- воспользоваться правом можно в любой момент до экспирации
\end{itemize}
\end{frame}



\begin{frame}{Пример: колл-опцион}
\justifying
Колл-опцион на покупку долларов за рубли со страйком 72.

\justifying
\centering
	\begin{tikzpicture}
		\begin{axis}[
			domain=64:80,
			%axis lines=middle,
			xtick={64,66,...,80},
			ytick={0,1,2,...,8},
			xmin=64, xmax=80,
			ymin=-1, ymax=8,
			grid = major,
			xlabel={Курс в дату экспирации},
			ylabel={Выплата (payoff)},
		]
  \addplot[Set1-A, very thick] {(\x > 72)*(\x - 72)};
  
  \draw[thick, color=black] (axis cs: 60, 0) -- (axis cs: 90, 0);
\end{axis}
\end{tikzpicture}
\end{frame}



\begin{frame}{Пример: покупаем колл-опцион}
\justifying
Покупка колл-опциона со страйком 72 за премию 0.5 руб.

\justifying
\centering
	\begin{tikzpicture}
		\begin{axis}[
			domain=64:80,
			%axis lines=middle,
			xtick={64,66,...,80},
			ytick={0,1,2,...,8},
			xmin=64, xmax=80,
			ymin=-1, ymax=8,
			%x label style={at={(axis description cs: 0.5, -0.1)}, anchor=north},
			%y label style={at={(axis description cs:-0.1,1)},anchor=south},
			grid = major,
			xlabel={Курс в дату экспирации},
			ylabel={Прибыль (profit'n'loss)},
			%scaled x ticks=false
		]
  \addplot[Set1-A, very thick] {(\x > 72)*(\x - 72) - 0.5};
  
  \draw[thick, color=black] (axis cs: 60, 0) -- (axis cs: 90, 0);
\end{axis}
\end{tikzpicture}
\end{frame}



\begin{frame}{Пример: покупаем пут-опцион}
\justifying
Покупка пут-опциона со страйком 72 за премию 0.5 руб.

\justifying
\centering
	\begin{tikzpicture}
		\begin{axis}[
			domain=64:80,
			%axis lines=middle,
			xtick={64,66,...,80},
			ytick={0,1,2,...,8},
			xmin=64, xmax=80,
			ymin=-1, ymax=8,
			%x label style={at={(axis description cs: 0.5, -0.1)}, anchor=north},
			%y label style={at={(axis description cs:-0.1,1)},anchor=south},
			grid = major,
			xlabel={Курс в дату экспирации},
			ylabel={Прибыль (profit'n'loss)},
			%scaled x ticks=false
		]
  \addplot[Set1-A, very thick] {(\x < 72)*(72 - \x) - 0.5};
  
  \draw[thick, color=black] (axis cs: 60, 0) -- (axis cs: 90, 0);
\end{axis}
\end{tikzpicture}
\end{frame}



\begin{frame}{Пример: продаём колл-опцион}
\justifying
Продажа колл-опциона со страйком 72 за премию 0.5 руб.

\justifying
\centering
	\begin{tikzpicture}
		\begin{axis}[
			domain=64:80,
			%axis lines=middle,
			xtick={64,66,...,80},
			ytick={-8,-7,...,0,1},
			xmin=64, xmax=80,
			ymin=-8, ymax=1,
			%x label style={at={(axis description cs: 0.5, -0.1)}, anchor=north},
			%y label style={at={(axis description cs:-0.1,1)},anchor=south},
			grid = major,
			xlabel={Курс в дату экспирации},
			ylabel={Прибыль (profit'n'loss)},
			%scaled x ticks=false
		]
  \addplot[Set1-A, very thick] {(\x > 72)*(72 - \x) + 0.5};
  
  \draw[thick, color=black] (axis cs: 60, 0) -- (axis cs: 90, 0);
\end{axis}
\end{tikzpicture}
\end{frame}



\begin{frame}{Пример: продаём пут-опцион}
\justifying
Продажа пут-опциона со страйком 72 за премию 0.5 руб.

\justifying
\centering
	\begin{tikzpicture}
		\begin{axis}[
			domain=64:80,
			%axis lines=middle,
			xtick={64,66,...,80},
			ytick={-8,-7,...,0,1},
			xmin=64, xmax=80,
			ymin=-8, ymax=1,
			%x label style={at={(axis description cs: 0.5, -0.1)}, anchor=north},
			%y label style={at={(axis description cs:-0.1,1)},anchor=south},
			grid = major,
			xlabel={Курс в дату экспирации},
			ylabel={Прибыль (profit'n'loss)},
			%scaled x ticks=false
		]
  \addplot[Set1-A, very thick] {(\x < 72)*(\x - 72) + 0.5};
  
  \draw[thick, color=black] (axis cs: 60, 0) -- (axis cs: 90, 0);
\end{axis}
\end{tikzpicture}
\end{frame}
\end{document}