\documentclass{beamer}

\usepackage{cmap}				% To be able to copy-paste russian text from pdf
\usepackage[T2A]{fontenc}
\usepackage[utf8]{inputenc}
\usepackage[russian]{babel}
\usepackage{textpos}
\usepackage{ragged2e}
\usepackage{amssymb}
\usepackage{ulem}
\usepackage{tikz}
\usepackage{pgfplots}
\usepackage{color}
\usepackage{cancel}
\usepackage{multirow}
\pgfplotsset{compat=1.17}
\usetikzlibrary{arrows,snakes,backgrounds,shapes}
\usepgfplotslibrary{groupplots,colorbrewer,dateplot,statistics}
\usepackage{animate}

\usepackage{amsfonts}
\usepackage{amsmath}
\usepackage{amssymb}
\usepackage{graphicx}
\usepackage{setspace}
\usepackage{cancel}

\usepackage{enumitem}
\setitemize{label=\usebeamerfont*{itemize item}%
  \usebeamercolor[fg]{itemize item}
  \usebeamertemplate{itemize item}}

% remove navigation bar
\setbeamertemplate{navigation symbols}{} 

\usepackage{eurosym}
\renewcommand{\EUR}[1]{\textup{\euro}#1}

\title{Опционы}
\author{Артём Бакулин}
\date{21 марта 2022 г.}

\usetheme{Warsaw}
\usecolortheme{beaver}

\setbeamertemplate{page number in head/foot}[totalframenumber] 

\newcommand{\ru}[1]{\begin{otherlanguage}{russian}#1\end{otherlanguage}}
\newcommand{\en}[1]{\begin{otherlanguage}{english}#1\end{otherlanguage}}
\newcommand{\ruen}[2]{#1 (\en{#2})}

\begin{document}



\begin{frame}
\titlepage
\end{frame}



\begin{frame}{Ванильные опционы}
\justifying
\alert{Опцион колл (пут)} --- это контракт, который даёт владельцу право, но не обязанность, купить (продать) базовый актив по заранее оговоренной цене в заранее определённую дату. 

\justify
Пример: колл-опцион, который даёт право (но не обязанность) купить 1\,000 долларов за 72\,000 рублей рублей через 3 месяца. Сегодня за такой опцион нужно заплатить 500 рублей.
\end{frame}



\begin{frame}{Ванильные опционы}
\begin{itemize}
\justifying
\item Колл (call) --- право купить актив, пут (put) --- право продать.
\item Страйк (strike) --- фиксированная цена, по которой можно будет купить или продать базовый актив.
\item Премия (premium) --- количество денег, которые нужно заплатить сегодня, чтобы получить опцион.
\item Выписать опцион (write an option) --- то же самое, что продать опцион.
\item Дата экспирации (expiration date) --- дата, в которую опцион прекращает действовать, если владелец его не использовал.
\item Европейский опцион (\en{European option}) --- воспользоваться правом можно только в дату экспирации.
\item Американский опцион (\en{American option}) --- воспользоваться правом можно в любой день до экспирации.
\end{itemize}
\end{frame}



\newcommand{\circlewithtext}[3]{
    \node[circle, fill, inner sep = 1.5pt] at (#1, #2) {};
    \node[anchor = north] at (#1, #2) {#3};
}

\begin{frame}{Ванильные опционы}
\justify
\centering
\begin{tikzpicture}
		\draw [->,>=triangle 90] (0, 0) -- (9.5, 0);
		
		\draw [->, >=triangle 45] (0.5, 0) .. controls (1.0, 1) and (2.0, 1) .. (2.5, 0) node[pos=0.5, anchor=south]{+1 день};
		
		\draw [->, >=triangle 45] (2.5, 0) .. controls (3, 2) and (8, 2) .. (8.5, 0) node[pos=0.5, anchor=north]{+3 месяца};
		
		\draw [->, >=triangle 45] (8.5, 0) .. controls (8, 0.5) and (7, 0.5) .. (6.5, 0) node[pos=0.5, anchor=south]{$-1$ день};
		
		\circlewithtext{0.5}{0}{Сегодня};
		\circlewithtext{2.5}{0}{\begin{tabular}{c}Спот \\ $-500$ р. \end{tabular}};
		\circlewithtext{6.5}{0}{Экспирация};
		\circlewithtext{8.5}{0}{\begin{tabular}{c}Поставка \\ $+\$1\,000$ \\ $-72\,000$ р.\end{tabular}};
	\end{tikzpicture}
	
\justify
\begin{itemize}
\justifying
\item Сегодня --- дата заключения договора.
\item Спот-дата (\en{spot date}) --- дата выплаты премии.
\item Экспирация (\en{expiration date}) --- дата, в которую покупатель опциона принимает решение об исполнении опциона.
\item Дата поставки (\en{settlement date}) --- дата, в которую произойдёт обмен валютой, если опцион будет исполнен.
\end{itemize}

\justify
Дальше в курсе мы будем считать, что премия платится в день сделки, а даты экспирации и поставки совпадают.
\end{frame}



\begin{frame}{Ванильные опционы}
\justify
Европейские опционы колл и пут называют <<\alert{ванильными}>> (\en{plain vanilla}), потому что самое простое мороженое без ничего --- ванильное.

\justify
<<Первый канал>> об убытках банка \en{Societe General}, 25.01.2008:

\justify
<<Источник агентства Reuters в самом банке на условиях анонимности сообщил, что в последнее время трейдер, о котором идет речь, покупал и продавал контракты на поставку обычной ванили. \alert{Той самой, которую используют в кулинарии}. Правда, ванили на 7 миллиардов долларов, в год не производят и во всем мире. Фактически речь идет о торговле воздухом. Хотя это вполне законно и используется повсеместно.>>

\justify
\url{https://www.newsru.com/russia/25Jan2008/vanilla.html}
\end{frame}



\begin{frame}{Пример: колл-опцион}
\justifying
Колл-опцион на покупку долларов за рубли со страйком 72.

\justifying
\centering
	\begin{tikzpicture}
		\begin{axis}[
			domain=64:80,
			%axis lines=middle,
			xtick={64,66,...,80},
			ytick={0,1,2,...,8},
			xmin=64, xmax=80,
			ymin=-1, ymax=8,
			grid = major,
			xlabel={Курс в дату экспирации},
			ylabel={Выплата (payoff)},
		]
  \addplot[Set1-A, very thick] {(\x > 72)*(\x - 72) + 0.05};
  
  \draw[thick, color=black] (axis cs: 60, 0) -- (axis cs: 90, 0);
\end{axis}
\end{tikzpicture}
\end{frame}



\begin{frame}{Пример: покупаем колл-опцион}
\justifying
Покупка колл-опциона со страйком 72 за премию 0.5 руб.

\justifying
\centering
	\begin{tikzpicture}
		\begin{axis}[
			domain=64:80,
			%axis lines=middle,
			xtick={64,66,...,80},
			ytick={0,1,2,...,8},
			xmin=64, xmax=80,
			ymin=-1, ymax=8,
			%x label style={at={(axis description cs: 0.5, -0.1)}, anchor=north},
			%y label style={at={(axis description cs:-0.1,1)},anchor=south},
			grid = major,
			xlabel={Курс в дату экспирации},
			ylabel={Прибыль (profit'n'loss)},
			%scaled x ticks=false
		]
  \addplot[Set1-A, very thick] {(\x > 72)*(\x - 72) - 0.5};
  
  \draw[thick, color=black] (axis cs: 60, 0) -- (axis cs: 90, 0);
\end{axis}
\end{tikzpicture}
\end{frame}



\begin{frame}{Пример: покупаем пут-опцион}
\justifying
Покупка пут-опциона со страйком 72 за премию 0.5 руб.

\justifying
\centering
	\begin{tikzpicture}
		\begin{axis}[
			domain=64:80,
			%axis lines=middle,
			xtick={64,66,...,80},
			ytick={0,1,2,...,8},
			xmin=64, xmax=80,
			ymin=-1, ymax=8,
			%x label style={at={(axis description cs: 0.5, -0.1)}, anchor=north},
			%y label style={at={(axis description cs:-0.1,1)},anchor=south},
			grid = major,
			xlabel={Курс в дату экспирации},
			ylabel={Прибыль (profit'n'loss)},
			%scaled x ticks=false
		]
  \addplot[Set1-A, very thick] {(\x < 72)*(72 - \x) - 0.5};
  
  \draw[thick, color=black] (axis cs: 60, 0) -- (axis cs: 90, 0);
\end{axis}
\end{tikzpicture}
\end{frame}



\begin{frame}{Пример: продаём колл-опцион}
\justifying
Продажа колл-опциона со страйком 72 за премию 0.5 руб.

\justifying
\centering
	\begin{tikzpicture}
		\begin{axis}[
			domain=64:80,
			%axis lines=middle,
			xtick={64,66,...,80},
			ytick={-8,-7,...,0,1},
			xmin=64, xmax=80,
			ymin=-8, ymax=1,
			%x label style={at={(axis description cs: 0.5, -0.1)}, anchor=north},
			%y label style={at={(axis description cs:-0.1,1)},anchor=south},
			grid = major,
			xlabel={Курс в дату экспирации},
			ylabel={Прибыль (profit'n'loss)},
			%scaled x ticks=false
		]
  \addplot[Set1-A, very thick] {(\x > 72)*(72 - \x) + 0.5};
  
  \draw[thick, color=black] (axis cs: 60, 0) -- (axis cs: 90, 0);
\end{axis}
\end{tikzpicture}
\end{frame}



\begin{frame}{Пример: продаём пут-опцион}
\justifying
Продажа пут-опциона со страйком 72 за премию 0.5 руб.

\justifying
\centering
	\begin{tikzpicture}
		\begin{axis}[
			domain=64:80,
			%axis lines=middle,
			xtick={64,66,...,80},
			ytick={-8,-7,...,0,1},
			xmin=64, xmax=80,
			ymin=-8, ymax=1,
			%x label style={at={(axis description cs: 0.5, -0.1)}, anchor=north},
			%y label style={at={(axis description cs:-0.1,1)},anchor=south},
			grid = major,
			xlabel={Курс в дату экспирации},
			ylabel={Прибыль (profit'n'loss)},
			%scaled x ticks=false
		]
  \addplot[Set1-A, very thick] {(\x < 72)*(\x - 72) + 0.5};
  
  \draw[thick, color=black] (axis cs: 60, 0) -- (axis cs: 90, 0);
\end{axis}
\end{tikzpicture}
\end{frame}



\begin{frame}{Хэджирование с помощью опционов}
\justify
Через три месяца нам понадобится \$1\,000 на отпуск. Форвардный курс на три месяца 72.0. Если купить форвард, мы зафиксируем курс, но что делать, если рубль укрепится до 50? Будет обидно покупать доллары по 72 при рыночном курсе 50.

\justify
Решение: можно купить колл-опцион со страйком 72 на \$1\,000 за премию 500 рублей. Если доллар еще вырастет, мы исполняем опцион и покупаем по 72. Если рубль внезапно укрепляется до 50, опцион пропадает, а мы покупаем по рыночному курсу 50.

\justify
Плюс: не так обидно при укреплении рубля. Минус: прямо сейчас нужно заплатить за опцион. Опцион работает как страховка от падения рубля.
\end{frame}




\begin{frame}{Хэджирование с помощью опционов --- 2}
\justify
Через три месяца нам придут дивиденды \$1\,000 от иностранных инвестиций. Тогда же нам нужно будет заплатить рубли за ипотеку. Форвардный курс на три месяца 72.0. Если продать форвард, мы зафиксируем курс, но что делать, если рубль ослабнет до 90? Будет обидно продавать доллары по 72 при рыночном курсе 90.

\justify
Решение: можно купить пут-опцион со страйком 72 на \$1\,000 за премию 500 рублей. Если доллар упадёт до 50, мы исполняем опцион и продаём доллары по 72. Если доллар укрепится до 90, опцион пропадает, а мы продадим доллары по рыночному курсу 90.

\justify
Плюс: не так обидно при укреплении доллара. Минус: прямо сейчас нужно заплатить за опцион. Опцион работает как страховка от укрепления рубля.
\end{frame}



\begin{frame}{Спекуляция с помощью опционов}
\justify
Хрустальный шар показал, что за три месяца доллар сильно укрепится относительного текущего форвардного курса 72. Мы хотим на этом заработать, но \sout{жена} акционеры нашего хэдж-фонда выделили нам всего 500 рублей.

\justify
Решение: можно купить колл-опцион со страйком 72 на \$1\,000 за премию 500 рублей. Если доллар укрепится до 80 рублей, мы заработаем $(80 - 72) \cdot 1\,000 - 500 = 7\,500$ рублей. Если этого не случится, мы просто потеряем нашу ставку.

\justify
Опцион позволяет взять <<плечо>> --- увеличить выигрыш при хорошем исходе (как будто у нас есть целая \$1,000) ценой потерь в худшем исходе (можно потерять всю премию). Однако, проигрыш строго ограничен снизу, а выигрыш может быть бесконечно большим.
\end{frame}



\begin{frame}{Спекуляция с помощью опционов --- 2}
\justify
Хрустальный шар показал, что за три месяца доллар сильно упадёт относительного текущего форвардного курса 72. Мы хотим на этом заработать, но \sout{жена} акционеры нашего хэдж-фонда выделили нам всего 500 рублей.

\justify
Решение: можно купить пут-опцион со страйком 72 на \$1\,000 за премию 500 рублей. Если доллар упадёт до 64 рублей, мы заработаем $(72 - 64) \cdot 1\,000 - 500 = 7\,500$ рублей. Если этого не случится, мы просто потеряем нашу ставку.

\justify
Опцион пут позволяет сделать ставку на падение базового актива (доллара).
\end{frame}



\begin{frame}{Денежность опционов}
\justify
Несколько стандартных терминов для <<денежности>> (\en{moneyness}) колл-опционов:

\begin{itemize}
\justifying
\item В деньгах (\en{in the money, ITM}) --- текущая цена базового актива выше страйка (опцион было бы выгодно исполнить прямо сейчас).
\item На деньгах (\en{at the money, ATM}) --- текущая цена базового актива равна страйку.
\item Вне денег (\en{out of the money, OTM}) --- текущая цена базового актива ниже страйка (было бы не выгодно исполнять опцион прямо сейчас).
\end{itemize}

\justify
Для пут-опционов терминология симметричная.
\end{frame}



\begin{frame}{Комбинации: коллар}
\justifying
Коллар (\en{collar}) --- проданный пут с меньшим страйком и купленный колл с большим страйком.

\justifying
\centering
	\begin{tikzpicture}
		\begin{axis}[
			domain=64:80,
			%axis lines=middle,
			xtick={64,66,...,80},
			ytick={-8,-7,...,6},
			xmin=64, xmax=80,
			ymin=-6, ymax=6,
			%x label style={at={(axis description cs: 0.5, -0.1)}, anchor=north},
			%y label style={at={(axis description cs:-0.1,1)},anchor=south},
			grid = major,
			xlabel={Курс в дату экспирации},
			ylabel={Прибыль (profit'n'loss)},
			%scaled x ticks=false
		]
	
  \addplot[Set1-A, very thick, dashed] {-(\x < 70)*(70 - \x) + 0.5};
  \addplot[Set1-B, very thick, dashed] {(\x > 74)*(\x - 74) - 0.5};
  \addplot[Set1-C, very thick] {-(\x < 70)*(70 - \x) + 0.5 + (\x > 74)*(\x - 74) - 0.5 + 0.05};	

  \draw[thick, color=black] (axis cs: 60, 0) -- (axis cs: 90, 0);
\end{axis}
\end{tikzpicture}
\end{frame}



\begin{frame}{Комбинации: колл-спред}
\justifying
Колл-спред (\en{call spread}) --- купленный колл с меньшим страйком, проданный колл с большим страйком.

\justifying
\centering
	\begin{tikzpicture}
		\begin{axis}[
			domain=64:80,
			%axis lines=middle,
			xtick={64,66,...,80},
			ytick={-8,-7,...,6},
			xmin=64, xmax=80,
			ymin=-6, ymax=6,
			%x label style={at={(axis description cs: 0.5, -0.1)}, anchor=north},
			%y label style={at={(axis description cs:-0.1,1)},anchor=south},
			grid = major,
			xlabel={Курс в дату экспирации},
			ylabel={Прибыль (profit'n'loss)},
			%scaled x ticks=false
		]
		
	  \addplot[Set1-A, very thick, dashed] {(\x > 70)*(\x - 70) - 0.5};
  \addplot[Set1-B, very thick, dashed] {-(\x > 74)*(\x - 74) + 0.25};
  \addplot[Set1-C, very thick] {(\x > 70)*(\x - 70) - 0.5 -(\x > 74)*(\x - 74) + 0.25};
  
  \draw[thick, color=black] (axis cs: 60, 0) -- (axis cs: 90, 0);
\end{axis}
\end{tikzpicture}
\end{frame}



\begin{frame}{Комбинации: стрэддл}
\justifying
Стрэддл (\en{straddle}) --- купленные колл и пут с одинаковыми страйками.

\justifying
\centering
	\begin{tikzpicture}
		\begin{axis}[
			domain=64:80,
			%axis lines=middle,
			xtick={64,66,...,80},
			ytick={-8,-7,...,6},
			xmin=64, xmax=80,
			ymin=-6, ymax=6,
			%x label style={at={(axis description cs: 0.5, -0.1)}, anchor=north},
			%y label style={at={(axis description cs:-0.1,1)},anchor=south},
			grid = major,
			xlabel={Курс в дату экспирации},
			ylabel={Прибыль (profit'n'loss)},
			%scaled x ticks=false
		]
		
	  \addplot[Set1-A, very thick, dashed] {(\x < 72)*(72 - \x) - 0.5};
  \addplot[Set1-B, very thick, dashed] {(\x > 72)*(\x - 72) - 0.5};
  \addplot[Set1-C, very thick] {(\x < 72)*(72 - \x) - 0.5 + (\x > 72)*(\x - 72) - 0.5};
  
  \draw[thick, color=black] (axis cs: 60, 0) -- (axis cs: 90, 0);
\end{axis}
\end{tikzpicture}
\end{frame}



\begin{frame}{Комбинации: колл-опцион и пут-опцион}
\justifying
Купленные колл и проданный пут с одинаковым страйком (без учёта премии).

\justifying
\centering
	\begin{tikzpicture}
		\begin{axis}[
			domain=64:80,
			%axis lines=middle,
			xtick={64,66,...,80},
			ytick={-6,-5,...,6},
			xmin=64, xmax=80,
			ymin=-6, ymax=6,
			%x label style={at={(axis description cs: 0.5, -0.1)}, anchor=north},
			%y label style={at={(axis description cs:-0.1,1)},anchor=south},
			grid = major,
			xlabel={Курс в дату экспирации},
			ylabel={Выплата (payoff)},
			%scaled x ticks=false
		]
		
	\addplot[Set1-A, very thick, dashed] {(\x > 72)*(\x - 72) + 0.1};
  	\addplot[Set1-B, very thick, dashed] {-(\x < 72)*(72-\x) - 0.1};
  	\addplot[Set1-C, very thick] {(\x > 72)*(\x - 72) -(\x < 72)*(72-\x) };
 
   \draw[thick, color=black] (axis cs: 60, 0) -- (axis cs: 90, 0);
\end{axis}
\end{tikzpicture}
\end{frame}



\begin{frame}{Комбинации: базовый актив и долг}
\justifying
1 доллар и долг в 72 рубля.

\justifying
\centering
	\begin{tikzpicture}
		\begin{axis}[
			domain=64:80,
			%axis lines=middle,
			xtick={64,66,...,80},
			ytick={-6,-5,...,6},
			xmin=64, xmax=80,
			ymin=-6, ymax=6,
			%x label style={at={(axis description cs: 0.5, -0.1)}, anchor=north},
			%y label style={at={(axis description cs:-0.1,1)},anchor=south},
			grid = major,
			xlabel={Курс в дату экспирации},
			ylabel={Выплата (payoff)},
			%scaled x ticks=false
		]
		
  	\addplot[Set1-C, very thick] {\x - 72};
 
   \draw[thick, color=black] (axis cs: 60, 0) -- (axis cs: 90, 0);
\end{axis}
\end{tikzpicture}
\end{frame}



\begin{frame}{Паритет опционов колл и пут}
\justify
Следующие два портфеля дают одинаковую выплату в день экспирации $T$:
\begin{itemize}
\justifying
\item Купленный колл со страйком $K$ и проданный пут со страйком $K$.
\item Базовый актив ценой $S(T)$ и долг в $K$ денег.
\end{itemize}

\justify
Следовательно, в любой день $t$ до даты экспирации ($t<T$) они тоже должны стоить одинаковых денег с учётом ставки дисконтирования $r$:
\begin{align*}
C_K(t) - P_K(t) = S(t) - Ke^{-r(T-t)}
\end{align*}

\justify
Это свойство называется \alert{паритет опционов колл и пут} (\en{call-put parity}). Оно позволяет выразить цену колла через пут или наоборот.
\end{frame}



\begin{frame}{Кто продаёт опционы?}
\justify
Опционы --- удобный инструмент и для хэджирования, и для спекуляции. Возможные потери ограничены снизу размером премии, а возможный выигрыш --- нет (для колл-опциона). Кто же тогда продаёт опционы, то есть соглашается на ограниченную прибыль в обмен на неограниченные убытки?
\end{frame}



\begin{frame}{Кто продаёт опционы?}
\justify
Несколько стратегий в среднем на дистанции давали положительную доходность.

\justify
\begin{itemize}
\justifying
\item Covered call: длинная позиция в индексе акций и проданный колл-опцион с высоким страйком.
\item Short straddle: проданный стрэддл с динамическим дельта-хэджированием (об этом позже).
\item Short OTM put: продажа опционов пут на индекс акций с низким страйком --- страховок от падения рынка.
\end{itemize}

\justify
Все три стратегии <<продают волатильность>> --- делают ставку на то, что в будущем рынок будет колебаться не так сильно, как ожидают покупатели опционов. Опционы похожи на страховку и стоят достаточно дорого, чтобы на дистанции продавцы страховок в среднем зарабатывали деньги.
\end{frame}



\newcommand{\drawStockNode}[5]{

	\node (#5)
	[
		draw,
		rectangle,
		rounded corners,
		inner sep = 0pt,
		outer sep = 0pt,
		minimum width = 2.4cm,
		minimum height = 0.55cm,
		align = center
	]
	at (#3, #4)
	{
		\begin{tabular}{c|c}
		#1 & #2
		\end{tabular}
	};
}

\newcommand{\drawStockLink}[4]{

	\draw[
		->,
		>=triangle 90
	]
	(#1.east) -- (#2.west)
	node[
		pos = 0.5,
		anchor = #4
	]
	{#3};
}

\newcommand{\drawOneStepBinomialTree}{
	\drawStockNode{\$100}{?}{0}{0}{S0_node}
	\drawStockNode{\$120}{\$20}{4}{ 1}{Su_node}
	\drawStockNode{\$80}{\$0}{4}{-1}{Sd_node}
	
	\drawStockLink{S0_node}{Su_node}{$90\%$}{south east}	
	\drawStockLink{S0_node}{Sd_node}{$10\%$}{north east}
}

\begin{frame}{Биномиальная модель}
\centering
\begin{tikzpicture}
	\drawOneStepBinomialTree
	\draw (S0_node.east) [dashed] -- (3, 0) node[anchor=west] {K=\$100};
\end{tikzpicture}

\justify
Текущая цена акции $\$100$. Завтра она может стоить либо $\$120$ с вероятностью $90\%$, либо $\$80$ с вероятностью $10\%$. Безрисковая процентная ставка $r=0\%$.

\justify
Сколько стоит европейский колл-опцион со страйком $K=\$100$, истекающий завтра?

\justify
Математическое ожидание:
$$\mathbb{E}(Payoff) = 0.9\cdot \$20 + 0.1\cdot \$0 = \$18 $$
\end{frame}



\begin{frame}{Биномиальная модель}
\centering
\begin{tikzpicture}
	\drawOneStepBinomialTree
	\draw (S0_node.east) [dashed] -- (3, 0) node[anchor=west] {K=\$100};
\end{tikzpicture}

\justify
Рассмотрим портфель $\Pi$, который состоит из $0.5$ акции и долга в $\$40$, который нужно отдать завтра.

\justify
Если акция будет стоить \$120, то портфель будет стоить
\begin{align*}\Pi = 0.5\cdot\$120 - \$40 = \$20\end{align*}

\justify
Если акция будет стоить \$80, то портфель будет стоить
$$\Pi = 0.5 \cdot \$80 - \$40 = \$0$$

\justify
Портфель $\Pi$ \alert{реплицирует} опцион!
\end{frame}



\begin{frame}{Биномиальная модель}
\centering
\begin{tikzpicture}
	\drawOneStepBinomialTree
	\draw (S0_node.east) [dashed] -- (3, 0) node[anchor=west] {K=\$100};
\end{tikzpicture}


\justify
Если портфель из долга и $0.5$ акции \alert{завтра} принесёт столько же денег, сколько и колл-опцион, то \alert{сегодня} они должны стоить одинаково.
\begin{align*}
C_{K=100} = \Pi_0 = 0.5 \cdot \$100 - \$40 = \$10
\end{align*}

Ответ не зависит от вероятностей 90\% и 10\%! Мы устранили всю неопределённость, связанную с будущей ценой акции.
\end{frame}



\begin{frame}{Биномиальная модель}
\centering
\begin{tikzpicture}
	\drawOneStepBinomialTree
	\draw (S0_node.east) [dashed] -- (3, 0) node[anchor=west] {K=\$100};
\end{tikzpicture}

\justify
Если опцион стоит \$18, то возможен арбитраж:
\begin{itemize}
\item Продать (выписать) опцион за \$18.
\item Взять в долг \$32.
\item Купить 0.5 акции за \$50.
\item Подождать до завтра.
\item Если акция стоит \$80, то продать 0.5 акции за \$40, выплатить долг \$32. Прибыль \$8.
\item Если акция стоит \$120, то купить еще 0.5 акции за \$60, продать держателю опциона за \$100, выплатить долг \$32. Прибыль \$8.
\end{itemize}
\end{frame}



\renewcommand{\drawOneStepBinomialTree}{
	\drawStockNode{$S_0$}{?}{0}{0}{S0_node}
	\drawStockNode{$S_0u$}{$V_u$}{4}{ 1}{Su_node}
	\drawStockNode{$S_0d$}{$V_d$}{4}{-1}{Sd_node}
	
	\drawStockLink{S0_node}{Su_node}{$p$}{south east}	
	\drawStockLink{S0_node}{Sd_node}{$1 - p$}{north east}
}

\begin{frame}{Биномиальная модель}
\centering
\begin{tikzpicture}
	\drawOneStepBinomialTree
\end{tikzpicture}

\justify
\begin{itemize}
\item Текущая цена акции $S_0$.
\item Цена акции может либо вырасти до $S_0\cdot u$ (u>1) либо снизиться до $S_0 \cdot d$ (d<1).
\item Длина одного периода $\tau$ лет, безрисковая  процентная ставка $r$, причём $d < 1+r\tau < u$.
\item В случае роста или падения акции опцион принесет (будет иметь value) либо $V_u$, либо $V_d$.
\end{itemize}
\end{frame}



\begin{frame}{Биномиальная модель}
\centering
\begin{tikzpicture}
	\drawOneStepBinomialTree
\end{tikzpicture}

\justify
Рассмотрим портфель, состоящий из $\Delta$ акций и долга $L$. 
\begin{equation*}
\begin{cases}
L(1+r\tau) + \Delta S_0 u = V_u \\
L(1+r\tau) + \Delta S_0 d = V_d
\end{cases}
\end{equation*}

\begin{equation*}
\begin{cases}
\Delta = \dfrac{V_u - V_d}{S_0(u-d)} \\
L = \dfrac{V_du - V_ud}{(1+r\tau)(u-d)}
\end{cases}
\end{equation*}
\end{frame}



\begin{frame}{Биномиальная модель}
\centering
\begin{tikzpicture}
\drawOneStepBinomialTree
\end{tikzpicture}

\justify
Цена опциона равна цене реплицирующего портфеля:
\begin{align*}
C &= \Delta S_0 +L = \\
 &= \dfrac{V_u-V_d}{(u-d)\cancel{S_0}}\cancel{S_0} + \dfrac{V_du -V_ud}{(1+r\tau)(u-d)} = \\
 &= \dfrac{qV_u +(1-q)V_d}{1+r\tau},
\end{align*}
где
\begin{equation*}
q = \dfrac{1+r\tau - d}{u-d} \text{ --- <<риск-нейтральная вероятность>>}
\end{equation*}
\end{frame}



\renewcommand{\drawStockLink}[2]{

	\draw[
		->,
		>=triangle 45
	]
	(#1.east) -- (#2.west)
	{};
}

\renewcommand{\drawStockNode}[5]{

	\node (#5)
	[
		draw,
		rectangle,
		rounded corners,
		inner sep = 1pt,
		outer sep = 0pt,
		minimum width = 1.5cm
	]
	at (#3, #4)
	{
		\centering
		\begin{tabular}{c}
		#1 \\ \hline #2
		\end{tabular}
	};
}

\newcommand{\nodeVerticalStep}{0.7}
\newcommand{\nodeHorizontalStep}{2.75}

\begin{frame}{Биномиальная модель}
\centering
\begin{tikzpicture}
\drawStockNode{$\$100$}{\only<1-7>{?}\only<8->{\$14.8}}{0}{0}{S0_node}

\drawStockNode{$\$120$}{\only<1-5>{?}\only<6->{\$25.8}}{\nodeHorizontalStep}{\nodeVerticalStep}{Su_node}
\drawStockNode{$\$80$}{\only<1-6>{?}\only<7->{\$3.8}}{\nodeHorizontalStep}{-\nodeVerticalStep}{Sd_node}

\drawStockNode{$\$144$}{\only<1-2>{?}\only<3->{\$44}}{2*\nodeHorizontalStep}{2*\nodeVerticalStep}{Suu_node}
\drawStockNode{$\$96$}{\only<1-3>{?}\only<4->{\$7.6}}{2*\nodeHorizontalStep}{0}{Sud_node}
\drawStockNode{$\$64$}{\only<1-4>{?}\only<5->{\$0}}{2*\nodeHorizontalStep}{-2*\nodeVerticalStep}{Sdd_node}

\drawStockNode{$\$172.8$}{\only<1>{?}\only<2->{\$72.8}}{3*\nodeHorizontalStep}{3*\nodeVerticalStep}{Suuu_node}
\drawStockNode{$\$115.2$}{\only<1>{?}\only<2->{\$15.2}}{3*\nodeHorizontalStep}{\nodeVerticalStep}{Suud_node}
\drawStockNode{$\$76.8$}{\only<1>{?}\only<2->{\$0}}{3*\nodeHorizontalStep}{-\nodeVerticalStep}{Sudd_node}
\drawStockNode{$\$51.2$}{\only<1>{?}\only<2->{\$0}}{3*\nodeHorizontalStep}{-3*\nodeVerticalStep}{Sddd_node}

\drawStockLink{S0_node}{Su_node}
\drawStockLink{S0_node}{Sd_node}

\drawStockLink{Su_node}{Suu_node}
\drawStockLink{Su_node}{Sud_node}

\drawStockLink{Sd_node}{Sud_node}
\drawStockLink{Sd_node}{Sdd_node}

\drawStockLink{Suu_node}{Suuu_node}
\drawStockLink{Suu_node}{Suud_node}

\drawStockLink{Sud_node}{Suud_node}
\drawStockLink{Sud_node}{Sudd_node}

\drawStockLink{Sdd_node}{Sudd_node}
\drawStockLink{Sdd_node}{Sddd_node}
\end{tikzpicture}

\justify
Предположим, что $u=1.2$, $d=0.8$, $S_0=\$100$, $r=0\%$. Сколько стоит колл со страйком $K=100$?

\justify
<<Риск-нейтральная вероятность>>:
\begin{align*}
q = \dfrac{1+r\tau - d}{u - d} = \dfrac{1 - 0.8}{1.2 - 0.8} = 0.5
\end{align*}
\end{frame}



\begin{frame}{Биномиальная модель}
\centering
\begin{tikzpicture}
\drawStockNode{$\$100$}{$\Delta=0.58$}{0}{0}{S0_node}

\drawStockNode{$\$120$}{$\Delta=0.76$}{\nodeHorizontalStep}{\nodeVerticalStep}{Su_node}
\drawStockNode{$\$80$}{$\Delta=0.24$}{\nodeHorizontalStep}{-\nodeVerticalStep}{Sd_node}

\drawStockNode{$\$144$}{$\Delta=1.0$}{2*\nodeHorizontalStep}{2*\nodeVerticalStep}{Suu_node}
\drawStockNode{$\$96$}{$\Delta=0.4$}{2*\nodeHorizontalStep}{0}{Sud_node}
\drawStockNode{$\$64$}{$\Delta=0.0$}{2*\nodeHorizontalStep}{-2*\nodeVerticalStep}{Sdd_node}

\drawStockNode{$\$172.8$}{$\Delta=1$}{3*\nodeHorizontalStep}{3*\nodeVerticalStep}{Suuu_node}
\drawStockNode{$\$115.2$}{$\Delta=1$}{3*\nodeHorizontalStep}{\nodeVerticalStep}{Suud_node}
\drawStockNode{$\$76.8$}{$\Delta=0$}{3*\nodeHorizontalStep}{-\nodeVerticalStep}{Sudd_node}
\drawStockNode{$\$51.2$}{$\Delta=0$}{3*\nodeHorizontalStep}{-3*\nodeVerticalStep}{Sddd_node}

\drawStockLink{S0_node}{Su_node}
\drawStockLink{S0_node}{Sd_node}

\drawStockLink{Su_node}{Suu_node}
\drawStockLink{Su_node}{Sud_node}

\drawStockLink{Sd_node}{Sud_node}
\drawStockLink{Sd_node}{Sdd_node}

\drawStockLink{Suu_node}{Suuu_node}
\drawStockLink{Suu_node}{Suud_node}

\drawStockLink{Sud_node}{Suud_node}
\drawStockLink{Sud_node}{Sudd_node}

\drawStockLink{Sdd_node}{Sudd_node}
\drawStockLink{Sdd_node}{Sddd_node}
\end{tikzpicture}

\justify
Справедливая безарбитражная цена опциона не зависит от вероятностей изменения цены вверх и вниз. Если мы можем \alert{динамически} ребалансировать портфель, состоящий из акций и долга, то мы можем реплицировать любой опцион.

\justify
Эта стратегия называется \alert{дельта-хэджирование} (\en{delta hedging}).
\end{frame}



\newcommand{\highlightStockLink}[6]{
	\draw[
		color=#4,
		very thick,
		->,
		>=triangle 45
	]
	(#1.east) -- (#2.west)
	node[
		pos=#5,
		anchor=#6
	]
	{#3};
}

\newcommand{\highlightStockLinkUp}[3]{
	\highlightStockLink{#1}{#2}{$q$}{#3}{0.5}{south}
}

\newcommand{\highlightStockLinkDown}[3]{
	\highlightStockLink{#1}{#2}{$1-q$}{#3}{0.15}{west}
}

\begin{frame}{Биномиальная модель}
\centering
\begin{tikzpicture}
\drawStockNode{$S_0$}{?}{0}{0}{S0_node}

\drawStockNode{$S_0u$}{?}{\nodeHorizontalStep}{\nodeVerticalStep}{Su_node}
\drawStockNode{$S_0d$}{?}{\nodeHorizontalStep}{-\nodeVerticalStep}{Sd_node}

\drawStockNode{$S_0u^2$}{?}{2*\nodeHorizontalStep}{2*\nodeVerticalStep}{Suu_node}
\drawStockNode{$S_0ud$}{?}{2*\nodeHorizontalStep}{0}{Sud_node}
\drawStockNode{$S_0d^2$}{?}{2*\nodeHorizontalStep}{-2*\nodeVerticalStep}{Sdd_node}

\drawStockNode{$S_0u^3$}{$V_3$}{3*\nodeHorizontalStep}{3*\nodeVerticalStep}{Suuu_node}
\drawStockNode{$S_0u^2d$}{$V_2$}{3*\nodeHorizontalStep}{\nodeVerticalStep}{Suud_node}
\drawStockNode{$S_0ud^2$}{$V_1$}{3*\nodeHorizontalStep}{-\nodeVerticalStep}{Sudd_node}
\drawStockNode{$S_0d^3$}{$V_0$}{3*\nodeHorizontalStep}{-3*\nodeVerticalStep}{Sddd_node}

\only<1-2>{
	\drawStockLink{S0_node}{Su_node}
	\drawStockLink{S0_node}{Sd_node}

	\drawStockLink{Su_node}{Suu_node}
	\drawStockLink{Su_node}{Sud_node}

	\drawStockLink{Sd_node}{Sud_node}
	\drawStockLink{Sd_node}{Sdd_node}

	\drawStockLink{Suu_node}{Suuu_node}
	\drawStockLink{Suu_node}{Suud_node}

	\drawStockLink{Sud_node}{Suud_node}
	\drawStockLink{Sud_node}{Sudd_node}

	\drawStockLink{Sdd_node}{Sudd_node}
	\drawStockLink{Sdd_node}{Sddd_node}
}

\only<3>{
	\highlightStockLinkUp{S0_node}{Su_node}{Set1-A}
	\highlightStockLinkUp{Su_node}{Suu_node}{Set1-A}
	\highlightStockLinkDown{Suu_node}{Suud_node}{Set1-A}
}

\only<4>{
	\highlightStockLinkUp{S0_node}{Su_node}{Set1-A}
	\highlightStockLinkDown{Su_node}{Sud_node}{Set1-A}
	\highlightStockLinkUp{Sud_node}{Suud_node}{Set1-A}
}

\only<5>{
	\highlightStockLinkDown{S0_node}{Sd_node}{Set1-A}
	\highlightStockLinkUp{Sd_node}{Sud_node}{Set1-A}
	\highlightStockLinkUp{Sud_node}{Suud_node}{Set1-A}
}

\end{tikzpicture}

\justify
Риск-нейтральная вероятность: $q = \dfrac{1 + rT - d}{u - d}$.

\justify
Цена дериватива сегодня:
\begin{align*}
V = \frac{q^3V_3 + \only<1>{3q^2(1-q)}\only<2->{\alert{3q^2(1-q)}}V_2 + 3q(1-q)^2V_1 + (1-q)^3V_0}{(1+rT)^3}
\end{align*}
\end{frame}



\begin{frame}{Риск-нейтральная вероятность}
\justify
Если представить, что $q$ --- вероятность движения акции вверх, то $3q^2(1-q)$ --- вероятность того, что акция дважды вырастет и один раз упадёт (не важно, в каком порядке). Тогда акция будет стоить $S_0u^2d$, а дериватив принесёт прибыль $V_2$.

\justify
\centering
\begin{tabular}{l|l|l}
Цена акции & Прибыль & <<Вероятность>> \\ \hline
$S_0u^3$   & $V_3$   & $q^3$ \\
$S_0u^2d$  & $V_2$   & $3q^2(1-q)$ \\
$S_0ud^2$  & $V_1$   & $3q(1-q)^2$ \\ 
$S_0d^3$   & $V_0$   & $(1-q)^3$ 
\end{tabular}

\justify
Цена дериватива похожа на дисконтированное <<математическое ожидание>> прибыли.
\begin{align*}
V = \frac{q^3V_3 + 3q^2(1-q)V_2 + 3q(1-q)^2V_1 + (1-q)^3V_0}{(1+rT)^3}
\end{align*}
\end{frame}



\begin{frame}{Биномиальная модель}
Для дерева, состоящего из $n$ шагов:
\begin{align*}
C &= \dfrac{\sum\limits_{k=0}^{n} C^k_nq^k(1-q)^{n-k}V(S_0u^kd^{n-k})}{(1+r\tau)^n} \\
C^k_n &= \dfrac{n!}{k!(n-k)!}
\end{align*}

\justify
Что будет, если вместо произвольной функции $V(S)$ взять функцию $max(S-K,0)$, как в колл-опционе, и устремить $n$ к бесконечности? Получится формула Блэка-Шоулза.

\vspace{\baselineskip}
Строгое доказательство с помощью закона больших чисел и центральной
предельной теоремы здесь (28 страниц):

\url{http://www.math.cmu.edu/~handron/21_370/BS.pdf}
\end{frame}



\begin{frame}{Демонстрация: доска Гальтона}

\url{https://www.mathsisfun.com/data/quincunx.html}
\end{frame}



\begin{frame}{Геометрическое броуновское движение}
\justify
Предположим, что до экспирации опциона осталось $T$ лет, и мы построили биномиальное дерево из $N$ шагов. Длина каждого шага $\Delta t = T/N$ лет.

\justify
На каждом самом маленьком шаге цена акции растёт на $\mu$ процентов годовых в среднем, а также колеблется вокруг этого среднего со стандартным отклонением (волатильностью) $\sigma$.
\begin{align*}
\frac{S_{i+1} - S_i}{S_i} = \mu \Delta t + \sigma \sqrt{\Delta t}\xi_i, \quad \xi_i \sim \mathcal{N}(0, 1)
\end{align*}

\justify
Это геометрическое броуновское движение. Такой процесс можно получить, если в биномиальном дереве положить
\begin{align*}
u &= 1 + \mu \Delta t + \sigma\sqrt{\Delta t} \\
d &= 1 + \mu \Delta t - \sigma\sqrt{\Delta t}
\end{align*}
\end{frame}



\newcommand{\plotBrownianMotion}[2] {
	
	\addplot[
		color = #2,
		mark = none,
		thick
	]
	table[
		x=t,
		y=s,
		col sep=comma
	]
	{#1};
	
	\addplot[
		color = #2,
		mark = none,
		thick,
		dashed,
		forget plot
	]
	table[
		x=t,
		y=trend,
		col sep=comma
	]
	{#1};
}



\begin{frame}{Геометрическое броуновское движение}
\centering
\begin{tikzpicture}
\begin{axis}[
  width=\textwidth,
  height=\textheight - 1cm,
  xlabel near ticks,
  ylabel near ticks,
  xmin=0, xmax=1,
  legend entries = {
  	   {$\mu=5\%, \sigma=10\%, N=100$},
      {$\mu=10\%, \sigma=40\%, N=500$}
  },
  legend cell align={left},
  legend style={at={(0.97,0.03)},anchor=south east}
]

	\plotBrownianMotion{gbm_sample_100.csv}{Set1-A}
	\plotBrownianMotion{gbm_sample_500.csv}{Set1-B}
\end{axis}
\end{tikzpicture}
\end{frame}



\begin{frame}{Геометрическое броуновское движение}
\justify
Если геометрическое броуновское движение начинается с цены базового актива $S_0$ с трендом $\mu$ и волатильностью $\sigma$, то цена базового актива через время $T$ $S_T$ есть случайная величина:
\begin{align*}
S_T = S_0\cdot \exp\left((\mu - \sigma^2/2)T + \sigma\sqrt{T}\xi\right) \quad \xi \sim \mathcal{N}(0, 1)
\end{align*} 

\justify
Логарифм изменения цены (грубо: процентный прирост) есть случайная величина, которая следует нормальному распределению:
\begin{align*}
\ln\left(\frac{S_T}{S_0}\right) = \left(\mu - \frac{\sigma^2}{2}\right)T + \sigma\sqrt{T}\xi \quad \xi \sim \mathcal{N}(0, 1)
\end{align*}

\justify
Поэтому говорят, что цена базового актива следует лог-нормальному распределению.
\end{frame}



\begin{frame}{Геометрическое броуновское движение}
\centering
\begin{tikzpicture}
	\begin{axis}[
		width = \textwidth,
		height = \textheight - 1cm,
		xmin = -5, xmax = 5,
		ymin = 0, ymax = 0.7,
		grid = major,
		legend entries = {Нормальное, Лог-нормальное},
		legend pos = north west
	]
		
		\addplot[color=Set1-A, thick] table[x=x, y=norm_density, col sep=comma] {norm_and_lognorm_density.csv};
		
				\addplot[color=Set1-B, thick] table[x=x, y=lognorm_density, col sep=comma] {norm_and_lognorm_density.csv};
	\end{axis}
\end{tikzpicture}
\end{frame}



\begin{frame}{Модель Блэка-Шоулза}
\justify
Рассмотрим следующую модель:
\begin{itemize}
\item Базовый актив не платит дивидендов, его цена $S_0$.
\item Европейский колл-опцион со сроком экспирации $T$ лет и страйком $K$.
\item Безрисковая процентная ставка $r$ постоянна.
\item Если биномиальное дерево состоит из $N$ шагов, то на каждом шаге цена акции умножается либо на $u = \mu T/N + \sigma\sqrt{T/N}$ либо на $d=\mu T/N - \sigma\sqrt{T/N}$, то есть цена следует геометрическому броуновскому движению. Здесь $\sigma=const$ --- волатильность, $\mu$ --- тренд.
\item Идеальный ликвидный рынок без транзакционных издержек и без возможностей для арбитража.
\end{itemize}
\end{frame}



\begin{frame}{Модель Блэка-Шоулза}
Цена европейского колл-опциона задается формулой:
\begin{align*}
C_{BS} &= S_0N(d_1) - Ke^{-rT}N(d_2)
\end{align*}
где
\begin{align*}
d_1 &= \dfrac{1}{\sigma\sqrt{T}}\left( \ln\left(\dfrac{S_0}{K}\right) + \left(r + \dfrac{\sigma^2}{2}\right)T\right) \\
d_2 &= \dfrac{1}{\sigma\sqrt{T}}\left( \ln\left(\dfrac{S_0}{K}\right) + \left(r - \dfrac{\sigma^2}{2}\right)T\right) \\
N(x) &= \dfrac{1}{\sqrt{2\pi}}\int\limits_{-\infty}^x e^{-\frac{t^2}{2}}dt
\end{align*}

\justify
Цена не зависит от тренда $\mu$!
\end{frame}



\begin{frame}{Модель Блэка-Шоулза}
\justify
Если базовый актив имеет дивидендную доходность $q$  и дивиденды выплачиваются непрерывно (валюта, индекс акций, нефть в хранилище):
\begin{align*}
C_{BS} &= S_0e^{-qT}N(d_1) - Ke^{-rT}N(d_2)
\end{align*}
где
\begin{align*}
d_1 &= \dfrac{1}{\sigma\sqrt{T}}\left( \ln\left(\dfrac{S_0}{K}\right) + \left(r -q + \dfrac{\sigma^2}{2}\right)T\right) \\
d_2 &= \dfrac{1}{\sigma\sqrt{T}}\left( \ln\left(\dfrac{S_0}{K}\right) + \left(r -q- \dfrac{\sigma^2}{2}\right)T\right)
\end{align*}
\end{frame}



\begin{frame}{Модель Блэка-Шоулза}
\justify
Пример: колл-опцион со страйком $K=72$, сроком $T=0.25$ лет, $r=6\%$, $q=0\%$.

\centering
\begin{tikzpicture}
\begin{axis}[
			domain=64:80,
			xtick={64,66,...,80},
			ytick={0,1,2,...,10},
			xmin=64, xmax=80,
			ymin=0, ymax=10,
			grid = major,
			xlabel={Курс сегодня ($S_0$)},
			ylabel={Цена опциона},
  legend entries = {
  	   $\sigma=25\%$,
  	   $\sigma=10\%$,
  	   $\sigma=5\%$,
  	   $\sigma=0\%$
  },
  legend cell align={left},
  legend style={at={(0.03,0.97)},anchor=north west}
]

	\addplot[color = Set1-A, mark = none, thick]
	table[
		x=S,
		y=C_25,
		col sep=comma
	]
	{call_price.csv};
	
	\addplot[color = Set1-B, mark = none, thick]
	table[
		x=S,
		y=C_10,
		col sep=comma
	]
	{call_price.csv};
	
	\addplot[color = Set1-C, mark = none, thick]
	table[
		x=S,
		y=C_5,
		col sep=comma
	]
	{call_price.csv};
	
	\addplot[Set1-D, very thick, dashed] {(\x >= 72)*(\x - 72) + 0.05};
\end{axis}
\end{tikzpicture}
\end{frame}



\begin{frame}{Модель Блэка-Шоулза}
\justify
Величину $\max(S - K, 0)$ называют внутренней стоимостью колл-опциона (\en{intrinsic value}). Столько денег мы бы заработали на опционе, если бы могли исполнить его прямо сейчас.

\justify
Если $C$ --- цена колл-опциона, то величина $C - \max(S - K, 0)$ называется временной стоимостью (\en{time value}). Она показывает, насколько опцион дороже, чем простая внутренняя стоимость. Чем выше волатильность, и чем больше времени осталось до экспирации, тем выше временная стоимость.
\end{frame}


\begin{frame}{Модель Блэка-Щоулза}
\begin{align*}
C_{BS} &= S_0N(d_1) - Ke^{-rT}N(d_2)
\end{align*}

\justify
$S_0N(d_1)$ --- цена опциона <<актив или ничего>> (\en{asset-or-nothing call}). Этот опцион выплачивает 1 единицу базового актива, если цена выше страйка.

\justify
$e^{-rT}N(d_2)$ --- цена опциона <<деньги или ничего>> (\en{cash-or-nothing call}). Этот опцион выплачивает 1 единицу денег, если цена актива выше страйка.
\end{frame}



\begin{frame}{Приближённое вычисление: ATMF-колл}
\justify
Рассмотрим \en{at-the-money-forward (ATMF)}\ колл-опцион со страйком $K = F = S_0e^{(r-q)T}$.

\begin{align*}
d_1 &= \dfrac{1}{\sigma\sqrt{T}}\left( \ln\left(\dfrac{S_0}{S_0e^{(r-q)T}}\right) + \left(r -q + \dfrac{\sigma^2}{2}\right)T\right) = \frac{\sigma \sqrt{T}}{2} \\
d_2 &= \dfrac{1}{\sigma\sqrt{T}}\left( \ln\left(\dfrac{S_0}{S_0e^{(r-q)T}}\right) + \left(r -q- \dfrac{\sigma^2}{2}\right)T\right) = -\frac{\sigma \sqrt{T}}{2}
\end{align*}

Тогда:
\begin{align*}
C_{atmf} &= S_0e^{-qT}N(d_1) - S_0e^{(r-q)T}e^{-rT}N(d_2) = \\
&= S_0e^{-qT}\left[
N\left(\frac{\sigma \sqrt{T}}{2}\right) - N\left(-\frac{\sigma \sqrt{T}}{2}\right)
\right]
\end{align*}
\end{frame}



\begin{frame}{Приближённое вычисление: ATMF-колл}
\justify
Разложим $N(x)$ в ряд Тейлора:
\begin{align*}
N(x) \approx \frac{x}{\sqrt{2\pi}} \quad \Rightarrow
N(x) - N(-x) \approx \frac{2x}{\sqrt{2\pi}}
\end{align*}

Тогда:
\begin{align*}
N\left(\frac{\sigma \sqrt{T}}{2}\right) - N\left(-\frac{\sigma \sqrt{T}}{2}\right)
\approx
\frac{\sigma\sqrt{T}}{\sqrt{2\pi}}
\approx
0.4\sigma\sqrt{T}
\end{align*}

Если также предположить, что $q\approx 0\%$ и $e^{-qT} \approx 1$, то
\begin{align*}
C_{atmf} \approx 0.4 S_0 \sigma \sqrt{T}
\end{align*}
\end{frame}



\begin{frame}{Приближённое вычисление: ATMF-колл}
\justify
Пример: спот-курс EURRUB равен $S=80$. Форвард на три месяца стоит $F=81$. Волатильность равна $\sigma=10\%$. Сколько стоит ATMF-колл со страйком $K=81$?

\begin{align*}
C_{atmf} \approx 0.4 \cdot S_0 \cdot \sigma \cdot \sqrt{T} = \\
0.4 \cdot 80 \cdot 0.1 \cdot \sqrt{\dfrac{1}{4}} \approx 1.6
\end{align*}

Единицы измерения --- рубли за право купить 1 евро.
\end{frame}

\end{document}