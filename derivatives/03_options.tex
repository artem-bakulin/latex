\documentclass{beamer}

\usepackage{cmap}				% To be able to copy-paste russian text from pdf
\usepackage[T2A]{fontenc}
\usepackage[utf8]{inputenc}
\usepackage[russian]{babel}
\usepackage{textpos}
\usepackage{ragged2e}
\usepackage{amssymb}
\usepackage{ulem}
\usepackage{tikz}
\usepackage{pgfplots}
\usepackage{color}
\usepackage{cancel}
\usepackage{multirow}
\pgfplotsset{compat=1.17}
\usetikzlibrary{arrows,snakes,backgrounds,shapes}
\usepgfplotslibrary{groupplots,colorbrewer,dateplot,statistics}
\usepackage{animate}

\usepackage{amsfonts}
\usepackage{amsmath}
\usepackage{amssymb}
\usepackage{graphicx}
\usepackage{setspace}

\usepackage{enumitem}
\setitemize{label=\usebeamerfont*{itemize item}%
  \usebeamercolor[fg]{itemize item}
  \usebeamertemplate{itemize item}}

% remove navigation bar
\setbeamertemplate{navigation symbols}{} 

\usepackage{eurosym}
\renewcommand{\EUR}[1]{\textup{\euro}#1}

\title{Лекция 3. Опционы}
\author{Артём Бакулин}
\date{21 октября 2021 г.}

\usetheme{Warsaw}
\usecolortheme{beaver}

\newcommand{\ru}[1]{\begin{otherlanguage}{russian}#1\end{otherlanguage}}
\newcommand{\en}[1]{\begin{otherlanguage}{english}#1\end{otherlanguage}}
\newcommand{\ruen}[2]{#1 (\en{#2})}

\begin{document}



\begin{frame}
\titlepage
\end{frame}



\begin{frame}{Ванильные опционы}
\justifying
\alert{Опцион колл (пут)} --- это контракт, который даёт владельцу \textit{право}, но не обязанность, купить (продать) базовый актив по заранее оговоренной цене в заранее определённую дату. 

\justify
Пример: колл-опцион, который даёт право (но не обязанность) купить 1\,000 долларов за 72\,000 рублей рублей через 3 месяца. Сегодня за такой опцион нужно заплатить 500 рублей.
\end{frame}



\begin{frame}{Ванильные опционы}
\begin{itemize}
\justifying
\item Колл (call) --- право купить актив, пут (put) --- право продать.
\item Страйк (strike) --- фиксированная цена, по которой можно будет купить или продать базовый актив.
\item Премия (premium) --- количество денег, которые нужно заплатить сегодня, чтобы получить опцион.
\item Выписать опцион (write an option) --- то же самое, что продать опцион.
\item Дата экспирации (expiration date) --- дата, в которую опцион прекращает действовать, если владелец его не использовал.
\item Европейский опцион (\en{European option}) --- воспользоваться правом можно только в дату экспирации.
\item Американский опцион (\en{American option}) --- воспользоваться правом можно в любой день до экспирации.
\end{itemize}
\end{frame}



\newcommand{\circlewithtext}[3]{
    \node[circle, fill, inner sep = 1.5pt] at (#1, #2) {};
    \node[anchor = north] at (#1, #2) {#3};
}

\begin{frame}{Ванильные опционы}
\justify
\centering
\begin{tikzpicture}
		\draw [->,>=triangle 90] (0, 0) -- (9.5, 0);
		
		\draw [->, >=triangle 45] (0.5, 0) .. controls (1.0, 1) and (2.0, 1) .. (2.5, 0) node[pos=0.5, anchor=south]{+1 день};
		
		\draw [->, >=triangle 45] (2.5, 0) .. controls (3, 2) and (8, 2) .. (8.5, 0) node[pos=0.5, anchor=north]{+3 месяца};
		
		\draw [->, >=triangle 45] (8.5, 0) .. controls (8, 0.5) and (7, 0.5) .. (6.5, 0) node[pos=0.5, anchor=south]{-1 день};
		
		\circlewithtext{0.5}{0}{Сегодня};
		\circlewithtext{2.5}{0}{\begin{tabular}{c}Спот \\ $-500$ р. \end{tabular}};
		\circlewithtext{6.5}{0}{Экспирация};
		\circlewithtext{8.5}{0}{\begin{tabular}{c}Поставка \\ $+\$1\,000$ \\ $-72\,000$ р.\end{tabular}};
	\end{tikzpicture}
	
\justify
\begin{itemize}
\justifying
\item Сегодня --- дата заключения договора.
\item Спот-дата (\en{spot date}) --- дата выплаты премии.
\item Экспирация (\en{expiration date}) --- дата, в которую покупатель опциона принимает решение об исполнении опциона.
\item Дата поставки (\en{settlement date}) --- дата, в которую произойдёт обмен валютой, если опцион будет исполнен.
\end{itemize}

\justify
Дальше в курсе мы будем считать, что премия платится в день сделки, а даты экспирации и поставки совпадают.
\end{frame}



\begin{frame}{Ванильные опционы}
\justify
Европейские опционы колл и пут называют <<\alert{ванильными}>> (\en{plain vanilla}), потому что самое простое мороженое без ничего --- ванильное.

\justify
<<Первый канал>> об убытках банка \en{Societe General}, 25.01.2008:

\justify
<<Источник агентства Reuters в самом банке на условиях анонимности сообщил, что в последнее время трейдер, о котором идет речь, покупал и продавал контракты на поставку обычной ванили. \alert{Той самой, которую используют в кулинарии}. Правда, ванили на 7 миллиардов долларов, в год не производят и во всем мире. Фактически речь идет о торговле воздухом. Хотя это вполне законно и используется повсеместно.>>

\justify
\url{https://www.newsru.com/russia/25Jan2008/vanilla.html}
\end{frame}



\begin{frame}{Пример: колл-опцион}
\justifying
Колл-опцион на покупку долларов за рубли со страйком 72.

\justifying
\centering
	\begin{tikzpicture}
		\begin{axis}[
			domain=64:80,
			%axis lines=middle,
			xtick={64,66,...,80},
			ytick={0,1,2,...,8},
			xmin=64, xmax=80,
			ymin=-1, ymax=8,
			grid = major,
			xlabel={Курс в дату экспирации},
			ylabel={Выплата (payoff)},
		]
  \addplot[Set1-A, very thick] {(\x > 72)*(\x - 72) + 0.05};
  
  \draw[thick, color=black] (axis cs: 60, 0) -- (axis cs: 90, 0);
\end{axis}
\end{tikzpicture}
\end{frame}



\begin{frame}{Пример: покупаем колл-опцион}
\justifying
Покупка колл-опциона со страйком 72 за премию 0.5 руб.

\justifying
\centering
	\begin{tikzpicture}
		\begin{axis}[
			domain=64:80,
			%axis lines=middle,
			xtick={64,66,...,80},
			ytick={0,1,2,...,8},
			xmin=64, xmax=80,
			ymin=-1, ymax=8,
			%x label style={at={(axis description cs: 0.5, -0.1)}, anchor=north},
			%y label style={at={(axis description cs:-0.1,1)},anchor=south},
			grid = major,
			xlabel={Курс в дату экспирации},
			ylabel={Прибыль (profit'n'loss)},
			%scaled x ticks=false
		]
  \addplot[Set1-A, very thick] {(\x > 72)*(\x - 72) - 0.5};
  
  \draw[thick, color=black] (axis cs: 60, 0) -- (axis cs: 90, 0);
\end{axis}
\end{tikzpicture}
\end{frame}



\begin{frame}{Пример: покупаем пут-опцион}
\justifying
Покупка пут-опциона со страйком 72 за премию 0.5 руб.

\justifying
\centering
	\begin{tikzpicture}
		\begin{axis}[
			domain=64:80,
			%axis lines=middle,
			xtick={64,66,...,80},
			ytick={0,1,2,...,8},
			xmin=64, xmax=80,
			ymin=-1, ymax=8,
			%x label style={at={(axis description cs: 0.5, -0.1)}, anchor=north},
			%y label style={at={(axis description cs:-0.1,1)},anchor=south},
			grid = major,
			xlabel={Курс в дату экспирации},
			ylabel={Прибыль (profit'n'loss)},
			%scaled x ticks=false
		]
  \addplot[Set1-A, very thick] {(\x < 72)*(72 - \x) - 0.5};
  
  \draw[thick, color=black] (axis cs: 60, 0) -- (axis cs: 90, 0);
\end{axis}
\end{tikzpicture}
\end{frame}



\begin{frame}{Пример: продаём колл-опцион}
\justifying
Продажа колл-опциона со страйком 72 за премию 0.5 руб.

\justifying
\centering
	\begin{tikzpicture}
		\begin{axis}[
			domain=64:80,
			%axis lines=middle,
			xtick={64,66,...,80},
			ytick={-8,-7,...,0,1},
			xmin=64, xmax=80,
			ymin=-8, ymax=1,
			%x label style={at={(axis description cs: 0.5, -0.1)}, anchor=north},
			%y label style={at={(axis description cs:-0.1,1)},anchor=south},
			grid = major,
			xlabel={Курс в дату экспирации},
			ylabel={Прибыль (profit'n'loss)},
			%scaled x ticks=false
		]
  \addplot[Set1-A, very thick] {(\x > 72)*(72 - \x) + 0.5};
  
  \draw[thick, color=black] (axis cs: 60, 0) -- (axis cs: 90, 0);
\end{axis}
\end{tikzpicture}
\end{frame}



\begin{frame}{Пример: продаём пут-опцион}
\justifying
Продажа пут-опциона со страйком 72 за премию 0.5 руб.

\justifying
\centering
	\begin{tikzpicture}
		\begin{axis}[
			domain=64:80,
			%axis lines=middle,
			xtick={64,66,...,80},
			ytick={-8,-7,...,0,1},
			xmin=64, xmax=80,
			ymin=-8, ymax=1,
			%x label style={at={(axis description cs: 0.5, -0.1)}, anchor=north},
			%y label style={at={(axis description cs:-0.1,1)},anchor=south},
			grid = major,
			xlabel={Курс в дату экспирации},
			ylabel={Прибыль (profit'n'loss)},
			%scaled x ticks=false
		]
  \addplot[Set1-A, very thick] {(\x < 72)*(\x - 72) + 0.5};
  
  \draw[thick, color=black] (axis cs: 60, 0) -- (axis cs: 90, 0);
\end{axis}
\end{tikzpicture}
\end{frame}



\begin{frame}{Хэджирование с помощью опционов}
\justify
Через три месяца нам понадобится \$1\,000 на отпуск. Форвардный курс на три месяца 72.0. Если купить форвард, мы зафиксируем курс, но что делать, если рубль укрепится до 50? Будет обидно покупать доллары по 72 при рыночном курсе 50.

\justify
Решение: можно купить колл-опцион со страйком 72 на \$1\,000 за премию 500 рублей. Если доллар еще вырастет, мы исполняем опцион и покупаем по 72. Если рубль внезапно укрепляется до 50, опцион пропадает, а мы покупаем по рыночному курсу 50.

\justify
Плюс: не так обидно при укреплении рубля. Минус: прямо сейчас нужно заплатить за опцион. Опцион работает как страховка от падения рубля.
\end{frame}




\begin{frame}{Хэджирование с помощью опционов --- 2}
\justify
Через три месяца нам придут дивиденды \$1\,000 от иностранных инвестиций. Тогда же нам нужно будет заплатить рубли за ипотеку. Форвардный курс на три месяца 72.0. Если продать форвард, мы зафиксируем курс, но что делать, если рубль ослабнет до 90? Будет обидно продавать доллары по 72 при рыночном курсе 90.

\justify
Решение: можно купить пут-опцион со страйком 72 на \$1\,000 за премию 500 рублей. Если доллар упадёт до 50, мы исполняем опцион и продаём доллары по 72. Если доллар укрепится до 90, опцион пропадает, а мы продадим доллары по рыночному курсу 90.

\justify
Плюс: не так обидно при укреплении доллара. Минус: прямо сейчас нужно заплатить за опцион. Опцион работает как страховка от укрепления рубля.
\end{frame}



\begin{frame}{Спекуляция с помощью опционов}
\justify
Хрустальный шар показал, что за три месяца доллар сильно укрепится относительного текущего форвардного курса 72. Мы хотим на этом заработать, но \sout{жена} акционеры нашего хэдж-фонда выделили нам всего 500 рублей.

\justify
Решение: можно купить колл-опцион со страйком 72 на \$1\,000 за премию 500 рублей. Если доллар укрепится до 80 рублей, мы заработаем $(80 - 72) \cdot 1\,000 - 500 = 7\,500$ рублей. Если этого не случится, мы просто потеряем нашу ставку.

\justify
Опцион позволяет взять <<плечо>> --- увеличить выигрыш при хорошем исходе (как будто у нас есть целая \$1,000) ценой потерь в худшем исходе (можно потерять всю премию). Однако, проигрыш строго ограничен снизу, а выигрыш может быть бесконечно большим.
\end{frame}



\begin{frame}{Спекуляция с помощью опционов --- 2}
\justify
Хрустальный шар показал, что за три месяца доллар сильно упадёт относительного текущего форвардного курса 72. Мы хотим на этом заработать, но \sout{жена} акционеры нашего хэдж-фонда выделили нам всего 500 рублей.

\justify
Решение: можно купить пут-опцион со страйком 72 на \$1\,000 за премию 500 рублей. Если доллар упадёт до 64 рублей, мы заработаем $(72 - 64) \cdot 1\,000 - 500 = 7\,500$ рублей. Если этого не случится, мы просто потеряем нашу ставку.

\justify
Опцион пут позволяет сделать ставку на падение базового актива (доллара).
\end{frame}



\begin{frame}{Комбинации: коллар}
\justifying
Коллар (\en{collar}) --- проданный пут с меньшим страйком и купленный колл с большим страйком.

\justifying
\centering
	\begin{tikzpicture}
		\begin{axis}[
			domain=64:80,
			%axis lines=middle,
			xtick={64,66,...,80},
			ytick={-8,-7,...,6},
			xmin=64, xmax=80,
			ymin=-6, ymax=6,
			%x label style={at={(axis description cs: 0.5, -0.1)}, anchor=north},
			%y label style={at={(axis description cs:-0.1,1)},anchor=south},
			grid = major,
			xlabel={Курс в дату экспирации},
			ylabel={Прибыль (profit'n'loss)},
			%scaled x ticks=false
		]
	
  \addplot[Set1-A, very thick, dashed] {-(\x < 70)*(70 - \x) + 0.5};
  \addplot[Set1-B, very thick, dashed] {(\x > 74)*(\x - 74) - 0.5};
  \addplot[Set1-C, very thick] {-(\x < 70)*(70 - \x) + 0.5 + (\x > 74)*(\x - 74) - 0.5 + 0.05};	

  \draw[thick, color=black] (axis cs: 60, 0) -- (axis cs: 90, 0);
\end{axis}
\end{tikzpicture}
\end{frame}



\begin{frame}{Комбинации: колл-спред}
\justifying
Колл-спред (\en{call spread}) --- купленный колл с меньшим страйком, проданный колл с большим страйком.

\justifying
\centering
	\begin{tikzpicture}
		\begin{axis}[
			domain=64:80,
			%axis lines=middle,
			xtick={64,66,...,80},
			ytick={-8,-7,...,6},
			xmin=64, xmax=80,
			ymin=-6, ymax=6,
			%x label style={at={(axis description cs: 0.5, -0.1)}, anchor=north},
			%y label style={at={(axis description cs:-0.1,1)},anchor=south},
			grid = major,
			xlabel={Курс в дату экспирации},
			ylabel={Прибыль (profit'n'loss)},
			%scaled x ticks=false
		]
		
	  \addplot[Set1-A, very thick, dashed] {(\x > 70)*(\x - 70) - 0.5};
  \addplot[Set1-B, very thick, dashed] {-(\x > 74)*(\x - 74) + 0.25};
  \addplot[Set1-C, very thick] {(\x > 70)*(\x - 70) - 0.5 -(\x > 74)*(\x - 74) + 0.25};
  
  \draw[thick, color=black] (axis cs: 60, 0) -- (axis cs: 90, 0);
\end{axis}
\end{tikzpicture}
\end{frame}



\begin{frame}{Комбинации: стрэддл}
\justifying
Стрэддл (\en{straddle}) --- купленные колл и пут с одинаковыми страйками.

\justifying
\centering
	\begin{tikzpicture}
		\begin{axis}[
			domain=64:80,
			%axis lines=middle,
			xtick={64,66,...,80},
			ytick={-8,-7,...,6},
			xmin=64, xmax=80,
			ymin=-6, ymax=6,
			%x label style={at={(axis description cs: 0.5, -0.1)}, anchor=north},
			%y label style={at={(axis description cs:-0.1,1)},anchor=south},
			grid = major,
			xlabel={Курс в дату экспирации},
			ylabel={Прибыль (profit'n'loss)},
			%scaled x ticks=false
		]
		
	  \addplot[Set1-A, very thick, dashed] {(\x < 72)*(72 - \x) - 0.5};
  \addplot[Set1-B, very thick, dashed] {(\x > 72)*(\x - 72) - 0.5};
  \addplot[Set1-C, very thick] {(\x < 72)*(72 - \x) - 0.5 + (\x > 72)*(\x - 72) - 0.5};
  
  \draw[thick, color=black] (axis cs: 60, 0) -- (axis cs: 90, 0);
\end{axis}
\end{tikzpicture}
\end{frame}



\begin{frame}{Комбинации: колл-опцион и пут-опцион}
\justifying
Купленные колл и проданный пут с одинаковым страйком (без учёта премии).

\justifying
\centering
	\begin{tikzpicture}
		\begin{axis}[
			domain=64:80,
			%axis lines=middle,
			xtick={64,66,...,80},
			ytick={-6,-5,...,6},
			xmin=64, xmax=80,
			ymin=-6, ymax=6,
			%x label style={at={(axis description cs: 0.5, -0.1)}, anchor=north},
			%y label style={at={(axis description cs:-0.1,1)},anchor=south},
			grid = major,
			xlabel={Курс в дату экспирации},
			ylabel={Выплата (payoff)},
			%scaled x ticks=false
		]
		
	\addplot[Set1-A, very thick, dashed] {(\x > 72)*(\x - 72) + 0.1};
  	\addplot[Set1-B, very thick, dashed] {-(\x < 72)*(72-\x) - 0.1};
  	\addplot[Set1-C, very thick] {(\x > 72)*(\x - 72) -(\x < 72)*(72-\x) };
 
   \draw[thick, color=black] (axis cs: 60, 0) -- (axis cs: 90, 0);
\end{axis}
\end{tikzpicture}
\end{frame}



\begin{frame}{Комбинации: базовый актив и долг}
\justifying
1 доллар и долг в 72 рубля.

\justifying
\centering
	\begin{tikzpicture}
		\begin{axis}[
			domain=64:80,
			%axis lines=middle,
			xtick={64,66,...,80},
			ytick={-6,-5,...,6},
			xmin=64, xmax=80,
			ymin=-6, ymax=6,
			%x label style={at={(axis description cs: 0.5, -0.1)}, anchor=north},
			%y label style={at={(axis description cs:-0.1,1)},anchor=south},
			grid = major,
			xlabel={Курс в дату экспирации},
			ylabel={Выплата (payoff)},
			%scaled x ticks=false
		]
		
  	\addplot[Set1-C, very thick] {\x - 72};
 
   \draw[thick, color=black] (axis cs: 60, 0) -- (axis cs: 90, 0);
\end{axis}
\end{tikzpicture}
\end{frame}



\begin{frame}{Паритет опционов колл и пут}
\justify
Следующие два портфеля дают одинаковую выплату в день экспирации $T$:
\begin{itemize}
\justifying
\item Купленный колл со страйком $K$ и проданный пут со страйком $K$.
\item Базовый актив ценой $S(T)$ и $K$ денег на счету.
\end{itemize}

\justify
Следовательно, в любой день $t$ до даты экспирации ($t<T$) они тоже должны стоить одинаковых денег с учётом ставки дисконтирования $r$:
\begin{align*}
C_K(t) - P_K(t) = S(t) + Ke^{-r(T-t)}
\end{align*}

\justify
Это свойство называется \alert{паритет опционов колл и пут} (\en{call-put parity}). Оно позволяет выразить цену колла через пут или наоборот.
\end{frame}

\end{document}