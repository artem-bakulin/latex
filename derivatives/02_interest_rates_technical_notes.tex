\documentclass{beamer}

\usepackage{cmap}				% To be able to copy-paste russian text from pdf
\usepackage[T2A]{fontenc}
\usepackage[utf8]{inputenc}
\usepackage[russian]{babel}
\usepackage{textpos}
\usepackage{ragged2e}
\usepackage{amssymb}
\usepackage{ulem}
\usepackage{tikz}
\usepackage{pgfplots}
\usepackage{color}
\usepackage{cancel}
\usepackage{multirow}
\pgfplotsset{compat=1.17}
\usetikzlibrary{arrows,snakes,backgrounds,shapes}
\usepgfplotslibrary{groupplots,colorbrewer,dateplot,statistics}
\usepackage{animate}

\usepackage{amsfonts}
\usepackage{amsmath}
\usepackage{amssymb}
\usepackage{graphicx}
\usepackage{setspace}

\usepackage{enumitem}
\setitemize{label=\usebeamerfont*{itemize item}%
  \usebeamercolor[fg]{itemize item}
  \usebeamertemplate{itemize item}}

% remove navigation bar
\setbeamertemplate{navigation symbols}{} 

\usepackage{eurosym}
\renewcommand{\EUR}[1]{\textup{\euro}#1}

\title{Технические детали процентных ставок}
\author{Артём Бакулин}
\date{11 марта 2024 г.}

\usetheme{Warsaw}
\usecolortheme{beaver}

% remove navigation bar
\setbeamertemplate{navigation symbols}{}

\setbeamertemplate{page number in head/foot}[totalframenumber] 

\newcommand{\ru}[1]{\begin{otherlanguage}{russian}#1\end{otherlanguage}}
\newcommand{\en}[1]{\begin{otherlanguage}{english}#1\end{otherlanguage}}
\newcommand{\ruen}[2]{#1 (\en{#2})}

\begin{document}



\begin{frame}
\titlepage
\end{frame}

\begin{frame}{Капитализация процентов}
\justify
<<Сложные проценты --- самая могущественная формула сила во Вселенной>> 
(ошибочно приписывается Альберту Эйнштейну).

\justify
Пусть депозит под ставку (\en{rate}) $r$ действует (\en{term}) $T$ лет с 
частотой капитализации (\en{frequency})\ $f$ раз в год. Начальная сумма вклада 
$N_0$. В конце срока мы получим
\begin{align*}
N &= N_0\underbrace{\left(1 + \frac{r}{f}\right) \cdot \left(1 + \frac{r}{f}\right) \cdot ... \cdot \left(1 + \frac{r}{f}\right)}_{fT \text{\ множителей}} = 
N_0\left(1 + \frac{r}{f}\right)^{fT}
\end{align*}

Например, депозит на 1\,000\,000 под 12\% на 1 год с ежемесячной капитализацией:
\begin{align*}
1\,000\,000 \cdot \left(1 + \frac{0.12}{12}\right)^{12} \approx 1\,126\,825
\end{align*}
\end{frame}



\begin{frame}{Общеупотребительные методы капитализации}
\centering
\begin{tabular}{l|l|l}
Частота ($f$) & Русское название & Английское название \\ \hline 
0 & Без капитализации* & Zero \\
1 & Ежегодно & Annually \\
2 & Раз в полгода & Semi-annually \\
4 & Ежеквартально & Quarterly \\
12 & Ежемесячно & Monthly \\
365 & Ежедневно & Daily \\
$\infty$ & Непрерывно* & Continuously 
\end{tabular}

\justify
Простые проценты без капитализации (\en{zero compounding}), $f=0$:
\begin{align*}
N = N_0 (1 + rT)
\end{align*}

Непрерывная капитализация (\en{continuous compounding}), $f=\infty$:
\begin{align*}
N = N_0e^{rT}
\end{align*}
\end{frame}



\begin{frame}{Непрерывная капитализация процентов}
\justify
Предположим, что совсем-совсем небольшой процент прибавляется ко вкладу каждый 
час, каждую минуту, каждую миллисекунду. Частота капитализации $f$ стремится к 
бесконечности.
\begin{align*}
N &= N_0 \lim_{f \to +\infty} \left(1 + \frac{r}{f}\right)^{fT} = 
\begin{Bmatrix}x= f/r \\ f=xr \\ x \to +\infty \\ f \to +\infty\end{Bmatrix} = \\
&= N_0\lim_{x \to +\infty} \left(1 + \frac{r}{xr}\right) ^ {xrT} =
N_0{\underbrace{\left(\lim_{x \to +\infty} \left( 1 + \frac{1}{x} \right) ^ x\right)}_{e \approx 2.71828}}  ^ {rT} = \\
&= N_0e^{rT} \approx N_0(1 + rT)
\end{align*}
<<Непрерывные>> проценты не встречаются в реальной жизни, но заметно упрощают 
формулы в теоретических моделях.
\end{frame}



\begin{frame}{Капитализация процентов - 1}
\justify
Рассмотрим депозиты на 1\,000\,000 рублей под ставку 12\% на срок 1 год и 5 лет. 
Как конечная сумма вклада зависит от капитализации процентов?


\begin{table}
\centering
\begin{tabular}{l|r|r|r}
Начисление процентов & $f$ & 1 год & 5 лет \\ \hline
Без капитализации & 0         & 1\,120\,000 & 1\,600\,000 \\
Ежегодно          & 1         & 1\,120\,000 & 1\,762\,342 \\
Раз в полгода     & 2         & 1\,123\,600 & 1\,790\,848 \\
Ежеквартально     & 4         & 1\,125\,509 & 1\,806\,111 \\
Ежемесячно        & 12        & 1\,126\,825 & 1\,816\,697 \\
Ежедневно         & 365       & 1\,127\,475 & 1\,821\,939 \\
Непрерывно        & $+\infty$ & 1\,127\,497 & 1\,822\,119
\end{tabular}
\end{table}

\justifying
Чем выше ставка и дольше срок, тем важнее роль капитализации. Ежедневная капитализация процентов достаточно близка к непрерывной.
\end{frame}



\begin{frame}{Капитализация процентов - 2}

	\centering
	\begin{tikzpicture}
		\begin{axis}[
			xlabel=\text{Время, месяцы},
			ylabel=\text{Сумма вклада, рубли},
			xmin=0,
			xmax=24.5,
			width=\textwidth,
			height=\textheight - 1.5cm,
			ymin=100,
			ymax=130,
			legend style={at={(0.01, 0.99)}, anchor=north west}
		]
	

		\addplot[const plot, samples at={0,0.1,...,60}, very thick, color=Set1-A] {100 * (1 + 0.12/4) ^ (4*3*floor(x/3)/12.0)};
		\addlegendentry{$f=4$}
	
		\addplot[const plot, samples at={0,0.5,...,60}, very thick, color=Set1-B] {100 * (1 + 0.12/12) ^ (12*floor(x)/12.0)};
		\addlegendentry{$f=12$}
		
		\addplot[domain=0:60, color=Set1-C, very thick] {100 * exp(0.12*x/12)};
		\addlegendentry{$f=+\infty$}
		\end{axis}
	\end{tikzpicture}
	\scriptsize{Рост вклада с течением времени, $N_0=100$, $r=12\%$}
\end{frame}



\begin{frame}{Соглашение о подсчёте дней}
\justify
Сколько лет между 1 марта 2024 года и 1 апреля 2024 года?

\centering
\begin{tabular}{l|l}
Конвенция & Кол-во лет \\ \hline
ACT/365 & $31/365 \approx 0.08493$ \\
ACT/360 & $31/360 \approx 0.08611$ \\
30/360  & $30/360 \approx 0.08333$ \\
BUS/252 & $21/252 \approx 0.08333$ 
\end{tabular}

\justify
ACT --- считаем календарные дни. 30 --- считаем, что в месяце 30 дней. BUS --- считаем только рабочие дни.

\justify
Выбор способа начисления процентов --- рыночная конвенция. Правило большого пальца: в странах Британского Содружества ACT/365, в других странах --- ACT/360, в Бразилии --- BUS/252. Читайте мелкий шрифт! 
\end{frame}



\begin{frame}{Конвенция 30/360}
\justify
Сколько лет между датами $D_1.M_1.Y_1$ и $D_2.M_2.Y_2$?
\begin{align*}
T = \frac{360(Y_2-Y_1) + 30(M_2-M_1) + (D_2-D_1)}{360}
\end{align*}

\justify
Конвенция 30/360 возникла в докомпьютерную эпоху и позволяет быстро прикинуть результат в уме. Она по-прежнему живее всех живых (например, на рынке облигаций).
\begin{itemize}
\item Обратная совместимость
\item Размер платежей по кредитам не меняется в течение года
\end{itemize}

\vspace{\baselineskip}
\justify
*Несколько вариантов (30/360 Bond, 30/360 US, 30E/360, 30E/360 ISDA) отличаются тем, как они обрабатывают февраль и месяцы с 31 днями. Скажем спасибо библиотекам финансовых вычислений за то, что прячут от нас эти детали!
\end{frame}



\begin{frame}{Конвенция о переносе дней}
\justify
Кредит на три месяца начался 29 марта 2024 года (пт). Когда он закончится: 28 июня (пт) или 1 июля (пн)?

\justify
Конвенция \en{modified following / business month end}:

\justify
1) Если дата начала --- последний рабочий день текущего месяца, то дата окончания --- тоже последний рабочий день месяца. Пример: 29.02.2024 (чт) --- 31.05.2024 (пт).

\justify
2) Если дата окончания выпала на выходной, то она сдвигается вперёд на ближайший 
рабочий день, если только ближайший рабочий день --- не в следующем месяце. Пример: 
15.03.2024 (пт) --- 17.06.2024 (пн).

\justify
3) В противном случае дата окончания сдвигается назад. Пример: 29.03.2024 (пт) --- 28.06.2024 (пт).
\end{frame}



\begin{frame}{Переход между способами капитализации}
\justify
Если вам нужна <<непрерывная>> ставка $r^*$, а есть только <<обычная>> $r$ (или наоборот), то всегда можно выразить одну через другую.
\begin{align*}
e^{r^*T} &= \left(1 + \frac{r}{f}\right)^{fT} \\
r^* &= f\ln \left(1 + \frac{r}{f}\right) \\
r &= f\left(e^{r^*/f} - 1\right)
\end{align*}

\justify
То же самое для <<простых>> процентов без капитализации:
\begin{align*}
e^{r^*T} &= 1+rT \\
r^* &= \frac{\ln(1+rT)}{T} \\
r &= \frac{e^{r^*T}-1}{T}
\end{align*}

\end{frame}



\begin{frame}{Отложенный депозит}
\justify
Предположим, что на рынке доступны два депозита с <<нулевой>> капитализацией:
\begin{itemize}
\item На $T_1=1$ год под ставку $r_1=12\%$ 
\item На $T_2=2$ года под ставку $r_2=8\%$ 
\end{itemize}

Банк предлагает <<отложенный>> депозит: вы обязаны принести деньги через год на 1 год, а банк обязан зафиксировать ставку по депозиту $x$ уже сегодня. На какую ставку вы согласитесь?

\centering
\begin{tikzpicture}
		\draw [->,>=triangle 90] (0, 0) -- (8.5, 0);

		\draw [->,>=triangle 45] (1,0) node[anchor=north east]{$0$} .. controls (1.5, 1) and (3.5, 1) .. (4,0) node[anchor=north]{$T_1=1$} node[pos=0.5,anchor=south]{$r_1=12\%$};

		\draw [->,>=triangle 45] (4,0) .. controls (4.5, 1) and (6.5, 1) .. (7,0) node[anchor=north west]{$T_2=2$} node[pos=0.5,anchor=south]{$x$};

		\draw [->,>=triangle 45] (1,0) .. controls (1.5, -1) and (6.5, -1) .. (7,0) node[pos=0.5,anchor=north]{$r_2=8\%$};
	\end{tikzpicture}
\end{frame}



\begin{frame}{Отложенный депозит - 2}
\centering
\begin{tikzpicture}
		\draw [->,>=triangle 90] (0, 0) -- (8.5, 0);

		\draw [->,>=triangle 45] (1,0) node[anchor=north east]{$0$} .. controls (1.5, 1) and (3.5, 1) .. (4,0) node[anchor=north]{$T_1=1$} node[pos=0.5,anchor=south]{$r_1=12\%$};

		\draw [->,>=triangle 45] (4,0) .. controls (4.5, 1) and (6.5, 1) .. (7,0) node[anchor=north west]{$T_2=2$} node[pos=0.5,anchor=south]{$x$};

		\draw [->,>=triangle 45] (1,0) .. controls (1.5, -1) and (6.5, -1) .. (7,0) node[pos=0.5,anchor=north]{$r_2=8\%$};
	\end{tikzpicture}
	
\justifying
Не должно быть разницы между депозитом на два года и цепочкой из двух депозитов:
\begin{align*}
(1 + r_1T_1)\Big(1+x(T_2-T_1)\Big) = 1 + r_2T_2
\end{align*}
\begin{align*}
x = \frac{\dfrac{1+r_2T_2}{1+r_1T_1} - 1}{T_2-T_1} = \frac{\dfrac{1 + 0.08 \cdot 2}{1 + 0.12} - 1}{2-1} \approx 3.57\%
\end{align*}

При любой другой ставке <<отложенный>> депозит будет не выгоден либо вкладчикам (и они не будут им пользоваться), либо банку (и он не будет его предлагать).
\end{frame}



\begin{frame}{Непрерывная капитализация процентов}
\justify
Пересчитаем простые проценты в непрерывные:
\begin{align*}
r_1^* &= \frac{\ln(1+r_1T_1)}{T_1} = \frac{\ln 1.12}{1} \approx 11.333\% \\
r_2^* &= \frac{\ln(1+r_2T_2)}{T_2} = \frac{\ln 1.14}{2} \approx 7.421\%
\end{align*}

\centering
\begin{tikzpicture}
		\draw [->,>=triangle 90] (0, 0) -- (8.5, 0);

		\draw [->,>=triangle 45] (1,0) node[anchor=north east]{$0$} .. controls (1.5, 1) and (3.5, 1) .. (4,0) node[anchor=north]{$T_1=1$} node[pos=0.5,anchor=south]{$r_1^*=11.333\%$};

		\draw [->,>=triangle 45] (4,0) .. controls (4.5, 1) and (6.5, 1) .. (7,0) node[anchor=north west]{$T_2=2$} node[pos=0.5,anchor=south]{$x^*$};

		\draw [->,>=triangle 45] (1,0) .. controls (1.5, -1) and (6.5, -1) .. (7,0) node[pos=0.5,anchor=north]{$r_2^*=7.421\%$};
	\end{tikzpicture}
\end{frame}



\begin{frame}{Непрерывная капитализация процентов - 2}
\centering
\begin{tikzpicture}
		\draw [->,>=triangle 90] (0, 0) -- (8.5, 0);

		\draw [->,>=triangle 45] (1,0) node[anchor=north east]{$0$} .. controls (1.5, 1) and (3.5, 1) .. (4,0) node[anchor=north]{$T_1=1$} node[pos=0.5,anchor=south]{$r_1^*=11.333\%$};

		\draw [->,>=triangle 45] (4,0) .. controls (4.5, 1) and (6.5, 1) .. (7,0) node[anchor=north west]{$T_2=2$} node[pos=0.5,anchor=south]{$x^*$};

		\draw [->,>=triangle 45] (1,0) .. controls (1.5, -1) and (6.5, -1) .. (7,0) node[pos=0.5,anchor=north]{$r_2^*=7.421\%$};
	\end{tikzpicture}

\justifying
Не должно быть разницы между цепочкой из <<непрерывных>> годовых депозитов и двухгодовым депозитом:
\begin{align*}
e^{r_1^*T_1}e^{x^*(T_2-T_1)} = e^{r_2^*T_2}
\end{align*}
\begin{align*}
x^* = \frac{r_2^*T_2-r_1^*T_1}{T_2-T_1} = \frac{7.421\%\cdot2 - 11.333\%}{2 - 1} \approx 3.509\%
\end{align*}

Пересчитаем <<непрерывные>> обратно в простые:
\begin{align*}
x = \frac{e^{x^*(T_2-T_1)} - 1}{T_2 - T_1} = e^{0.03509} - 1 \approx 3.57\% 
\end{align*}
\end{frame}



\begin{frame}{Непрерывная капитализация процентов - 3}
\justify
Наблюдаемая действительность --- то, что люди готовы вложить $100$ рублей, чтобы получить $112$ рублей через год. Это можно интерпретировать и как <<простую>> ставку 12\% , и как <<непрерывную>> ставку 11.33\%, и как коэффициент дисконтирования $1/1.12 = 0.8929$. Это разные точки зрения (или разные единицы измерения) на одно и то же явление.

\justify 
В зависимости от контекста может быть удобнее пользоваться разными абстракциями. Можно показывать пользователям <<простые>> проценты, математику будущих ставок программировать в терминах <<непрерывных>> ставок, а \en{present value}\ будущих выплат вычислять с помощью коэффициентов дисконтирования.
\end{frame}

\end{document}